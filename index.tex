% Options for packages loaded elsewhere
\PassOptionsToPackage{unicode}{hyperref}
\PassOptionsToPackage{hyphens}{url}
\PassOptionsToPackage{dvipsnames,svgnames,x11names}{xcolor}
%
\documentclass[
  letterpaper,
  DIV=11,
  numbers=noendperiod]{scrreprt}

\usepackage{amsmath,amssymb}
\usepackage{iftex}
\ifPDFTeX
  \usepackage[T1]{fontenc}
  \usepackage[utf8]{inputenc}
  \usepackage{textcomp} % provide euro and other symbols
\else % if luatex or xetex
  \usepackage{unicode-math}
  \defaultfontfeatures{Scale=MatchLowercase}
  \defaultfontfeatures[\rmfamily]{Ligatures=TeX,Scale=1}
\fi
\usepackage{lmodern}
\ifPDFTeX\else  
    % xetex/luatex font selection
\fi
% Use upquote if available, for straight quotes in verbatim environments
\IfFileExists{upquote.sty}{\usepackage{upquote}}{}
\IfFileExists{microtype.sty}{% use microtype if available
  \usepackage[]{microtype}
  \UseMicrotypeSet[protrusion]{basicmath} % disable protrusion for tt fonts
}{}
\makeatletter
\@ifundefined{KOMAClassName}{% if non-KOMA class
  \IfFileExists{parskip.sty}{%
    \usepackage{parskip}
  }{% else
    \setlength{\parindent}{0pt}
    \setlength{\parskip}{6pt plus 2pt minus 1pt}}
}{% if KOMA class
  \KOMAoptions{parskip=half}}
\makeatother
\usepackage{xcolor}
\setlength{\emergencystretch}{3em} % prevent overfull lines
\setcounter{secnumdepth}{5}
% Make \paragraph and \subparagraph free-standing
\makeatletter
\ifx\paragraph\undefined\else
  \let\oldparagraph\paragraph
  \renewcommand{\paragraph}{
    \@ifstar
      \xxxParagraphStar
      \xxxParagraphNoStar
  }
  \newcommand{\xxxParagraphStar}[1]{\oldparagraph*{#1}\mbox{}}
  \newcommand{\xxxParagraphNoStar}[1]{\oldparagraph{#1}\mbox{}}
\fi
\ifx\subparagraph\undefined\else
  \let\oldsubparagraph\subparagraph
  \renewcommand{\subparagraph}{
    \@ifstar
      \xxxSubParagraphStar
      \xxxSubParagraphNoStar
  }
  \newcommand{\xxxSubParagraphStar}[1]{\oldsubparagraph*{#1}\mbox{}}
  \newcommand{\xxxSubParagraphNoStar}[1]{\oldsubparagraph{#1}\mbox{}}
\fi
\makeatother

\usepackage{color}
\usepackage{fancyvrb}
\newcommand{\VerbBar}{|}
\newcommand{\VERB}{\Verb[commandchars=\\\{\}]}
\DefineVerbatimEnvironment{Highlighting}{Verbatim}{commandchars=\\\{\}}
% Add ',fontsize=\small' for more characters per line
\usepackage{framed}
\definecolor{shadecolor}{RGB}{241,243,245}
\newenvironment{Shaded}{\begin{snugshade}}{\end{snugshade}}
\newcommand{\AlertTok}[1]{\textcolor[rgb]{0.68,0.00,0.00}{#1}}
\newcommand{\AnnotationTok}[1]{\textcolor[rgb]{0.37,0.37,0.37}{#1}}
\newcommand{\AttributeTok}[1]{\textcolor[rgb]{0.40,0.45,0.13}{#1}}
\newcommand{\BaseNTok}[1]{\textcolor[rgb]{0.68,0.00,0.00}{#1}}
\newcommand{\BuiltInTok}[1]{\textcolor[rgb]{0.00,0.23,0.31}{#1}}
\newcommand{\CharTok}[1]{\textcolor[rgb]{0.13,0.47,0.30}{#1}}
\newcommand{\CommentTok}[1]{\textcolor[rgb]{0.37,0.37,0.37}{#1}}
\newcommand{\CommentVarTok}[1]{\textcolor[rgb]{0.37,0.37,0.37}{\textit{#1}}}
\newcommand{\ConstantTok}[1]{\textcolor[rgb]{0.56,0.35,0.01}{#1}}
\newcommand{\ControlFlowTok}[1]{\textcolor[rgb]{0.00,0.23,0.31}{\textbf{#1}}}
\newcommand{\DataTypeTok}[1]{\textcolor[rgb]{0.68,0.00,0.00}{#1}}
\newcommand{\DecValTok}[1]{\textcolor[rgb]{0.68,0.00,0.00}{#1}}
\newcommand{\DocumentationTok}[1]{\textcolor[rgb]{0.37,0.37,0.37}{\textit{#1}}}
\newcommand{\ErrorTok}[1]{\textcolor[rgb]{0.68,0.00,0.00}{#1}}
\newcommand{\ExtensionTok}[1]{\textcolor[rgb]{0.00,0.23,0.31}{#1}}
\newcommand{\FloatTok}[1]{\textcolor[rgb]{0.68,0.00,0.00}{#1}}
\newcommand{\FunctionTok}[1]{\textcolor[rgb]{0.28,0.35,0.67}{#1}}
\newcommand{\ImportTok}[1]{\textcolor[rgb]{0.00,0.46,0.62}{#1}}
\newcommand{\InformationTok}[1]{\textcolor[rgb]{0.37,0.37,0.37}{#1}}
\newcommand{\KeywordTok}[1]{\textcolor[rgb]{0.00,0.23,0.31}{\textbf{#1}}}
\newcommand{\NormalTok}[1]{\textcolor[rgb]{0.00,0.23,0.31}{#1}}
\newcommand{\OperatorTok}[1]{\textcolor[rgb]{0.37,0.37,0.37}{#1}}
\newcommand{\OtherTok}[1]{\textcolor[rgb]{0.00,0.23,0.31}{#1}}
\newcommand{\PreprocessorTok}[1]{\textcolor[rgb]{0.68,0.00,0.00}{#1}}
\newcommand{\RegionMarkerTok}[1]{\textcolor[rgb]{0.00,0.23,0.31}{#1}}
\newcommand{\SpecialCharTok}[1]{\textcolor[rgb]{0.37,0.37,0.37}{#1}}
\newcommand{\SpecialStringTok}[1]{\textcolor[rgb]{0.13,0.47,0.30}{#1}}
\newcommand{\StringTok}[1]{\textcolor[rgb]{0.13,0.47,0.30}{#1}}
\newcommand{\VariableTok}[1]{\textcolor[rgb]{0.07,0.07,0.07}{#1}}
\newcommand{\VerbatimStringTok}[1]{\textcolor[rgb]{0.13,0.47,0.30}{#1}}
\newcommand{\WarningTok}[1]{\textcolor[rgb]{0.37,0.37,0.37}{\textit{#1}}}

\providecommand{\tightlist}{%
  \setlength{\itemsep}{0pt}\setlength{\parskip}{0pt}}\usepackage{longtable,booktabs,array}
\usepackage{calc} % for calculating minipage widths
% Correct order of tables after \paragraph or \subparagraph
\usepackage{etoolbox}
\makeatletter
\patchcmd\longtable{\par}{\if@noskipsec\mbox{}\fi\par}{}{}
\makeatother
% Allow footnotes in longtable head/foot
\IfFileExists{footnotehyper.sty}{\usepackage{footnotehyper}}{\usepackage{footnote}}
\makesavenoteenv{longtable}
\usepackage{graphicx}
\makeatletter
\newsavebox\pandoc@box
\newcommand*\pandocbounded[1]{% scales image to fit in text height/width
  \sbox\pandoc@box{#1}%
  \Gscale@div\@tempa{\textheight}{\dimexpr\ht\pandoc@box+\dp\pandoc@box\relax}%
  \Gscale@div\@tempb{\linewidth}{\wd\pandoc@box}%
  \ifdim\@tempb\p@<\@tempa\p@\let\@tempa\@tempb\fi% select the smaller of both
  \ifdim\@tempa\p@<\p@\scalebox{\@tempa}{\usebox\pandoc@box}%
  \else\usebox{\pandoc@box}%
  \fi%
}
% Set default figure placement to htbp
\def\fps@figure{htbp}
\makeatother
% definitions for citeproc citations
\NewDocumentCommand\citeproctext{}{}
\NewDocumentCommand\citeproc{mm}{%
  \begingroup\def\citeproctext{#2}\cite{#1}\endgroup}
\makeatletter
 % allow citations to break across lines
 \let\@cite@ofmt\@firstofone
 % avoid brackets around text for \cite:
 \def\@biblabel#1{}
 \def\@cite#1#2{{#1\if@tempswa , #2\fi}}
\makeatother
\newlength{\cslhangindent}
\setlength{\cslhangindent}{1.5em}
\newlength{\csllabelwidth}
\setlength{\csllabelwidth}{3em}
\newenvironment{CSLReferences}[2] % #1 hanging-indent, #2 entry-spacing
 {\begin{list}{}{%
  \setlength{\itemindent}{0pt}
  \setlength{\leftmargin}{0pt}
  \setlength{\parsep}{0pt}
  % turn on hanging indent if param 1 is 1
  \ifodd #1
   \setlength{\leftmargin}{\cslhangindent}
   \setlength{\itemindent}{-1\cslhangindent}
  \fi
  % set entry spacing
  \setlength{\itemsep}{#2\baselineskip}}}
 {\end{list}}
\usepackage{calc}
\newcommand{\CSLBlock}[1]{\hfill\break\parbox[t]{\linewidth}{\strut\ignorespaces#1\strut}}
\newcommand{\CSLLeftMargin}[1]{\parbox[t]{\csllabelwidth}{\strut#1\strut}}
\newcommand{\CSLRightInline}[1]{\parbox[t]{\linewidth - \csllabelwidth}{\strut#1\strut}}
\newcommand{\CSLIndent}[1]{\hspace{\cslhangindent}#1}

\usepackage{booktabs}
\usepackage{longtable}
\usepackage{array}
\usepackage{multirow}
\usepackage{wrapfig}
\usepackage{float}
\usepackage{colortbl}
\usepackage{pdflscape}
\usepackage{tabu}
\usepackage{threeparttable}
\usepackage{threeparttablex}
\usepackage[normalem]{ulem}
\usepackage{makecell}
\usepackage{xcolor}
\KOMAoption{captions}{tableheading}
\makeatletter
\@ifpackageloaded{tcolorbox}{}{\usepackage[skins,breakable]{tcolorbox}}
\@ifpackageloaded{fontawesome5}{}{\usepackage{fontawesome5}}
\definecolor{quarto-callout-color}{HTML}{909090}
\definecolor{quarto-callout-note-color}{HTML}{0758E5}
\definecolor{quarto-callout-important-color}{HTML}{CC1914}
\definecolor{quarto-callout-warning-color}{HTML}{EB9113}
\definecolor{quarto-callout-tip-color}{HTML}{00A047}
\definecolor{quarto-callout-caution-color}{HTML}{FC5300}
\definecolor{quarto-callout-color-frame}{HTML}{acacac}
\definecolor{quarto-callout-note-color-frame}{HTML}{4582ec}
\definecolor{quarto-callout-important-color-frame}{HTML}{d9534f}
\definecolor{quarto-callout-warning-color-frame}{HTML}{f0ad4e}
\definecolor{quarto-callout-tip-color-frame}{HTML}{02b875}
\definecolor{quarto-callout-caution-color-frame}{HTML}{fd7e14}
\makeatother
\makeatletter
\@ifpackageloaded{bookmark}{}{\usepackage{bookmark}}
\makeatother
\makeatletter
\@ifpackageloaded{caption}{}{\usepackage{caption}}
\AtBeginDocument{%
\ifdefined\contentsname
  \renewcommand*\contentsname{Table of contents}
\else
  \newcommand\contentsname{Table of contents}
\fi
\ifdefined\listfigurename
  \renewcommand*\listfigurename{List of Figures}
\else
  \newcommand\listfigurename{List of Figures}
\fi
\ifdefined\listtablename
  \renewcommand*\listtablename{List of Tables}
\else
  \newcommand\listtablename{List of Tables}
\fi
\ifdefined\figurename
  \renewcommand*\figurename{Figure}
\else
  \newcommand\figurename{Figure}
\fi
\ifdefined\tablename
  \renewcommand*\tablename{Table}
\else
  \newcommand\tablename{Table}
\fi
}
\@ifpackageloaded{float}{}{\usepackage{float}}
\floatstyle{ruled}
\@ifundefined{c@chapter}{\newfloat{codelisting}{h}{lop}}{\newfloat{codelisting}{h}{lop}[chapter]}
\floatname{codelisting}{Listing}
\newcommand*\listoflistings{\listof{codelisting}{List of Listings}}
\makeatother
\makeatletter
\makeatother
\makeatletter
\@ifpackageloaded{caption}{}{\usepackage{caption}}
\@ifpackageloaded{subcaption}{}{\usepackage{subcaption}}
\makeatother

\usepackage{bookmark}

\IfFileExists{xurl.sty}{\usepackage{xurl}}{} % add URL line breaks if available
\urlstyle{same} % disable monospaced font for URLs
\hypersetup{
  pdftitle={Answering questions with data},
  pdfauthor={Mallory Barnes},
  colorlinks=true,
  linkcolor={blue},
  filecolor={Maroon},
  citecolor={Blue},
  urlcolor={Blue},
  pdfcreator={LaTeX via pandoc}}


\title{Answering questions with data}
\author{Mallory Barnes}
\date{2025-08-21}

\begin{document}
\maketitle
\begin{abstract}
This comprehensive resource offers a free, accessible textbook for
environmental science students embarking on introductory statistics. The
package includes a practical lab manual and a dedicated course website,
all provided under a CC BY-SA 4.0 license.
\end{abstract}

\renewcommand*\contentsname{Table of contents}
{
\hypersetup{linkcolor=}
\setcounter{tocdepth}{2}
\tableofcontents
}

\bookmarksetup{startatroot}

\chapter*{Preface}\label{preface}
\addcontentsline{toc}{chapter}{Preface}

\markboth{Preface}{Preface}

Second Draft (version 0.1 = August 18th, 2025)

Welcome to the second edition of this Open Educational Resource (OER)
textbook, specifically adapted for Environmental Science students
enrolled in the SPEA E-538 statistics course at Indiana University (IU).

This textbook is an adaptation of a thorough introductory statistics
textbook originally developed for undergraduate Psychology students by
Matthew Crump and colleagues (refer to Acknowledgements for more
details). As part of IU's Course Materials Fellowship Program
(\href{https://libraries.indiana.edu/course-material-fellowship-program-1}{CMFP}),
I've had the opportunity to mold this material, refining it to serve as
a specialized resource for students studying Environmental Science.

\textbf{Online
Textbook}:\url{https://malloryb.github.io/statistics_E538/}

\textbf{Citation for original textbook}: Crump, M. J. C., Navarro, D.
J., \& Suzuki, J. (2019, June 5). Answering Questions with Data
(Textbook): Introductory Statistics for Psychology Students.
https://doi.org/10.17605/OSF.IO/JZE52

All resources are released under a creative commons licence
\href{https://creativecommons.org/licenses/by-sa/4.0/}{CC BY-SA 4.0}.
Click the link to read more about the license, or read more below in the
license section.

\subsection*{Acknowledgements}\label{acknowledgements}
\addcontentsline{toc}{subsection}{Acknowledgements}

I wish to express my deepest appreciation to the contributors of the
original textbook, without whom this adaptation would not have been
possible. I am deeply grateful for the expertise and vision of Matthew
Crump, Alla Chavarga, Anjali Krishnan, Jeffrey Suzuki, and Stephen Volz.
Their exceptional groundwork laid the foundation for this project.

My heartfelt thanks also go to the Course Materials Fellowship Program
(CMFP) at Indiana University (IU), which has been instrumental in
supporting this adaptation effort. The CMFP is an initiative designed to
incentivize the discovery, implementation, and creation of
cost-effective course materials. Its aim is to foster the use of Open
Educational Resources (OERs)---freely accessible and customizable
learning materials that make education more equitable and accessible.

I am particularly grateful to Sarah Hare, the Bloomington lead for the
CMFP program, and Adam Mazel, a digital publishing librarian at IU.
Their guidance, expertise, and steadfast support have been crucial to
this project's success. Their contributions to the advancement of
affordable and accessible education are truly commendable.

\subsection*{CC BY-SA 4.0 license}\label{cc-by-sa-4.0-license}
\addcontentsline{toc}{subsection}{CC BY-SA 4.0 license}

This license means that you are free to:

\begin{itemize}
\tightlist
\item
  Share: copy and redistribute the material in any medium or format
\item
  Adapt: remix, transform, and build upon the material for any purpose,
  even commercially.
\end{itemize}

The licensor cannot revoke these freedoms as long as you follow the
license terms.

Under the following terms:

\begin{itemize}
\tightlist
\item
  Attribution: You must give appropriate credit, provide a link to the
  license, and indicate if changes were made. You may do so in any
  reasonable manner, but not in any way that suggests the licensor
  endorses you or your use.
\item
  ShareAlike: If you remix, transform, or build upon the material, you
  must distribute your contributions under the same license as the
  original.
\item
  No additional restrictions: You may not apply legal terms or
  technological measures that legally restrict others from doing
  anything the license permits.
\end{itemize}

\section*{Copying the textbook}\label{copying-the-textbook}
\addcontentsline{toc}{section}{Copying the textbook}

\markright{Copying the textbook}

This textbook was written in R-Studio, using R Markdown, and compiled
into a web-book format using the bookdown package. In general, I thank
the larger R community for all of the amazing tools they made, and for
making those tools open, so that I could use them to make this thing.

All of the source code for compiling the book is available from the
GitHub repository for the original textbook:

\url{https://github.com/CrumpLab/statistics}

and my github repository:

\url{https://github.com/malloryb/statistics_E538}

In principle, anyone can fork or download the E-538 textbook, or the
original textbook, which is what I did. You can load the Rproj file in
RStudio, compile the entire book, and then edit the individual .Rmd
files for content and style to fit your needs.

If you'd like to contribute to this version, you can submit pull
requests on GitHub or use the issues tab to share suggestions.

\subsection*{The vision behind this
textbook}\label{the-vision-behind-this-textbook}
\addcontentsline{toc}{subsection}{The vision behind this textbook}

The aim of this textbook is twofold: to make core statistical concepts
accessible to Environmental Science students, and to promote the use of
open-source tools like R as flexible, transparent resources for learning
and research.

\bookmarksetup{startatroot}

\chapter{Why Statistics?}\label{why-statistics}

Adapted to environmental science by Mallory Barnes. Portions adapted
from Chapters 1 and 2 in Navarro, D. J. ``Learning Statistics with R.''
\url{https://compcogscisydney.org/learning-statistics-with-r/}

\hfill\break

\begin{quote}
To call in statisticians after the experiment is done may be no more
than asking them to perform a post-mortem examination: They may be able
to say what the experiment died of.*\\
-- Sir Ronald Fisher
\end{quote}

\section{The Role of Statistics in Environmental
Science}\label{the-role-of-statistics-in-environmental-science}

Many students come into statistics classes with a mix of nerves and low
expectations. It is not always the most eagerly anticipated part of an
environmental science degree. But statistics is one of the most
important tools we have for understanding complex systems. Environmental
data are messy: ecosystems shift, climate varies, and our judgments are
often biased. Without statistical methods we risk telling convenient
stories. With them, we can detect real patterns, test competing
explanations, and make stronger arguments for environmental decisions.

\subsection{Why Numbers Matter}\label{why-numbers-matter}

Common sense is useful, but it is prone to bias. When we already believe
something to be true, we are more likely to interpret new evidence as
supporting it, even if it does not. This tendency can lead us astray in
environmental science.

For example, if you are convinced that industrial farming is the main
driver of bee declines, you might see every new drop in bee abundance as
confirmation of that belief. Maybe you are right, but maybe disease or
weather are stronger contributors. Statistics gives us tools to test
these ideas systematically. It helps us separate patterns from noise,
avoid being misled by our own expectations, and draw conclusions that
are more likely to hold up under scrutiny. It's not magic, but it makes
our conclusions far more reliable.

Another challenge is that data can tell very different stories depending
on how they are aggregated. \textbf{Simpson's paradox} occurs when a
trend seen in combined data reverses after you split the data into
meaningful groups.

Here's a recent example. During the COVID-19 pandemic, early data showed
that Italy's overall case fatality rate was higher than China's. But,
once researchers \emph{disaggregated by age group,} the trend flipped:
within every age group, the fatality rate was actually higher in China.
The apparent contradiction was explained by differences in the age
distribution of cases (von Kügelgen, Gresele, and Schölkopf 2021).

Using case fatality rate (CFR), here are some illustrative numbers (made
up for teaching).

\begin{figure}

\centering{

\includegraphics[width=0.75\linewidth,height=\textheight,keepaspectratio]{01-Science_Data_files/figure-pdf/fig-simpson-covid-overall-1.pdf}

}

\caption{\label{fig-simpson-covid-overall}Crude case fatality rates.
Italy appears to have a higher fatality rate than China when all cases
are combined.}

\end{figure}%

At face value, Italy looks worse. But what happens when we stratify by
age group? Because age strongly affects COVID fatality, splitting cases
into 10-year bands reveals a different picture. When we do, China's
fatality rate is higher in every band.

\begin{figure}

\centering{

\includegraphics[width=0.75\linewidth,height=\textheight,keepaspectratio]{01-Science_Data_files/figure-pdf/fig-simpson-by-age-group-1.pdf}

}

\caption{\label{fig-simpson-by-age-group}Fatality rates by age group
(same made-up numbers). Within each age band, China's CFR exceeds
Italy's.}

\end{figure}%

Why the reversal? Italy's cases skew older while China's skew younger,
and baseline risk rises steeply with age. The differing age
distributions weight the overall averages in opposite ways. Aggregating
without age adjustment hid the within-group pattern.

\begin{figure}

\centering{

\includegraphics[width=0.75\linewidth,height=\textheight,keepaspectratio]{01-Science_Data_files/figure-pdf/fig-age-distributions-1.pdf}

}

\caption{\label{fig-age-distributions}Case age distribution by country.
Italy's cases are older on average; China's are younger.}

\end{figure}%

\begin{tcolorbox}[enhanced jigsaw, title=\textcolor{quarto-callout-note-color}{\faInfo}\hspace{0.5em}{Takeaway}, colframe=quarto-callout-note-color-frame, colbacktitle=quarto-callout-note-color!10!white, bottomtitle=1mm, leftrule=.75mm, rightrule=.15mm, titlerule=0mm, arc=.35mm, colback=white, opacitybacktitle=0.6, toprule=.15mm, toptitle=1mm, bottomrule=.15mm, coltitle=black, breakable, left=2mm, opacityback=0]

Statistics will not answer every scientific question, but it gives us a
disciplined way to sort signal from bias so our inferences age well.

\end{tcolorbox}

\subsection{What Statistics Add}\label{what-statistics-add}

Environmental science produces a \emph{lot} of data. Every day
satellites measure surface temperature, field stations log rainfall, and
sensors track air quality. A single study can generate thousands of rows
of numbers. Without statistical tools, these numbers would be
overwhelming. With them, we can summarize, compare, and test ideas
systematically.

Statistics matters here because environmental systems are messy. River
flow changes by the hour, forests grow unevenly, and human actions layer
on additional variability. Patterns are rarely obvious by eye.
Statistics helps us sort signal from noise and ask: is this apparent
change real, or just random variation?

You might think, ``can't a statistician just handle the math?'' But
knowing the basics yourself is essential for three reasons:

\begin{enumerate}
\def\labelenumi{\arabic{enumi}.}
\item
  \textbf{Design and analysis go together.} A good study starts long
  before you run a statistical test. If you want to study how fertilizer
  affects crop yields, your sampling plan and your analysis are
  inseparable. Poor design, like measuring only in unusually wet fields,
  can't be rescued by sophisticated analysis.
\item
  \textbf{Understanding the Science:} Scientific papers on climate
  change, biodiversity, or pollution are built on statistical results.
  To interpret them, you need to know what the numbers mean.
\item
  \textbf{Practicality:} Hiring a specialist for every question isn't
  realistic. A working knowledge of statistics makes you more
  self-sufficient as a scientist.
\end{enumerate}

\begin{quote}
\emph{``We are drowning in information, but we are starved for
knowledge''}

-- Various authors, original probably John Naisbitt
\end{quote}

Finally, statistics is not only for researchers. Weather forecasts, air
quality alerts, and wildlife population trends are all communicated
through numbers. A basic knowledge of statistics helps you tell whether
those numbers support the claims being made, or whether someone is
stretching the truth. In that sense, statistics is part of being an
informed citizen in a data-saturated world.

\section{Introduction to
measurements}\label{introduction-to-measurements}

Every dataset begins with measurement. Measurement means assigning
numbers, labels, or categories to aspects of the world so they can be
recorded and analyzed.

\textbf{Why measurement matters}

Environmental science often deals with broad ideas that need to be
defined before they can be studied. For example, \emph{soil health} or
\emph{forest change} are meaningful, but vague. To analyze them, we have
to decide exactly what we mean and how to capture it. That process is
called \textbf{operationalization}: turning a general concept into
something measurable.

Before moving on, let's clarify how operationalization fits with three
related terms:

\begin{itemize}
\tightlist
\item
  \textbf{Operationalization}: the step where you define exactly how a
  broad idea will be captured with a measure
\item
  \textbf{Measure}: the tool or method used to make observations
\item
  \textbf{Variable}: the actual values you record once the measure is
  applied. Variables are the actual data that end up in our dataset.
\end{itemize}

\textbf{Examples:}

\begin{itemize}
\tightlist
\item
  Concept of interest: \emph{Soil carbon sequestration}

  \begin{itemize}
  \tightlist
  \item
    \textbf{Operationalization:} concentration of organic carbon in the
    top 30 cm
  \item
    \textbf{Measure:} laboratory analysis of soil samples
  \item
    \textbf{Variable:} recorded carbon concentration values for each
    plot
  \end{itemize}
\item
  Concept of interest: \emph{Deforestation}

  \begin{itemize}
  \tightlist
  \item
    \textbf{Operationalization}: change in forest cover between the
    current year and five years ago
  \item
    \textbf{Measure:} aerial image classification
  \item
    \textbf{Variable:} change in forest cover (\%) per unit area
  \end{itemize}
\end{itemize}

Operationalization is rarely straightforward, and there is no single
correct way to do it. The best choice depends on the question you are
asking and how the data will be used. In many fields, scientists have
developed common practices, but every project still requires
case-by-case judgment. Even so, some principles of good
operationalization apply across studies: be precise about what you mean,
how you will measure it, and what values are possible.

\section{Scales of measurement}\label{scales-of-measurement}

As the previous section indicates, the outcome of a measurement is
called a \textbf{variable}. Not all variables are the same type, and
knowing the type matters because it determines which statistical tools
make sense. The four classic scales are \textbf{nominal, ordinal,
interval,} and \textbf{ratio}.

\subsection{Nominal scale}\label{nominal-scale}

A \textbf{nominal scale} variable (also referred to as a
\textbf{categorical} variable) is one in which the values are just
names. They do not have an inherent order, and it makes no sense to
average them.

\emph{Example:} eye color. Eyes can be blue, green, or brown, but none
is ``greater'' than another, and there's no such thing as an ``average
eye color.''

\subsection{Ordinal scale}\label{ordinal-scale}

\textbf{Ordinal scale} variables have a meaningful order, but the
spacing between values is not defined mathematically. You can rank the
values, but you can't assume equal steps between them, nor calculate a
meaningful average.

\emph{Example:} finishing position in a race. First place comes before
second, and second before third, but you don't know how much faster one
runner was than another. Saying the ``average place'' of the group is
2.3 doesn't mean anything.

\subsection{Interval scale}\label{interval-scale}

Interval variables have equal intervals between values, so differences
are meaningful. However, zero is arbitrary, so multiplication and
division are not valid.

\emph{Example:} temperature in degrees Celsius. A difference of
3\(^\circ\) means the same regardless of whether it is 7 → 10 or 15 →
18. But 0\(^\circ\) does not mean ``no temperature.'' That zero is
defined by the freezing point of water. So while you can say today is
3\(^\circ\) warmer than yesterday, you cannot say 20\(^\circ\) is
``twice as hot'' as 10\(^\circ\).

(Alternate example: latitude. A difference of 10° latitude is
meaningful, but zero latitude---the equator---doesn't mean ``no
latitude.'' Twice 45° is 90°, but that doesn't mean one place has
``twice as much latitude'' as another.)In contrast to nominal and
ordinal scale variables, \textbf{interval scale} and ratio scale
variables are variables for which the numerical value is genuinely
meaningful. In the case of interval scale variables, the
\emph{differences} between the numbers are interpretable, but the
variable doesn't have a ``natural'' zero value. You can add and
subtract, but ratios don't make sense.

\subsection{Ratio scale}\label{ratio-scale}

Ratio scale variables have all the properties of interval variables,
plus a true zero. This makes multiplication and division valid, as well
as addition and subtraction.

\emph{Example:} age in years. Zero means no age at all, and someone who
is 20 years old really is twice as old as someone who is 10. Differences
(20 -- 10 = 10 years) and ratios (20 ÷ 10 = 2) are both meaningful.

\begin{longtable}[]{@{}
  >{\raggedright\arraybackslash}p{(\linewidth - 8\tabcolsep) * \real{0.2000}}
  >{\centering\arraybackslash}p{(\linewidth - 8\tabcolsep) * \real{0.2000}}
  >{\centering\arraybackslash}p{(\linewidth - 8\tabcolsep) * \real{0.2000}}
  >{\centering\arraybackslash}p{(\linewidth - 8\tabcolsep) * \real{0.2000}}
  >{\raggedright\arraybackslash}p{(\linewidth - 8\tabcolsep) * \real{0.2000}}@{}}
\toprule\noalign{}
\begin{minipage}[b]{\linewidth}\raggedright
\end{minipage} & \begin{minipage}[b]{\linewidth}\centering
can rank
\end{minipage} & \begin{minipage}[b]{\linewidth}\centering
can subtract/add
\end{minipage} & \begin{minipage}[b]{\linewidth}\centering
can multiply/divide
\end{minipage} & \begin{minipage}[b]{\linewidth}\raggedright
example
\end{minipage} \\
\midrule\noalign{}
\endhead
\bottomrule\noalign{}
\endlastfoot
nominal & & & & eye color \\
ordinal & x & & & race position \\
interval & x & x & & temperature (°C) \\
ratio & x & x & x & age \\
\end{longtable}

\subsection{Continuous versus discrete
variables}\label{continuous-versus-discrete-variables}

Another useful distinction is whether a variable can take on values in
between others.

\begin{itemize}
\item
  A \textbf{continuous variable} can, in principle, take on any value
  within a range. For example, consider height. If you are 72 inches
  tall and your friend Cameron is 71 inches tall, Alan could be 71.4
  inches and David 71.49 inches. Because we can always imagine a new
  value in between two others, height is continuous.
\item
  A \textbf{discrete variable} is, in effect, a variable that isn't
  continuous. Discrete variables have separate, distinct values with
  nothing in between. Nominal variables are always discrete: there is no
  eye color that falls ``between'' green and blue in the same way that
  71.4 falls between 71 and 72. Ordinal variables are also discrete:
  although second place falls between first and third, nothing can
  logically fall between first and second.
\end{itemize}

Interval and ratio variables can be either. Height (a ratio variable) is
continuous. But the number of people living in a household (a ratio
variable) is discrete: you cannot have 4.2 people. Temperature in
degrees Celsius (an interval variable) is also continuous. But the year
you started college (an interval variable) is discrete: there is no year
between 2022 and 2023.

The table below shows how the scales of measurement relate to this
distinction. Cells with an ``x'' mark what is possible.

\begin{longtable}[]{@{}lcc@{}}
\toprule\noalign{}
& continuous & discrete \\
\midrule\noalign{}
\endhead
\bottomrule\noalign{}
\endlastfoot
nominal & & x \\
ordinal & & x \\
interval & x & x \\
ratio & x & x \\
\end{longtable}

\subsection{A note on real data}\label{a-note-on-real-data}

These categories are guides, not hard rules. For example, survey
responses on a 1--5 ``strongly disagree'' to ``strongly agree'' scale
are technically \textbf{ordinal}, but researchers often treat them as
``quasi-interval'' because the spacing is assumed to be roughly equal.

\section{Assessing the reliability of a
measurement}\label{assessing-the-reliability-of-a-measurement}

When we measure something, two questions matter: \emph{is the
measurement consistent, and is it accurate?}

\begin{itemize}
\item
  \textbf{Reliability} means consistency. If you measure the same thing
  again, do you get the same result?
\item
  \textbf{Validity} means accuracy. Does the measurement reflect the
  real thing you care about?
\end{itemize}

These aren't the same. A soil moisture probe that is miscalibrated might
always read 5\% too high. It is highly \textbf{reliable}, you get the
same number each time, but not \textbf{valid}, since it doesn't reflect
true soil moisture. On the other hand, a set of volunteer bird counts
might fluctuate from one observer to the next (\textbf{low
reliability}), but when averaged across many counts, the results may
still approximate true abundance (\textbf{reasonable validity}).

In practice, a measure that is very unreliable usually ends up being
invalid too, because we can't tell which result is right. This is why
reliability is often considered a prerequisite for validity.

\begin{itemize}
\item
  Reliability can show up in different ways:

  \begin{itemize}
  \item
    \textbf{Over time} (\emph{test--retest}): if you sample soil
    moisture today and tomorrow under the same conditions, do you get
    the same value?
  \item
    \textbf{Across people} (\emph{inter-rater}): if two field crews
    count birds at the same site, do their tallies agree?
  \item
    \textbf{Across tools} (\emph{parallel forms}): if you use two
    different rain gauges in the same spot, do they record the same
    rainfall?
  \item
    \textbf{Within a test} (\emph{internal consistency}): if multiple
    survey questions are meant to capture the same attitude, do they
    give similar answers?
  \end{itemize}
\end{itemize}

Not every measurement needs (or can even possibly have) every form of
reliability. The point is that \textbf{reliability is necessary but not
sufficient for validity}: a method can be consistently wrong, but if it
isn't consistent at all, it's almost impossible to know whether it's
right. We'll discuss how we assess validity a bit later in this chapter.

\section{The role of variables: predictors and
outcomes}\label{the-role-of-variables-predictors-and-outcomes}

One last piece of terminology before we leave variables behind. In most
studies we have many variables, but when we analyze them, we usually
split them into two roles: the \textbf{thing we're trying to explain}
and the \textbf{thing doing the explaining}. To keep it straight, we use
\(Y\) for the variable being explained, and a \(X_1\), \(X_2\), etc. for
the variables used to explain it.

Traditionally, \(X\) is called the \textbf{indepdendent variable (IV)}
and \(Y\) is the \textbf{dependent variable (DV).} The logic is that if
there's a relationship, \(Y\) depends on \(X\). These terms can be
clunky and can be confusing because: (a) IVs are rarely actually
``independent of everything else'' and (b) if there's no relationship,
then the DV doesn't actually ``depend'' on the IV at all.

Alternative terminology is often clearer. In experiments, IVs are
\textbf{manipulations}, and DVs are \textbf{measurements:}

\begin{longtable}[]{@{}
  >{\raggedright\arraybackslash}p{(\linewidth - 4\tabcolsep) * \real{0.4225}}
  >{\raggedright\arraybackslash}p{(\linewidth - 4\tabcolsep) * \real{0.3803}}
  >{\raggedright\arraybackslash}p{(\linewidth - 4\tabcolsep) * \real{0.1972}}@{}}
\toprule\noalign{}
\begin{minipage}[b]{\linewidth}\raggedright
role of the variable
\end{minipage} & \begin{minipage}[b]{\linewidth}\raggedright
classical name
\end{minipage} & \begin{minipage}[b]{\linewidth}\raggedright
alternative
\end{minipage} \\
\midrule\noalign{}
\endhead
\bottomrule\noalign{}
\endlastfoot
``to be explained'' (\(Y\)) & dependent variable (DV) & measurement \\
``to do the explaining'' (\(X\)) & independent variable (IV) &
manipulation \\
\end{longtable}

Another useful pair is \textbf{predictors} and \textbf{outcomes}. The
idea here is that what we use \(X\) to make predict or guess about
\(Y\):

\begin{longtable}[]{@{}
  >{\raggedright\arraybackslash}p{(\linewidth - 4\tabcolsep) * \real{0.4286}}
  >{\raggedright\arraybackslash}p{(\linewidth - 4\tabcolsep) * \real{0.3857}}
  >{\raggedright\arraybackslash}p{(\linewidth - 4\tabcolsep) * \real{0.1857}}@{}}
\toprule\noalign{}
\begin{minipage}[b]{\linewidth}\raggedright
role of the variable
\end{minipage} & \begin{minipage}[b]{\linewidth}\raggedright
classical name
\end{minipage} & \begin{minipage}[b]{\linewidth}\raggedright
alternative
\end{minipage} \\
\midrule\noalign{}
\endhead
\bottomrule\noalign{}
\endlastfoot
``to be explained'' (\(Y\)) & dependent variable (DV) & outcome \\
``to do the explaining'' (\(X\)) & independent variable (IV) &
predictor \\
\end{longtable}

\section{Experimental and non-experimental
research}\label{experimental-and-non-experimental-research}

A central distinction in research is between experimental and
non-experimental studies. What matters here is the degree of control the
researcher has.

In \textbf{experimental research}, the researcher deliberately
manipulates something (the predictor/IV) and measures its effect on
outcomes (the outcome/DV). The goal is to isolate causal effects. To
avoid the problem of ``something else'' influencing the outcome,
researchers try to hold other factors constant. In practice, it's almost
impossible to identify \emph{everything} that might matter, much less
keep it constant. The standard solution is \textbf{randomization}.
Randomization doesn't eliminate confounds, but it makes them less likely
to systematically bias results.

For instance, suppose we wanted to know if smoking causes lung cancer.
Observing smokers and non-smokers can only get us so far, because those
groups differ in many ways besides smoking, like occupation, income, and
diet. A true experiment would require randomly assigning people to smoke
or not. You can see how that would be deeply unethical. The same problem
comes up in medicine: we know surprisingly little about how certain
drugs or exposures affect pregnant people, precisely because we cannot
ethically assign them to risky conditions.

In \textbf{environmental science}, the limitation isn't usually ethics
but feasibility: there's no ``control planet'' we can set aside as a
baseline. Nor can researchers manipulate fundamental drivers like
precipitation or temperature. As a result, much of the field relies on
\textbf{non-experimental research} --- quasi-experiments, long-term time
series, or detailed case studies that allow scientists to tease out
effects in complex systems.

Much environmental science fits a \textbf{quasi-experimental design}.
For example, researchers may wish to study the effects of industrial
pollution on a river system but cannot directly control the pollutants
emitted. Instead, they observe existing conditions and make careful
comparisons, often using statistical tools to account for confounding
variables.

Another important quasi-experimental tool is \textbf{time series
analysis}, since many environmental processes unfold over decades or
centuries. Tracking changes in global temperature or sea level requires
years of consistent data, and statistical methods help isolate signals
from the noise of natural variability.

Environmental science also makes extensive use of \textbf{case studies},
which provide in-depth insights into specific events or locations. A
case study might explore the aftermath of a natural disaster, the
ecology of a threatened habitat, or the consequences of an environmental
policy. While a single case is tied to a particular context,
well-designed studies can uncover mechanisms that apply more broadly.
For example, a drought in one forest might reveal how water potential
and transpiration respond to stress, helping researchers anticipate
plant responses in other ecosystems.

\section{Assessing the validity of a
study}\label{assessing-the-validity-of-a-study}

Earlier we drew a line between \textbf{reliability} (consistency) and
\textbf{validity} (accuracy). Reliability tells us whether our
measurement is stable and repeatable. Validity asks the harder question:
are we actually measuring what we think we're measuring, and can we
trust the conclusions we draw from it?

\subsection{Internal validity}\label{internal-validity}

Internal validity is about whether the relationship you see is really
causal. From an earlier example: if smokers have more lung cancer than
non-smokers, does that mean smoking causes cancer? Not necessarily.
Smokers may differ from non-smokers in income, occupation, diet, or
other ways. These ``confounds'' muddy causal claims. Experiments with
random assignment improve internal validity by balancing out such
factors, but in environmental science, we often rely on careful design
and statistical controls instead.

\subsection{External validity}\label{external-validity}

External validity is about generalization. Does what you found in one
study hold elsewhere? A soil-warming experiment in one forest plot may
not capture how all forests respond. Or water-quality measurements from
a single river reach may not represent conditions across the whole
watershed. Strong internal validity can still leave you with weak
external validity if the setting, population, or conditions are narrow.

\subsection{Construct validity}\label{construct-validity}

Construct validity is about whether your measure really matches the
concept. If you care about soil fertility but measure only soil
moisture, you have high reliability but weak construct validity: you're
consistently measuring the wrong thing. Getting this right requires
aligning theory, measurement, and data.

\section{Confounds, artifacts and other threats to
validity}\label{confounds-artifacts-and-other-threats-to-validity}

Broadly, the two biggest threats to validity are \emph{confounds} and
\emph{artifacts}:

\begin{itemize}
\item
  \textbf{Confound}: An additional variable, often unmeasured, that
  masks or distorts the relationshp between the predictor and the
  outcome. Confounds threaten \emph{internal validity} because you can't
  tell whether the predictor really causes the outcome. Two variables
  are said to be \textbf{confounded} if their effects on a response
  variable cannot be distinguished.

  \begin{itemize}
  \tightlist
  \item
    \textbf{Example:} In a study of tree growth near roads, trees closer
    to the road might appear to grow faster. But if those same trees
    also receive more runoff water from pavement, then water
    availability (not proximity to the road) could explain the growth
    difference.
  \end{itemize}
\item
  \textbf{Artifact}: An aspect of the experimental setup or apparatus
  that biases the results. An artifact gives the appearance of measuring
  the phenomenon of interest, but is actually measuring something else
  introduced by the method itself. Artifacts often undermine
  \emph{external validity,} and sometimes internal validity too.

  \begin{itemize}
  \tightlist
  \item
    \textbf{Example:} In soft-sediment studies, repeated coring to
    sample buried organisms disturbs the sediment, exposing animals and
    disrupting microbial and structural assemblages; later samples then
    reflect the disturbance caused by earlier sampling rather than true
    ecological patterns (Skilleter 1996)
  \end{itemize}
\end{itemize}

Here are some more types of threats to validity:

\begin{itemize}
\item
  \textbf{History Effects:} Events outside the study can shift the
  outcome. Long-term ecosystem monitoring is especially vulnerable:
  imagine tracking stream chemistry for years, then a wildfire in the
  watershed alters nutrient fluxes. Your ``trend'' may just reflect that
  disturbance rather than the process you meant to study.
\item
  \textbf{Repeated Testing Effects:} Even measurements themselves can
  have an effect. For example, extracting multiple tree-ring cores from
  the same tree could wound it and reduce growth in later years,
  influencing the very variable you were trying to measure.
\item
  \textbf{Selection bias:} Environmental research often focuses on sites
  that are easiest to study, like forests near R1 universities,
  long-established research stations, or accessible watersheds. But
  those places aren't necessarily representative. If they differ
  systematically from the broader set of ecosystems, conclusions may not
  generalize.
\item
  \textbf{Differential Attrition:} Long-term field studies depend on
  cooperation, and dropout can skew results. For example, in a
  cover-cropping trial, if only the most motivated farmers stick with
  the practice, the observed benefits won't generalize to all farms.
\item
  \textbf{Non-response bias:} Surveys face the same problem: those who
  answer aren't random. For instance, a survey on attitudes toward
  conservation easements may mostly attract landowners already
  supportive of conservation, biasing the results.
\item
  \textbf{Regression to the mean:} Extreme cases tend to moderate over
  time. If you select the fastest-growing trees in a stand for special
  study, you'll probably find their growth slows later --- not
  necessarily because of your treatment, but simply because growth rates
  fluctuate and extremes rarely persist.
\item
  \textbf{Experimenter bias:} Researchers bring expectations to the
  field. Choices about where to sample a river, or when to measure soil
  respiration, can unconsciously favor the expected outcome. The classic
  cautionary tale is Clever Hans, the horse who ``did math'' by
  responding to subtle human cues (Pfungst 1911; Hothersall 2004). The
  lesson: be aware of how your own expectations can shape results.
\item
  \textbf{Reactivity and demand effects:} Sometimes behavior changes
  because people know they're being observed. A field crew told they are
  testing the ``impact'' of a method may record data more carefully than
  usual, producing results that reflect observer effort rather than the
  phenomenon itself.
\end{itemize}

\subsection{Fraud, deception, and
self-deception}\label{fraud-deception-and-self-deception}

\begin{quote}
\emph{It is difficult to get a man to understand something, when his
salary depends on his not understanding it.}\\
-- Upton Sinclair
\end{quote}

Most scientists are honest, but self-deception is common. Researchers
can over-interpret noisy results, design studies that all but guarantee
an effect, or (consciously or not) hide inconvenient variables.
Publication bias makes it worse: null results often go unreported, so
the literature overstates effects. And sometimes, as in Simpson's
paradox, aggregated data can mislead if you don't dig deeper. The point
isn't to be cynical, it's to recognize that science is done by humans,
and humans bring bias, incentives, and blind spots.

\section{Summary}\label{summary}

In this chapter, we have explored essential aspects of research
methodology pertinent to environmental statistics:

\begin{itemize}
\item
  \textbf{The role of statistics in environmental science:} Statistics
  does not answer every substantive question, but it provides the
  discipline to separate signal from noise and make credible inferences
  from messy data.
\item
  \textbf{Operationalization and Measurement}: Defining theoretical
  constructs and deciding how to measure them is foundational.
\item
  \textbf{Scales of measurement and types of variables}: Distinguishing
  discrete from continuous data, and recognizing nominal, ordinal,
  interval, and ratio scales.
\item
  \textbf{Reliability of a measurement}: If I measure the ``same'' thing
  twice, should I expect the same result? In what sense (test--retest,
  inter-rater, internal consistency)?
\item
  \textbf{Terminology: predictors and outcomes}: Can I clearly explain
  the roles variables play in an analysis (predictor vs.~outcome,
  independent vs.~dependent)?
\item
  \textbf{Experimental and non-experimental research designs}:
  Identifying what constitutes an experiment, and how researchers rely
  on quasi-experiments, time series, and case studies when full control
  is not possible.
\item
  \textbf{Validity and Threats}: Does the study truly measure what it
  claims to? What pitfalls---such as confounds or artifacts---could bias
  results?
\end{itemize}

Study design is a cornerstone of environmental research methodology.
While many textbooks provide fuller coverage, e.g., Campbell and Stanley
(1963), this chapter highlights the unique challenges of applying these
principles in environmental science, where ethical and practical
constraints often preclude perfect control. By linking statistics to
study design, we see how careful methods make it possible to draw
credible inferences from complex, real-world systems.

\begin{center}\rule{0.5\linewidth}{0.5pt}\end{center}

\section{Videos}\label{videos}

\subsection{Terms of Statistics}\label{terms-of-statistics}

\bookmarksetup{startatroot}

\chapter{Describing Data}\label{DescribingData}

\begin{quote}
Far better an approximate answer to the right question, which is often
vague, than an exact answer to the wrong question, which can always be
made precise. *\\
---John W. Tukey
\end{quote}

Statistics often begins with the problem of \textbf{too much data}.
Modern environmental science generates vast amounts of information:
rainfall at hundreds of stations, thousands of tree measurements in a
forest survey, millions of satellite pixels. Looking at raw numbers
alone is overwhelming and uninformative. We need tools to compress,
summarize, and visualize. These are \textbf{descriptive statistics}.

Keep in mind:

\begin{enumerate}
\def\labelenumi{\arabic{enumi}.}
\tightlist
\item
  There is more than one useful way to describe data.
\item
  You choose a description depending on what helps reveal the pattern
  you care about.
\item
  All methods were invented for that reason. They are helpful shortcuts.
\end{enumerate}

Imagine running a survey of 500 people's happiness levels. The full
dataset is just a long list of values. Staring at hundreds of numbers
tells us little. How happy are most people? How unhappy are the
unhappiest? Raw data alone can't answer, we need to \textbf{summarize.}

\section{Looking at the data}\label{looking-at-the-data}

\subsection{Scatter of raw values}\label{scatter-of-raw-values}

One first step is simply to \textbf{plot every value} instead of listing
them. Let's look at them!

\begin{figure}

\centering{

\includegraphics[width=0.75\linewidth,height=\textheight,keepaspectratio]{02-Describing_Data_files/figure-pdf/fig-happyPlot-1.pdf}

}

\caption{\label{fig-happyPlot}Pretend happiness ratings from 500 people}

\end{figure}%

Figure~\ref{fig-happyPlot} shows 500 measurements of happiness. The
horizontal axis is an index of participants (1--500), and the vertical
axis shows their reported happiness. Each dot represents one person's
score.

The way we choose to plot data makes some patterns easier to see and
others harder. What do we notice here? There are indeed 500 dots, spread
across a wide range: some values climb as high as about 1500, others
fall below --1500. Most points cluster somewhere near zero, with fewer
at the extremes.

\textbf{Take-home:} plotting all the values at once is already much more
useful than staring at a raw table of numbers.

Still, the scatter doesn't make it easy to answer questions like ``are
most people generally happy or unhappy?'' For that, we need a different
view, a \textbf{histogram}.

\subsection{Histograms}\label{histograms}

Making a histogram summarizes data by grouping numbers rather than
looking at individual data points. Figure~\ref{fig-happyHist} displays
500 happiness scores grouped into bins. Each bar shows how many people's
responses fall within a range. The bar between 0 and 500 indicates about
150 people in that range. The y-axis shows the frequency count of each
bar.

\begin{figure}

\centering{

\includegraphics[width=0.75\linewidth,height=\textheight,keepaspectratio]{02-Describing_Data_files/figure-pdf/fig-happyHist-1.pdf}

}

\caption{\label{fig-happyHist}A histogram of the happiness ratings}

\end{figure}%

The histogram makes patterns clearer than the scatterplot. Most of the
responses fall between --500 and +500, with only a few extreme highs or
lows. We can also see the overall spread of the data: roughly --1500 to
+1500.

When making histograms, the width of the bins matters. If the bins are
wide, the plot looks smooth but may hide detail. If they are narrow, the
plot shows more variation but can appear noisy.
Figure~\ref{fig-manyhistbin} compares the same data with different bin
widths.

\begin{figure}

\centering{

\includegraphics[width=0.75\linewidth,height=\textheight,keepaspectratio]{02-Describing_Data_files/figure-pdf/fig-manyhistbin-1.pdf}

}

\caption{\label{fig-manyhistbin}Four histograms of the same data using
different bin widths}

\end{figure}%

All of the histograms show the same general shape: few values at the
extremes and many near zero. Narrow bins make the bars jump up and down,
adding noise. Wider bins smooth the pattern, but if they're too wide,
important variation gets lost.

\textbf{Take-home:} histograms summarize individual values into a
distribution, revealing both the typical range and how the data are
spread. Bin width is a choice you make, and it affects what you see.

\section{Important Ideas: Distribution, Central Tendency, and
Variance}\label{important-ideas-distribution-central-tendency-and-variance}

Three terms will keep coming up: \textbf{distribution, central tendency,
and variance.} Their everyday meanings aren't far from the statistical
ones.

\begin{itemize}
\item
  \textbf{Distribution}: how values are spread across their range. A
  histogram is one way of showing a distribution. The shape of the
  distribution tells us where values are concentrated, where they're
  rare, and how far the extremes reach. Later we'll see that
  distributions don't just describe data, we also use them to
  \textbf{generate theoretical data}, which is the basis for sampling
  distributions and tools like \emph{t}-tests.
\item
  \textbf{Central tendency} is about \emph{sameness}: where values
  cluster. In the happiness histogram, most responses are near zero, so
  we say the data have a central tendency around zero. Central tendency
  doesn't mean ``everything is the same'', just that many values gather
  in a common area. Some datasets even have more than one cluster, and
  therefore more than one central tendency.
\item
  \textbf{Variance} is about \emph{differentness}: how spread out the
  values are. If all responses were nearly identical, variance would be
  low. In our happiness example, values range widely from strongly
  negative to strongly positive, so variance is high. In practice,
  variance is a family of measures we use to quantify differences, and
  we'll introduce those next.
\end{itemize}

\section{Measures of Central Tendency
(Sameness)}\label{measures-of-central-tendency-sameness}

Plots and histograms show where values fall, but lack precision. We can
summarize data with a single number representing its ``center'' - these
\textbf{measures of central tendency} show what values are generally
like.

\subsection{Mode}\label{mode}

The \textbf{mode} is the most frequently occurring value in a dataset.
How do you find it? You have to count the number of times each number
appears, then whichever one occurs the most, is the mode.

\begin{quote}
Example: 1 1 1 2 3 4 5 6
\end{quote}

The mode is 1, since it appears three times.

If two or more values tie for most frequent, the dataset has multiple
modes:

\begin{quote}
Example: 1 1 1 2 2 2 3 4 5 6
\end{quote}

Here, 1 and 2 both occur three times each. So, there are two modes, and
they are 1 and 2.

Is the mode a good measure of central tendency? That depends on your
numbers. The mode can be useful when a dataset has clear repeating
values, but it doesn't always capture what's ``typical.'' For example:

\begin{quote}
1 1 2 3 4 5 6 7 8 9
\end{quote}

Here, the mode is 1, but most numbers are larger. Like any statistical
tool, consider if the mode suits your dataset and justify that choice.

\subsection{Median}\label{median}

The \textbf{median} is the exact middle of the data once they are
ordered from smallest to largest. For example:

\begin{quote}
1 5 4 3 6 7 9
\end{quote}

Before we can compute the median, we need to order the numbers from
smallest to largest. Ordered:

\begin{quote}
1 3 4 \textbf{5} 6 7 9
\end{quote}

The median is 5, with three numbers on each side.

With an even number of values, the median is the midpoint between the
two in the center:

\begin{quote}
1 2 3 4 5 6
\end{quote}

Here we have six numbers, so there isn't a single middle value. Instead,
the median is the midpoint between the two central numbers, 3 and 4,
which gives 3.5.

The median is often a useful measure of central tendency because it
stays put even if some values are extreme. For example:

\begin{quote}
1 2 3 4 4 4 \textbf{5} 6 6 6 7 7 1000
\end{quote}

Most of these numbers are modest, but 1000 is far from the rest. The
median is still 5, which reflects the bulk of the data despite that
extreme value. This shows why the median often does a good job of
representing a dataset even when one or two values are very different.

In this case, 1000 would be considered an \textbf{outlier}: a value much
farther from the others. Outliers can strongly affect some summaries
(like the mean) but leave the median unchanged. How to handle outliers
is a topic we return to later in the course.

\subsection{Mean}\label{mean}

Have you noticed this statistics textbook hasn't used a formula yet?
That's about to change, but don't worry if you have formula anxiety -
we'll explain them. The \textbf{mean} is also called the average. You
probably know it's the sum of numbers divided by how many numbers there
are.

\textbf{Here's the formula:}

\(Mean = \bar{X} = \frac{\sum_{i=1}^{n} x_{i}}{N}\)

\begin{itemize}
\item
  The \(\sum\) symbol is called \textbf{sigma}, and it stands for the
  operation of summing.
\item
  The ``i'' and ``n'' refer to all numbers in the set, from first to
  last.
\item
  The \(x_{i}\) refers to individual numbers in the set.
\item
  \(\bar{X}\) refers to the mean
\item
  We sum them all, then divide by \(N\), the total count.
\end{itemize}

I\textbf{n simpler terms:}

\(mean = \frac{\text{Sum of my numbers}}{\text{Count of my numbers}}\)

Let's compute the mean for these five numbers:

\begin{quote}
3 7 9 2 6
\end{quote}

Add them:

\begin{quote}
3+7+9+2+6 = 27
\end{quote}

Count them:

\begin{quote}
\(i_{1}\) = 3, \(i_{2}\) = 7, \(i_{3}\) = 9, \(i_{4}\) = 2, \(i_{5}\) =
6; N=5, because \(i\) went from 1 to 5
\end{quote}

Divide:

\begin{quote}
mean = 27 / 5 = 5.4
\end{quote}

Or, putting everything in the formula:

\(Mean = \bar{X} = \frac{\sum_{i=1}^{n} x_{i}}{N} = \frac{3+7+9+2+6}{5} = \frac{27}{5} = 5.4\)

That's how to compute the mean. You probably knew this already, and if
not, now you do. But is the mean a good measure of central tendency? By
now, you should know: it depends.

\subsection{What does the mean mean?}\label{what-does-the-mean-mean}

It's not enough to know how to calculate a mean, you also need to
understand what it represents. The formula divides the sum of all values
by the number of values, but what does that operation actually do?

Think about division. If we compute

\(\frac{12}{3} = 4\)

we're splitting 12 into three equal parts. Each part is 4. Division
equalizes the numerator into identical pieces.

``The same logic applies to the mean. Suppose you add up all 500
happiness ratings into one large total. If that total were redistributed
equally across every person, each would get the same value. That value
is the mean. Some individuals were originally higher or lower, but the
mean is the balance point created by spreading the total evenly.

This is why the mean is more than just arithmetic. It is the one number
that can replace every observation so that, when all those replacements
are added together, you get back the original total.

\begin{tcolorbox}[enhanced jigsaw, title=\textcolor{quarto-callout-note-color}{\faInfo}\hspace{0.5em}{Takeaway}, colframe=quarto-callout-note-color-frame, colbacktitle=quarto-callout-note-color!10!white, bottomtitle=1mm, leftrule=.75mm, rightrule=.15mm, titlerule=0mm, arc=.35mm, colback=white, opacitybacktitle=0.6, toprule=.15mm, toptitle=1mm, bottomrule=.15mm, coltitle=black, breakable, left=2mm, opacityback=0]

The mean is unique: it is the only single value that can replace every
observation so that the total stays the same.

\end{tcolorbox}

\subsection{Comparing the mean, median, and
mode}\label{comparing-the-mean-median-and-mode}

Figure~\ref{fig-meanmodemed} shows a histogram of simulated data with
three vertical lines: the mode (blue, ≈ 1), the median (green, ≈ 6), and
the mean (red, ≈ 8).

Notice that these three measures of central tendency do not agree. The
\textbf{mean} is pulled to the right because of a few very large values
in the tail. The \textbf{median} is more stable, landing near the middle
of the bulk of the data. The \textbf{mode} marks the most common single
value, which here is close to zero.

\begin{figure}

\centering{

\includegraphics[width=0.75\linewidth,height=\textheight,keepaspectratio]{02-Describing_Data_files/figure-pdf/fig-meanmodemed-1.pdf}

}

\caption{\label{fig-meanmodemed}A histogram with the mean (red), the
median (green), and the mode (blue)}

\end{figure}%

\begin{tcolorbox}[enhanced jigsaw, title=\textcolor{quarto-callout-note-color}{\faInfo}\hspace{0.5em}{Takeaway}, colframe=quarto-callout-note-color-frame, colbacktitle=quarto-callout-note-color!10!white, bottomtitle=1mm, leftrule=.75mm, rightrule=.15mm, titlerule=0mm, arc=.35mm, colback=white, opacitybacktitle=0.6, toprule=.15mm, toptitle=1mm, bottomrule=.15mm, coltitle=black, breakable, left=2mm, opacityback=0]

When data are skewed---as they are here with a long right tail---the
mean, median, and mode can give very different answers. Understanding
how each behaves is essential for choosing the right summary.

\end{tcolorbox}

\section{\texorpdfstring{Measures of Variation
(Different\emph{ness})}{Measures of Variation (Differentness)}}\label{measures-of-variation-differentness}

Central tendency tells us what values have in common. Measures of
\textbf{variation} tell us how they differ. Any dataset with more than
one value will show some variation, and summarizing that spread is as
important as finding the center.

\subsection{The Range}\label{the-range}

Consider these 10 ordered numbers:

\begin{quote}
1 3 4 5 5 6 7 8 9 24
\end{quote}

The smallest is 1, the largest is 24. Together they define the
\textbf{range}. The range quickly shows the boundaries of the data and
can flag possible outliers. For instance, if you expected values between
1 and 7 but found one at 340,500, you'd know something unusual happened
and might investigate.

\subsection{Difference Scores}\label{difference-scores}

It would be nice to summarize the amount of different\emph{ness} in the
data. Think about these 10 ordered numbers:

\begin{quote}
1 3 4 5 5 6 7 8 9 24
\end{quote}

We can compute the differences between each pair of numbers and put them
in a matrix:

\begin{longtable}[]{@{}lrrrrrrrrrr@{}}
\toprule\noalign{}
& 1 & 3 & 4 & 5 & 5 & 6 & 7 & 8 & 9 & 24 \\
\midrule\noalign{}
\endhead
\bottomrule\noalign{}
\endlastfoot
1 & 0 & 2 & 3 & 4 & 4 & 5 & 6 & 7 & 8 & 23 \\
3 & -2 & 0 & 1 & 2 & 2 & 3 & 4 & 5 & 6 & 21 \\
4 & -3 & -1 & 0 & 1 & 1 & 2 & 3 & 4 & 5 & 20 \\
5 & -4 & -2 & -1 & 0 & 0 & 1 & 2 & 3 & 4 & 19 \\
5 & -4 & -2 & -1 & 0 & 0 & 1 & 2 & 3 & 4 & 19 \\
6 & -5 & -3 & -2 & -1 & -1 & 0 & 1 & 2 & 3 & 18 \\
7 & -6 & -4 & -3 & -2 & -2 & -1 & 0 & 1 & 2 & 17 \\
8 & -7 & -5 & -4 & -3 & -3 & -2 & -1 & 0 & 1 & 16 \\
9 & -8 & -6 & -5 & -4 & -4 & -3 & -2 & -1 & 0 & 15 \\
24 & -23 & -21 & -20 & -19 & -19 & -18 & -17 & -16 & -15 & 0 \\
\end{longtable}

In the top left, the difference between 1 and itself is 0. One column
over, 3--1 = 2, and so on. With 10 numbers this produces 100
differences. With 500 numbers, you'd have 250,000---too many to be
useful.

If all the numbers were the same, every difference would be 0, making
the lack of variation obvious. But with different numbers, we end up
with a large table. How can we summarize it? One idea is to apply what
we learned about central tendency and take the \textbf{average
difference}.

Let's try it with three numbers:

\begin{quote}
1 2 3
\end{quote}

\begin{longtable}[]{@{}lrrr@{}}
\toprule\noalign{}
& 1 & 2 & 3 \\
\midrule\noalign{}
\endhead
\bottomrule\noalign{}
\endlastfoot
1 & 0 & 1 & 2 \\
2 & -1 & 0 & 1 \\
3 & -2 & -1 & 0 \\
\end{longtable}

The mean of these nine difference scores is:

\(\text{mean of difference scores} = \frac{0+1+2-1+0+1-2-1+0}{9} = \frac{0}{9} = 0\)

This will always happen: the positives and negatives cancel. So the mean
of raw difference scores is not a useful measure of variation.

Notice also that the matrix is redundant: the diagonal is always zero,
and values above and below the diagonal are the same except for sign.

These problems motivate why we compute \textbf{variance} and
\textbf{standard deviation}. They solve the cancellation issue and give
us a practical summary of spread.

\subsection{The Variance}\label{the-variance}

We've used the words variability, variation, and variance a lot. They
all point to the same big idea: numbers differ. When numbers are
different, they have variance.{]}

The word \emph{variance} is used in two ways. First, it can mean the
general idea of differences between numbers---when values vary, there is
variance. Second, it refers to a specific summary statistic: the
\textbf{mean of the squared deviations from the mean}.

To calculate it, we take each score, subtract the mean to find its
\textbf{difference score}, then square those differences, then average
them. Squaring is the mathematical ``trick'' that keeps positive and
negative deviations from canceling each other out. Finally, we average
the squared deviations. In short:

\(variance = \frac{\text{Sum of squared difference scores}}{\text{Number of Scores}}\)

Later we'll distinguish between dividing by \(N\) (all data =
population) or \(N-1\) (sample). For now, just use \(N\).

\subsubsection{Deviations from the mean}\label{deviations-from-the-mean}

Earlier we compared every number to every other number, which quickly
became unmanageable. A simpler approach is to compare each score to the
mean.

\begin{itemize}
\item
  Step 1: Find the mean.
\item
  Step 2: Subtract the mean from each score.
\end{itemize}

This tells us:

\begin{enumerate}
\def\labelenumi{\arabic{enumi}.}
\item
  How well the mean represents the data
\item
  How much spread there is around that mean.
\end{enumerate}

Here's an example:

\begin{longtable}[]{@{}llll@{}}
\toprule\noalign{}
scores & values & mean & Difference\_from\_Mean \\
\midrule\noalign{}
\endhead
\bottomrule\noalign{}
\endlastfoot
1 & 1 & 4.5 & -3.5 \\
2 & 6 & 4.5 & 1.5 \\
3 & 4 & 4.5 & -0.5 \\
4 & 2 & 4.5 & -2.5 \\
5 & 6 & 4.5 & 1.5 \\
6 & 8 & 4.5 & 3.5 \\
Sums & 27 & 27 & 0 \\
Means & 4.5 & 4.5 & 0 \\
\end{longtable}

The mean is \(4.5\):

\(\frac{1+6+4+2+6+8}{6} = \frac{27}{6} = 4.5\).

The third column simply repeats this mean for each row. That looks odd,
but it shows the important property I mentioned earlier: the mean
distributes the total equally across all points. Six copies of 4.5 add
back to the total of 27.

The fourth column shows the \textbf{deviation from the mean:}
\(X_{i}-\bar{X}\). For example, 1, is -3.5 from the mean, 6, is +1.5,
and so on.

Now notice the problem: the deviations add up to zero. The positive and
negative differences cancel, which makes it seem like there's no
variation. Clearly that isn't true, so we'll need another step, squaring
the deviations, to solve it. But first let's investigate \emph{why} this
happens.

\begin{center}\rule{0.5\linewidth}{0.5pt}\end{center}

\subsubsection{Mean as the Balancing
Point}\label{mean-as-the-balancing-point}

The mean is the balancing point of the data. Imagine laying a ruler
across your finger. The place where it balances is the point where the
weight on each side is equal. Data works the same way: if we treat each
value like a weight, the mean is where the total ``mass'' to the left
and right cancel out.

This balancing property explains why the deviations always sum to zero.
The values below the mean contribute negative deviations, the values
above contribute positive ones, and together they cancel:

\(-x + x = 0\)

To see this more concretely, consider the numbers

\(X = (1,2,6,7,9)\)

\begin{figure}

\centering{

\includegraphics[width=0.75\linewidth,height=\textheight,keepaspectratio]{02-Describing_Data_files/figure-pdf/fig-number-line-1.pdf}

}

\caption{\label{fig-number-line}Values on a number line.}

\end{figure}%

Now imagine the number line as a teeter-totter. Only when the fulcrum is
placed at 5 does the board balance:

\begin{figure}

\centering{

\includegraphics[width=0.75\linewidth,height=\textheight,keepaspectratio]{02-Describing_Data_files/figure-pdf/fig-mean-balance-panels-1.pdf}

}

\caption{\label{fig-mean-balance-panels}Mean as the unique balancing
point. Only at the mean do signed deviations cancel.}

\end{figure}%

Formally, this means the signed distances from the mean always cancel
out:

\((1−5)+(2−5)+(6−5)+(7−5)+(9−5)=0\)

This makes the mean unique. It is the \textbf{only value} for which the
sum of deviations equals zero:

\(\sum_{i=1}^{n}x_{{i}-a} = 0 \quad \Rightarrow \quad a = \bar{x}\)

The mean is not just a ``typical'' value, it is the one point where the
data balance. And that balancing property is exactly why the raw
deviations always sum to zero. To summarize variation, we need a way
around this problem.

\subsubsection{The squared deviations}\label{the-squared-deviations}

The standard trick is to square the deviations. Squaring converts
negatives to positives:

\(2^2 = 4\)

\(-2^2 = 4\).\\
\strut \\
Since a squared number is always non-negative, the deviations no longer
cancel. We call these \textbf{squared deviations}: differences from the
mean that have been squared.

Let's revisit our table, this time adding a column for squared
deviations:

\begin{longtable}[]{@{}lllll@{}}
\toprule\noalign{}
scores & values & mean & Difference\_from\_Mean & Squared\_Deviations \\
\midrule\noalign{}
\endhead
\bottomrule\noalign{}
\endlastfoot
1 & 1 & 4.5 & -3.5 & 12.25 \\
2 & 6 & 4.5 & 1.5 & 2.25 \\
3 & 4 & 4.5 & -0.5 & 0.25 \\
4 & 2 & 4.5 & -2.5 & 6.25 \\
5 & 6 & 4.5 & 1.5 & 2.25 \\
6 & 8 & 4.5 & 3.5 & 12.25 \\
Sums & 27 & 27 & 0 & 35.5 \\
Means & 4.5 & 4.5 & 0 & 5.91666666666667 \\
\end{longtable}

For example, the first score is 1, which is 3.5 below the mean. Its
deviation is \(−3.5\), and its squared deviation is \((−3.5)^2=12.25\).

Now that all deviations are positive, we can add them up. The result is
the \textbf{sum of squares (SS)}, the sum of squared deviations from the
mean. You'll see this quantity again in the ANOVA chapter, but the idea
is simple: it's just the total of those squared deviations, nothing
more.

\subsubsection{Finally, the variance}\label{finally-the-variance}

Guess what, we've already computed the variance. Maybe you didn't
notice.

Let's remind ourselves of the goal: we want a single number that
summarizes how spread out the data are. Deviations from the mean show
those differences, but there are as many deviations as data points. To
simplify, we want the \emph{average} squared deviation.

Look back at the table. We added up the squared deviations (the
\emph{sum of squares}, SS), and then we divided by the number of
observations. That's \textbf{the variance}. The variance is the mean of
the sum of the squared deviations:

\(variance = \frac{SS}{N}\)

where SS is the sum of squared deviations, and N is the number of
observations.

For our data, this came out to 5.916 (repeating).

So what does that number mean? Honestly, not much on its own. The
problem is that it's on the \emph{squared} scale, remember, we squared
the deviations before averaging. Squaring inflates the values, so the
variance is no longer in the same units as the original data.

The fix is straightforward: take the square root. For our example,

\(\sqrt{5.916} ≈2.43\)

\subsection{The Standard Deviation}\label{the-standard-deviation}

We did it again, we already computed the \textbf{standard deviation
(SD)}, without calling it by name. The standard deviation is just the
square root of the variance:

\(\text{standard deviation} = \sqrt{Variance} = \sqrt{\frac{SS}{N}}\).

or, written out fully,

\(\text{standard deviation} = \sqrt{\frac{\sum_{i}^{n}({x_{i}-\bar{x})^2}}{N}}\)

Those square root signs simply ``unsquare'' the variance, bringing the
measure of spread back to the original scale of the data. Let's look at
our table again:

\begin{longtable}[]{@{}lllll@{}}
\toprule\noalign{}
scores & values & mean & Difference\_from\_Mean & Squared\_Deviations \\
\midrule\noalign{}
\endhead
\bottomrule\noalign{}
\endlastfoot
1 & 1 & 4.5 & -3.5 & 12.25 \\
2 & 6 & 4.5 & 1.5 & 2.25 \\
3 & 4 & 4.5 & -0.5 & 0.25 \\
4 & 2 & 4.5 & -2.5 & 6.25 \\
5 & 6 & 4.5 & 1.5 & 2.25 \\
6 & 8 & 4.5 & 3.5 & 12.25 \\
Sums & 27 & 27 & 0 & 35.5 \\
Means & 4.5 & 4.5 & 0 & 5.91666666666667 \\
\end{longtable}

We calculated our standard deviation as:

\(\sqrt{5.916} ≈2.43\)

This value makes much more sense than the variance. A variance of 5.916
feels too large for this dataset, because it's on the squared scale. But
a standard deviation of 2.43 lines up with the actual differences from
the mean---most scores are within about ±2.4 of 4.5.

So if someone told you their dataset had a mean of 4.5 and a standard
deviation of 2.4, you'd already have a good sense of it: the numbers
cluster around 4.5, but not exactly at 4.5, and the typical spread is
about 2 to 3 units in either direction.

That's the power of the standard deviation: it gives you a single number
that describes the \emph{typical deviation} from the mean, while staying
on the same scale as the original data.

\subsection{Mean Absolute Deviation}\label{mean-absolute-deviation}

So far we've handled the ``differences cancel out'' problem by squaring
deviations. But there's another simple option: take the \textbf{absolute
value} of each deviation from the mean. That way, every difference
becomes positive, and when we add them up we don't end up at zero.

Here's what that looks like for our data:

\begin{longtable}[]{@{}lllll@{}}
\toprule\noalign{}
scores & values & mean & Difference\_from\_Mean &
Absolute\_Deviations \\
\midrule\noalign{}
\endhead
\bottomrule\noalign{}
\endlastfoot
1 & 1 & 4.5 & -3.5 & 3.5 \\
2 & 6 & 4.5 & 1.5 & 1.5 \\
3 & 4 & 4.5 & -0.5 & 0.5 \\
4 & 2 & 4.5 & -2.5 & 2.5 \\
5 & 6 & 4.5 & 1.5 & 1.5 \\
6 & 8 & 4.5 & 3.5 & 3.5 \\
Sums & 27 & 27 & 0 & 13 \\
Means & 4.5 & 4.5 & 0 & 2.16666666666667 \\
\end{longtable}

The last column shows absolute deviations. If we take their mean, we get
the mean absolute deviation (MAD). Notice this acts a lot like the
standard deviation: it's the average size of a deviation from the mean,
only without squaring.Both approaches---squaring or taking absolute
values---solve the same problem in slightly different ways. Squaring is
more common because it leads to nice mathematical properties (like in
regression and ANOVA), but the MAD can be more intuitive to interpret.

\section{Remember to look at your
data}\label{remember-to-look-at-your-data}

Descriptive statistics are useful, but they are also compressed
summaries. They reduce a dataset to a few numbers, which means they
always lose detail. That's fine if the summary captures the important
features, but sometimes it doesn't.

The safest habit is to \textbf{always combine descriptive statistics
with graphs}. Graphs show you the shape of the data and reveal patterns
summaries can hide.

\subsection{Anscombe's Quartet}\label{anscombes-quartet}

Francis Anscombe (Anscombe (1973)) made this point with a famous
example, now called \emph{Anscombe's Quartet}. Each panel in Figure
Figure~\ref{fig-anscombe} shows pairs of \(x\) and \(y\) values as a
scatterplot.

Visually, the four panels are very different: one looks linear, one
curved, one is mostly linear except for an outlier, and one forms a
near-vertical line except for one point.

\begin{figure}

\centering{

\includegraphics[width=0.75\linewidth,height=\textheight,keepaspectratio]{02-Describing_Data_files/figure-pdf/fig-anscombe-1.pdf}

}

\caption{\label{fig-anscombe}Anscombe's Quartet}

\end{figure}%

Now here's the kicker:

\begin{longtable}[]{@{}lrrrr@{}}
\toprule\noalign{}
quartet & mean\_x & var\_x & mean\_y & var\_y \\
\midrule\noalign{}
\endhead
\bottomrule\noalign{}
\endlastfoot
1 & 9 & 11 & 7.500909 & 4.127269 \\
2 & 9 & 11 & 7.500909 & 4.127629 \\
3 & 9 & 11 & 7.500000 & 4.122620 \\
4 & 9 & 11 & 7.500909 & 4.123249 \\
\end{longtable}

All four datasets have the \textbf{exact same descriptive statistics}:

same mean and variance for \(x\)

same mean and variance for \(y\)

same correlation between \(x\) and \(y\)

Yet the scatterplots could not look more different.

\begin{tcolorbox}[enhanced jigsaw, title=\textcolor{quarto-callout-note-color}{\faInfo}\hspace{0.5em}{Takeaway}, colframe=quarto-callout-note-color-frame, colbacktitle=quarto-callout-note-color!10!white, bottomtitle=1mm, leftrule=.75mm, rightrule=.15mm, titlerule=0mm, arc=.35mm, colback=white, opacitybacktitle=0.6, toprule=.15mm, toptitle=1mm, bottomrule=.15mm, coltitle=black, breakable, left=2mm, opacityback=0]

Descriptive statistics alone can be misleading. Always look at the
graph. Numbers may be identical, but patterns can be radically
different.

\end{tcolorbox}

\subsection{Datasaurus Dozen}\label{datasaurus-dozen}

If you thought that Anscombe's quartet was neat, you should take a look
at the
\href{https://www.autodeskresearch.com/publications/samestats}{Datasaurus
Dozen} (Matejka and Fitzmaurice 2017). hey constructed 13 datasets with
nearly identical means, variances, and correlations --- but when
plotted, the points form shapes like a star, a circle, or even a
dinosaur. Another reminder: your numbers might look like a dinosaur if
you don't plot them.

\section{Videos}\label{videos-1}

\subsection{Measures of center: Mode}\label{measures-of-center-mode}

\subsection{Measures of center: Median and
Mean}\label{measures-of-center-median-and-mean}

\subsection{Standard deviation part I}\label{standard-deviation-part-i}

\subsection{Standard deviation part
II}\label{standard-deviation-part-ii}

\bookmarksetup{startatroot}

\chapter{Correlation}\label{Correlation}

\begin{quote}
Correlation does not equal causation ---Every Statistics and Research
Methods Instructor Ever
\end{quote}

In the last chapter, we worked with data that felt overwhelming at
first. We used plots and histograms to make the data visible, and
descriptive statistics to summarize their central tendency and
variability.

The reason we learned those tools is simple: we want to use data to
answer questions. That theme runs through this course. As you read each
section, keep asking yourself: how does this concept help me answer
questions with data?

In Chapter 2, we imagined collecting self-reports of happiness from a
group of people. The data were just numbers, but they already raised
useful questions. How spread out are people's happiness ratings? Do most
people cluster near the middle, or are there groups of especially happy
or unhappy individuals?

We also saw that numbers are imperfect reflections of experience. A
rating from 0 to 100 is just one way of measuring happiness. It may not
capture the full construct, but for now we will treat it as meaningful
enough to work with.

This sets up the next step: moving from description to explanation.
Rather than just asking how much happiness people report, we might ask
what predicts differences in happiness? For example, does happiness vary
with weather, income, or education? With social connections or cultural
factors? Or with something lighter, like access to chocolate? Questions
like these push us beyond describing data toward identifying
relationships --- the core of correlation analysis.

\section{If something caused something else to change, what would that
look
like?}\label{if-something-caused-something-else-to-change-what-would-that-look-like}

To use data for explanation, we need tools for recognizing when two
things move together. Sometimes that reflects a real cause-and-effect
relationship, and sometimes it does not. The challenge is learning to
tell the difference. In this chapter we'll build intuition for what data
look like when two variables are unrelated, when they move together, and
when the pattern might mislead us.

\subsection{Charlie and the Chocolate
factory}\label{charlie-and-the-chocolate-factory}

Let's imagine that a person's supply of chocolate has a causal influence
on their level of happiness. Let's further imagine that, like Charlie,
the more chocolate you have the more happy you will be, and the less
chocolate you have, the less happy you will be. Finally, because we
suspect happiness is caused by lots of other things in a person's life,
we anticipate that the relationship between chocolate supply and
happiness won't be perfect. What do these assumptions mean for how the
data should look?

Our first step is to collect some imaginary data from 100 people. We
walk around and ask the first 100 people we meet to answer two
questions:

\begin{enumerate}
\def\labelenumi{\arabic{enumi}.}
\tightlist
\item
  how much chocolate do you have, and
\item
  how happy are you.
\end{enumerate}

For convenience, both the scales will go from 0 to 100. For the
chocolate scale, 0 means no chocolate, 100 means lifetime supply of
chocolate. Any other number is somewhere in between. For the happiness
scale, 0 means no happiness, 100 means all of the happiness, and in
between means some amount in between.

Here is some sample data from the first 10 imaginary subjects.

\begin{longtable}[]{@{}rrr@{}}
\toprule\noalign{}
subject & chocolate & happiness \\
\midrule\noalign{}
\endhead
\bottomrule\noalign{}
\endlastfoot
1 & 1 & 1 \\
2 & 2 & 2 \\
3 & 2 & 3 \\
4 & 3 & 4 \\
5 & 4 & 5 \\
6 & 6 & 3 \\
7 & 5 & 5 \\
8 & 6 & 6 \\
9 & 8 & 6 \\
10 & 8 & 9 \\
\end{longtable}

We asked each subject two questions so there are two scores for each
subject, one for their chocolate supply, and one for their level of
happiness. You might already notice some relationships between amount of
chocolate and level of happiness in the table. To make those
relationships even more clear, let's plot all of the data in a graph.

\subsection{Scatter plots}\label{scatter-plots}

When you have two measurements worth of data, you can always turn them
into dots and plot them in a scatter plot. A scatter plot has a
horizontal x-axis, and a vertical y-axis. You get to choose which
measurement goes on which axis. Let's put chocolate supply on the
x-axis, and happiness level on the y-axis. Figure~\ref{fig-3scatter1}
shows 100 dots for each subject.

\begin{figure}

\centering{

\includegraphics[width=0.75\linewidth,height=\textheight,keepaspectratio]{03-Correlation_files/figure-pdf/fig-3scatter1-1.pdf}

}

\caption{\label{fig-3scatter1}Imaginary data showing a positive
correlation between amount of chocolate and amount happiness}

\end{figure}%

You might be wondering, why are there only 100 dots for the data. Didn't
we collect 100 measures for chocolate, and 100 measures for happiness,
shouldn't there be 200 dots? Nope. Each dot is for one subject, there
are 100 subjects, so there are 100 dots.

What do the dots mean? Each dot has two coordinates, an x-coordinate for
chocolate, and a y-coordinate for happiness. The first dot, all the way
on the bottom left is the first subject in the table, who had close to 0
chocolate and close to zero happiness. You can look at any dot, then
draw a straight line down to the x-axis: that will tell you how much
chocolate that subject has. You can draw a straight line left to the
y-axis: that will tell you how much happiness the subject has.

Looking at the \textbf{scatter plot}, we can see that the dots show a
pattern. People with less chocolate tend to report lower happiness,
while those with more chocolate tend to report higher happiness. It
looks like the more chocolate you have the happier you will be, and
vice-versa. This kind of relationship is called a \textbf{positive
correlation}.

\subsection{Positive, Negative, and
No-Correlation}\label{positive-negative-and-no-correlation}

Seeing as we are in the business of imagining data, let's imagine some
more. We've already imagined what data would look like if larger
chocolate supplies increase happiness. We'll show that again in a bit.
What do you imagine the scatter plot would look like if the relationship
was reversed, and larger chocolate supplies decreased happiness. Or,
what do you imagine the scatter plot would look like if there was no
relationship, and the amount of chocolate that you have doesn't do
anything to your happiness. We invite your imagination to look at
Figure~\ref{fig-3posnegrand}:

\begin{figure}

\centering{

\includegraphics[width=0.75\linewidth,height=\textheight,keepaspectratio]{03-Correlation_files/figure-pdf/fig-3posnegrand-1.pdf}

}

\caption{\label{fig-3posnegrand}Three scatterplots showing negative,
positive, and zero correlation}

\end{figure}%

The first panel shows a \textbf{negative correlation}. Happiness goes
down as chocolate supply increases. Negative correlation occurs when one
thing goes up and the other thing goes down; or, when more of X is less
of Y, and vice-versa. The second panel shows a \textbf{positive
correlation}. Happiness goes up as chocolate as chocolate supply
increases. Positive correlation occurs when both things go up together,
and go down together: more of X is more of Y, and vice-versa. The third
panel shows \textbf{no correlation}. Here, there doesn't appear to be
any obvious relationship between chocolate supply and happiness. The
dots are scattered all over the place, the truest of the scatter plots.

\begin{tcolorbox}[enhanced jigsaw, title=\textcolor{quarto-callout-note-color}{\faInfo}\hspace{0.5em}{Note}, colframe=quarto-callout-note-color-frame, colbacktitle=quarto-callout-note-color!10!white, bottomtitle=1mm, leftrule=.75mm, rightrule=.15mm, titlerule=0mm, arc=.35mm, colback=white, opacitybacktitle=0.6, toprule=.15mm, toptitle=1mm, bottomrule=.15mm, coltitle=black, breakable, left=2mm, opacityback=0]

We are wading into the idea that measures of two things can be related,
or correlated with one another. It is possible for the relationships to
be more complicated than just going up, or going down. For example, we
could have a relationship that where the dots go up for the first half
of X, and then go down for the second half.

\end{tcolorbox}

Zero correlation occurs when one thing is not related in any way to
another things: changes in X do not relate to any changes in Y, and
vice-versa.

\section{Pearson's r}\label{pearsons-r}

``So you've examined your scatter plots and now you might be wondering
how to quantify what you see. We've already covered how to generate
descriptive statistics for individual variables---think of single
measures like happiness levels or chocolate consumption, summarized
through means, variances, and so on. But what if you want to capture the
relationship between two such variables in a single descriptive
statistic? Is that even possible? Karl Pearson to the rescue.

\begin{tcolorbox}[enhanced jigsaw, title=\textcolor{quarto-callout-note-color}{\faInfo}\hspace{0.5em}{Note}, colframe=quarto-callout-note-color-frame, colbacktitle=quarto-callout-note-color!10!white, bottomtitle=1mm, leftrule=.75mm, rightrule=.15mm, titlerule=0mm, arc=.35mm, colback=white, opacitybacktitle=0.6, toprule=.15mm, toptitle=1mm, bottomrule=.15mm, coltitle=black, breakable, left=2mm, opacityback=0]

The stories about the invention of various statistics are very
interesting, you can read more about them in the book, ``The Lady
Tasting Tea'' (Salsburg 2001)

\end{tcolorbox}

There's a statistic for that, and Karl Pearson invented it. Everyone now
calls it, ``Pearson's \(r\)''. We will find out later that Karl Pearson
was a big-wig editor at Biometrika in the 1930s. He took a hating to
another big-wig statistician, Sir Ronald Fisher (who we learn about
later), and they had some statistics fights. Even in the stats world,
not everyone plays nice in the sandbox.

How does Pearson's \(r\) work? Let's look again at the first 10 subjects
in our fake experiment:

\begin{longtable}[]{@{}lll@{}}
\toprule\noalign{}
subject & chocolate & happiness \\
\midrule\noalign{}
\endhead
\bottomrule\noalign{}
\endlastfoot
1 & 1 & 1 \\
2 & 2 & 2 \\
3 & 2 & 3 \\
4 & 3 & 4 \\
5 & 4 & 5 \\
6 & 6 & 3 \\
7 & 5 & 5 \\
8 & 6 & 6 \\
9 & 8 & 6 \\
10 & 8 & 9 \\
Sums & 45 & 44 \\
Means & 4.5 & 4.4 \\
\end{longtable}

What could we do to these numbers to produce a single summary value that
represents the relationship between the chocolate supply and happiness?

\subsection{The idea of co-variance}\label{the-idea-of-co-variance}

Let's revisit the idea of variance. Variance tells us how much a set of
numbers differs --- some values are larger, some are smaller. We've seen
variance for a single variable, like happiness ratings. Covariance
extends this idea to two variables: it describes how their changes
relate to each other.

We can see that there is variance in chocolate supply across the 10
subjects. We can see that there is variance in happiness across the 10
subjects. We also saw in the scatter plot, that happiness increases as
chocolate supply increases; which is a positive relationship, a positive
correlation. What does this have to do with variance? Well, it means
there is a relationship between the variance in chocolate supply, and
the variance in happiness levels. The two measures vary together don't
they? When we have two measures that vary together, they are like a
happy couple who share their variance. This is what co-variance refers
to, the idea that the pattern of varying numbers in one measure is
shared by the pattern of varying numbers in another measure.

\textbf{Co-variance} is very important. It underlies correlation and
many other statistical tools. At first the concept may feel abstract,
especially if variance itself still feels new. But working with it
repeatedly will build the intuition you need for understanding
relationships in data.

\begin{quote}
Pro tip: You can think of covariance with a three-legged race. When two
people move forward together, their steps are aligned - a positive
relationship. If one moves forward while the other moves back, their
movements oppose each other - a negative relationship. If each person
moves randomly without coordination, there's still variation, but no
consistent relationship between them.
\end{quote}

\section{Turning the numbers into a measure of
co-variance}\label{turning-the-numbers-into-a-measure-of-co-variance}

``OK, so if you are saying that co-variance is just another word for
correlation or relationship between two measures, I'm good with that. I
suppose we would need some way to measure that.'' Correct, back to our
table\ldots notice anything new?

\begin{longtable}[]{@{}llll@{}}
\toprule\noalign{}
subject & chocolate & happiness & Chocolate\_X\_Happiness \\
\midrule\noalign{}
\endhead
\bottomrule\noalign{}
\endlastfoot
1 & 1 & 1 & 1 \\
2 & 2 & 2 & 4 \\
3 & 2 & 3 & 6 \\
4 & 3 & 4 & 12 \\
5 & 4 & 5 & 20 \\
6 & 6 & 3 & 18 \\
7 & 5 & 5 & 25 \\
8 & 6 & 6 & 36 \\
9 & 8 & 6 & 48 \\
10 & 8 & 9 & 72 \\
Sums & 45 & 44 & 242 \\
Means & 4.5 & 4.4 & 24.2 \\
\end{longtable}

We've added a new column called \texttt{Chocolate\_X\_Happiness}, which
translates to Chocolate scores multiplied by Happiness scores. Each row
in the new column, is the product, or multiplication of the chocolate
and happiness score for that row. Yes, but why would we do this?

Last chapter we took you back to Elementary school and had you think
about division. Now it's time to do the same thing with multiplication.
We assume you know how that works. One number times another, means
taking the first number, and adding it as many times as the second says
to do,

\(2*2= 2+2=4\)

\(2*6= 2+2+2+2+2+2 = 12\), or \(6+6=12\), same thing.

Yes, you already know how multiplication works. The key point here is
that multiplying pairs of numbers gives us a way to summarize how two
variables change together. Products of large-with-large or
small-with-small values accumulate differently than mixed pairs, and
this difference forms the basis of covariance.

We know how to multiple numbers, and all we have to next is think about
the consequences of multiplying sets of numbers together. For example,
what happens when you multiply two small numbers together, compared to
multiplying two big numbers together? The first product should be
smaller than the second product right? How about things like multiplying
a small number by a big number? Those products should be in between
right?.

Then next step is to think about how the products of two measures sum
together, depending on how they line up. Let's look at another table:

\begin{longtable}[]{@{}lllllll@{}}
\toprule\noalign{}
scores & X & Y & A & B & XY & AB \\
\midrule\noalign{}
\endhead
\bottomrule\noalign{}
\endlastfoot
1 & 1 & 1 & 1 & 10 & 1 & 10 \\
2 & 2 & 2 & 2 & 9 & 4 & 18 \\
3 & 3 & 3 & 3 & 8 & 9 & 24 \\
4 & 4 & 4 & 4 & 7 & 16 & 28 \\
5 & 5 & 5 & 5 & 6 & 25 & 30 \\
6 & 6 & 6 & 6 & 5 & 36 & 30 \\
7 & 7 & 7 & 7 & 4 & 49 & 28 \\
8 & 8 & 8 & 8 & 3 & 64 & 24 \\
9 & 9 & 9 & 9 & 2 & 81 & 18 \\
10 & 10 & 10 & 10 & 1 & 100 & 10 \\
Sums & 55 & 55 & 55 & 55 & 385 & 220 \\
Means & 5.5 & 5.5 & 5.5 & 5.5 & 38.5 & 22 \\
\end{longtable}

Look at the \(X\) and \(Y\) column. The scores for \(X\) and \(Y\)
perfectly co-vary. When \(X\) is 1, \(Y\) is 1; when \(X\) is 2, \(Y\)
is 2, etc. They are perfectly aligned. The scores for \(A\) and \(B\)
also perfectly co-vary, just in the opposite manner. When \(A\) is 1,
\(B\) is 10; when \(A\) is 2, \(B\) is 9, etc. \(B\) is a reversed copy
of \(A\).

Now, look at the column \(XY\). These are the products we get when we
multiply the values of \(X\) across with the values of \(Y\). Also, look
at the column \(AB\). These are the products we get when we multiply the
values of A across with the values of B. So far so good.

Now, look at the \texttt{Sums} for the \(XY\) and \(AB\) columns. Not
the same. The sum of the \(XY\) products is 385, and the sum of the
\(AB\) products is 220. For this specific set of data, the numbers 385
and 220 are very important. They represent the biggest possible sum of
products (385), and the smallest possible sum of products (220). There
is no way of re-ordering the numbers 1 to 10, say for \(X\), and the
numbers 1 to 10 for \(Y\), that would ever produce larger or smaller
numbers. Don't believe me? Check this out:

\begin{figure}

\centering{

\includegraphics[width=0.75\linewidth,height=\textheight,keepaspectratio]{03-Correlation_files/figure-pdf/fig-3simsum-1.pdf}

}

\caption{\label{fig-3simsum}Simulated sums of products showing the kinds
of values than can be produced by randomly ordering the numbers in X and
Y.}

\end{figure}%

Figure~\ref{fig-3simsum} shows 1000 computer simulations. I convinced my
computer to randomly order the numbers 1 to 10 for X, and randomly order
the numbers 1 to 10 for Y. Then, I multiplied X and Y, and added the
products together. I did this 1000 times. The dots show the sum of the
products for each simulation. The two black lines show the maximum
possible sum (385), and the minimum possible sum (220), for this set of
numbers. Notice, how all of the dots are in between the maximum and
minimum possible values. Told you so.

``OK fine, you told me so\ldots So what, who cares?''. We've been
looking for a way to summarize the co-variance between two measures
right? Well, for these numbers, we have found one, haven't we. It's the
sum of the products. We know that when the sum of the products is 385,
we have found a perfect, positive correlation. We know, that when the
sum of the products is 220, we have found a perfect negative
correlation. What about the numbers in between. What could we conclude
about the correlation if we found the sum of the products to be 350.
Well, it's going to be positive, because it's close to 385, and that's
perfectly positive. If the sum of the products was 240, that's going to
be negative, because it's close to the perfectly negatively correlating
220. What about no correlation? Well, that's going to be in the middle
between 220 and 385 right.

We have just come up with a data-specific summary measure for the
correlation between the numbers 1 to 10 in X, and the numbers 1 to 10 in
Y, it's the sum of the products. We know the maximum (385) and minimum
values (220), so we can now interpret any product sum for this kind of
data with respect to that scale.

\begin{quote}
Pro tip: When the correlation between two measures increases in the
positive direction, the sum of their products increases to its maximum
possible value. This is because the bigger numbers in X will tend to
line up with the bigger numbers in Y, creating the biggest possible sum
of products. When the correlation between two measures increases in the
negative direction, the sum of their products decreases to its minimum
possible value. This is because the bigger numbers in X will tend to
line up with the smaller numbers in Y, creating the smallest possible
sum of products. When there is no correlation, the big numbers in X will
be randomly lined up with the big and small numbers in Y, making the sum
of the products, somewhere in the middle.
\end{quote}

\subsection{Co-variance, the measure}\label{co-variance-the-measure}

We took some time to see what happens when you multiply sets of numbers
together. We found that \(big*big = bigger\) and
\(small*small=\text{still small}\), and
\(big*small=\text{in the middle}\). The purpose of this was to give you
some conceptual idea of how the co-variance between two measures is
reflected in the sum of their products. We did something very
straightforward. We just multiplied X with Y, and looked at how the
product sums get big and small, as X and Y co-vary in different ways.

Now, we can get a little bit more formal. In statistics,
\textbf{co-variance} is not just the straight multiplication of values
in X and Y. Instead, it's the multiplication of the deviations in X from
the mean of X, and the deviation in Y from the mean of Y. Remember those
difference scores from the mean we talked about last chapter? They're
coming back to haunt you know, but in a good way like Casper the
friendly ghost.

Let's see what this look like in a table:

\begin{longtable}[]{@{}llllll@{}}
\toprule\noalign{}
subject & chocolate & happiness & C\_d & H\_d & Cd\_x\_Hd \\
\midrule\noalign{}
\endhead
\bottomrule\noalign{}
\endlastfoot
1 & 1 & 1 & -3.5 & -3.4 & 11.9 \\
2 & 2 & 2 & -2.5 & -2.4 & 6 \\
3 & 2 & 3 & -2.5 & -1.4 & 3.5 \\
4 & 3 & 4 & -1.5 & -0.4 & 0.6 \\
5 & 4 & 5 & -0.5 & 0.6 & -0.3 \\
6 & 6 & 3 & 1.5 & -1.4 & -2.1 \\
7 & 5 & 5 & 0.5 & 0.6 & 0.3 \\
8 & 6 & 6 & 1.5 & 1.6 & 2.4 \\
9 & 8 & 6 & 3.5 & 1.6 & 5.6 \\
10 & 8 & 9 & 3.5 & 4.6 & 16.1 \\
Sums & 45 & 44 & 0 & 0 & 44 \\
Means & 4.5 & 4.4 & 0 & 0 & 4.4 \\
\end{longtable}

We have computed the deviations from the mean for the chocolate scores
(column \texttt{C\_d}), and the deviations from the mean for the
happiness scores (column \texttt{H\_d}). Then, we multiplied them
together (last column). Finally, you can see the mean of the products
listed in the bottom right corner of the table, the official \textbf{the
covariance}.

The formula for the co-variance is:

\(cov(X,Y) = \frac{\sum_{i}^{n}(x_{i}-\bar{X})(y_{i}-\bar{Y})}{N}\)

OK, so now we have a formal single number to calculate the relationship
between two variables. This is great, it's what we've been looking for.
However, there is a problem. Remember when we learned how to compute
just the plain old \textbf{variance}. We looked at that number, and we
didn't know what to make of it. It was squared, it wasn't in the same
scale as the original data. So, we square rooted the \textbf{variance}
to produce the \textbf{standard deviation}, which gave us a more
interpretable number in the range of our data. The \textbf{co-variance}
has a similar problem. When you calculate the co-variance as we just
did, we don't know immediately know its scale. Is a 3 big? is a 6 big?
is a 100 big? How big or small is this thing?

From our prelude discussion on the idea of co-variance, we learned the
sum of products between two measures ranges between a maximum and
minimum value. The same is true of the co-variance. For a given set of
data, there is a maximum possible positive value for the co-variance
(which occurs when there is perfect positive correlation). And, there is
a minimum possible negative value for the co-variance (which occurs when
there is a perfect negative correlation). When there is zero
co-variation, guess what happens. Zeroes. So, at the very least, when we
look at a co-variation statistic, we can see what direction it points,
positive or negative. But, we don't know how big or small it is compared
to the maximum or minimum possible value, so we don't know the relative
size, which means we can't say how strong the correlation is. What to
do?

\subsection{Pearson's r we there yet}\label{pearsons-r-we-there-yet}

Yes, we are here now. Wouldn't it be nice if we could force our measure
of co-variation to be between -1 and +1?

-1 would be the minimum possible value for a perfect negative
correlation. +1 would be the maximum possible value for a perfect
positive correlation. 0 would mean no correlation. Everything in between
0 and -1 would be increasingly large negative correlations. Everything
between 0 and +1 would be increasingly large positive correlations. It
would be a fantastic, sensible, easy to interpret system. If only we
could force the co-variation number to be between -1 and 1. Fortunately,
for us, this episode is brought to you by Pearson's \(r\), which does
precisely this wonderful thing.

Let's take a look at a formula for Pearson's \(r\):

\(r = \frac{cov(X,Y)}{\sigma_{X}\sigma_{Y}} = \frac{cov(X,Y)}{SD_{X}SD_{Y}}\)

We see the symbol \(\sigma\) here, that's more Greek for you. \(\sigma\)
is often used as a symbol for the standard deviation (SD). If we read
out the formula in English, we see that r is the co-variance of X and Y,
divided by the product of the standard deviation of X and the standard
deviation of Y. Why are we dividing the co-variance by the product of
the standard deviations. This operation has the effect of
\textbf{normalizing} the co-variance into the range -1 to 1.

\begin{tcolorbox}[enhanced jigsaw, title=\textcolor{quarto-callout-note-color}{\faInfo}\hspace{0.5em}{Note}, colframe=quarto-callout-note-color-frame, colbacktitle=quarto-callout-note-color!10!white, bottomtitle=1mm, leftrule=.75mm, rightrule=.15mm, titlerule=0mm, arc=.35mm, colback=white, opacitybacktitle=0.6, toprule=.15mm, toptitle=1mm, bottomrule=.15mm, coltitle=black, breakable, left=2mm, opacityback=0]

But, we will fill this part in as soon as we can\ldots promissory note
to explain the magic. FYI, it's not magic. Brief explanation here is
that dividing each measure by its standard deviation ensures that the
values in each measure are in the same range as one another.

\end{tcolorbox}

For now, we will call this mathematical magic. It works, but we don't
have space to tell you why it works right now.

There are several equivalent formulas for computing Pearson's \(r\).
Some are written to simplify hand calculations, others are more compact
algebraically. They all yield the same result. Since we rely on software
for calculations, our focus will be on what \(r\) means and how to
interpret it, rather than practicing manual computation. You'll see how
to run the calculation in R during lab.

Does Pearson's \(r\) really stay between -1 and 1 no matter what? It's
true, take a look at the following simulation. Here I randomly ordered
the numbers 1 to 10 for an X measure, and did the same for a Y measure.
Then, I computed Pearson's \(r\), and repeated this process 1000 times.
As you can see from Figure~\ref{fig-3onethousandr} all of the dots are
between -1 and 1. Neat huh.

\begin{figure}

\centering{

\includegraphics[width=0.75\linewidth,height=\textheight,keepaspectratio]{03-Correlation_files/figure-pdf/fig-3onethousandr-1.pdf}

}

\caption{\label{fig-3onethousandr}A simulation of of correlations. Each
dot represents the r-value for the correlation between an X and Y
variable that each contain the numbers 1 to 10 in random orders. The
figure ilustrates that many r-values can be obtained by this random
process}

\end{figure}%

\section{Examples with Data}\label{examples-with-data}

In the lab for correlation you will be shown how to compute correlations
in real data-sets using software. To give you a brief preview, let's
look at some data from the \href{http://worldhappiness.report}{world
happiness report} (2018).

This report measured various attitudes across people from different
countries. For example, one question asked about how much freedom people
thought they had to make life choices. Another question asked how
confident people were in their national government.
\textbf{?@fig-3hrsdata} is a scatterplot showing the relationship
between these two measures. Each dot represents means for different
countries.

\begin{figure}

\centering{

\includegraphics[width=0.75\linewidth,height=\textheight,keepaspectratio]{03-Correlation_files/figure-pdf/fig-3hsrdata-1.pdf}

}

\caption{\label{fig-3hsrdata}Relationship between freedom to make life
choices and confidence in national government. Data from the world
happiness report for 2018}

\end{figure}%

We put a blue line on the scatterplot to summarize the positive
relationship. It appears that as ``freedom to make life choices goes
up'', so to does confidence in national government. It's a positive
correlation.

The actual correlation, as measured by Pearson's \(r\) is:

\begin{verbatim}
#> [1] 0.4080963
\end{verbatim}

You will do a lot more of this kind of thing in the lab. Looking at the
graph you might start to wonder: Does freedom to make life choices cause
changes how confident people are in their national government? Our does
it work the other way? Does being confident in your national government
give you a greater sense of freedom to make life choices? Or, is this
just a random relationship that doesn't mean anything? All good
questions. These data do not provide the answers, they just suggest a
possible relationship.

\section{Regression: A mini intro}\label{regression-a-mini-intro}

We're going to spend the next little bit adding one more thing to our
understanding of correlation. It's called \textbf{linear regression}. It
sounds scary, and it really is. You'll find out much later in your
Statistics education that everything we will be soon be talking about
can be thought of as a special case of regression. But, we don't want to
scare you off, so right now we just introduce the basic concepts.

First, let's look at a linear regression. This way we can see what we're
trying to learn about. Figure~\ref{fig-3regression} shows the same
scatter plots as before with something new: lines!

\begin{figure}

\centering{

\includegraphics[width=0.75\linewidth,height=\textheight,keepaspectratio]{03-Correlation_files/figure-pdf/fig-3regression-1.pdf}

}

\caption{\label{fig-3regression}Three scatterplots showing negative,
positive, and a random correlation (where the r-value is expected to be
0), along with the best fit regression line}

\end{figure}%

\subsection{The best fit line}\label{the-best-fit-line}

Notice anything about these blue lines? Hopefully you can see, at least
for the first two panels, that they go straight through the data. We
call these lines \textbf{best fit} lines, because according to our
definition (soon we promise) there are no other lines that you could
draw that would do a better job of going straight throw the data.

One big idea here is that we are using the line as a kind of mean to
describe the relationship between the two variables. When we only have
one variable, that variable exists on a single dimension, it's 1D. So,
it is appropriate that we only have one number, like the mean, to
describe it's central tendency. When we have two variables, and plot
them together, we now have a two-dimensional space. So, for two
dimensions we could use a bigger thing that is 2d, like a line, to
summarize the central tendency of the relationship between the two
variables.

What do we want out of our line? Well, if you had a pencil, and a
printout of the data, you could draw all sorts of straight lines any way
you wanted. Your lines wouldn't even have to go through the data, or
they could slant through the data with all sorts of angles. Would all of
those lines be very good a describing the general pattern of the dots?
Most of them would not. The best lines would go through the data
following the general shape of the dots. Of the best lines, however,
which one is the best? How can we find out, and what do we mean by that?
In short, the best fit line is the one that has the least error.

\begin{tcolorbox}[enhanced jigsaw, title=\textcolor{quarto-callout-note-color}{\faInfo}\hspace{0.5em}{Note}, colframe=quarto-callout-note-color-frame, colbacktitle=quarto-callout-note-color!10!white, bottomtitle=1mm, leftrule=.75mm, rightrule=.15mm, titlerule=0mm, arc=.35mm, colback=white, opacitybacktitle=0.6, toprule=.15mm, toptitle=1mm, bottomrule=.15mm, coltitle=black, breakable, left=2mm, opacityback=0]

R code for plotting residuals thanks to Simon Jackson's blog post:
\url{https://drsimonj.svbtle.com/visualising-residuals}

\end{tcolorbox}

Check out this next plot, it shows a line through some dots. But, it
also shows some teeny tiny lines. These lines drop down from each dot,
and they land on the line. Each of these little lines is called a
\textbf{residual}. They show you how far off the line is for different
dots. It's measure of error, it shows us just how wrong the line is.
After all, it's pretty obvious that not all of the dots are on the line.
This means the line does not actually represent all of the dots. The
line is wrong. But, the best fit line is the least wrong of all the
wrong lines.

\begin{figure}

\centering{

\includegraphics[width=0.75\linewidth,height=\textheight,keepaspectratio]{03-Correlation_files/figure-pdf/fig-3regressionResiduals-1.pdf}

}

\caption{\label{fig-3regressionResiduals}Black dots represent data
points. The blue line is the best fit regression line. The white dots
are repesent the predicted location of each black dot. The red lines
show the error between each black dot and the regression line. The blue
line is the best fit line because it minimizes the error shown by the
red lines.}

\end{figure}%

There's a lot going on in Figure~\ref{fig-3regressionResiduals}. First,
we are looking at a scatter plot of two variables, an X and Y variable.
Each of the black dots are the actual values from these variables. You
can see there is a negative correlation here, as X increases, Y tends to
decrease. We drew a regression line through the data, that's the blue
line. There's these little white dots too. This is where the line thinks
the black dots should be. The red lines are the important residuals
we've been talking about. Each black dot has a red line that drops
straight down, or straight up from the location of the black dot, and
lands directly on the line. We can already see that many of the dots are
not on the line, so we already know the line is ``off'' by some amount
for each dot. The red line just makes it easier to see exactly how off
the line is.

The important thing that is happening here, is that the the blue line is
drawn is such a way, that it minimizes the total length of the red
lines. For example, if we wanted to know how wrong this line was, we
could simply gather up all the red lines, measure how long they are, and
then add all the wrongness together. This would give us the total amount
of wrongness. We usually call this the error. In fact, we've already
talked about this idea before when we discussed standard deviation. What
we will actually be doing with the red lines, is computing the sum of
the squared deviations from the line. That sum is the total amount of
error. Now, this blue line here minimizes the sum of the squared
deviations. Any other line would produce a larger total error.

\textbf{?@fig-3regressionGIF} is an animation to see this in action. The
animations compares the best fit line in blue, to some other possible
lines in black. The black line moves up and down. The red lines show the
error between the black line and the data points. As the black line
moves toward the best fit line, the total error, depicted visually by
the grey area shrinks to it's minimum value. The total error expands as
the black line moves away from the best fit line.

Whenever the black line does not overlap with the blue line, it fits the
data less well. The blue regression line is the one that minimizes the
overall error --- not too high, not too low, but the line with the least
total deviation.

Figure~\ref{fig-3minimizeSS} shows how the sum of squared deviations
(the sum of the squared lengths of the red lines) behaves as we move the
line up and down. What's going on here is that we are computing a
measure of the total error as the black line moves through the best fit
line. This represents the sum of the squared deviations. In other words,
we square the length of each red line from the above animation, then we
add up all of the squared red lines, and get the total error (the total
sum of the squared deviations). The graph below shows what the total
error looks like as the black line approaches then moves away from the
best fit line. Notice, the dots in this graph start high on the left
side, then they swoop down to a minimum at the bottom middle of the
graph. When they reach their minimum point, we have found a line that
minimizes the total error. This is the best fit regression line.

\begin{figure}

\centering{

\includegraphics[width=0.75\linewidth,height=\textheight,keepaspectratio]{03-Correlation_files/figure-pdf/fig-3minimizeSS-1.pdf}

}

\caption{\label{fig-3minimizeSS}A plot of the sum of the squared
deviations for different lines moving up and down, through the best fit
line. The best fit line occurs at the position that minimizes the sum of
the sqaured deviations.}

\end{figure}%

OK, so we haven't talked about the y-intercept yet. But, what this graph
shows us is how the total error behaves as we move the line up and down.
The y-intercept here is the thing we change that makes our line move up
and down. As you can see the dots go up when we move the line down from
0 to -5, and the dots go up when we move the line up from 0 to +5. The
best line, that minimizes the error occurs right in the middle, when we
don't move the blue regression line at all.

\subsection{Lines}\label{lines}

OK, fine you say. So, there is one magic line that will go through the
middle of the scatter plot and minimize the sum of the squared
deviations. How do I find this magic line? We'll show you. But, to be
completely honest, you'll almost never do it the way we'll show you
here. Instead, it's much easier to use software and make your computer
do it for. You'll learn how to that in the labs.

Before we show you how to find the regression line, it's worth
refreshing your memory about how lines work, especially in 2 dimensions.
Remember this?

\(y = ax + b\), or also \(y = mx + b\) (sometimes a or m is used for the
slope)

This is the formula for a line. Another way of writing it is:

\(y = slope * x + \text{y-intercept}\)

The slope is the slant of the line, and the y-intercept is where the
line crosses the y-axis. Let's look at the lines in
Figure~\ref{fig-3twolines}.

\begin{figure}

\centering{

\includegraphics[width=0.5\linewidth,height=\textheight,keepaspectratio]{03-Correlation_files/figure-pdf/fig-3twolines-1.pdf}

}

\caption{\label{fig-3twolines}Two different lines with different
y-intercepts (where the line crosses the y-axis), and different slopes.
A positive slope makes the line go up from left to right. A negative
slope makes the line go down from left to right.}

\end{figure}%

The formula for the blue line is \(y = 1*x + 5\). Let's talk about that.
When x = 0, where is the blue line on the y-axis? It's at five. That
happens because 1 times 0 is 0, and then we just have the five left
over. How about when x = 5? In that case y =10. You just need the plug
in the numbers to the formula, like this:

\(y = 1*x + 5\) \(y = 1*5 + 5 = 5+5 =10\)

The point of the formula is to tell you where y will be, for any number
of x. The slope of the line tells you whether the line is going to go up
or down, as you move from the left to the right. The blue line has a
positive slope of one, so it goes up as x goes up. How much does it go
up? It goes up by one for everyone one of x! If we made the slope a 2,
it would be much steeper, and go up faster. The red line has a negative
slope, so it slants down. This means \(y\) goes down, as \(x\) goes up.
When there is no slant, and we want to make a perfectly flat line, we
set the slope to 0. This means that y doesn't go anywhere as x gets
bigger and smaller.

That's lines.

\subsection{Computing the best fit
line}\label{computing-the-best-fit-line}

If you have a scatter plot showing the locations of scores from two
variables, the real question is how can you find the slope and the
y-intercept for the best fit line? What are you going to do? Draw
millions of lines, add up the residuals, and then see which one was
best? That would take forever. Fortunately, there are computers, and
when you don't have one around, there's also some handy formulas.

\begin{tcolorbox}[enhanced jigsaw, title=\textcolor{quarto-callout-note-color}{\faInfo}\hspace{0.5em}{Note}, colframe=quarto-callout-note-color-frame, colbacktitle=quarto-callout-note-color!10!white, bottomtitle=1mm, leftrule=.75mm, rightrule=.15mm, titlerule=0mm, arc=.35mm, colback=white, opacitybacktitle=0.6, toprule=.15mm, toptitle=1mm, bottomrule=.15mm, coltitle=black, breakable, left=2mm, opacityback=0]

It's worth noting that many formulas in statistics were developed for
convenience in hand calculations, before computers were available. Today
we rely on software for computation, but seeing one worked example by
hand helps illustrate the logic behind the formulas.

\end{tcolorbox}

Next we'll look at the formulas for calculating the slope and intercept
of a regression line, and work through one example by hand. The goal is
not to memorize the algebra, but to see how the pieces fit together
before we shift to using software.

Here are two formulas we can use to calculate the slope and the
intercept, straight from the data. We won't go into why these formulas
do what they do. These ones are for ``easy'' calculation.

\(intercept = b = \frac{\sum{y}\sum{x^2}-\sum{x}\sum{xy}}{n\sum{x^2}-(\sum{x})^2}\)

\(slope = m = \frac{n\sum{xy}-\sum{x}\sum{y}}{n\sum{x^2}-(\sum{x})^2}\)

In these formulas, the \(x\) and the \(y\) refer to the individual
scores. Here's a table showing you how everything fits together.

\begin{longtable}[]{@{}llllll@{}}
\toprule\noalign{}
scores & x & y & x\_squared & y\_squared & xy \\
\midrule\noalign{}
\endhead
\bottomrule\noalign{}
\endlastfoot
1 & 1 & 2 & 1 & 4 & 2 \\
2 & 4 & 5 & 16 & 25 & 20 \\
3 & 3 & 1 & 9 & 1 & 3 \\
4 & 6 & 8 & 36 & 64 & 48 \\
5 & 5 & 6 & 25 & 36 & 30 \\
6 & 7 & 8 & 49 & 64 & 56 \\
7 & 8 & 9 & 64 & 81 & 72 \\
Sums & 34 & 39 & 200 & 275 & 231 \\
\end{longtable}

We see 7 sets of scores for the x and y variable. We calculated \(x^2\)
by squaring each value of x, and putting it in a column. We calculated
\(y^2\) by squaring each value of y, and putting it in a column. Then we
calculated \(xy\), by multiplying each \(x\) score with each \(y\)
score, and put that in a column. Then we added all the columns up, and
put the sums at the bottom. These are all the number we need for the
formulas to find the best fit line. Here's what the formulas look like
when we put numbers in them:

\(intercept = b = \frac{\sum{y}\sum{x^2}-\sum{x}\sum{xy}}{n\sum{x^2}-(\sum{x})^2} = \frac{39 * 200 - 34*231}{7*200-34^2} = -.221\)

\(slope = m = \frac{n\sum{xy}-\sum{x}\sum{y}}{n\sum{x^2}-(\sum{x})^2} = \frac{7*231-34*39}{7*275-34^2} = 1.19\)

Great, now we can check our work, let's plot the scores in a scatter
plot and draw a line through it with slope = 1.19, and a y-intercept of
-.221. As shown in Figure~\ref{fig-3corwithLine}, the line should go
through the middle of the dots.

\begin{figure}

\centering{

\includegraphics[width=0.75\linewidth,height=\textheight,keepaspectratio]{03-Correlation_files/figure-pdf/fig-3corwithLine-1.pdf}

}

\caption{\label{fig-3corwithLine}An example regression line with
confidence bands going through a few data points in a scatterplot}

\end{figure}%

\section{Interpreting Correlations}\label{interpreting-correlations}

What does the presence or the absence of a correlation between two
measures mean? How should correlations be interpreted? What kind of
inferences can be drawn from correlations? These are all very good
questions. A first piece of advice is to use caution when interpreting
correlations. Here's why.

\subsection{Correlation does not equal
causation}\label{correlation-does-not-equal-causation}

Perhaps you have heard that correlation does not equal causation. Why
not? There are lots of reasons why not. However, before listing some of
the reasons let's start with a case where we would expect a causal
connection between two measurements. Consider, buying a snake plant for
your home. Snake plants are supposed to be easy to take care of because
you can mostly ignore them.

Like most plants, snake plants need some water to stay alive. However,
they also need just the right amount of water. Imagine an experiment
where 1000 snake plants were grown in a house. Each snake plant is given
a different amount of water per day, from zero teaspoons of water per
day to 1000 teaspoons of water per day. We will assume that water is
part of the causal process that allows snake plants to grow. The amount
of water given to each snake plant per day can also be one of our
measures. Imagine further that every week the experimenter measures
snake plant growth, which will be the second measurement. Now, can you
imagine for yourself what a scatter plot of weekly snake plant growth by
tablespoons of water would look like?

\subsubsection{Even when there is causation, there might not be obvious
correlation}\label{even-when-there-is-causation-there-might-not-be-obvious-correlation}

The first plant given no water at all would have a very hard time and
eventually die. It should have the least amount of weekly growth. How
about the plants given only a few teaspoons of water per day. This could
be just enough water to keep the plants alive, so they will grow a
little bit but not a lot. If you are imagining a scatter plot, with each
dot being a snake plant, then you should imagine some dots starting in
the bottom left hand corner (no water \& no plant growth), moving up and
to the right (a bit of water, and a bit of growth). As we look at snake
plants getting more and more water, we should see more and more plant
growth, right? ``Sure, but only up to a point''. Correct, there should
be a trend for a positive correlation with increasing plant growth as
amount of water per day increases. But, what happens when you give snake
plants too much water? Eventually, they die. So, at some point, the dots
in the scatter plot will start moving back down again. Snake plants that
get way too much water will not grow very well.

The imaginary scatter plot you should be envisioning could have an
upside U shape. Going from left to right, the dot's go up, they reach a
maximum, then they go down again reaching a minimum. Computing Pearson's
\(r\) for data like this can give you \(r\) values close to zero. The
scatter plot could look something like Figure~\ref{fig-3snakeplant}.

\begin{figure}

\centering{

\includegraphics[width=0.75\linewidth,height=\textheight,keepaspectratio]{03-Correlation_files/figure-pdf/fig-3snakeplant-1.pdf}

}

\caption{\label{fig-3snakeplant}Illustration of a possible relationship
between amount of water and snake plant growth. Growth goes up with
water, but eventually goes back down as too much water makes snake
plants die.}

\end{figure}%

Granted this looks more like an inverted V, than an inverted U, but you
get the picture right? There is clearly a relationship between watering
and snake plant growth. But, the correlation isn't in one direction. As
a result, when we compute the correlation in terms of Pearson's r, we
get a value suggesting no relationship.

\begin{verbatim}
#> [1] 0.008735413
\end{verbatim}

What this really means is there is no linear relationship that can be
described by a single straight line. When we need lines or curves going
in more than one direction, we have a nonlinear relationship.

This example illustrates some conundrums in interpreting correlations.
We already know that water is needed for plants to grow, so we are
rightly expecting there to be a relationship between our measure of
amount of water and plant growth. If we look at the first half of the
data we see a positive correlation, if we look at the last half of the
data we see a negative correlation, and if we look at all of the data we
see no correlation. Yikes. So, even when there is a causal connection
between two measures, we won't necessarily obtain clear evidence of the
connection just by computing a correlation coefficient.

\begin{quote}
Pro Tip: This is one reason why plotting your data is so important. If
you see an upside U shape pattern, then a correlation analysis is
probably not the best analysis for your data.
\end{quote}

\subsubsection{Confounding variable, or Third variable
problem}\label{confounding-variable-or-third-variable-problem}

Anybody can correlate any two things that can be quantified and
measured. For example, we could find a hundred people, ask them all
sorts of questions like:

\begin{enumerate}
\def\labelenumi{\arabic{enumi}.}
\tightlist
\item
  how happy are you
\item
  how old are you
\item
  how tall are you
\item
  how much money do you make per year
\item
  how long are your eyelashes
\item
  how many books have you read in your life
\item
  how loud is your inner voice
\end{enumerate}

Let's say we found a positive correlation between yearly salary and
happiness. Note, we could have just as easily computed the same
correlation between happiness and yearly salary. If we found a
correlation, would you be willing to infer that yearly salary causes
happiness? Perhaps it does play a small part. But, something like
happiness probably has a lot of contributing causes. Money could
directly cause some people to be happy. But, more likely, money buys
people access to all sorts of things, and some of those things might
contribute happiness. These ``other'' things are called \textbf{third}
variables. For example, perhaps people living in nicer places in more
expensive houses are more happy than people in worse places in cheaper
houses. In this scenario, money isn't causing happiness, it's the places
and houses that money buys. But, even is this were true, people can
still be more or less happy in lots of different situations.

The lesson here is that a correlation can occur between two measures
because of a third variable that is not directly measured. So, just
because we find a correlation, does not mean we can conclude anything
about a causal connection between two measurements.

\subsection{Correlation and Random
chance}\label{correlation-and-random-chance}

Another very important aspect of correlations is the fact that they can
be produced by random chance. This means that you can find a positive or
negative correlation between two measures, even when they have
absolutely nothing to do with one another. You might have hoped to find
zero correlation when two measures are totally unrelated to each other.
Although this certainly happens, unrelated measures can accidentally
produce \textbf{spurious} correlations, just by chance alone.

Let's demonstrate how correlations can occur by chance when there is no
causal connection between two measures. Imagine two participants. One is
at the North pole with a lottery machine full of balls with numbers from
1 to 10. The other is at the south pole with a different lottery machine
full of balls with numbers from 1 to 10. There are an endless supply of
balls in the machine, so every number could be picked for any ball. Each
participant randomly chooses 10 balls, then records the number on the
ball. In this situation we will assume that there is no possible way
that balls chosen by the first participant could causally influence the
balls chosen by the second participant. They are on the other side of
the world. We should assume that the balls will be chosen by chance
alone.

Here is what the numbers on each ball could look like for each
participant:

\begin{longtable}[]{@{}rrr@{}}
\toprule\noalign{}
Ball & North\_pole & South\_pole \\
\midrule\noalign{}
\endhead
\bottomrule\noalign{}
\endlastfoot
1 & 1 & 8 \\
2 & 6 & 5 \\
3 & 6 & 2 \\
4 & 2 & 7 \\
5 & 6 & 5 \\
6 & 7 & 6 \\
7 & 9 & 5 \\
8 & 8 & 6 \\
9 & 9 & 4 \\
10 & 2 & 1 \\
\end{longtable}

In this one case, if we computed Pearson's \(r\), we would find that
\(r =\) -0.1306809. But, we already know that this value does not tell
us anything about the relationship between the balls chosen in the north
and south pole. We know that relationship should be completely random,
because that is how we set up the game.

The better question here is to ask what can random chance do? For
example, if we ran our game over and over again thousands of times, each
time choosing new balls, and each time computing the correlation, what
would we find?First, we will find fluctuation. The r value will
sometimes be positive, sometimes be negative, sometimes be big and
sometimes be small. Second, we will see what the fluctuation looks like.
This will give us a window into the kinds of correlations that chance
alone can produce. Let's see what happens.

\subsubsection{Monte-carlo simulation of random
correlations}\label{monte-carlo-simulation-of-random-correlations}

It is possible to use a computer to simulate our game as many times as
we want. This process is often termed \textbf{monte-carlo simulation}.

Below is a script written for the programming language R. We won't go
into the details of the code here. However, let's briefly explain what
is going on. Notice, the part that says \texttt{for(sim\ in\ 1:1000)}.
This creates a loop that repeats our game 1000 times. Inside the loop
there are variables named \texttt{North\_pole} and \texttt{South\_pole}.
During each simulation, we sample 10 random numbers (between 1 to 10)
into each variable. These random numbers stand for the numbers that
would have been on the balls from the lottery machine. Once we have 10
random numbers for each, we then compute the correlation using
\texttt{cor(North\_pole,South\_pole)}. Then, we save the correlation
value and move on to the next simulation. At the end, we will have 1000
individual Pearson \(r\) values.

\begin{Shaded}
\begin{Highlighting}[]
\NormalTok{simulated\_correlations }\OtherTok{\textless{}{-}} \FunctionTok{length}\NormalTok{(}\DecValTok{0}\NormalTok{)}
\ControlFlowTok{for}\NormalTok{(sim }\ControlFlowTok{in} \DecValTok{1}\SpecialCharTok{:}\DecValTok{1000}\NormalTok{)\{}
\NormalTok{  North\_pole }\OtherTok{\textless{}{-}} \FunctionTok{runif}\NormalTok{(}\DecValTok{10}\NormalTok{,}\DecValTok{1}\NormalTok{,}\DecValTok{10}\NormalTok{)}
\NormalTok{  South\_pole }\OtherTok{\textless{}{-}} \FunctionTok{runif}\NormalTok{(}\DecValTok{10}\NormalTok{,}\DecValTok{1}\NormalTok{,}\DecValTok{10}\NormalTok{)}
\NormalTok{  simulated\_correlations[sim] }\OtherTok{\textless{}{-}} \FunctionTok{cor}\NormalTok{(North\_pole,South\_pole)}
\NormalTok{\}}

\NormalTok{sim\_df }\OtherTok{\textless{}{-}} \FunctionTok{data.frame}\NormalTok{(}\AttributeTok{sims=}\DecValTok{1}\SpecialCharTok{:}\DecValTok{1000}\NormalTok{,simulated\_correlations)}

\FunctionTok{ggplot}\NormalTok{(sim\_df, }\FunctionTok{aes}\NormalTok{(}\AttributeTok{x =}\NormalTok{ sims, }\AttributeTok{y =}\NormalTok{ simulated\_correlations))}\SpecialCharTok{+}
  \FunctionTok{geom\_point}\NormalTok{()}\SpecialCharTok{+}
  \FunctionTok{theme\_classic}\NormalTok{()}\SpecialCharTok{+}
  \FunctionTok{geom\_hline}\NormalTok{(}\AttributeTok{yintercept =} \SpecialCharTok{{-}}\DecValTok{1}\NormalTok{)}\SpecialCharTok{+}
  \FunctionTok{geom\_hline}\NormalTok{(}\AttributeTok{yintercept =} \DecValTok{1}\NormalTok{)}\SpecialCharTok{+}
  \FunctionTok{ggtitle}\NormalTok{(}\StringTok{"Simulation of 1000 r values"}\NormalTok{)}
\end{Highlighting}
\end{Shaded}

\begin{figure}[H]

\centering{

\includegraphics[width=0.75\linewidth,height=\textheight,keepaspectratio]{03-Correlation_files/figure-pdf/fig-3anotherthousand-1.pdf}

}

\caption{\label{fig-3anotherthousand}Another figure showing a range of
r-values that can be obtained by chance.}

\end{figure}%

Figure~\ref{fig-3anotherthousand} shows the 1000 Pearson \(r\) values
from the simulation. Does the figure below look familiar to you? We have
already conducted a similar kind of simulation before. Each dot in the
scatter plot shows the Pearson \(r\) for each simulation from 1 to 1000.
As you can see the dots are all over of the place, in between the range
-1 to 1. The important lesson here is that random chance produced all of
these correlations. This means we can find ``correlations'' in the data
that are completely meaningless, and do not reflect any causal
relationship between one measure and another.

Let's illustrate the idea of finding ``random'' correlations one more
time, with a little movie. This time, we will show you a scatter plot of
the random values sampled for the balls chosen from the North and South
pole. If there is no relationship we should see dots going everywhere.
If there happens to be a positive relationship (purely by chance), we
should see the dots going from the bottom left to the top right. If
there happens to be a negative relationship (purely by chance), we
should see the dots going from the top left down to the bottom right.

On more thing to prepare you for the movie. There are three scatter
plots below in Figure~\ref{fig-3reminder}, showing negative, positive,
and zero correlations between two variables. You've already seen this
graph before. We are just reminding you that the blue lines are helpful
for seeing the correlation.Negative correlations occur when a line goes
down from the top left to bottom right. Positive correlations occur when
a line goes up from the bottom left to the top right. Zero correlations
occur when the line is flat (doesn't go up or down).

\begin{figure}

\centering{

\includegraphics[width=0.75\linewidth,height=\textheight,keepaspectratio]{03-Correlation_files/figure-pdf/fig-3reminder-1.pdf}

}

\caption{\label{fig-3reminder}A reminder of what positive, negative, and
zero correlation looks like}

\end{figure}%

OK, now we are ready for the movie. \textbf{?@fig-3randcor10gif} shows
the process of sampling two sets of numbers randomly, one for the X
variable, and one for the Y variable. Each time we sample 10 numbers for
each, plot them, then draw a line through them. Remember, these numbers
are all completely random, so we should expect, on average that there
should be no correlation between the numbers. However, this is not what
happens. You can the line going all over the place. Sometimes we find a
negative correlation (line goes down), sometimes we see a positive
correlation (line goes up), and sometimes it looks like zero correlation
(line is more flat).

You might be thinking this is kind of disturbing. If we know that there
should be no correlation between two random variables, how come we are
finding correlations? This is a big problem right? I mean, if someone
showed me a correlation between two things, and then claimed one thing
was related to another, how could know I if it was true. After all, it
could be chance! Chance can do that too.

Fortunately, all is not lost. We can look at our simulated data in
another way, using a histogram. Remember, just before the movie, we
simulated 1000 different correlations using random numbers. By, putting
all of those \(r\) values into a histogram, we can get a better sense of
how chance behaves. We can see what kind of correlations chance is
likely or unlikely to produce. Figure~\ref{fig-3histrandcor} is a
histogram of the simulated \(r\) values.

\begin{figure}

\centering{

\includegraphics[width=0.75\linewidth,height=\textheight,keepaspectratio]{03-Correlation_files/figure-pdf/fig-3histrandcor-1.pdf}

}

\caption{\label{fig-3histrandcor}A histogram showing the frequency
distribution of r-values for completely random values between an X and Y
variable (sample-size=10). A rull range of r-values can be obtained by
chance alone. Larger r-values are less common than smaller r-values}

\end{figure}%

Notice that this histogram is not flat. Most of the simulated \(r\)
values are close to zero. Notice, also that the bars get smaller as you
move away from zero in the positive or negative direction. The general
take home here is that chance can produce a wide range of correlations.
However, not all correlations happen very often. For example, the bars
for -1 and 1 are very small. Chance does not produce nearly perfect
correlations very often. The bars around -.5 and .5 are smaller than the
bars around zero, as medium correlations do not occur as often as small
correlations by chance alone.

You can think of this histogram as the window of chance. It shows what
chance often does, and what it often does not do. If you found a
correlation under these very same circumstances (e.g., measured the
correlation between two sets of 10 random numbers), then you could
consult this window. What should you ask the window? How about, could my
observed correlation (the one that you found in your data) have come
from this window. Let's say you found a correlation of \(r = .1\). Could
a .1 have come from the histogram? Well, look at the histogram around
where the .1 mark on the x-axis is. Is there a big bar there? If so,
this means that chance produces this value fairly often. You might be
comfortable with the inference: Yes, this .1 could have been produced by
chance, because it is well inside the window of chance. How about
\(r = .5\)? The bar is much smaller here, you might think, ``well, I can
see that chance does produce .5 some times, so chance could have
produced my .5. Did it? Maybe, maybe not, not sure''. Here, your
confidence in a strong inference about the role of chance might start
getting a bit shakier.

How about an \(r = .95\)?. You might see that the bar for .95 is very
very small, perhaps too small to see. What does this tell you? It tells
you that chance does not produce .95 very often, hardly if at all,
pretty much never. So, if you found a .95 in your data, what would you
infer? Perhaps you would be comfortable inferring that chance did not
produce your .95, after .95 is mostly outside the window of chance.

\subsubsection{Increasing sample-size decreases opportunity for spurious
correlation}\label{increasing-sample-size-decreases-opportunity-for-spurious-correlation}

Before moving on, let's do one more thing with correlations. In our
pretend lottery game, each participant only sampled 10 balls each. We
found that this could lead to a range of correlations between the
numbers randomly drawn from either sides of the pole. Indeed, we even
found some correlations that were medium to large in size. If you were a
researcher who found such correlations, you might be tempted to believe
there was a relationship between your measurements. However, we know in
our little game, that those correlations would be spurious, just a
product of random sampling.

The good news is that, as a researcher, you get to make the rules of the
game. You get to determine how chance can play. This is all a little bit
metaphorical, so let's make it concrete.

We will see what happens in four different scenarios. First, we will
repeat what we already did. Each participant will draw 10 balls, then we
compute the correlation, and do this over 1000 times and look at a
histogram. Second, we will change the game so each participant draws 50
balls each, and then repeat our simulation. Third, and fourth, we will
change the game so each participant draws 100 balls each, and then 1000
balls each, and repeat etc.

Figure~\ref{fig-3corrandN} shows four different histograms of the
Pearson \(r\) values in each of the different scenarios. Each scenario
involves a different sample-size, from, 10, 50, 100 to 1000.

\begin{figure}

\centering{

\includegraphics[width=0.75\linewidth,height=\textheight,keepaspectratio]{03-Correlation_files/figure-pdf/fig-3corrandN-1.pdf}

}

\caption{\label{fig-3corrandN}Four histograms showing the frequency
distributions of r-values between completely random X and Y variables as
a function of sample-size. The width of the distributions shrink as
sample-size increases. Smaller sample-sizes are more likely to produce a
wider range of r-values by chance. Larger sample-sizes always produce a
narrow range of small r-values}

\end{figure}%

By inspecting the four histograms you should notice a clear pattern. The
width or range of each histogram shrinks as the sample-size increases.
What is going on here? Well, we already know that we can think of these
histograms as windows of chance. They tell us which \(r\) values occur
fairly often, which do not. When our sample-size is 10, lots of
different \(r\) values happen. That histogram is very flat and spread
out. However, as the sample-size increases, we see that the window of
chance gets pulled in. For example, by the time we get to 1000 balls
each, almost all of the Pearson \(r\) values are very close to 0.

One take home here, is that increasing sample-size narrows the window of
chance. So, for example, if you ran a study involving 1000 samples of
two measures, and you found a correlation of .5, then you can clearly
see in the bottom right histogram that .5 does not occur very often by
chance alone. In fact, there is no bar, because it didn't happen even
once in the simulation. As a result, when you have a large sample size
like n = 1000, you might be more confident that your observed
correlation (say of .5) was not a spurious correlation. If chance is not
producing your result, then something else is.

Finally, notice how your confidence about whether or not chance is
mucking about with your results depends on your sample size. If you only
obtained 10 samples per measurement, and found \(r = .5\), you should
not be as confident that your correlation reflects a real relationship.
Instead, you can see that \(r\)'s of .5 happen fairly often by chance
alone.

\begin{quote}
Pro tip: when you run an experiment you get to decide how many samples
you will collect, which means you can choose to narrow the window of
chance. Then, if you find a relationship in the data you can be more
confident that your finding is real, and not just something that
happened by chance.
\end{quote}

\subsection{Some more movies}\label{some-more-movies}

Let's ingrain these idea with some more movies. When our sample-size is
small (N is small), sampling error can cause all sort ``patterns'' in
the data. This makes it possible, and indeed common, for
``correlations'' to occur between two sets of numbers. When we increase
the sample-size, sampling error is reduced, making it less possible for
``correlations'' to occur just by chance alone. When N is large, chance
has less of an opportunity to operate.

\subsubsection{Watching how correlation behaves when there is no
correlation}\label{watching-how-correlation-behaves-when-there-is-no-correlation}

Below we randomly sample numbers for two variables, plot them, and show
the correlation using a line. There are four panels, each showing the
number of observations in the samples, from 10, 50, 100, to 1000 in each
sample.

Remember, because we are randomly sampling numbers, there should be no
relationship between the X and Y variables. But, as we have been
discussing, because of chance, we can sometimes observe a correlation
(due to chance). The important thing to watch is how the line behaves
across the four panels in \textbf{?@fig-3corRandfour}. The line twirls
around in all directions when the sample size is 10. It is also moves
around quite a bit when the sample size is 50 or 100. It still moves a
bit when the sample size is 1000, but much less. In all cases we expect
that the line should be flat, but every time we take new samples,
sometimes the line shows us pseudo patterns.

Which line should you trust? Well, hopefully you can see that the line
for 1000 samples is the most stable. It tends to be very flat every
time, and it does not depend so much on the particular sample. The line
with 10 observations per sample goes all over the place. The take home
here, is that if someone told you that they found a correlation, you
should want to know how many observations they hand in their sample. If
they only had 10 observations, how could you trust the claim that there
was a correlation? You can't!!! Not now that you know samples that are
that small can do all sorts of things by chance alone. If instead, you
found out the sample was very large, then you might trust that finding a
little bit more. For example, in the above movie you can see that when
there are 1000 samples, we never see a strong or weak correlation; the
line is always flat. This is because chance almost never produces strong
correlations when the sample size is very large.

In the above example, we sampled numbers random numbers from a uniform
distribution. Many examples of real-world data will come from a normal
or approximately normal distribution. We can repeat the above, but
sample random numbers from the same normal distribution. There will
still be zero actual correlation between the X and Y variables, because
everything is sampled randomly. \textbf{?@fig-3normCorfour} shows the
same behavior. The computed correlation for small sample-sizes fluctuate
wildly, and large sample sizes do not.

OK, so what do things look like when there actually is a correlation
between variables?

\subsubsection{Watching correlations behave when there really is a
correlation}\label{watching-correlations-behave-when-there-really-is-a-correlation}

Sometimes there really are correlations between two variables that are
not caused by chance. \textbf{?@fig-3realcorFour} shows a movie of four
scatter plots. Each shows the correlation between two variables. Again,
we change the sample-size in steps of 10, 50 100, and 1000. The data
have been programmed to contain a real positive correlation. So, we
should expect that the line will be going up from the bottom left to the
top right. However, there is still variability in the data. So this
time, sampling error due to chance will fuzz the correlation. We know it
is there, but sometimes chance will cause the correlation to be
eliminated.

Notice that in the top left panel (sample-size = 10), the line is
twirling around much more than the other panels. Every new set of
samples produces different correlations. Sometimes, the line even goes
flat or downward. However, as we increase sample-size, we can see that
the line doesn't change very much, it is always going up showing a
positive correlation.

The main takeaway here is that even when there is a positive correlation
between two things, you might not be able to see it if your sample size
is small. For example, you might get unlucky with the one sample that
you measured. Your sample could show a negative correlation, even when
the actual correlation is positive! Unfortunately, in the real world we
usually only have the sample that we collected, so we always have to
wonder if we got lucky or unlucky. Fortunately, if you want to remove
luck, all you need to do is collect larger samples. Then you will be
much more likely to observe the real pattern, rather the pattern that
can be introduced by chance.

\section{Summary}\label{summary-1}

In this section we have talked about correlation, and started to build
some intuitions about \textbf{inferential statistics}, which is the
major topic of the remaining chapters. For now, the main ideas are:

\begin{enumerate}
\def\labelenumi{\arabic{enumi}.}
\tightlist
\item
  We can measure relationships in data using things like correlation
\item
  The correlations we measure can be produced by numerous things, so
  they are hard to to interpret
\item
  Correlations can be produced by chance, so have the potential to be
  completely meaningless.
\item
  However, we can create a model of exactly what chance can do. The
  model tells us whether chance is more or less likely to produce
  correlations of different sizes
\item
  We can use the chance model to help us make decisions about our own
  data. We can compare the correlation we found in our data to the
  model, then ask whether or not chance could have or was likely to have
  produced our results.
\end{enumerate}

\bookmarksetup{startatroot}

\chapter{Probability, Sampling, and
Estimation}\label{probability-sampling-and-estimation}

Sections 4.1 \& 4.9 - Adapted text by Danielle Navarro Section 4.10 -
4.11 \& 4.13 - Mix of Matthew Crump \& Danielle Navarro Section 4.12 -
4.13 - Adapted text by Danielle Navarro, all sections modified by
Mallory Barnes.

\hfill\break

\begin{quote}
I have studied many languages-French, Spanish and a little Italian, but
no one told me that Statistics was a foreign language. ---Charmaine J.
Forde
\end{quote}

Up to this point in the book, we've discussed some of the key ideas in
experimental design, and we've talked a little about how you can
summarize a data set. To a lot of people, this is all there is to
statistics: it's about calculating averages, collecting all the numbers,
drawing pictures, and putting them all in a report somewhere. Kind of
like stamp collecting, but with numbers. However, statistics covers much
more than that. In fact, descriptive statistics is one of the smallest
parts of statistics, and one of the least powerful. The bigger and more
useful part of statistics is that it provides tools \textbf{that let you
make inferences about data}.

Once you start thinking about statistics in these terms -- that
statistics is there to help us draw inferences from data -- you start
seeing examples of it everywhere. For instance, here's a tiny extract
from a newspaper article in the Sydney Morning Herald (30 Oct 2010):

\begin{quote}
``I have a tough job,'' the Premier said in response to a poll which
found her government is now the most unpopular Labor administration in
polling history, with a primary vote of just 23 per cent.
\end{quote}

This kind of remark is entirely unremarkable in the papers or in
everyday life, but let's have a think about what it entails. A polling
company has conducted a survey, usually a pretty big one because they
can afford it. I'm too lazy to track down the original survey, so let's
just imagine that they called 1000 voters at random, and 230 (23\%) of
those claimed that they intended to vote for the party. For the 2010
Federal election, the Australian Electoral Commission reported 4,610,795
enrolled voters in New South Whales; so the opinions of the remaining
4,609,795 voters (about 99.98\% of voters) remain unknown to us. Even
assuming that no-one lied to the polling company the only thing we can
say with 100\% confidence is that the true primary vote is somewhere
between 230/4610795 (about 0.005\%) and 4610025/4610795 (about 99.83\%).
So, on what basis is it legitimate for the polling company, the
newspaper, and the readership to conclude that the ALP primary vote is
only about 23\%?

The answer to the question is pretty obvious: if I call 1000 people at
random, and 230 of them say they intend to vote for the ALP, then it
seems very unlikely that these are the \textbf{only} 230 people out of
the entire voting public who actually intend to do so. In other words,
we assume that the data collected by the polling company is pretty
representative of the population at large. But how representative? Would
we be surprised to discover that the true ALP primary vote is actually
24\%? 29\%? 37\%? At this point everyday intuition starts to break down
a bit. No-one would be surprised by 24\%, and everybody would be
surprised by 37\%, but it's a bit hard to say whether 29\% is plausible.
We need some more powerful tools than just looking at the numbers and
guessing.

\textbf{Inferential statistics} provides the tools that we need to
answer these sorts of questions, and since these kinds of questions lie
at the heart of the scientific enterprise, they take up the lions share
of every introductory course on statistics and research methods.
However, our tools for making statistical inferences are 1) built on top
of \textbf{probability theory}, and 2) require an understanding of how
samples behave when you take them from distributions (defined by
probability theory\ldots). So, this chapter has two main parts. A brief
introduction to probability theory, and an introduction to sampling from
distributions.

\section{How are probability and statistics
different?}\label{how-are-probability-and-statistics-different}

Before we start talking about probability theory, it's helpful to spend
a moment thinking about the relationship between probability and
statistics. The two disciplines are closely related but they're not
identical. Probability theory is ``the doctrine of chances''. It's a
branch of mathematics that tells you how often different kinds of events
will happen. For example, all of these questions are things you can
answer using probability theory:

\begin{itemize}
\item
  What are the chances of a fair coin coming up heads 10 times in a row?
\item
  If I roll two six sided dice, how likely is it that I'll roll two
  sixes?
\item
  How likely is it that five cards drawn from a perfectly shuffled deck
  will all be hearts?
\item
  What are the chances that I'll win the lottery?
\end{itemize}

Notice that all of these questions have something in common. In each
case the ``truth of the world'' is known, and my question relates to the
``what kind of events'' will happen. In the first question I
\textbf{know} that the coin is fair, so there's a 50\% chance that any
individual coin flip will come up heads. In the second question, I
\textbf{know} that the chance of rolling a 6 on a single die is 1 in 6.
In the third question I \textbf{know} that the deck is shuffled
properly. And in the fourth question, I \textbf{know} that the lottery
follows specific rules. You get the idea. The critical point is that
probabilistic questions start with a known \emph{model} of the world,
and we use that model to do some calculations.

The underlying model can be quite simple. For instance, in the coin
flipping example, we can write down the model like this:
\(P(\mbox{heads}) = 0.5\) which you can read as ``the probability of
heads is 0.5''.

As we'll see later, in the same way that percentages are numbers that
range from 0\% to 100\%, probabilities are just numbers that range from
0 to 1. When using this probability model to answer the first question,
I don't actually know exactly what's going to happen. Maybe I'll get 10
heads, like the question says. But maybe I'll get three heads. That's
the key thing: in probability theory, the \textbf{model} is known, but
the \textbf{data} are not.

So that's probability. What about statistics? Statistical questions work
the other way around. In statistics, we know the truth about the world.
All we have is the data, and it is from the data that we want to
\textbf{learn} the truth about the world. Statistical questions tend to
look more like these:

\begin{itemize}
\item
  If my friend flips a coin 10 times and gets 10 heads, are they playing
  a trick on me?
\item
  If five cards off the top of the deck are all hearts, how likely is it
  that the deck was shuffled?
\item
  If the lottery commissioner's spouse wins the lottery, how likely is
  it that the lottery was rigged?
\end{itemize}

This time around, the only thing we have are data. What I \textbf{know}
is that I saw my friend flip the coin 10 times and it came up heads
every time. And what I want to \emph{infer} is whether or not I should
conclude that what I just saw was actually a fair coin being flipped 10
times in a row, or whether I should suspect that my friend is playing a
trick on me. The data I have look like this:

\begin{verbatim}
H H H H H H H H H H H
\end{verbatim}

and what I'm trying to do is work out which ``model of the world'' I
should put my trust in. If the coin is fair, then the model I should
adopt is one that says that the probability of heads is 0.5; that is,
\(P(\mbox{heads}) = 0.5\). If the coin is not fair, then I should
conclude that the probability of heads is \textbf{not} 0.5, which we
would write as \(P(\mbox{heads}) \neq 0.5\). In other words, the
statistical inference problem is to figure out which of these
probability models is right. Clearly, the statistical question isn't the
same as the probability question, but they're deeply connected to one
another. Because of this, a good introduction to statistical theory will
start with a discussion of what probability is and how it works.

\section{What does probability mean?}\label{what-does-probability-mean}

Let's start with the first of these questions. What is ``probability''?
It might seem surprising to you, but while statisticians and
mathematicians (mostly) agree on what the \textbf{rules} of probability
are, there's much less of a consensus on what the word really
\textbf{means}. It seems weird because we're all very comfortable using
words like ``chance'', ``likely'', ``possible'' and ``probable'', and it
doesn't seem like it should be a very difficult question to answer. If
you had to explain ``probability'' to a five year old, you could do a
pretty good job. But if you've ever had that experience in real life,
you might walk away from the conversation feeling like you didn't quite
get it right, and that (like many everyday concepts) it turns out that
you don't \textbf{really} know what it's all about.

So I'll have a go at it. Let's suppose I want to bet on a soccer game
between two teams of robots, \textbf{Arduino Arsenal} and \textbf{C
Milan}. After thinking about it, I decide that there is an 80\%
probability that \textbf{Arduino Arsenal} winning. What do I mean by
that? Here are three possibilities\ldots{}

\begin{itemize}
\item
  They're robot teams, so I can make them play over and over again, and
  if I did that, \textbf{Arduino Arsenal} would win 8 out of every 10
  games on average.
\item
  For any given game, I would only agree that betting on this game is
  only ``fair'' if a \$1 bet on \textbf{C Milan} gives a \$5 payoff
  (i.e.~I get my \$1 back plus a \$4 reward for being correct), as would
  a \$4 bet on \textbf{Arduino Arsenal} (i.e., my \$4 bet plus a \$1
  reward).
\item
  My subjective ``belief'' or ``confidence'' in an \textbf{Arduino
  Arsenal} victory is four times as strong as my belief in a \textbf{C
  Milan} victory.
\end{itemize}

Each of these seems sensible. However they're not identical, and not
every statistician would endorse all of them. The reason is that there
are different statistical ideologies (yes, really!) and depending on
which one you subscribe to, you might say that some of those statements
are meaningless or irrelevant. In this section, I give a brief
introduction the two main approaches that exist in the literature. These
are by no means the only approaches, but they're the two big ones.

\subsection{The frequentist view}\label{the-frequentist-view}

The first of the two major approaches to probability, and the more
dominant one in statistics, is referred to as the \emph{frequentist
view}, and it defines probability as a \emph{long-run frequency}.
Suppose we were to try flipping a fair coin, over and over again. By
definition, this is a coin that has \(P(H) = 0.5\). What might we
observe? One possibility is that the first 20 flips might look like
this:

\begin{verbatim}
T,H,H,H,H,T,T,H,H,H,H,T,H,H,T,T,T,T,T,H
\end{verbatim}

In this case 11 of these 20 coin flips (55\%) came up heads. Now suppose
that I'd been keeping a running tally of the number of heads (which I'll
call \(N_H\)) that I've seen, across the first \(N\) flips, and
calculate the proportion of heads \(N_H / N\) every time. Here's what
I'd get (I did literally flip coins to produce this!):

\begin{longtable}[]{@{}
  >{\raggedright\arraybackslash}p{(\linewidth - 20\tabcolsep) * \real{0.2537}}
  >{\centering\arraybackslash}p{(\linewidth - 20\tabcolsep) * \real{0.0746}}
  >{\centering\arraybackslash}p{(\linewidth - 20\tabcolsep) * \real{0.0746}}
  >{\centering\arraybackslash}p{(\linewidth - 20\tabcolsep) * \real{0.0746}}
  >{\centering\arraybackslash}p{(\linewidth - 20\tabcolsep) * \real{0.0746}}
  >{\centering\arraybackslash}p{(\linewidth - 20\tabcolsep) * \real{0.0746}}
  >{\centering\arraybackslash}p{(\linewidth - 20\tabcolsep) * \real{0.0746}}
  >{\centering\arraybackslash}p{(\linewidth - 20\tabcolsep) * \real{0.0746}}
  >{\centering\arraybackslash}p{(\linewidth - 20\tabcolsep) * \real{0.0746}}
  >{\centering\arraybackslash}p{(\linewidth - 20\tabcolsep) * \real{0.0746}}
  >{\centering\arraybackslash}p{(\linewidth - 20\tabcolsep) * \real{0.0746}}@{}}
\toprule\noalign{}
\endhead
\bottomrule\noalign{}
\endlastfoot
number of flips & 1 & 2 & 3 & 4 & 5 & 6 & 7 & 8 & 9 & 10 \\
number of heads & 0 & 1 & 2 & 3 & 4 & 4 & 4 & 5 & 6 & 7 \\
proportion & .00 & .50 & .67 & .75 & .80 & .67 & .57 & .63 & .67 &
.70 \\
\end{longtable}

\begin{longtable}[]{@{}
  >{\raggedright\arraybackslash}p{(\linewidth - 20\tabcolsep) * \real{0.2537}}
  >{\centering\arraybackslash}p{(\linewidth - 20\tabcolsep) * \real{0.0746}}
  >{\centering\arraybackslash}p{(\linewidth - 20\tabcolsep) * \real{0.0746}}
  >{\centering\arraybackslash}p{(\linewidth - 20\tabcolsep) * \real{0.0746}}
  >{\centering\arraybackslash}p{(\linewidth - 20\tabcolsep) * \real{0.0746}}
  >{\centering\arraybackslash}p{(\linewidth - 20\tabcolsep) * \real{0.0746}}
  >{\centering\arraybackslash}p{(\linewidth - 20\tabcolsep) * \real{0.0746}}
  >{\centering\arraybackslash}p{(\linewidth - 20\tabcolsep) * \real{0.0746}}
  >{\centering\arraybackslash}p{(\linewidth - 20\tabcolsep) * \real{0.0746}}
  >{\centering\arraybackslash}p{(\linewidth - 20\tabcolsep) * \real{0.0746}}
  >{\centering\arraybackslash}p{(\linewidth - 20\tabcolsep) * \real{0.0746}}@{}}
\toprule\noalign{}
\endhead
\bottomrule\noalign{}
\endlastfoot
number of flips & 11 & 12 & 13 & 14 & 15 & 16 & 17 & 18 & 19 & 20 \\
number of heads & 8 & 8 & 9 & 10 & 10 & 10 & 10 & 10 & 10 & 11 \\
proportion & .73 & .67 & .69 & .71 & .67 & .63 & .59 & .56 & .53 &
.55 \\
\end{longtable}

Notice that at the start of the sequence, the \textbf{proportion} of
heads fluctuates wildly, starting at .00 and rising as high as .80.
Later on, one gets the impression that it dampens out a bit, with more
and more of the values actually being pretty close to the ``right''
answer of .50. This is the frequentist definition of probability in a
nutshell: flip a fair coin over and over again, and as \(N\) grows large
(approaches infinity, denoted \(N\rightarrow \infty\)), the proportion
of heads will converge to 50\%. There are some subtle technicalities
that the mathematicians care about, but qualitatively speaking, that's
how the frequentists define probability. Unfortunately, I don't have an
infinite number of coins, or the infinite patience required to flip a
coin an infinite number of times. However, I do have a computer, and
computers excel at mindless repetitive tasks. So I asked my computer to
simulate flipping a coin 1000 times, and then drew a picture of what
happens to the proportion \(N_H / N\) as \(N\) increases. Actually, I
did it four times, just to make sure it wasn't a fluke. The results are
shown in Figure~\ref{fig-4FreqProb}. As you can see, the
\textbf{proportion of observed heads} eventually stops fluctuating, and
settles down; when it does, the number at which it finally settles is
the true probability of heads.

\begin{figure}

\centering{

\includegraphics[width=0.75\linewidth,height=\textheight,keepaspectratio]{imgs/navarro_img/probability/frequentistProb-eps-converted-to.png}

}

\caption{\label{fig-4FreqProb}An illustration of how frequentist
probability works. If you flip a fair coin over and over again, the
proportion of heads that you've seen eventually settles down, and
converges to the true probability of 0.5. Each panel shows four
different simulated experiments: in each case, we pretend we flipped a
coin 1000 times, and kept track of the proportion of flips that were
heads as we went along. Although none of these sequences actually ended
up with an exact value of .5, if we'd extended the experiment for an
infinite number of coin flips they would have.}

\end{figure}%

The frequentist definition of probability has some desirable
characteristics. First, it is objective: the probability of an event is
\textbf{necessarily} grounded in the world. The only way that
probability statements can make sense is if they refer to (a sequence
of) events that occur in the physical universe. Second, it is
unambiguous: any two people watching the same sequence of events unfold,
trying to calculate the probability of an event, must inevitably come up
with the same answer.

However, it also has undesirable characteristics. Infinite sequences
don't exist in the physical world. Suppose you picked up a coin from
your pocket and started to flip it. Every time it lands, it impacts on
the ground. Each impact wears the coin down a bit; eventually, the coin
will be destroyed. So, one might ask whether it really makes sense to
pretend that an ``infinite'' sequence of coin flips is even a meaningful
concept, or an objective one. We can't say that an ``infinite sequence''
of events is a real thing in the physical universe, because the physical
universe doesn't allow infinite anything.

More seriously, the frequentist definition has a narrow scope. There are
lots of things out there that human beings are happy to assign
probability to in everyday language, but cannot (even in theory) be
mapped onto a hypothetical sequence of events. For instance, if a
meteorologist comes on TV and says, ``the probability of rain in
Adelaide on 2 November 2048 is 60\%'' we humans are happy to accept
this. But it's not clear how to define this in frequentist terms.
There's only one city of Adelaide, and only 2 November 2048. There's no
infinite sequence of events here, just a once-off thing. Frequentist
probability genuinely \textbf{forbids} us from making probability
statements about a single event. From the frequentist perspective, it
will either rain tomorrow or it will not; there is no ``probability''
that attaches to a single non-repeatable event. Now, it should be said
that there are some very clever tricks that frequentists can use to get
around this. One possibility is that what the meteorologist means is
something like this: ``There is a category of days for which I predict a
60\% chance of rain; if we look only across those days for which I make
this prediction, then on 60\% of those days it will actually rain''.
It's very weird and counter intuitive to think of it this way, but you
do see frequentists do this sometimes.

\subsection{The Bayesian view}\label{the-bayesian-view}

The \textbf{Bayesian view} of probability is often called the
subjectivist view, and it is a minority view among statisticians, but
one that has been steadily gaining traction for the last several
decades. There are many flavors of Bayesianism, making hard to say
exactly what ``the'' Bayesian view is. The most common way of thinking
about subjective probability is to define the probability of an event as
the \textbf{degree of belief} that an intelligent and rational agent
assigns to that truth of that event. From that perspective,
probabilities don't exist in the world, but rather in the thoughts and
assumptions of people and other intelligent beings. However, in order
for this approach to work, we need some way of operationalising ``degree
of belief''. One way that you can do this is to formalize it in terms of
``rational gambling'', though there are many other ways. Suppose that I
believe that there's a 60\% probability of rain tomorrow. If someone
offers me a bet: if it rains tomorrow, then I win \$5, but if it doesn't
rain then I lose \$5. Clearly, from my perspective, this is a pretty
good bet. On the other hand, if I think that the probability of rain is
only 40\%, then it's a bad bet to take. Thus, we can operationalize the
notion of a ``subjective probability'' in terms of what bets I'm willing
to accept.

What are the advantages and disadvantages to the Bayesian approach? The
main advantage is that it allows you to assign probabilities to any
event you want to. You don't need to be limited to those events that are
repeatable. The main disadvantage (to many people) is that we can't be
purely objective -- specifying a probability requires us to specify an
entity that has the relevant degree of belief. This entity might be a
human, an alien, a robot, or even a statistician, but there has to be an
intelligent agent out there that believes in things. To many people this
is uncomfortable: it seems to make probability arbitrary. While the
Bayesian approach does require that the agent in question be rational
(i.e., obey the rules of probability), it does allow everyone to have
their own beliefs; I can believe the coin is fair and you don't have to,
even though we're both rational. The frequentist view doesn't allow any
two observers to attribute different probabilities to the same event:
when that happens, then at least one of them must be wrong. The Bayesian
view does not prevent this from occurring. Two observers with different
background knowledge can legitimately hold different beliefs about the
same event. In short, where the frequentist view is sometimes considered
to be too narrow (forbids lots of things that that we want to assign
probabilities to), the Bayesian view is sometimes thought to be too
broad (allows too many differences between observers).

\subsection{What's the difference? And who is
right?}\label{whats-the-difference-and-who-is-right}

Now that you've seen each of these two views independently, it's useful
to make sure you can compare the two. Go back to the hypothetical robot
soccer game at the start of the section. What do you think a frequentist
and a Bayesian would say about these three statements? Which statement
would a frequentist say is the correct definition of probability? Which
one would a Bayesian do? Would some of these statements be meaningless
to a frequentist or a Bayesian? If you've understood the two
perspectives, you should have some sense of how to answer those
questions.

Okay, assuming you understand the different, you might be wondering
which of them is \textbf{right}? Honestly, I don't know that there is a
right answer. As far as I can tell there's nothing mathematically
incorrect about the way frequentists think about sequences of events,
and there's nothing mathematically incorrect about the way that
Bayesians define the beliefs of a rational agent. In fact, when you dig
down into the details, Bayesians and frequentists actually agree about a
lot of things. Many frequentist methods lead to decisions that Bayesians
agree a rational agent would make. Many Bayesian methods have very good
frequentist properties.

For the most part, I'm a pragmatist so I'll use any statistical method
that I trust. As it turns out, that makes me prefer Bayesian methods,
for reasons I'll explain towards the end of the book, but I'm not
fundamentally opposed to frequentist methods. Not everyone is quite so
relaxed. For instance, consider Sir Ronald Fisher, one of the towering
figures of 20th century statistics and a vehement opponent to all things
Bayesian, whose paper on the mathematical foundations of statistics
referred to Bayesian probability as ``an impenetrable jungle {[}that{]}
arrests progress towards precision of statistical concepts'' Fisher
(1922, 311). Or the psychologist Paul Meehl, who suggests that relying
on frequentist methods could turn you into ``a potent but sterile
intellectual rake who leaves in his merry path a long train of ravished
maidens but no viable scientific offspring'' Meehl (1967, 114). The
history of statistics, as you might gather, is not devoid of
entertainment.

\section{Basic probability theory}\label{basic-probability-theory}

Ideological arguments between Bayesians and frequentists
notwithstanding, it turns out that people mostly agree on the rules that
probabilities should obey. There are lots of different ways of arriving
at these rules. The most commonly used approach is based on the work of
Andrey Kolmogorov, one of the great Soviet mathematicians of the 20th
century. I won't go into a lot of detail, but I'll try to give you a bit
of a sense of how it works. And in order to do so, I'm going to have to
talk about my pants.

\subsection{Introducing probability
distributions}\label{introducing-probability-distributions}

One of the disturbing truths about my life is that I only own 5 pairs of
pants: three pairs of jeans, the bottom half of a suit, and a pair of
tracksuit pants. Even sadder, I've given them names: I call them
\(X_1\), \(X_2\), \(X_3\), \(X_4\) and \(X_5\). I really do: that's why
they call me Mister Imaginative. Now, on any given day, I pick out
exactly one of pair of pants to wear. Not even I'm so stupid as to try
to wear two pairs of pants, and thanks to years of training I never go
outside without wearing pants anymore. If I were to describe this
situation using the language of probability theory, I would refer to
each pair of pants (i.e., each \(X\)) as an \emph{elementary event}. The
key characteristic of elementary events is that every time we make an
observation (e.g., every time I put on a pair of pants), then the
outcome will be one and only one of these events. Like I said, these
days I always wear exactly one pair of pants, so my pants satisfy this
constraint. Similarly, the set of all possible events is called a
\emph{sample space}. Granted, some people would call it a ``wardrobe'',
but that's because they're refusing to think about my pants in
probabilistic terms. Sad.

Okay, now that we have a sample space (a wardrobe), which is built from
lots of possible elementary events (pants), what we want to do is assign
a \emph{probability} of one of these elementary events. For an event
\(X\), the probability of that event \(P(X)\) is a number that lies
between 0 and 1. The bigger the value of \(P(X)\), the more likely the
event is to occur. So, for example, if \(P(X) = 0\), it means the event
\(X\) is impossible (i.e., I never wear those pants). On the other hand,
if \(P(X) = 1\) it means that event \(X\) is certain to occur (i.e., I
always wear those pants). For probability values in the middle, it means
that I sometimes wear those pants. For instance, if \(P(X) = 0.5\) it
means that I wear those pants half of the time.

At this point, we're almost done. The last thing we need to recognize is
that ``something always happens''. Every time I put on pants, I really
do end up wearing pants (crazy, right?). What this somewhat trite
statement means, in probabilistic terms, is that the probabilities of
the elementary events need to add up to 1. This is known as the
\emph{law of total probability}, not that any of us really care. More
importantly, if these requirements are satisfied, then what we have is a
\emph{probability distribution}. For example, this is an example of a
probability distribution

\begin{longtable}[]{@{}lcc@{}}
\toprule\noalign{}
Which pants? & Label & Probability \\
\midrule\noalign{}
\endhead
\bottomrule\noalign{}
\endlastfoot
Blue jeans & \(X_1\) & \(P(X_1) = .5\) \\
Grey jeans & \(X_2\) & \(P(X_2) = .3\) \\
Black jeans & \(X_3\) & \(P(X_3) = .1\) \\
Black suit & \(X_4\) & \(P(X_4) = 0\) \\
Blue tracksuit & \(X_5\) & \(P(X_5) = .1\) \\
\end{longtable}

Each of the events has a probability that lies between 0 and 1, and if
we add up the probability of all events, they sum to 1. Awesome. We can
even draw a nice bar graph to visualize this distribution, as shown in
Figure~\ref{fig-4pantsprob}. And at this point, we've all achieved
something. You've learned what a probability distribution is, and I've
finally managed to find a way to create a graph that focuses entirely on
my pants. Everyone wins!

\begin{figure}

\centering{

\includegraphics[width=0.75\linewidth,height=\textheight,keepaspectratio]{imgs/navarro_img/probability/pantsDistribution-eps-converted-to.png}

}

\caption{\label{fig-4pantsprob}A visual depiction of the pants
probability distribution. There are five elementary events,
corresponding to the five pairs of pants that I own. Each event has some
probability of occurring: this probability is a number between 0 to 1.
The sum of these probabilities is 1.}

\end{figure}%

The only other thing that I need to point out is that probability theory
allows you to talk about \emph{non elementary events} as well as
elementary ones. The easiest way to illustrate the concept is with an
example. In the pants example, it's perfectly legitimate to refer to the
probability that I wear jeans. In this scenario, the ``Dan wears jeans''
event said to have happened as long as the elementary event that
actually did occur is one of the appropriate ones; in this case ``blue
jeans'', ``black jeans'' or ``grey jeans''. In mathematical terms, we
defined the ``jeans'' event \(E\) to correspond to the set of elementary
events \((X_1, X_2, X_3)\). If any of these elementary events occurs,
then \(E\) is also said to have occurred. Having decided to write down
the definition of the \(E\) this way, it's pretty straightforward to
state what the probability \(P(E)\) is: we just add everything up. In
this particular case \[P(E) = P(X_1) + P(X_2) + P(X_3)\] and, since the
probabilities of blue, grey and black jeans respectively are .5, .3 and
.1, the probability that I wear jeans is equal to .9.

At this point you might be thinking that this is all terribly obvious
and simple and you'd be right. All we've really done is wrap some basic
mathematics around a few common sense intuitions. However, from these
simple beginnings it's possible to construct some extremely powerful
mathematical tools. I'm definitely not going to go into the details in
this book, but what I will do is list some of the other rules that
probabilities satisfy. These rules can be derived from the simple
assumptions that I've outlined above, but since we don't actually use
these rules for anything in this book, I won't do so here.

\begin{longtable}[]{@{}lrll@{}}
\caption{Some basic rules that probabilities must satisfy. You don't
really need to know these rules in order to understand the analyses that
we'll talk about later in the book, but they are important if you want
to understand probability theory a bit more deeply.}\tabularnewline
\toprule\noalign{}
English & Notation & & Formula \\
\midrule\noalign{}
\endfirsthead
\toprule\noalign{}
English & Notation & & Formula \\
\midrule\noalign{}
\endhead
\bottomrule\noalign{}
\endlastfoot
not \(A\) & \(P(\neg A)\) & \(=\) & \(1-P(A)\) \\
\(A\) or \(B\) & \(P(A \cup B)\) & \(=\) &
\(P(A) + P(B) - P(A \cap B)\) \\
\(A\) and \(B\) & \(P(A \cap B)\) & \(=\) & \(P(A|B) P(B)\) \\
\end{longtable}

Now that we have the ability to ``define'' non-elementary events in
terms of elementary ones, we can actually use this to construct (or, if
you want to be all mathematicallish, ``derive'') some of the other rules
of probability. These rules are listed above, and while I'm pretty
confident that very few of my readers actually care about how these
rules are constructed, I'm going to show you anyway: even though it's
boring and you'll probably never have a lot of use for these
derivations, if you read through it once or twice and try to see how it
works, you'll find that probability starts to feel a bit less
mysterious, and with any luck a lot less daunting. So here goes.
Firstly, in order to construct the rules I'm going to need a sample
space \(X\) that consists of a bunch of elementary events \(x\), and two
non-elementary events, which I'll call \(A\) and \(B\). Let's say:
\[\begin{array}{rcl}
X &=& (x_1, x_2, x_3, x_4, x_5) \\
A &=& (x_1, x_2, x_3) \\
B &=& (x_3, x_4)
\end{array}\] To make this a bit more concrete, let's suppose that we're
still talking about the pants distribution. If so, \(A\) corresponds to
the event ``jeans'', and \(B\) corresponds to the event ``black'':
\[\begin{array}{rcl}
\mbox{``jeans''} &=& (\mbox{``blue jeans''}, \mbox{``grey jeans''}, \mbox{``black jeans''}) \\
\mbox{``black''} &=& (\mbox{``black jeans''}, \mbox{``black suit''})
\end{array}\] So now let's start checking the rules that I've listed in
the table.

In the first line, the table says that \[P(\neg A) = 1- P(A)\] and what
it \textbf{means} is that the probability of ``not \(A\)'' is equal to 1
minus the probability of \(A\). A moment's thought (and a tedious
example) make it obvious why this must be true. If \(A\) corresponds to
the even that I wear jeans (i.e., one of \(x_1\) or \(x_2\) or \(x_3\)
happens), then the only meaningful definition of ``not \(A\)'' (which is
mathematically denoted as \(\neg A\)) is to say that \(\neg A\) consists
of \textbf{all} elementary events that don't belong to \(A\). In the
case of the pants distribution it means that \(\neg A = (x_4, x_5)\),
or, to say it in English: ``not jeans'' consists of all pairs of pants
that aren't jeans (i.e., the black suit and the blue tracksuit).
Consequently, every single elementary event belongs to either \(A\) or
\(\neg A\), but not both. Okay, so now let's rearrange our statement
above: \[P(\neg A) + P(A) = 1\] which is a trite way of saying either I
do wear jeans or I don't wear jeans: the probability of ``not jeans''
plus the probability of ``jeans'' is 1. Mathematically:
\[\begin{array}{rcl}
P(\neg A) &=&  P(x_4) + P(x_5) \\
P(A) &=& P(x_1) + P(x_2) + P(x_3) 
\end{array}\] so therefore \[\begin{array}{rcl} 
P(\neg A) + P(A) &=& P(x_1) + P(x_2) + P(x_3) + P(x_4) + P(x_5) \\
&=& \sum_{x \in X} P(x) \\
&=& 1
\end{array}\] Excellent. It all seems to work.

Wow, I can hear you saying. That's a lot of \(x\)s to tell me the
freaking obvious. And you're right: this \textbf{is} freaking obvious.
The whole \textbf{point} of probability theory to to formalize and
mathematize a few very basic common sense intuitions. So let's carry
this line of thought forward a bit further. In the last section I
defined an event corresponding to \textbf{not} A, which I denoted
\(\neg A\). Let's now define two new events that correspond to important
everyday concepts: \(A\) \textbf{and} \(B\), and \(A\) \textbf{or}
\(B\). To be precise:

\begin{longtable}[]{@{}lc@{}}
\toprule\noalign{}
English statement: & Mathematical notation: \\
\midrule\noalign{}
\endhead
\bottomrule\noalign{}
\endlastfoot
``\(A\) and \(B\)'' both happen & \(A \cap B\) \\
at least one of ``\(A\) or \(B\)'' happens & \(A \cup B\) \\
\end{longtable}

Since \(A\) and \(B\) are both defined in terms of our elementary events
(the \(x\)s) we're going to need to try to describe \(A \cap B\) and
\(A \cup B\) in terms of our elementary events too. Can we do this? Yes
we can The only way that both \(A\) and \(B\) can occur is if the
elementary event that we observe turns out to belong to both \(A\) and
\(B\). Thus ``\(A \cap B\)'' includes only those elementary events that
belong to both \(A\) and \(B\)\ldots{} \[\begin{array}{rcl}
A &=& (x_1, x_2, x_3) \\
B &=& (x_3, x_4) \\
A \cap B & = & (x_3) 
\end{array}\] So, um, the only way that I can wear ``jeans''
\((x_1, x_2, x_3)\) and ``black pants'' \((x_3, x_4)\) is if I wear
``black jeans'' \((x_3)\). Another victory for the bloody obvious.

At this point, you're not going to be at all shocked by the definition
of \(A \cup B\), though you're probably going to be extremely bored by
it. The only way that I can wear ``jeans'' or ``black pants'' is if the
elementary pants that I actually do wear belongs to \(A\) or to \(B\),
or to both. So\ldots{} \[\begin{array}{rcl}
A &=& (x_1, x_2, x_3) \\
B &=& (x_3, x_4) \\
A \cup B & = & (x_1, x_2, x_3, x_4) 
\end{array}\] Oh yeah baby. Mathematics at its finest.

So, we've defined what we mean by \(A \cap B\) and \(A \cup B\). Now
let's assign probabilities to these events. More specifically, let's
start by verifying the rule that claims that:
\[P(A \cup B) = P(A) + P(B) - P(A \cap B)\] Using our definitions
earlier, we know that \(A \cup B = (x_1, x_2, x_3, x_4)\), so
\[P(A \cup B) = P(x_1) + P(x_2) + P(x_3) + P(x_4)\] and making similar
use of the fact that we know what elementary events belong to \(A\),
\(B\) and \(A \cap B\)\ldots. \[\begin{array}{rcl}
P(A)  &=&   P(x_1) + P(x_2) + P(x_3)    \\
P(B) &=&  P(x_3) + P(x_4)  \\
P(A \cap B) &=&  P(x_3)
\end{array}\] and therefore \[\begin{array}{rcl}
P(A) + P(B) - P(A \cap B)
&=&  P(x_1) + P(x_2) + P(x_3) +  P(x_3) + P(x_4) -  P(x_3) \\
&=& P(x_1) + P(x_2) + P(x_3) + P(x_4) \\
&=& P(A \cup B)
\end{array}\] Done.

The next concept we need to define is the notion of ``\(B\) given
\(A\)'', which is typically written \(B | A\). Here's what I mean:
suppose that I get up one morning, and put on a pair of pants. An
elementary event \(x\) has occurred. Suppose further I yell out to my
wife (who is in the other room, and so cannot see my pants) ``I'm
wearing jeans today!''. Assuming that she believes that I'm telling the
truth, she knows that \(A\) is true. \textbf{Given} that she knows that
\(A\) has happened, what is the \textbf{conditional probability} that
\(B\) is also true? Well, let's think about what she knows. Here are the
facts:

\begin{itemize}
\item
  \textbf{The non-jeans events are impossible}. If \(A\) is true, then
  we know that the only possible elementary events that could have
  occurred are \(x_1\), \(x_2\) and \(x_3\) (i.e.,the jeans). The
  non-jeans events \(x_4\) and \(x_5\) are now impossible, and must be
  assigned probability zero. In other words, our \textbf{sample space}
  has been restricted to the jeans events. But it's still the case that
  the probabilities of these these events \textbf{must} sum to 1: we
  know for sure that I'm wearing jeans.
\item
  \textbf{She's learned nothing about which jeans I'm wearing}. Before I
  made my announcement that I was wearing jeans, she already knew that I
  was five times as likely to be wearing blue jeans (\(P(x_1) = 0.5\))
  than to be wearing black jeans (\(P(x_3) = 0.1\)). My announcement
  doesn't change this\ldots{} I said \textbf{nothing} about what color
  my jeans were, so it must remain the case that \(P(x_1) / P(x_3)\)
  stays the same, at a value of 5.
\end{itemize}

There's only one way to satisfy these constraints: set the impossible
events to have zero probability (i.e., \(P(x | A) = 0\) if \(x\) is not
in \(A\)), and then divide the probabilities of all the others by
\(P(A)\). In this case, since \(P(A) = 0.9\), we divide by 0.9. This
gives:

\begin{longtable}[]{@{}
  >{\raggedright\arraybackslash}p{(\linewidth - 6\tabcolsep) * \real{0.2162}}
  >{\raggedright\arraybackslash}p{(\linewidth - 6\tabcolsep) * \real{0.2432}}
  >{\centering\arraybackslash}p{(\linewidth - 6\tabcolsep) * \real{0.2432}}
  >{\centering\arraybackslash}p{(\linewidth - 6\tabcolsep) * \real{0.2973}}@{}}
\toprule\noalign{}
\begin{minipage}[b]{\linewidth}\raggedright
which pants?
\end{minipage} & \begin{minipage}[b]{\linewidth}\raggedright
elementary event
\end{minipage} & \begin{minipage}[b]{\linewidth}\centering
old prob, \(P(x)\)
\end{minipage} & \begin{minipage}[b]{\linewidth}\centering
new prob, \(P(x | A)\)
\end{minipage} \\
\midrule\noalign{}
\endhead
\bottomrule\noalign{}
\endlastfoot
blue jeans & \(x_1\) & 0.5 & 0.556 \\
grey jeans & \(x_2\) & 0.3 & 0.333 \\
black jeans & \(x_3\) & 0.1 & 0.111 \\
black suit & \(x_4\) & 0 & 0 \\
blue tracksuit & \(x_5\) & 0.1 & 0 \\
\end{longtable}

In mathematical terms, we say that \[P(x | A) = \frac{P(x)}{P(A)}\] if
\(x \in A\), and \(P(x|A) = 0\) otherwise. And therefore\ldots{}
\[\begin{array}{rcl}
P(B | A) &=& P(x_3 | A)  + P(x_4 | A) \\ \\
&=&  \displaystyle\frac{P(x_3)}{P(A)} + 0    \\ \\
&=& \displaystyle\frac{P(x_3)}{P(A)}
\end{array}\] Now, recalling that \(A \cap B = (x_3)\), we can write
this as \[P(B | A) = \frac{P(A \cap B)}{P(A)}\] and if we multiply both
sides by \(P(A)\) we obtain: \[P(A \cap B) = P(B| A) P(A)\] which is the
third rule that we had listed in the table.

\section{The binomial distribution}\label{the-binomial-distribution}

As you might imagine, probability distributions vary enormously, and
there's an enormous range of distributions out there. However, they
aren't all equally important. In fact, the vast majority of the content
in this book relies on one of five distributions: the binomial
distribution, the normal distribution, the \(t\) distribution, the
\(\chi^2\) (``chi-square'') distribution and the \(F\) distribution.
Given this, what I'll do over the next few sections is provide a brief
introduction to all five of these, paying special attention to the
binomial and the normal. I'll start with the binomial distribution,
since it's the simplest of the five.

\subsection{Introducing the binomial}\label{introducing-the-binomial}

The theory of probability originated in the attempt to describe how
games of chance work, so it seems fitting that our discussion of the
\emph{binomial distribution} should involve a discussion of rolling dice
and flipping coins. Let's imagine a simple ``experiment'': in my hand
I'm holding 20 identical six-sided dice. On one face of each die there's
a picture of a skull; the other five faces are all blank. If I proceed
to roll all 20 dice, what's the probability that I'll get exactly 4
skulls? Assuming that the dice are fair, we know that the chance of any
one die coming up skulls is 1 in 6; to say this another way, the skull
probability for a single die is approximately \(.167\). This is enough
information to answer our question, so let's have a look at how it's
done.

As usual, we'll want to introduce some names and some notation. We'll
let \(N\) denote the number of dice rolls in our experiment; which is
often referred to as the \emph{size parameter} of our binomial
distribution. We'll also use \(\theta\) to refer to the the probability
that a single die comes up skulls, a quantity that is usually called the
\emph{success probability} of the binomial. Finally, we'll use \(X\) to
refer to the results of our experiment, namely the number of skulls I
get when I roll the dice. Since the actual value of \(X\) is due to
chance, we refer to it as a \emph{random variable}. In any case, now
that we have all this terminology and notation, we can use it to state
the problem a little more precisely. The quantity that we want to
calculate is the probability that \(X = 4\) given that we know that
\(\theta = .167\) and \(N=20\). The general ``form'' of the thing I'm
interested in calculating could be written as \[P(X \ | \ \theta, N)\]
and we're interested in the special case where \(X=4\),
\(\theta = .167\) and \(N=20\). There's only one more piece of notation
I want to refer to before moving on to discuss the solution to the
problem. If I want to say that \(X\) is generated randomly from a
binomial distribution with parameters \(\theta\) and \(N\), the notation
I would use is as follows: \[X \sim \mbox{Binomial}(\theta, N)\]

Yeah, yeah. I know what you're thinking: notation, notation, notation.
Really, who cares? Very few readers of this book are here for the
notation, so I should probably move on and talk about how to use the
binomial distribution. To that end, Figure Figure~\ref{fig-4binomial1}
plots the binomial probabilities for all possible values of \(X\) for
our dice rolling experiment, from \(X=0\) (no skulls) all the way up to
\(X=20\) (all skulls). Note that this is basically a bar chart, and is
no different to the ``pants probability'' plot I drew in
Figure~\ref{fig-4pantsprob}. On the horizontal axis we have all the
possible events, and on the vertical axis we can read off the
probability of each of those events. So, the probability of rolling 4
skulls out of 20 times is about 0.20 (the actual answer is 0.2022036, as
we'll see in a moment). In other words, you'd expect that to happen
about 20\% of the times you repeated this experiment.

\begin{figure}

\centering{

\includegraphics[width=0.75\linewidth,height=\textheight,keepaspectratio]{imgs/navarro_img/probability/binomSkulls20-eps-converted-to.png}

}

\caption{\label{fig-4binomial1}The binomial distribution with size
parameter of N =20 and an underlying success probability of 1/6. Each
vertical bar depicts the probability of one specific outcome (i.e., one
possible value of X). Because this is a probability distribution, each
of the probabilities must be a number between 0 and 1, and the heights
of the bars must sum to 1 as well.}

\end{figure}%

\subsection{Working with the binomial distribution in
R}\label{working-with-the-binomial-distribution-in-r}

R has a function called \texttt{dbinom} that calculates binomial
probabilities for us. The main arguments to the function are

\begin{itemize}
\item
  \texttt{x} This is a number, or vector of numbers, specifying the
  outcomes whose probability you're trying to calculate.
\item
  \texttt{size} This is a number telling R the size of the experiment.
\item
  \texttt{prob} This is the success probability for any one trial in the
  experiment.
\end{itemize}

So, in order to calculate the probability of getting skulls, from an
experiment of trials, in which the probability of getting a skull on any
one trial is \ldots{} well, the command I would use is simply this:

\begin{Shaded}
\begin{Highlighting}[]
\FunctionTok{dbinom}\NormalTok{( }\AttributeTok{x =} \DecValTok{4}\NormalTok{, }\AttributeTok{size =} \DecValTok{20}\NormalTok{, }\AttributeTok{prob =} \DecValTok{1}\SpecialCharTok{/}\DecValTok{6}\NormalTok{ )}
\CommentTok{\#\textgreater{} [1] 0.2022036}
\end{Highlighting}
\end{Shaded}

To give you a feel for how the binomial distribution changes when we
alter the values of \(\theta\) and \(N\), let's suppose that instead of
rolling dice, I'm actually flipping coins. This time around, my
experiment involves flipping a fair coin repeatedly, and the outcome
that I'm interested in is the number of heads that I observe. In this
scenario, the success probability is now \(\theta = 1/2\). Suppose I
were to flip the coin \(N=20\) times. In this example, I've changed the
success probability, but kept the size of the experiment the same. What
does this do to our binomial distribution?

\begin{figure}

\centering{

\includegraphics[width=0.75\linewidth,height=\textheight,keepaspectratio]{imgs/navarro_img/probability/Binomial2.png}

}

\caption{\label{fig-4binomial2}Two binomial distributions, involving a
scenario in which I'm flipping a fair coin, so the underlying success
probability is 1/2. In panel (a), we assume I'm flipping the coin N = 20
times. In panel (b) we assume that the coin is flipped N = 100 times.}

\end{figure}%

Well, as Figure~\ref{fig-4binomial2} \(a\) shows, the main effect of
this is to shift the whole distribution, as you'd expect. Okay, what if
we flipped a coin \(N=100\) times? Well, in that case, we get
Figure~\ref{fig-4binomial2} \(b\). The distribution stays roughly in the
middle, but there's a bit more variability in the possible outcomes.

At this point, I should probably explain the name of the \texttt{dbinom}
function. Obviously, the ``binom'' part comes from the fact that we're
working with the binomial distribution, but the ``d'' prefix is probably
a bit of a mystery. In this section I'll give a partial explanation:
specifically, I'll explain why there is a prefix. As for why it's a
``d'' specifically, you'll have to wait until the next section. What's
going on here is that R actually provides \textbf{four} functions in
relation to the binomial distribution. These four functions are
\texttt{dbinom}, \texttt{pbinom}, \texttt{rbinom} and \texttt{qbinom},
and each one calculates a different quantity of interest. Not only that,
R does the same thing for \textbf{every} probability distribution that
it implements. No matter what distribution you're talking about, there's
a \texttt{d} function, a \texttt{p} function, \texttt{r} a function and
a \texttt{q} function.

Let's have a look at what all four functions do. Firstly, all four
versions of the function require you to specify the \texttt{size} and
\texttt{prob} arguments: no matter what you're trying to get R to
calculate, it needs to know what the parameters are. However, they
differ in terms of what the other argument is, and what the output is.
So let's look at them one at a time.

\begin{itemize}
\item
  The \texttt{d} form we've already seen: you specify a particular
  outcome \texttt{x}, and the output is the probability of obtaining
  exactly that outcome. (the ``d'' is short for \emph{density}, but
  ignore that for now).
\item
  The \texttt{p} form calculates the \emph{cumulative probability}. You
  specify a particular quantile \texttt{q} , and it tells you the
  probability of obtaining an outcome \textbf{smaller than or equal to}
  \texttt{q}.
\item
  The \texttt{q} form calculates the \emph{quantiles} of the
  distribution. You specify a probability value \texttt{p}, and it gives
  you the corresponding percentile. That is, the value of the variable
  for which there's a probability \texttt{p} of obtaining an outcome
  lower than that value.
\item
  The \texttt{r} form is a \emph{random number generator}: specifically,
  it generates \texttt{n} random outcomes from the distribution.
\end{itemize}

This is a little abstract, so let's look at some concrete examples.
Again, we've already covered \texttt{dbinom} so let's focus on the other
three versions. We'll start with \texttt{pbinom}, and we'll go back to
the skull-dice example. Again, I'm rolling 20 dice, and each die has a 1
in 6 chance of coming up skulls. Suppose, however, that I want to know
the probability of rolling 4 \textbf{or fewer} skulls. If I wanted to, I
could use the \texttt{dbinom} function to calculate the exact
probability of rolling 0 skulls, 1 skull, 2 skulls, 3 skulls and 4
skulls and then add these up, but there's a faster way. Instead, I can
calculate this using the \texttt{pbinom} function. Here's the command:

\begin{Shaded}
\begin{Highlighting}[]
\FunctionTok{pbinom}\NormalTok{( }\AttributeTok{q=} \DecValTok{4}\NormalTok{, }\AttributeTok{size =} \DecValTok{20}\NormalTok{, }\AttributeTok{prob =} \DecValTok{1}\SpecialCharTok{/}\DecValTok{6}\NormalTok{)}
\CommentTok{\#\textgreater{} [1] 0.7687492}
\end{Highlighting}
\end{Shaded}

In other words, there is a 76.9\% chance that I will roll 4 or fewer
skulls. Or, to put it another way, R is telling us that a value of 4 is
actually the 76.9th percentile of this binomial distribution.

Next, let's consider the \texttt{qbinom} function. Let's say I want to
calculate the 75th percentile of the binomial distribution. If we're
sticking with our skulls example, I would use the following command to
do this:

\begin{Shaded}
\begin{Highlighting}[]
\FunctionTok{qbinom}\NormalTok{( }\AttributeTok{p =} \FloatTok{0.75}\NormalTok{, }\AttributeTok{size =} \DecValTok{20}\NormalTok{, }\AttributeTok{prob =} \DecValTok{1}\SpecialCharTok{/}\DecValTok{6}\NormalTok{ )}
\CommentTok{\#\textgreater{} [1] 4}
\end{Highlighting}
\end{Shaded}

Hm. There's something odd going on here. Let's think this through. What
the \texttt{qbinom} function appears to be telling us is that the 75th
percentile of the binomial distribution is 4, even though we saw from
the function that 4 is \textbf{actually} the 76.9th percentile. And it's
definitely the \texttt{pbinom} function that is correct. I promise. The
weirdness here comes from the fact that our binomial distribution
doesn't really \textbf{have} a 75th percentile. Not really. Why not?
Well, there's a 56.7\% chance of rolling 3 or fewer skulls (you can type
\texttt{pbinom(3,\ 20,\ 1/6)} to confirm this if you want), and a 76.9\%
chance of rolling 4 or fewer skulls. So there's a sense in which the
75th percentile should lie ``in between'' 3 and 4 skulls. But that makes
no sense at all! You can't roll 20 dice and get 3.9 of them come up
skulls. This issue can be handled in different ways: you could report an
in between value (or \textbf{interpolated} value, to use the technical
name) like 3.9, you could round down (to 3) or you could round up (to
4).

The \texttt{qbinom} function rounds upwards: if you ask for a percentile
that doesn't actually exist (like the 75th in this example), R finds the
smallest value for which the the percentile rank is \textbf{at least}
what you asked for. In this case, since the ``true'' 75th percentile
(whatever that would mean) lies somewhere between 3 and 4 skulls, R
Rounds up and gives you an answer of 4. This subtlety is tedious, I
admit, but thankfully it's only an issue for discrete distributions like
the binomial. The other distributions that I'll talk about (normal,
\(t\), \(\chi^2\) and \(F\)) are all continuous, and so R can always
return an exact quantile whenever you ask for it.

Finally, we have the random number generator. To use the \texttt{rbinom}
function, you specify how many times R should ``simulate'' the
experiment using the \texttt{n} argument, and it will generate random
outcomes from the binomial distribution. So, for instance, suppose I
were to repeat my die rolling experiment 100 times. I could get R to
simulate the results of these experiments by using the following
command:

\begin{Shaded}
\begin{Highlighting}[]
\FunctionTok{rbinom}\NormalTok{( }\AttributeTok{n =} \DecValTok{100}\NormalTok{, }\AttributeTok{size =} \DecValTok{20}\NormalTok{, }\AttributeTok{prob =} \DecValTok{1}\SpecialCharTok{/}\DecValTok{6}\NormalTok{ )}
\CommentTok{\#\textgreater{}   [1] 2 5 8 2 6 2 2 4 4 3 7 5 3 5 3 3 1 3 7 1 2 3 4 7 4 1 3 6 3 5 7 1 3 5 3 6 4}
\CommentTok{\#\textgreater{}  [38] 7 3 2 3 5 1 4 4 2 3 6 3 5 2 6 2 1 2 3 3 2 4 1 6 2 2 5 2 2 3 5 5 0 6 2 4 5}
\CommentTok{\#\textgreater{}  [75] 3 7 4 3 1 5 6 1 5 2 0 5 4 2 5 2 2 4 3 4 2 1 6 0 6 3}
\end{Highlighting}
\end{Shaded}

As you can see, these numbers are pretty much what you'd expect given
the distribution shown in Figure~\ref{fig-4binomial1} . Most of the time
I roll somewhere between 1 to 5 skulls. There are a lot of subtleties
associated with random number generation using a computer, but for the
purposes of this book we don't need to worry too much about them.

\section{The normal distribution}\label{the-normal-distribution}

While the binomial distribution is conceptually the simplest
distribution to understand, it's not the most important one. That
particular honor goes to the \emph{normal distribution}, which is also
referred to as ``the bell curve'' or a ``Gaussian distribution''.

\begin{figure}

\centering{

\includegraphics[width=0.75\linewidth,height=\textheight,keepaspectratio]{imgs/navarro_img/probability/standardNormal-eps-converted-to.png}

}

\caption{\label{fig-4normal}The normal distribution with mean = 0 and
standard deviation = 1. The x-axis corresponds to the value of some
variable, and the y-axis tells us something about how likely we are to
observe that value. However, notice that the y-axis is labelled
Probability Density and not Probability. There is a subtle and somewhat
frustrating characteristic of continuous distributions that makes the y
axis behave a bit oddly: the height of the curve here isn't actually the
probability of observing a particular x value. On the other hand, it is
true that the heights of the curve tells you which x values are more
likely (the higher ones!).}

\end{figure}%

A normal distribution is described using two parameters, the mean of the
distribution \(\mu\) and the standard deviation of the distribution
\(\sigma\). The notation that we sometimes use to say that a variable
\(X\) is normally distributed is as follows:
\[X \sim \mbox{Normal}(\mu,\sigma)\] Of course, that's just notation. It
doesn't tell us anything interesting about the normal distribution
itself. The mathematical formula for the normal distribution is:

\begin{figure}[H]

{\centering \includegraphics[width=0.75\linewidth,height=\textheight,keepaspectratio]{imgs/navarro_img/probability/Normal_formula.png}

}

\caption{Formula for the normal distribution}

\end{figure}%

The formula is important enough that everyone who learns statistics
should at least look at it, but since this is an introductory text I
don't want to focus on it to much. Instead, we look at how R can be used
to work with normal distributions. The R functions for the normal
distribution are \emph{dnorm()}, \emph{pnorm()}, \emph{qnorm()} and
\emph{rnorm()}. However, they behave in pretty much exactly the same way
as the corresponding functions for the binomial distribution, so there's
not a lot that you need to know. The only thing that I should point out
is that the argument names for the parameters are \emph{mean} and
\emph{sd}. In pretty much every other respect, there's nothing else to
add.

Instead of focusing on the maths, let's try to get a sense for what it
means for a variable to be normally distributed. To that end, have a
look at Figure~\ref{fig-4normal}, which plots a normal distribution with
mean \(\mu = 0\) and standard deviation \(\sigma = 1\). You can see
where the name ``bell curve'' comes from: it looks a bit like a bell.
Notice that, unlike the plots that I drew to illustrate the binomial
distribution, the picture of the normal distribution in Figure
Figure~\ref{fig-4normal} shows a smooth curve instead of
``histogram-like'' bars. This isn't an arbitrary choice: the normal
distribution is continuous, whereas the binomial is discrete. For
instance, in the die rolling example from the last section, it was
possible to get 3 skulls or 4 skulls, but impossible to get 3.9 skulls.

With this in mind, let's see if we can get an intuition for how the
normal distribution works. First, let's have a look at what happens when
we play around with the parameters of the distribution. One parameter we
can change is the mean. This will shift the distribution to the right or
left. The animation in \textbf{?@fig-4normalMeanShift} shows a normal
distribution with mean = 0, moving up and down from mean = 0 to mean =
5. Note, when you change the mean the whole shape of the distribution
does not change, it just shifts from left to right. In the animation the
normal distribution bounces up and down a little, but that's just a
quirk of the animation (plus it looks fun that way).

In contrast, if we increase the standard deviation while keeping the
mean constant, the peak of the distribution stays in the same place, but
the distribution gets wider. The animation in
\textbf{?@fig-4normalSDShift} shows what happens when you start with a
small standard deviation (sd = 0.5), and move to larger and larger
standard deviation (up to sd = 5). As you can see, the distribution
spreads out and becomes wider as the standard deviation increases.

Notice that when we widen the distribution the height of the peak
shrinks. This has to happen: in the same way that the heights of the
bars that we used to draw a discrete binomial distribution have to
\emph{sum} to 1, the total \emph{area under the curve} for the normal
distribution must equal 1. Before moving on, I want to point out one
important characteristic of the normal distribution. Irrespective of
what the actual mean and standard deviation are, 68.3\% of the area
falls within 1 standard deviation of the mean. Similarly, 95.4\% of the
distribution falls within 2 standard deviations of the mean, and 99.7\%
of the distribution is within 3 standard deviations.

\subsection{Probability density}\label{probability-density}

There's something I've been trying to hide throughout my discussion of
the normal distribution, something that some introductory textbooks omit
completely. They might be right to do so: this ``thing'' that I'm hiding
is weird and counter intuitive even by the admittedly distorted
standards that apply in statistics. Fortunately, it's not something that
you need to understand at a deep level in order to do basic statistics:
rather, it's something that starts to become important later on when you
move beyond the basics. So, if it doesn't make complete sense, don't
worry: try to make sure that you follow the gist of it.

Throughout my discussion of the normal distribution, there's been one or
two things that don't quite make sense. Perhaps you noticed that the
\(y\)-axis in these figures is labelled ``Probability Density'' rather
than density. Maybe you noticed that I used \(p(X)\) instead of \(P(X)\)
when giving the formula for the normal distribution. Maybe you're
wondering why R uses the ``d'' prefix for functions like \emph{dnorm()}.
And maybe, just maybe, you've been playing around with the
\emph{dnorm()} function, and you accidentally typed in a command like
this:

\begin{Shaded}
\begin{Highlighting}[]
\FunctionTok{dnorm}\NormalTok{( }\AttributeTok{x =} \DecValTok{1}\NormalTok{, }\AttributeTok{mean =} \DecValTok{1}\NormalTok{, }\AttributeTok{sd =} \FloatTok{0.1}\NormalTok{ )}
\CommentTok{\#\textgreater{} [1] 3.989423}
\end{Highlighting}
\end{Shaded}

And if you've done the last part, you're probably very confused. I've
asked R to calculate the probability that \emph{x = 1}, for a normally
distributed variable with \emph{mean = 1} and standard deviation
\emph{sd = 0.1}; and it tells me that the probability is 3.99. But, as
we discussed earlier, probabilities \emph{can't} be larger than 1. So
either I've made a mistake, or that's not a probability.

As it turns out, the second answer is correct. What we've calculated
here isn't actually a probability: it's something else. To understand
what that something is, you have to spend a little time thinking about
what it really \emph{means} to say that \(X\) is a continuous variable.
Let's say we're talking about the temperature outside. The thermometer
tells me it's 23 degrees, but I know that's not really true. It's not
\emph{exactly} 23 degrees. Maybe it's 23.1 degrees, I think to myself.
But I know that that's not really true either, because it might actually
be 23.09 degrees. But, I know that\ldots{} well, you get the idea. The
tricky thing with genuinely continuous quantities is that you never
really know exactly what they are.

Now think about what this implies when we talk about probabilities.
Suppose that tomorrow's maximum temperature is sampled from a normal
distribution with mean 23 and standard deviation 1. What's the
probability that the temperature will be \emph{exactly} 23 degrees? The
answer is ``zero'', or possibly, ``a number so close to zero that it
might as well be zero''. Why is this?

It's like trying to throw a dart at an infinitely small dart board: no
matter how good your aim, you'll never hit it. In real life you'll never
get a value of exactly 23. It'll always be something like 23.1 or
22.99998 or something. In other words, it's completely meaningless to
talk about the probability that the temperature is exactly 23 degrees.
However, in everyday language, if I told you that it was 23 degrees
outside and it turned out to be 22.9998 degrees, you probably wouldn't
call me a liar. Because in everyday language, ``23 degrees'' usually
means something like ``somewhere between 22.5 and 23.5 degrees''. And
while it doesn't feel very meaningful to ask about the probability that
the temperature is exactly 23 degrees, it does seem sensible to ask
about the probability that the temperature lies between 22.5 and 23.5,
or between 20 and 30, or any other range of temperatures.

The point of this discussion is to make clear that, when we're talking
about continuous distributions, it's not meaningful to talk about the
probability of a specific value. However, what we \emph{can} talk about
is the \textbf{probability that the value lies within a particular range
of values}. To find out the probability associated with a particular
range, what you need to do is calculate the ``area under the curve''.

Okay, so that explains part of the story. I've explained a little bit
about how continuous probability distributions should be interpreted
(i.e., area under the curve is the key thing), but I haven't actually
explained what the \emph{dnorm()} function actually calculates.
Equivalently, what does the formula for \(p(x)\) that I described
earlier actually mean? Obviously, \(p(x)\) doesn't describe a
probability, but what is it? The name for this quantity \(p(x)\) is a
\emph{probability density}, and in terms of the plots we've been
drawing, it corresponds to the \emph{height} of the curve. The densities
themselves aren't meaningful in and of themselves: but they're
``rigged'' to ensure that the \emph{area} under the curve is always
interpretable as genuine probabilities. To be honest, that's about as
much as you really need to know for now.

\section{Other useful distributions}\label{other-useful-distributions}

There are many other useful distributions, these include the \texttt{t}
distribution, the \texttt{F} distribution, and the chi squared
distribution. We will soon discover more about the \texttt{t} and
\texttt{F} distributions when we discuss t-tests and ANOVAs in later
chapters.

\section{Summary of Probability}\label{summary-of-probability}

We've talked what probability means, and why statisticians can't agree
on what it means. We talked about the rules that probabilities have to
obey. And we introduced the idea of a probability distribution, and
spent a good chunk talking about some of the more important probability
distributions that statisticians work with. We talked about things like
this:

\begin{itemize}
\item
  Probability theory versus statistics
\item
  Frequentist versus Bayesian views of probability
\item
  Basics of probability theory
\item
  Binomial distribution, normal distribution
\end{itemize}

As you'd expect, this coverage is by no means exhaustive. Probability
theory is a large branch of mathematics in its own right, entirely
separate from its application to statistics and data analysis. As such,
there are thousands of books written on the subject and universities
generally offer multiple classes devoted entirely to probability theory.
Even the ``simpler'' task of documenting standard probability
distributions is a big topic.Fortunately for you, very little of this is
necessary. You're unlikely to need to know dozens of statistical
distributions when you go out and do real world data analysis, and you
definitely won't need them for this book, but it never hurts to know
that there's other possibilities out there.

Picking up on that last point, there's a sense in which this whole
chapter is something of a digression. Many statistics classes skim over
this content very quickly (I know mine did), and even the more advanced
classes will often ``forget'' to revisit the basic foundations of the
field. Many academics would not know the difference between probability
and density, and until recently very few would have been aware of the
difference between Bayesian and frequentist probability. However, I
think it's important to understand these things before moving onto the
applications. For example, there are a lot of rules about what you're
``allowed'' to say when doing statistical inference, and many of these
can seem arbitrary and weird. However, they start to make sense if you
understand that there is this Bayesian/frequentist distinction.

\section{Samples, populations and
sampling}\label{samples-populations-and-sampling}

Remember, the role of descriptive statistics is to concisely summarize
what we \textbf{do} know. In contrast, the purpose of inferential
statistics is to ``learn what we do not know from what we do''. What
kinds of things would we like to learn about? And how do we learn them?
These are the questions that lie at the heart of inferential statistics,
and they are traditionally divided into two ``big ideas'': estimation
and hypothesis testing. The goal in this chapter is to introduce the
first of these big ideas, estimation theory, but we'll talk about
sampling theory first because estimation theory doesn't make sense until
you understand sampling. So, this chapter divides into sampling theory,
and how to make use of sampling theory to discuss how statisticians
think about estimation. We have already done lots of sampling, so you
are already familiar with some of the big ideas.

\textbf{Sampling theory} plays a huge role in specifying the assumptions
upon which your statistical inferences rely. And in order to talk about
``making inferences'' the way statisticians think about it, we need to
be a bit more explicit about what it is that we're drawing inferences
\textbf{from} (the sample) and what it is that we're drawing inferences
\textbf{about} (the population).

In almost every situation of interest, what we have available to us as
researchers is a \textbf{sample} of data. We might have run experiment
with some number of participants; a polling company might have phoned
some number of people to ask questions about voting intentions; etc.
Regardless: the data set available to us is finite, and incomplete. We
can't possibly get every person in the world to do our experiment; a
polling company doesn't have the time or the money to ring up every
voter in the country etc. In our earlier discussion of descriptive
statistics, this sample was the only thing we were interested in. Our
only goal was to find ways of describing, summarizing and graphing that
sample. This is about to change.

\subsection{Defining a population}\label{defining-a-population}

A sample is a concrete thing. You can open up a data file, and there's
the data from your sample. A \textbf{population}, on the other hand, is
a more abstract idea. It refers to the set of all possible people, or
all possible observations, that you want to draw conclusions about, and
is generally \textbf{much} bigger than the sample. In an ideal world,
the researcher would begin the study with a clear idea of what the
population of interest is, since the process of designing a study and
testing hypotheses about the data that it produces does depend on the
population about which you want to make statements. However, that
doesn't always happen in practice: usually the researcher has a fairly
vague idea of what the population is and designs the study as best
he/she can on that basis.

Sometimes it's easy to state the population of interest. For instance,
in the ``polling company'' example, the population consisted of all
voters enrolled at the a time of the study -- millions of people. The
sample was a set of 1000 people who all belong to that population. In
most situations the situation is much less simple. In a typical a
psychological experiment, determining the population of interest is a
bit more complicated. Suppose I run an experiment using 100
undergraduate students as my participants. My goal, as a cognitive
scientist, is to try to learn something about how the mind works. So,
which of the following would count as ``the population'':

\begin{itemize}
\item
  All of the undergraduate psychology students at the University of
  Adelaide?
\item
  Undergraduate psychology students in general, anywhere in the world?
\item
  Australians currently living?
\item
  Australians of similar ages to my sample?
\item
  Anyone currently alive?
\item
  Any human being, past, present or future?
\item
  Any biological organism with a sufficient degree of intelligence
  operating in a terrestrial environment?
\item
  Any intelligent being?
\end{itemize}

Each of these defines a real group of mind-possessing entities, all of
which might be of interest to me as a cognitive scientist, and it's not
at all clear which one ought to be the true population of interest.

\subsection{Simple random samples}\label{simple-random-samples}

Irrespective of how we define the population, the critical point is that
the sample is a subset of the population, and our goal is to use our
knowledge of the sample to draw inferences about the properties of the
population. The relationship between the two depends on the
\textbf{procedure} by which the sample was selected. This procedure is
referred to as a \textbf{sampling method}, and it is important to
understand why it matters.

To keep things simple, imagine we have a bag containing 10 chips. Each
chip has a unique letter printed on it, so we can distinguish between
the 10 chips. The chips come in two colors, black and white.

\begin{figure}

\centering{

\includegraphics[width=0.75\linewidth,height=\textheight,keepaspectratio]{imgs/navarro_img/estimation/srs1.png}

}

\caption{\label{fig-srs1}Simple random sampling without replacement from
a finite population}

\end{figure}%

This set of chips is the population of interest, and it is depicted
graphically on the left of Figure~\ref{fig-srs1}.

As you can see from looking at the picture, there are 4 black chips and
6 white chips, but of course in real life we wouldn't know that unless
we looked in the bag. Now imagine you run the following ``experiment'':
you shake up the bag, close your eyes, and pull out 4 chips without
putting any of them back into the bag. First out comes the \(a\) chip
(black), then the \(c\) chip (white), then \(j\) (white) and then
finally \(b\) (black). If you wanted, you could then put all the chips
back in the bag and repeat the experiment, as depicted on the right hand
side of Figure~\ref{fig-srs1}. Each time you get different results, but
the procedure is identical in each case. The fact that the same
procedure can lead to different results each time, we refer to it as a
\textbf{random} process. However, because we shook the bag before
pulling any chips out, it seems reasonable to think that every chip has
the same chance of being selected. A procedure in which every member of
the population has the same chance of being selected is called a
\textbf{simple random sample}. The fact that we did \textbf{not} put the
chips back in the bag after pulling them out means that you can't
observe the same thing twice, and in such cases the observations are
said to have been sampled \textbf{without replacement}.

To help make sure you understand the importance of the sampling
procedure, consider an alternative way in which the experiment could
have been run. Suppose that my 5-year old son had opened the bag, and
decided to pull out four black chips without putting any of them back in
the bag. This \textbf{biased} sampling scheme is depicted in
Figure~\ref{fig-brs}.

\begin{figure}

\centering{

\includegraphics[width=0.75\linewidth,height=\textheight,keepaspectratio]{imgs/navarro_img/estimation/brs.png}

}

\caption{\label{fig-brs}Biased sampling without replacement from a
finite populations.}

\end{figure}%

Now consider the evidentiary value of seeing 4 black chips and 0 white
chips. Clearly, it depends on the sampling scheme, does it not? If you
know that the sampling scheme is biased to select only black chips, then
a sample that consists of only black chips doesn't tell you very much
about the population! For this reason, statisticians really like it when
a data set can be considered a simple random sample, because it makes
the data analysis \textbf{much} easier.

A third procedure is worth mentioning. This time around we close our
eyes, shake the bag, and pull out a chip. This time, however, we record
the observation and then put the chip back in the bag. Again we close
our eyes, shake the bag, and pull out a chip. We then repeat this
procedure until we have 4 chips. Data sets generated in this way are
still simple random samples, but because we put the chips back in the
bag immediately after drawing them it is referred to as a sample
\textbf{with replacement}. The difference between this situation and the
first one is that it is possible to observe the same population member
multiple times, as illustrated in Figure~\ref{fig-srs2}.

\begin{figure}

\centering{

\includegraphics[width=0.75\linewidth,height=\textheight,keepaspectratio]{imgs/navarro_img/estimation/srs2.png}

}

\caption{\label{fig-srs2}Simple random sampling with replacement from a
finite population.}

\end{figure}%

Most psychology experiments tend to be sampling without replacement,
because the same person is not allowed to participate in the experiment
twice. However, most statistical theory is based on the assumption that
the data arise from a simple random sample \textbf{with} replacement. In
real life, this very rarely matters. If the population of interest is
large (e.g., has more than 10 entities!) the difference between sampling
with- and without- replacement is too small to be concerned with. The
difference between simple random samples and biased samples, on the
other hand, is not such an easy thing to dismiss.

\subsection{Most samples are not simple random
samples}\label{most-samples-are-not-simple-random-samples}

As you can see from looking at the list of possible populations that I
showed above, it is almost impossible to obtain a simple random sample
from most populations of interest. When I run experiments, I'd consider
it a minor miracle if my participants turned out to be a random sampling
of the undergraduate psychology students at Adelaide university, even
though this is by far the narrowest population that I might want to
generalize to. A thorough discussion of other types of sampling schemes
is beyond the scope of this book, but to give you a sense of what's out
there I'll list a few of the more important ones:

\begin{itemize}
\item
  \textbf{Stratified sampling}. Suppose your population is (or can be)
  divided into several different sub-populations, or \textbf{strata}.
  Perhaps you're running a study at several different sites, for
  example. Instead of trying to sample randomly from the population as a
  whole, you instead try to collect a separate random sample from each
  of the strata. Stratified sampling is sometimes easier to do than
  simple random sampling, especially when the population is already
  divided into the distinct strata. It can also be more efficient that
  simple random sampling, especially when some of the sub-populations
  are rare. For instance, when studying schizophrenia it would be much
  better to divide the population into two strata (schizophrenic and
  not-schizophrenic), and then sample an equal number of people from
  each group. If you selected people randomly, you would get so few
  schizophrenic people in the sample that your study would be useless.
  This specific kind of of stratified sampling is referred to as
  \textbf{oversampling} because it makes a deliberate attempt to
  over-represent rare groups.
\item
  \textbf{Snowball sampling} is a technique that is especially useful
  when sampling from a ``hidden'' or hard to access population, and is
  especially common in social sciences. For instance, suppose the
  researchers want to conduct an opinion poll among transgender people.
  The research team might only have contact details for a few trans
  folks, so the survey starts by asking them to participate (stage 1).
  At the end of the survey, the participants are asked to provide
  contact details for other people who might want to participate. In
  stage 2, those new contacts are surveyed. The process continues until
  the researchers have sufficient data. The big advantage to snowball
  sampling is that it gets you data in situations that might otherwise
  be impossible to get any. On the statistical side, the main
  disadvantage is that the sample is highly non-random, and non-random
  in ways that are difficult to address. On the real life side, the
  disadvantage is that the procedure can be unethical if not handled
  well, because hidden populations are often hidden for a reason. I
  chose transgender people as an example here to highlight this: if you
  weren't careful you might end up outing people who don't want to be
  outed (very, very bad form), and even if you don't make that mistake
  it can still be intrusive to use people's social networks to study
  them. It's certainly very hard to get people's informed consent
  \textbf{before} contacting them, yet in many cases the simple act of
  contacting them and saying ``hey we want to study you'' can be
  hurtful. Social networks are complex things, and just because you can
  use them to get data doesn't always mean you should.
\item
  \textbf{Convenience sampling} is more or less what it sounds like. The
  samples are chosen in a way that is convenient to the researcher, and
  not selected at random from the population of interest. Snowball
  sampling is one type of convenience sampling, but there are many
  others. A common example in psychology are studies that rely on
  undergraduate psychology students. These samples are generally
  non-random in two respects: firstly, reliance on undergraduate
  psychology students automatically means that your data are restricted
  to a single sub-population. Secondly, the students usually get to pick
  which studies they participate in, so the sample is a self selected
  subset of psychology students not a randomly selected subset. In real
  life, most studies are convenience samples of one form or another.
  This is sometimes a severe limitation, but not always.
\end{itemize}

\subsection{How much does it matter if you don't have a simple random
sample?}\label{how-much-does-it-matter-if-you-dont-have-a-simple-random-sample}

Okay, so real world data collection tends not to involve nice simple
random samples. Does that matter? A little thought should make it clear
to you that it \textbf{can} matter if your data are not a simple random
sample: just think about the difference between Figure~\ref{fig-srs1}
and Figure~\ref{fig-brs}. However, it's not quite as bad as it sounds.
Some types of biased samples are entirely unproblematic. For instance,
when using a stratified sampling technique you actually \textbf{know}
what the bias is because you created it deliberately, often to
\textbf{increase} the effectiveness of your study, and there are
statistical techniques that you can use to adjust for the biases you've
introduced (not covered in this book!). So in those situations it's not
a problem.

More generally though, it's important to remember that random sampling
is a means to an end, not the end in itself. Let's assume you've relied
on a convenience sample, and as such you can assume it's biased. A bias
in your sampling method is only a problem if it causes you to draw the
wrong conclusions. When viewed from that perspective, I'd argue that we
don't need the sample to be randomly generated in \textbf{every}
respect: we only need it to be random with respect to the
psychologically-relevant phenomenon of interest. Suppose I'm doing a
study looking at working memory capacity. In study 1, I actually have
the ability to sample randomly from all human beings currently alive,
with one exception: I can only sample people born on a Monday. In study
2, I am able to sample randomly from the Australian population. I want
to generalize my results to the population of all living humans. Which
study is better? The answer, obviously, is study 1. Why? Because we have
no reason to think that being ``born on a Monday'' has any interesting
relationship to working memory capacity. In contrast, I can think of
several reasons why ``being Australian'' might matter. Australia is a
wealthy, industrialized country with a very well-developed education
system. People growing up in that system will have had life experiences
much more similar to the experiences of the people who designed the
tests for working memory capacity. This shared experience might easily
translate into similar beliefs about how to ``take a test'', a shared
assumption about how psychological experimentation works, and so on.
These things might actually matter. For instance, ``test taking'' style
might have taught the Australian participants how to direct their
attention exclusively on fairly abstract test materials relative to
people that haven't grown up in a similar environment; leading to a
misleading picture of what working memory capacity is.

There are two points hidden in this discussion. Firstly, when designing
your own studies, it's important to think about what population you care
about, and try hard to sample in a way that is appropriate to that
population. In practice, you're usually forced to put up with a ``sample
of convenience'' (e.g., psychology lecturers sample psychology students
because that's the least expensive way to collect data, and our coffers
aren't exactly overflowing with gold), but if so you should at least
spend some time thinking about what the dangers of this practice might
be.

Secondly, if you're going to criticize someone else's study because
they've used a sample of convenience rather than laboriously sampling
randomly from the entire human population, at least have the courtesy to
offer a specific theory as to \textbf{how} this might have distorted the
results. Remember, everyone in science is aware of this issue, and does
what they can to alleviate it. Merely pointing out that ``the study only
included people from group BLAH'' is entirely unhelpful, and borders on
being insulting to the researchers, who are aware of the issue. They
just don't happen to be in possession of the infinite supply of time and
money required to construct the perfect sample. In short, if you want to
offer a responsible critique of the sampling process, then be
\textbf{helpful}. Rehashing the blindingly obvious truisms that I've
been rambling on about in this section isn't helpful.

\subsection{Population parameters and sample
statistics}\label{population-parameters-and-sample-statistics}

Okay. Setting aside the thorny methodological issues associated with
obtaining a random sample, let's consider a slightly different issue. Up
to this point we have been talking about populations the way a scientist
might. To a psychologist, a population might be a group of people. To an
ecologist, a population might be a group of bears. In most cases the
populations that scientists care about are concrete things that actually
exist in the real world.

Statisticians, however, are a funny lot. On the one hand, they
\textbf{are} interested in real world data and real science in the same
way that scientists are. On the other hand, they also operate in the
realm of pure abstraction in the way that mathematicians do. As a
consequence, statistical theory tends to be a bit abstract in how a
population is defined. In much the same way that psychological
researchers operationalize our abstract theoretical ideas in terms of
concrete measurements, statisticians operationalize the concept of a
``population'' in terms of mathematical objects that they know how to
work with. You've already come across these objects they're called
probability distributions (remember, the place where data comes from).

The idea is quite simple. Let's say we're talking about IQ scores. To a
psychologist, the population of interest is a group of actual humans who
have IQ scores. A statistician ``simplifies'' this by operationally
defining the population as the probability distribution depicted in
Figure~\ref{fig-IQdist} \(a\).

\begin{figure}

\centering{

\includegraphics[width=1\linewidth,height=\textheight,keepaspectratio]{imgs/figures/navIQ.png}

}

\caption{\label{fig-IQdist}The population distribution of IQ scores
(panel a) and two samples drawn randomly from it. In panel b we have a
sample of 100 observations, and panel c we have a sample of 10,000
observations.}

\end{figure}%

IQ tests are designed so that the average IQ is 100, the standard
deviation of IQ scores is 15, and the distribution of IQ scores is
normal. These values are referred to as the \textbf{population
parameters} because they are characteristics of the entire population.
That is, we say that the population mean \(\mu\) is 100, and the
population standard deviation \(\sigma\) is 15.

Now suppose we collect some data. We select 100 people at random and
administer an IQ test, giving a simple random sample from the
population. The sample would consist of a collection of numbers like
this:

\texttt{106\ 101\ 98\ 80\ 74\ ...\ 107\ 72\ 100}

Each of these IQ scores is sampled from a normal distribution with mean
100 and standard deviation 15. So if I plot a histogram of the sample, I
get something like the one shown in Figure~\ref{fig-IQdist} \(b\). As
you can see, the histogram is \textbf{roughly} the right shape, but it's
a very crude approximation to the true population distribution shown in
Figure~\ref{fig-IQdist} \(a\). The mean of the sample is fairly close to
the population mean 100 but not identical. In this case, it turns out
that the people in the sample have a mean IQ of 98.5, and the standard
deviation of their IQ scores is 15.9. These \textbf{sample statistics}
are properties of the data set, and although they are fairly similar to
the true population values, they are not the same. \textbf{In general,
sample statistics are the things you can calculate from your data set,
and the population parameters are the things you want to learn about.}
Later on in this chapter we'll talk about how you can estimate
population parameters using your sample statistics and how to work out
how confident you are in your estimates but before we get to that
there's a few more ideas in sampling theory that you need to know about.

\section{The law of large numbers}\label{the-law-of-large-numbers}

We just looked at the results of one fictitious IQ experiment with a
sample size of \(N=100\). The results were somewhat encouraging: the
true population mean is 100, and the sample mean of 98.5 is a pretty
reasonable approximation to it. In many scientific studies that level of
precision is perfectly acceptable, but in other situations you need to
be a lot more precise. If we want our sample statistics to be much
closer to the population parameters, what can we do about it?

The obvious answer is to collect more data. Suppose that we ran a much
larger experiment, this time measuring the IQ's of 10,000 people. We can
simulate the results of this experiment using R, using the
\textbf{rnorm()} function, which generates random numbers sampled from a
normal distribution. For an experiment with a sample size of \textbf{n =
10000}, and a population with \textbf{mean = 100} and \textbf{sd = 15},
R produces our fake IQ data using these commands:

\begin{Shaded}
\begin{Highlighting}[]
\NormalTok{IQ }\OtherTok{\textless{}{-}} \FunctionTok{rnorm}\NormalTok{(}\AttributeTok{n=}\DecValTok{10000}\NormalTok{, }\AttributeTok{mean=}\DecValTok{100}\NormalTok{, }\AttributeTok{sd=}\DecValTok{15}\NormalTok{) }\CommentTok{\#generate IQ scores}
\NormalTok{IQ }\OtherTok{\textless{}{-}} \FunctionTok{round}\NormalTok{(IQ) }\CommentTok{\# make round numbers}
\end{Highlighting}
\end{Shaded}

Cool, we just generated 10,000 fake IQ scores. Where did they go? Well,
they went into the variable IQ on my computer. You can do the same on
your computer too by copying the above code. 10,000 numbers is too many
numbers to look at. We can look at the first 100 like this:

\begin{Shaded}
\begin{Highlighting}[]
\FunctionTok{print}\NormalTok{(IQ[}\DecValTok{1}\SpecialCharTok{:}\DecValTok{100}\NormalTok{])}
\CommentTok{\#\textgreater{}   [1]  99 108 133 106 127  91  95 104 120 123  92 108  93 125 124 109  94 100}
\CommentTok{\#\textgreater{}  [19] 104 121  89  83  83  91  92 102 106  93 111 114  78 128 104 104  92  98}
\CommentTok{\#\textgreater{}  [37]  99  99  85 130  88 131 108  69 126 114  92  87 114  87 107  95  81 111}
\CommentTok{\#\textgreater{}  [55] 112 115 111 127  95 107 109 111 111 112 110 118 100 112 109  99  83  98}
\CommentTok{\#\textgreater{}  [73] 124 113  82  87  70  95 106 104 107 107 100  94  72  93 112  96  87  69}
\CommentTok{\#\textgreater{}  [91] 101 109 104 100 106  92 101 123 114 118}
\end{Highlighting}
\end{Shaded}

We can compute the mean IQ using the command \textbf{mean(IQ)} and the
standard deviation using the command \textbf{sd(IQ)}, and draw a
histogram using \textbf{hist()}. The histogram of this much larger
sample is shown in Figure @ref(fig:IQdist)c.~Even a moment's inspections
makes clear that the larger sample is a much better approximation to the
true population distribution than the smaller one. This is reflected in
the sample statistics: the mean IQ for the larger sample turns out to be
99.9, and the standard deviation is 15.1. These values are now very
close to the true population.

I feel a bit silly saying this, but the thing I want you to take away
from this is that large samples generally give you better information. I
feel silly saying it because it's so bloody obvious that it shouldn't
need to be said. In fact, it's such an obvious point that when Jacob
Bernoulli -- one of the founders of probability theory -- formalized
this idea back in 1713, he was kind of a jerk about it. Here's how he
described the fact that we all share this intuition:

\begin{quote}
\textbf{For even the most stupid of men, by some instinct of nature, by
himself and without any instruction (which is a remarkable thing), is
convinced that the more observations have been made, the less danger
there is of wandering from one's goal} (see Stigler, 1986, p65).
\end{quote}

Okay, so the passage comes across as a bit condescending (not to mention
sexist), but his main point is correct: it really does feel obvious that
more data will give you better answers. The question is, why is this so?
Not surprisingly, this intuition that we all share turns out to be
correct, and statisticians refer to it as the \textbf{law of large
numbers}. The law of large numbers is a mathematical law that applies to
many different sample statistics, but the simplest way to think about it
is as a law about averages. The sample mean is the most obvious example
of a statistic that relies on averaging (because that's what the mean
is\ldots{} an average), so let's look at that. \textbf{When applied to
the sample mean, what the law of large numbers states is that as the
sample gets larger, the sample mean tends to get closer to the true
population mean.} Or, to say it a little bit more precisely, as the
sample size ``approaches'' infinity (written as
\(N \rightarrow \infty\)) the sample mean approaches the population mean
(\(\bar{X} \rightarrow \mu\)).

I don't intend to subject you to a proof that the law of large numbers
is true, but it's one of the most important tools for statistical
theory. The law of large numbers is the thing we can use to justify our
belief that collecting more and more data will eventually lead us to the
truth. For any particular data set, the sample statistics that we
calculate from it will be wrong, but the law of large numbers tells us
that if we keep collecting more data those sample statistics will tend
to get closer and closer to the true population parameters.

\section{Sampling distributions and the central limit
theorem}\label{sampling-distributions-and-the-central-limit-theorem}

The law of large numbers is a very powerful tool, but it's not going to
be good enough to answer all our questions. Among other things, all it
gives us is a ``long run guarantee''. In the long run, if we were
somehow able to collect an infinite amount of data, then the law of
large numbers guarantees that our sample statistics will be correct. But
as John Maynard Keynes famously argued in economics, a long run
guarantee is of little use in real life:

\begin{quote}
{[}**The{]} long run is a misleading guide to current affairs. In the
long run we are all dead. Economists set themselves too easy, too
useless a task, if in tempestuous seasons they can only tell us, that
when the storm is long past, the ocean is flat again.** Keynes (1923,
80)
\end{quote}

As in economics, so too in psychology and statistics. It is not enough
to know that we will \textbf{eventually} arrive at the right answer when
calculating the sample mean. Knowing that an infinitely large data set
will tell me the exact value of the population mean is cold comfort when
my \textbf{actual} data set has a sample size of \(N=100\). In real
life, then, we must know something about the behavior of the sample mean
when it is calculated from a more modest data set!

\subsection{Sampling distribution of the sample
means}\label{sampling-distribution-of-the-sample-means}

``Oh no, what is the sample distribution of the sample means? Is that
even allowed in English?''. Yes, unfortunately, this is allowed. The
\textbf{sampling distribution of the sample means} is the next most
important thing you will need to understand. IT IS SO IMPORTANT THAT IT
IS NECESSARY TO USE ALL CAPS. It is only confusing at first because it's
long and uses sampling and sample in the same phrase.

Don't worry, we've been prepping you for this. You know what a
distribution is right? It's where numbers comes from. It makes some
numbers occur more or less frequently, or the same as other numbers. You
know what a sample is right? It's the numbers we take from a
distribution. So, what could the sampling distribution of the sample
means refer to?

First, what do you think the sample means refers to? Well, if you took a
sample of numbers, you would have a bunch of numbers\ldots then, you
could compute the mean of those numbers. The sample mean is the mean of
the numbers in the sample. That is all. So, what is this distribution
you speak of? Well, what if you took a bunch of samples, put one here,
put one there, put some other ones other places. You have a lot of
different samples of numbers. You could compute the mean for each them.
Then you would have a bunch of means. What do those means look like?
Well, if you put them in a histogram, you could find out. If you did
that, you would be looking at (roughly) a distribution, AKA \textbf{the
sampling distribution of the sample means}.

``I'm following along sort of, why would I want to do this instead of
watching Netflix\ldots{}''. Because, the sampling distribution of the
sample means gives you another window into chance. A very useful one
that you can control, just like your remote control, by pressing the
right design buttons.

\subsection{Seeing the pieces}\label{seeing-the-pieces}

To make a sampling distribution of the sample means, we just need the
following:

\begin{enumerate}
\def\labelenumi{\arabic{enumi}.}
\tightlist
\item
  A distribution to take numbers from
\item
  A bunch of different samples from the distribution
\item
  The means of each of the samples
\item
  Get all of the sample means, and plot them in a histogram
\end{enumerate}

\begin{center}\rule{0.5\linewidth}{0.5pt}\end{center}

Question for yourself: What do you think the sampling distribution of
the sample means will look like? Will it tend to look the shape of the
distribution that the samples came from? Or not? Good question, think
about it.

\begin{center}\rule{0.5\linewidth}{0.5pt}\end{center}

Let's do those four things. We will sample numbers from the uniform
distribution. Figure~\ref{fig-4Unif} shows the uniform distribution for
sampling the set of integers from 1 to 10:

\begin{figure}

\centering{

\includegraphics[width=0.75\linewidth,height=\textheight,keepaspectratio]{04-SamplesPopulations_files/figure-pdf/fig-4Unif-1.pdf}

}

\caption{\label{fig-4Unif}A uniform distribution illustrating the
probabilites of sampling the numbers 1 to 10. In a uniform distribution,
all numbers have an equal probability of being sampled, so the line is
flat indicating all numbers have the same probability}

\end{figure}%

\textbf{?@fig-4sample20unif} animates the process of taking a bunch of
samples from the uniform distribution. We will set our sample-size to
20. It's easier to see how the sample mean behaves in a movie. Each
histogram shows a new sample. The red line shows where the mean of the
sample is. The samples are all very different from each other, but the
red line doesn't move around very much, it always stays near the middle.
However, the red line does move around a little bit, and this variance
is what we call the sampling distribution of the sample mean.

OK, what have we got here? We have an animation of 10 different samples.
Each sample has 20 observations and these are summarized in each of
histograms that show up in the animation. Each histogram has a red line.
The red line shows you where the mean of each sample is located. So, we
have found the sample means for the 10 different samples from a uniform
distribution.

First question. Are the sample means all the same? The answer is no.
They are all kind of similar to each other though, they are all around
five plus or minus a few numbers. This is interesting. Although all of
our samples look pretty different from one another, the means of our
samples look more similar than different.

Second question. What should we do with the means of our samples? Well,
how about we collect them them all, and then plot a histogram of them.
This would allow us to see what the distribution of the sample means
looks like. The next histogram is just this. Except, rather than taking
10 samples, we will take 10,000 samples. For each of them we will
compute the means. So, we will have 10,000 means.
Figure~\ref{fig-4unifmany} shows the histogram of the sample means:

\begin{figure}

\centering{

\includegraphics[width=0.75\linewidth,height=\textheight,keepaspectratio]{04-SamplesPopulations_files/figure-pdf/fig-4unifmany-1.pdf}

}

\caption{\label{fig-4unifmany}A histogram showing the sample means for
10,000 samples, each size 20, from the uniform distribution of numbers
from 1 to 10. The expected mean is 5.5, and the histogram is centered on
5.5. The mean of each sample is not always 5.5 because of sampling error
or chance}

\end{figure}%

``Wait what? This doesn't look right. I thought we were taking samples
from a uniform distribution. Uniform distributions are flat. THIS DOES
NOT LOOK LIKE A FLAT DISTRIBTUION, WHAT IS GOING ON, AAAAAGGGHH.''. We
feel your pain.

Remember, we are looking at the distribution of sample means. It is
indeed true that the distribution of sample means does not look the same
as the distribution we took the samples from. Our distribution of sample
means goes up and down. In fact, this will almost always be the case for
distributions of sample means. This fact is called the \textbf{central
limit theorem}, which we talk about later.

For now, let's talk about about what's happening. Remember, we have been
sampling numbers between the range 1 to 10. We are supposed to get each
number with roughly equal frequency, because we are sampling from a
uniform distribution. So, let's say we took a sample of 10 numbers, and
happened to get one of each from 1 to 10.

\texttt{1\ 2\ 3\ 4\ 5\ 6\ 7\ 8\ 9\ 10}

What is the mean of those numbers? Well, its 1+2+3+4+5+6+7+8+9+10 = 55 /
10 = 5.5. Imagine if we took a bigger sample, say of 20 numbers, and
again we got exactly 2 of each number. What would the mean be? It would
be (1+2+3+4+5+6+7+8+9+10)*2 = 110 / 20 = 5.5. Still 5.5. You can see
here, that the mean value of our uniform distribution is 5.5. Now that
we know this, we might expect that most of our samples will have a mean
near this number. We already know that every sample won't be perfect,
and it won't have exactly an equal amount of every number. So, we will
expect the mean of our samples to vary a little bit. The histogram that
we made shows the variation. Not surprisingly, the numbers vary around
the value 5.5.

\subsection{Sampling distributions exist for any sample
statistic!}\label{sampling-distributions-exist-for-any-sample-statistic}

One thing to keep in mind when thinking about sampling distributions is
that \textbf{any} sample statistic you might care to calculate has a
sampling distribution. For example, suppose that each time you sampled
some numbers from an experiment you wrote down the largest number in the
experiment. Doing this over and over again would give you a very
different sampling distribution, namely the \textbf{sampling
distribution of the maximum}. You could calculate the smallest number,
or the mode, or the median, of the variance, or the standard deviation,
or anything else from your sample. Then, you could repeat many times,
and produce the sampling distribution of those statistics. Neat!

Just for fun here are some different sampling distributions for
different statistics. We will take a normal distribution with mean =
100, and standard deviation =20. Then, we'll take lots of samples with n
= 50 (50 observations per sample). We'll save all of the sample
statistics, then plot their histograms in Figure~\ref{fig-4samplestats}.
Let's do it:

\begin{figure}

\centering{

\includegraphics[width=1\linewidth,height=\textheight,keepaspectratio]{04-SamplesPopulations_files/figure-pdf/fig-4samplestats-1.pdf}

}

\caption{\label{fig-4samplestats}Each panel shows a histogram of a
different sampling statistic}

\end{figure}%

We just computed 4 different sampling distributions, for the mean,
standard deviation, maximum value, and the median. If you just look
quickly at these histograms you might think they all basically look the
same. Hold up now. It's very important to look at the x-axes. They are
different. For example, the sample mean goes from about 90 to 110,
whereas the standard deviation goes from 15 to 25.

These sampling distributions are super important, and worth thinking
about. What should you think about? Well, here's a clue. These
distributions are telling you what to expect from your sample.
Critically, they are telling you what you should expect from a sample,
when you take one from the specific distribution that we used (normal
distribution with mean =100 and SD = 20). What have we learned. We've
learned a tonne. We've learned that we can expect our sample to have a
mean somewhere between 90 and 108ish. Notice, the sample means are never
more extreme. We've learned that our sample will usually have some
variance, and that the the standard deviation will be somewhere between
15 and 25 (never much more extreme than that). We can see that sometime
we get some big numbers, say between 120 and 180, but not much bigger
than that. And, we can see that the median is pretty similar to the
mean. If you ever took a sample of 50 numbers, and your descriptive
statistics were inside these windows, then perhaps they came from this
kind of normal distribution. If your sample statistics are very
different, then your sample probably did not come this distribution. By
using simulation, we can find out what samples look like when they come
from distributions, and we can use this information to make inferences
about whether our sample came from particular distributions.

\section{The central limit theorem}\label{the-central-limit-theorem}

OK, so now you've seen lots of sampling distributions, and you know what
the sampling distribution of the mean is. Here, we'll focus on
\textbf{how the sampling distribution of the mean changes as a function
of sample size.}

Intuitively, you already know part of the answer: if you only have a few
observations, the sample mean is likely to be quite inaccurate (you've
already seen it bounce around): if you replicate a small experiment and
recalculate the mean you'll get a very different answer. In other words,
the sampling distribution is quite wide. If you replicate a large
experiment and recalculate the sample mean you'll probably get the same
answer you got last time, so the sampling distribution will be very
narrow.

Let's give ourselves a nice movie to see everything in action. We're
going to sample numbers from a normal distribution.
\textbf{?@fig-4samplingmean} has four panels, each panel represents a
different sample size (n), including sample-sizes of 10, 50, 100, and
1000. The red line shows the shape of the normal distribution. The grey
bars show a histogram of each of the samples that we take. The red line
shows the mean of an individual sample (the middle of the grey bars). As
you can see, the red line moves around a lot, especially when the sample
size is small (10).

The new bits are the blue bars and the blue lines. The blue bars
represent the sampling distribution of the sample mean. For example, in
the panel for sample-size 10, we see a bunch of blue bars. This is a
histogram of 10 sample means, taken from 10 samples of size 10. In the
50 panel, we see a histogram of 50 sample means, taken from 50 samples
of size 50, and so on. The blue line in each panel is the mean of the
sample means (``aaagh, it's a mean of means'', yes it is).

What should you notice? Notice that the range of the blue bars shrinks
as sample size increases. The sampling distribution of the mean is quite
wide when the sample-size is 10, it narrows as sample-size increases to
50 and 100, and it's just one bar, right in the middle when sample-size
goes to 1000. What we are seeing is that the mean of the sampling
distribution approaches the mean of the population as sample-size
increases.

So, the sampling distribution of the mean is another distribution, and
it has some variance. It varies more when sample-size is small, and
varies less when sample-size is large. We can quantify this effect by
calculating the standard deviation of the sampling distribution, which
is referred to as the \textbf{standard error}. The standard error of a
statistic is often denoted SE, and since we're usually interested in the
standard error of the sample \textbf{mean}, we often use the acronym
SEM. As you can see just by looking at the movie, as the sample size
\(N\) increases, the SEM decreases.

Okay, so that's one part of the story. However, there's something we've
been glossing over a little bit. We've seen it already, but it's worth
looking at it one more time. Here's the thing: \textbf{no matter what
shape your population distribution is}, as \(N\) increases the sampling
distribution of the mean starts to look more like a normal distribution.
This is the central limit theorem.

To see the central limit theorem in action, we are going to look at some
histograms of sample means from different kinds of distributions. It is
very important to recognize that you are looking at distributions of
sample means, not distributions of individual samples.

Here we go, Figure~\ref{fig-4sampledistmeannorm} shows sampling from a
normal distribution. The red line is the normal distribution where each
sample is drawn from. The mean for each sample of numbers is computed,
and the distribution of sample means is shown by the blue bars. Note
that the shape of red line and the blue bars are similar, they both look
like a normal distribution.

\begin{figure}

\centering{

\includegraphics[width=0.75\linewidth,height=\textheight,keepaspectratio]{04-SamplesPopulations_files/figure-pdf/fig-4sampledistmeannorm-1.pdf}

}

\caption{\label{fig-4sampledistmeannorm}Comparison of two normal
distributions, and histograms for the sampling distribution of the mean
for different samples-sizes. The range of sampling distribution of the
mean shrinks as sample-size increases}

\end{figure}%

Let's do it again. This time we will sample from a flat uniform
distribution shown by the red line. However,
Figure~\ref{fig-4samplemeanunif} shows the distribution of sample means
represented by the blue bars is not flat, it looks like a normal
distribution.

\begin{figure}

\centering{

\includegraphics[width=1\linewidth,height=\textheight,keepaspectratio]{04-SamplesPopulations_files/figure-pdf/fig-4samplemeanunif-1.pdf}

}

\caption{\label{fig-4samplemeanunif}Illustration that the shape of the
sampling distribution of the mean is normal, even when the samples come
from a non-normal (uniform in this case) distribution}

\end{figure}%

One more time with an exponential distribution (shown in red) where
smaller numbers are more likely to be sampled than larger numbers. Even
though way more of the numbers in a given sample will be smaller than
larger, according to Figure~\ref{fig-4samplemeanExp} the sampling
distribution of the mean does not look the red line. Instead, the
sampling distribution of the mean looks like a bell-shaped normal curve.
This is the central limit theorem in action.

\begin{figure}

\centering{

\includegraphics[width=0.75\linewidth,height=\textheight,keepaspectratio]{04-SamplesPopulations_files/figure-pdf/fig-4samplemeanExp-1.pdf}

}

\caption{\label{fig-4samplemeanExp}Illustration that the shape of the
sampling distribution of the mean is normal, even when the samples come
from an exponential distribution}

\end{figure}%

On the basis of these figures, it seems like we have evidence for all of
the following claims about the sampling distribution of the mean:

\begin{itemize}
\item
  The mean of the sampling distribution is the same as the mean of the
  population
\item
  The standard deviation of the sampling distribution (i.e., the
  standard error) gets smaller as the sample size increases
\item
  The shape of the sampling distribution becomes normal as the sample
  size increases
\end{itemize}

As it happens, not only are all of these statements true, there is a
very famous theorem in statistics that proves all three of them, known
as the \textbf{central limit theorem}. Among other things, the central
limit theorem tells us that if the population distribution has mean
\(\mu\) and standard deviation \(\sigma\), then the sampling
distribution of the mean also has mean \(\mu\), and the standard error
of the mean is \[\mbox{SEM} = \frac{\sigma}{ \sqrt{N} }\] Because we
divide the population standard deviation \(\sigma\) by the square root
of the sample size \(N\), the SEM gets smaller as the sample size
increases. It also tells us that the shape of the sampling distribution
becomes normal.

This result is useful for all sorts of things. It tells us why large
experiments are more reliable than small ones, and because it gives us
an explicit formula for the standard error it tells us \textbf{how much}
more reliable a large experiment is. It tells us why the normal
distribution is, well, \textbf{normal}. In real experiments, many of the
things that we want to measure are actually averages of lots of
different quantities (e.g., arguably, ``general'' intelligence as
measured by IQ is an average of a large number of ``specific'' skills
and abilities), and when that happens, the averaged quantity should
follow a normal distribution. Because of this mathematical law, the
normal distribution pops up over and over again in real data.

\section{z-scores}\label{z-scores}

We are now in a position to combine some of things we've been talking
about in this chapter, and introduce you to a new tool,
\textbf{z-scores}. It turns out we won't use \textbf{z-scores} very much
in this textbook. However, you can't take a class on statistics and not
learn about \textbf{z-scores}.

We are going to look at a normal distribution in
Figure~\ref{fig-4normalSDspercents}, and draw lines through the
distribution at 0, +/- 1, +/-2, and +/- 3 standard deviations from the
mean:

\begin{figure}

\centering{

\includegraphics[width=0.75\linewidth,height=\textheight,keepaspectratio]{04-SamplesPopulations_files/figure-pdf/fig-4normalSDspercents-1.pdf}

}

\caption{\label{fig-4normalSDspercents}A normal distribution. Each line
represents a standard deviation from the mean. The labels show the
proportions of scores that fall between each bar.}

\end{figure}%

The figure shows a normal distribution with mean = 0, and standard
deviation = 1. We've drawn lines at each of the standard deviations: -3,
-2, -1, 0, 1, 2, and 3. We also show some numbers in the labels, in
between each line. These numbers are proportions. For example, we see
the proportion is .341 for scores that fall between the range 0 and 1.
Scores between 0 and 1 occur 34.1\% of the time. Scores in between -1
and 1, occur 68.2\% of the time, that's more than half of the scores.
Scores between 1 and occur about 13.6\% of the time, and scores between
2 and 3 occur even less, only 2.1\% of the time.

Normal distributions always have these properties, even when they have
different means and standard deviations. For example, take a look at the
normal distribution in Figure~\ref{fig-4normalSDspercentsB} that has a
mean = 100, and standard deviation = 25.

\begin{figure}

\centering{

\includegraphics[width=0.75\linewidth,height=\textheight,keepaspectratio]{04-SamplesPopulations_files/figure-pdf/fig-4normalSDspercentsB-1.pdf}

}

\caption{\label{fig-4normalSDspercentsB}A normal distribution. Each line
represents a standard deviation from the mean. The labels show the
proportions of scores that fall between each bar.}

\end{figure}%

Now we are looking at a normal distribution with mean = 100 and standard
deviation = 25. Notice that the region between 100 and 125 contains
34.1\% of the scores. This region is 1 standard deviation away from the
mean (the standard deviation is 25, the mean is 100, so 25 is one whole
standard deviation away from 100). As you can see, the very same
proportions occur between each of the standard deviations, as they did
when our standard deviation was set to 1 (with a mean of 0).

\subsection{Idea behind z-scores}\label{idea-behind-z-scores}

Sometimes it can be convenient to transform your original scores into
different scores that are easier to work with. For example, if you have
a bunch of proportions, like .3, .5, .6, .7, you might want to turn them
into percentages like 30\%, 50\%, 60\%, and 70\%. To do that you
multiply the proportions by a constant of 100. If you want to turn
percentages back into proportions, you divide by a constant of 100. This
kind of transformation just changes the scale of the numbers from
between 0-1, and between 0-100. Otherwise, the pattern in the numbers
stays the same.

The idea behind z-scores is a similar kind of transformation. The idea
is to express each raw score in terms of it's standard deviation. For
example, if I told you I got a 75\% on test, you wouldn't know how well
I did compared to the rest of the class. But, if I told you that I
scored 2 standard deviations above the mean, you'd know I did quite well
compared to the rest of the class, because you know that most scores (if
they are distributed normally) fall below 2 standard deviations of the
mean.

We also know, now thanks to the central limit theorem, that many of our
measures, such as sample means, will be distributed normally. So, it can
often be desirable to express the raw scores in terms of their standard
deviations.

Let's see how this looks in a table without showing you any formulas. We
will look at some scores that come from a normal distribution with mean
= 100, and standard deviation = 25. We will list some raw scores, along
with the z-scores

\begin{longtable}[]{@{}rr@{}}
\toprule\noalign{}
raw & z \\
\midrule\noalign{}
\endhead
\bottomrule\noalign{}
\endlastfoot
25 & -3 \\
50 & -2 \\
75 & -1 \\
100 & 0 \\
125 & 1 \\
150 & 2 \\
175 & 3 \\
\end{longtable}

Remember, the mean is 100, and the standard deviation is 25. How many
standard deviations away from the mean is a score of 100? The answer is
0, it's right on the mean. You can see the z-score for 100, is 0. How
many standard deviations is 125 away from the mean? Well the standard
deviation is 25, 125 is one whole 25 away from 100, that's a total of 1
standard deviation, so the z-score for 125 is 1. The z-score for 150 is
2, because 150 is two 25s away from 100. The z-score for 50 is -2,
because 50 is two 25s away from 100 in the opposite direction. All we
are doing here is re-expressing the raw scores in terms of how many
standard deviations they are from the mean. Remember, the mean is always
right on target, so the center of the z-score distribution is always 0.

\subsection{Calculating z-scores}\label{calculating-z-scores}

To calculate z-scores all you have to do is figure out how many standard
deviations from the mean each number is. Let's say the mean is 100, and
the standard deviation is 25. You have a score of 97. How many standard
deviations from the mean is 97?

First compute the difference between the score and the mean:

\(97-100 = -3\)

Alright, we have a total difference of -3. How many standard deviations
does -3 represent if 1 standard deviation is 25? Clearly -3 is much
smaller than 25, so it's going to be much less than 1. To figure it out,
just divide -3 by the standard deviation.

\(\frac{-3}{25} = -.12\)

Our z-score for 97 is -.12.

Here's the general formula:

\(z = \frac{\text{raw score} - \text{mean}}{\text{standard deviation}}\)

So, for example if we had these 10 scores from a normal distribution
with mean = 100, and standard deviation =25

\begin{verbatim}
#>  [1]  86.10  92.04 126.19 100.55  93.35  95.68 137.43  40.21 116.49  77.17
\end{verbatim}

The z-scores would be:

\begin{verbatim}
#>  [1] -0.5560 -0.3184  1.0476  0.0220 -0.2660 -0.1728  1.4972 -2.3916  0.6596
#> [10] -0.9132
\end{verbatim}

Once you have the z-scores, you could use them as another way to
describe your data. For example, now just by looking at a score you know
if it is likely or unlikely to occur, because you know how the area
under the normal curve works. z-scores between -1 and 1 happen pretty
often, scores greater than 1 or -1 still happen fairly often, but not as
often. And, scores bigger than 2 or -2 don't happen very often. This is
a convenient thing to do if you want to look at your numbers and get a
general sense of how often they happen.

Usually you do not know the mean or the standard deviation of the
population that you are drawing your sample scores from. So, you could
use the mean and standard deviation of your sample as an estimate, and
then use those to calculate z-scores.

Finally, z-scores are also called \textbf{standardized scores}, because
each raw score is described in terms of it's standard deviation. This
may well be the last time we talk about z-scores in this book. You might
wonder why we even bothered telling you about them. First, it's worth
knowing they are a thing. Second, they become important as your
statistical prowess becomes more advanced. Third, some statistical
concepts, like correlation, can be re-written in terms of z-scores, and
this illuminates aspects of those statistics. Finally, they are super
useful when you are dealing with a normal distribution that has a known
mean and standard deviation.

\section{Estimating population
parameters}\label{estimating-population-parameters}

Let's pause for a moment to get our bearings. We're about to go into the
topic of \textbf{estimation}. What is that, and why should you care?
First, population parameters are things about a distribution. For
example, distributions have means. The mean is a parameter of the
distribution. The standard deviation of a distribution is a parameter.
Anything that can describe a distribution is a potential parameter.

OK fine, who cares? This I think, is a really good question. There are
some good concrete reasons to care. And there are some great abstract
reasons to care. Unfortunately, most of the time in research, it's the
abstract reasons that matter most, and these can be the most difficult
to get your head around.

\subsection{Concrete population
parameters}\label{concrete-population-parameters}

First some concrete reasons. There are real populations out there, and
sometimes you want to know the parameters of them. For example, if you
are a shoe company, you would want to know about the population
parameters of feet size. As a first pass, you would want to know the
mean and standard deviation of the population. If your company knew
this, and other companies did not, your company would do better
(assuming all shoes are made equal). Why would your company do better,
and how could it use the parameters? Here's one good reason. As a shoe
company you want to meet demand with the right amount of supply. If you
make too many big or small shoes, and there aren't enough people to buy
them, then you're making extra shoes that don't sell. If you don't make
enough of the most popular sizes, you'll be leaving money on the table.
Right? Yes. So, what would be an optimal thing to do? Perhaps, you would
make different amounts of shoes in each size, corresponding to how the
demand for each shoe size. You would know something about the demand by
figuring out the frequency of each size in the population. You would
need to know the population parameters to do this.

Fortunately, it's pretty easy to get the population parameters without
measuring the entire population. Who has time to measure every-bodies
feet? Nobody, that's who. Instead, you would just need to randomly pick
a bunch of people, measure their feet, and then measure the parameters
of the sample. If you take a big enough sample, we have learned that the
sample mean gives a very good estimate of the population mean. We will
learn shortly that a version of the standard deviation of the sample
also gives a good estimate of the standard deviation of the population.
Perhaps shoe-sizes have a slightly different shape than a normal
distribution. Here too, if you collect a big enough sample, the shape of
the distribution of the sample will be a good estimate of the shape of
the populations. All of these are good reasons to care about estimating
population parameters. But, do you run a shoe company? Probably not.

\subsection{Abstract population
parameters}\label{abstract-population-parameters}

Even when we think we are talking about something concrete in
Psychology, it often gets abstract right away. Instead of measuring the
population of feet-sizes, how about the population of human happiness.
We all think we know what happiness is, everyone has more or less of it,
there are a bunch of people, so there must be a population of happiness
right? Perhaps, but it's not very concrete. The first problem is
figuring out how to measure happiness. Let's use a questionnaire.
Consider these questions:

\begin{quote}
How happy are you right now on a scale from 1 to 7? How happy are you in
general on a scale from 1 to 7? How happy are you in the mornings on a
scale from 1 to 7? How happy are you in the afternoons on a scale from 1
to 7?
\end{quote}

\begin{enumerate}
\def\labelenumi{\arabic{enumi}.}
\tightlist
\item
  = very unhappy
\item
  = unhappy
\item
  = sort of unhappy
\item
  = in the middle
\item
  = sort of happy
\item
  = happy
\item
  = very happy
\end{enumerate}

Forget about asking these questions to everybody in the world. Let's
just ask them to lots of people (our sample). What do you think would
happen? Well, obviously people would give all sorts of answers right. We
could tally up the answers and plot them in a histogram. This would show
us a distribution of happiness scores from our sample. ``Great,
fantastic!'', you say. Yes, fine and dandy.

So, on the one hand we could say lots of things about the people in our
sample. We could say exactly who says they are happy and who says they
aren't, after all they just told us!

But, what can we say about the larger population? Can we use the
parameters of our sample (e.g., mean, standard deviation, shape etc.) to
estimate something about a larger population. Can we infer how happy
everybody else is, just from our sample? HOLD THE PHONE.

\subsubsection{Complications with
inference}\label{complications-with-inference}

Before listing a bunch of complications, let me tell you what I think we
can do with our sample. Provided it is big enough, our sample parameters
will be a pretty good estimate of what another sample would look like.
Because of the following discussion, this is often all we can say. But,
that's OK, as you see throughout this book, we can work with that!

\textbf{Problem 1: Multiple populations}: If you looked at a large
sample of questionnaire data you will find evidence of multiple
distributions inside your sample. People answer questions differently.
Some people are very cautious and not very extreme. Their answers will
tend to be distributed about the middle of the scale, mostly 3s, 4s, and
5s. Some people are very bi-modal, they are very happy and very unhappy,
depending on time of day. These people's answers will be mostly 1s and
2s, and 6s and 7s, and those numbers look like they come from a
completely different distribution. Some people are entirely happy or
entirely unhappy. Again, these two ``populations'' of people's numbers
look like two different distributions, one with mostly 6s and 7s, and
one with mostly 1s and 2s. Other people will be more random, and their
scores will look like a uniform distribution. So, is there a single
population with parameters that we can estimate from our sample?
Probably not. Could be a mixture of lots of populations with different
distributions.

\textbf{Problem 2: What do these questions measure?}: If the whole point
of doing the questionnaire is to estimate the population's happiness, we
really need wonder if the sample measurements actually tell us anything
about happiness in the first place. Some questions: Are people accurate
in saying how happy they are? Does the measure of happiness depend on
the scale, for example, would the results be different if we used 0-100,
or -100 to +100, or no numbers? Does the measure of happiness depend on
the wording in the question? Does a measure like this one tell us
everything we want to know about happiness (probably not), what is it
missing (who knows? probably lots). In short, nobody knows if these
kinds of questions measure what we want them to measure. We just hope
that they do. Instead, we have a very good idea of the kinds of things
that they actually measure. It's really quite obvious, and staring you
in the face. Questionnaire measurements measure how people answer
questionnaires. In other words, how people behave and answer questions
when they are given a questionnaire. This might also measure something
about happiness, when the question has to do about happiness. But, it
turns out people are remarkably consistent in how they answer questions,
even when the questions are total nonsense, or have no questions at all
(just numbers to choose!) Maul (2017).

The take home complications here are that we can collect samples, but in
Psychology, we often don't have a good idea of the populations that
might be linked to these samples. There might be lots of populations, or
the populations could be different depending on who you ask. Finally,
the ``population'' might not be the one you want it to be.

\subsection{Experiments and Population
parameters}\label{experiments-and-population-parameters}

OK, so we don't own a shoe company, and we can't really identify the
population of interest in Psychology, can't we just skip this section on
estimation? After all, the ``population'' is just too weird and abstract
and useless and contentious. HOLD THE PHONE AGAIN!

It turns out we can apply the things we have been learning to solve lots
of important problems in research. These allow us to answer questions
with the data that we collect. Parameter estimation is one of these
tools. We just need to be a little bit more creative, and a little bit
more abstract to use the tools.

Here is what we know already. The numbers that we measure come from
somewhere, we have called this place ``distributions''. Distributions
control how the numbers arrive. Some numbers happen more than others
depending on the distribution. We assume, even if we don't know what the
distribution is, or what it means, that the numbers came from one.
Second, when get some numbers, we call it a sample. This entire chapter
so far has taught you one thing. When your sample is big, it resembles
the distribution it came from. And, when your sample is big, it will
resemble very closely what another big sample of the same thing will
look like. We can use this knowledge!

Very often as Psychologists what we want to know is what causes what. We
want to know if X causes something to change in Y. Does eating chocolate
make you happier? Does studying improve your grades? There a bazillions
of these kinds of questions. And, we want answers to them.

I've been trying to be mostly concrete so far in this textbook, that's
why we talk about silly things like chocolate and happiness, at least
they are concrete. Let's give a go at being abstract. We can do it.

So, we want to know if X causes Y to change. What is X? What is Y? X is
something you change, something you manipulate, the independent
variable. Y is something you measure. So, we will be taking samples from
Y. ``Oh I get it, we'll take samples from Y, then we can use the sample
parameters to estimate the population parameters of Y!'' NO, not really,
but yes sort of. We will take sample from Y, that is something we
absolutely do. In fact, that is really all we ever do, which is why
talking about the population of Y is kind of meaningless. We're more
interested in our samples of Y, and how they behave.

So, what would happen if we removed X from the universe altogether, and
then took a big sample of Y. We'll pretend Y measures something in a
Psychology experiment. So, we know right away that Y is variable. When
we take a big sample, it will have a distribution (because Y is
variable). So, we can do things like measure the mean of Y, and measure
the standard deviation of Y, and anything else we want to know about Y.
Fine. What would happen if we replicated this measurement. That is, we
just take another random sample of Y, just as big as the first. What
should happen is that our first sample should look a lot like our second
example. After all, we didn't do anything to Y, we just took two big
samples twice. Both of our samples will be a little bit different (due
to sampling error), but they'll be mostly the same. The bigger our
samples, the more they will look the same, especially when we don't do
anything to cause them to be different. In other words, we can use the
parameters of one sample to estimate the parameters of a second sample,
because they will tend to be the same, especially when they are large.

We are now ready for step two. You want to know if X changes Y. What do
you do? You make X go up and take a big sample of Y then look at it. You
make X go down, then take a second big sample of Y and look at it. Next,
you compare the two samples of Y. If X does nothing then what should you
find? We already discussed that in the previous paragraph. If X does
nothing, then both of your big samples of Y should be pretty similar.
However, if X does something to Y, then one of your big samples of Y
will be different from the other. You will have changed something about
Y. Maybe X makes the mean of Y change. Or maybe X makes the variation in
Y change. Or, maybe X makes the whole shape of the distribution change.
If we find any big changes that can't be explained by sampling error,
then we can conclude that something about X caused a change in Y! We
could use this approach to learn about what causes what!

The very important idea is still about estimation, just not population
parameter estimation exactly. We know that when we take samples they
naturally vary. So, when we estimate a parameter of a sample, like the
mean, we know we are off by some amount. When we find that two samples
are different, we need to find out if the size of the difference is
consistent with what sampling error can produce, or if the difference is
bigger than that. If the difference is bigger, then we can be confident
that sampling error didn't produce the difference. So, we can
confidently infer that something else (like an X) did cause the
difference. This bit of abstract thinking is what most of the rest of
the textbook is about. Determining whether there is a difference caused
by your manipulation. There's more to the story, there always is. We can
get more specific than just, is there a difference, but for introductory
purposes, we will focus on the finding of differences as a foundational
concept.

\subsection{Interim summary}\label{interim-summary}

We've talked about estimation without doing any estimation, so in the
next section we will do some estimating of the mean and of the standard
deviation. Formally, we talk about this as using a sample to estimate a
parameter of the population. Feel free to think of the ``population'' in
different ways. It could be concrete population, like the distribution
of feet-sizes. Or, it could be something more abstract, like the
parameter estimate of what samples usually look like when they come from
a distribution.

\subsection{Estimating the population
mean}\label{estimating-the-population-mean}

Suppose we go to Brooklyn and 100 of the locals are kind enough to sit
through an IQ test. The average IQ score among these people turns out to
be \(\bar{X}=98.5\). So what is the true mean IQ for the entire
population of Brooklyn? Obviously, we don't know the answer to that
question. It could be \(97.2\), but if could also be \(103.5\). Our
sampling isn't exhaustive so we cannot give a definitive answer.
Nevertheless if forced to give a ``best guess'' I'd have to say
\(98.5\). That's the essence of statistical estimation: giving a best
guess. We're using the sample mean as the best guess of the population
mean.

In this example, estimating the unknown population parameter is
straightforward. I calculate the sample mean, and I use that as my
\textbf{estimate of the population mean}. It's pretty simple, and in the
next section we'll explain the statistical justification for this
intuitive answer. However, for the moment let's make sure you recognize
that the sample statistic and the estimate of the population parameter
are conceptually different things. A sample statistic is a description
of your data, whereas the estimate is a guess about the population. With
that in mind, statisticians often use different notation to refer to
them. For instance, if true population mean is denoted \(\mu\), then we
would use \(\hat\mu\) to refer to our estimate of the population mean.
In contrast, the sample mean is denoted \(\bar{X}\) or sometimes \(m\).
However, in simple random samples, the estimate of the population mean
is identical to the sample mean: if I observe a sample mean of
\(\bar{X} = 98.5\), then my estimate of the population mean is also
\(\hat\mu = 98.5\). To help keep the notation clear, here's a handy
table:

\begin{longtable}[]{@{}
  >{\raggedright\arraybackslash}p{(\linewidth - 4\tabcolsep) * \real{0.2603}}
  >{\raggedright\arraybackslash}p{(\linewidth - 4\tabcolsep) * \real{0.3562}}
  >{\raggedright\arraybackslash}p{(\linewidth - 4\tabcolsep) * \real{0.3836}}@{}}
\toprule\noalign{}
\begin{minipage}[b]{\linewidth}\raggedright
Symbol
\end{minipage} & \begin{minipage}[b]{\linewidth}\raggedright
What is it?
\end{minipage} & \begin{minipage}[b]{\linewidth}\raggedright
Do we know what it is?
\end{minipage} \\
\midrule\noalign{}
\endhead
\bottomrule\noalign{}
\endlastfoot
\(\bar{X}\) & Sample mean & Yes, calculated from the raw data \\
\(\mu\) & True population mean & Almost never known for sure \\
\(\hat{\mu}\) & Estimate of the population mean & Yes, identical to the
sample mean \\
\end{longtable}

\subsection{Estimating the population standard
deviation}\label{estimating-the-population-standard-deviation}

So far, estimation seems pretty simple, and you might be wondering why I
forced you to read through all that stuff about sampling theory. In the
case of the mean, our estimate of the population parameter
(i.e.~\(\hat\mu\)) turned out to identical to the corresponding sample
statistic (i.e.~\(\bar{X}\)). However, that's not always true. To see
this, let's have a think about how to construct an \textbf{estimate of
the population standard deviation}, which we'll denote \(\hat\sigma\).
What shall we use as our estimate in this case? Your first thought might
be that we could do the same thing we did when estimating the mean, and
just use the sample statistic as our estimate. That's almost the right
thing to do, but not quite.

Here's why. Suppose I have a sample that contains a single observation.
For this example, it helps to consider a sample where you have no
intuitions at all about what the true population values might be, so
let's use something completely fictitious. Suppose the observation in
question measures the \textbf{cromulence} of my shoes. It turns out that
my shoes have a cromulence of 20. So here's my sample:

\texttt{20}

This is a perfectly legitimate sample, even if it does have a sample
size of \(N=1\). It has a sample mean of 20, and because every
observation in this sample is equal to the sample mean (obviously!) it
has a sample standard deviation of 0. As a description of the
\textbf{sample} this seems quite right: the sample contains a single
observation and therefore there is no variation observed within the
sample. A sample standard deviation of \(s = 0\) is the right answer
here. But as an estimate of the \textbf{population} standard deviation,
it feels completely insane, right? Admittedly, you and I don't know
anything at all about what ``cromulence'' is, but we know something
about data: the only reason that we don't see any variability in the
\textbf{sample} is that the sample is too small to display any
variation! So, if you have a sample size of \(N=1\), it \textbf{feels}
like the right answer is just to say ``no idea at all''.

Notice that you \textbf{don't} have the same intuition when it comes to
the sample mean and the population mean. If forced to make a best guess
about the population mean, it doesn't feel completely insane to guess
that the population mean is 20. Sure, you probably wouldn't feel very
confident in that guess, because you have only the one observation to
work with, but it's still the best guess you can make.

Let's extend this example a little. Suppose I now make a second
observation. My data set now has \(N=2\) observations of the cromulence
of shoes, and the complete sample now looks like this:

\texttt{20,\ 22}

This time around, our sample is \textbf{just} large enough for us to be
able to observe some variability: two observations is the bare minimum
number needed for any variability to be observed! For our new data set,
the sample mean is \(\bar{X}=21\), and the sample standard deviation is
\(s=1\). What intuitions do we have about the population? Again, as far
as the population mean goes, the best guess we can possibly make is the
sample mean: if forced to guess, we'd probably guess that the population
mean cromulence is 21. What about the standard deviation? This is a
little more complicated. The sample standard deviation is only based on
two observations, and if you're at all like me you probably have the
intuition that, with only two observations, we haven't given the
population ``enough of a chance'' to reveal its true variability to us.
It's not just that we suspect that the estimate is \textbf{wrong}: after
all, with only two observations we expect it to be wrong to some degree.
The worry is that the error is \textbf{systematic}.

If the error is systematic, that means it is \textbf{biased}. For
example, imagine if the sample mean was always smaller than the
population mean. If this was true (it's not), then we couldn't use the
sample mean as an estimator. It would be biased, we'd be using the wrong
number.

It turns out the sample standard deviation is a \textbf{biased
estimator} of the population standard deviation. We can sort of
anticipate this by what we've been discussing. When the sample size is
1, the standard deviation is 0, which is obviously to small. When the
sample size is 2, the standard deviation becomes a number bigger than 0,
but because we only have two sample, we suspect it might still be too
small. Turns out this intuition is correct.

It would be nice to demonstrate this somehow. There are in fact
mathematical proofs that confirm this intuition, but unless you have the
right mathematical background they don't help very much. Instead, what
I'll do is use R to simulate the results of some experiments. With that
in mind, let's return to our IQ studies. Suppose the true population
mean IQ is 100 and the standard deviation is 15. I can use the
\textbf{rnorm()} function to generate the the results of an experiment
in which I measure \(N=2\) IQ scores, and calculate the sample standard
deviation. If I do this over and over again, and plot a histogram of
these sample standard deviations, what I have is the \textbf{sampling
distribution of the standard deviation}. I've plotted this distribution
in Figure~\ref{fig-sampdistsd}.

\begin{figure}

\centering{

\includegraphics[width=0.75\linewidth,height=\textheight,keepaspectratio]{imgs/navarro_img/estimation/sampdistsd.png}

}

\caption{\label{fig-sampdistsd}The sampling distribution of the sample
standard deviation for a two IQ scores experiment. The true population
standard deviation is 15 (dashed line), but as you can see from the
histogram, the vast majority of experiments will produce a much smaller
sample standard deviation than this. On average, this experiment would
produce a sample standard deviation of only 8.5, well below the true
value! In other words, the sample standard deviation is a biased
estimate of the population standard deviation.}

\end{figure}%

Even though the true population standard deviation is 15, the average of
the \textbf{sample} standard deviations is only 8.5. Notice that this is
a very different from when we were plotting sampling distributions of
the sample mean, those were always centered around the mean of the
population.

Now let's extend the simulation. Instead of restricting ourselves to the
situation where we have a sample size of \(N=2\), let's repeat the
exercise for sample sizes from 1 to 10. If we plot the average sample
mean and average sample standard deviation as a function of sample size,
you get the following results.

@fig-estimatorbiasA shows the sample mean as a function of sample size.
Notice it's a flat line. The sample mean doesn't underestimate or
overestimate the population mean. It is an unbiased estimate!

\begin{figure}

\centering{

\includegraphics[width=0.75\linewidth,height=\textheight,keepaspectratio]{imgs/navarro_img/estimation/biasMean-eps-converted-to.png}

}

\caption{\label{fig-estimatorbiasA}An illustration of the fact that the
sample mean is an unbiased estimator of the population mean.}

\end{figure}%

Figure~\ref{fig-estimatorbiasB} shows the sample standard deviation as a
function of sample size. Notice it is not a flat line. The sample
standard deviation systematically underestimates the population standard
deviation!

\begin{figure}

\centering{

\includegraphics[width=0.75\linewidth,height=\textheight,keepaspectratio]{imgs/navarro_img/estimation/biasSD-eps-converted-to.png}

}

\caption{\label{fig-estimatorbiasB}An illustration of the fact that the
the sample standard deviation is a biased estimator of the population
standard deviation.}

\end{figure}%

In other words, if we want to make a ``best guess'' (\(\hat\sigma\), our
estimate of the population standard deviation) about the value of the
population standard deviation \(\sigma\), we should make sure our guess
is a little bit larger than the sample standard deviation \(s\).

The fix to this systematic bias turns out to be very simple. Here's how
it works. Before tackling the standard deviation, let's look at the
variance. If you recall from the second chapter, the sample variance is
defined to be the average of the squared deviations from the sample
mean. That is: \[s^2 = \frac{1}{N} \sum_{i=1}^N (X_i - \bar{X})^2\] The
sample variance \(s^2\) is a biased estimator of the population variance
\(\sigma^2\). But as it turns out, we only need to make a tiny tweak to
transform this into an unbiased estimator. All we have to do is divide
by \(N-1\) rather than by \(N\). If we do that, we obtain the following
formula: \[\hat\sigma^2 = \frac{1}{N-1} \sum_{i=1}^N (X_i - \bar{X})^2\]
This is an unbiased estimator of the population variance \(\sigma\).

A similar story applies for the standard deviation. If we divide by
\(N-1\) rather than \(N\), our estimate of the population standard
deviation becomes:
\[\hat\sigma = \sqrt{\frac{1}{N-1} \sum_{i=1}^N (X_i - \bar{X})^2}\].

It is worth pointing out that software programs make assumptions
\textbf{for you}, about which variance and standard deviation
\textbf{you} are computing. Some programs automatically divide by
\(N-1\), some do not. You need to check to figure out what they are
doing. Don't let the software tell you what to do. Software is for you
telling it what to do.

One final point: in practice, a lot of people tend to refer to
\(\hat{\sigma}\) (i.e., the formula where we divide by \(N-1\)) as the
\textbf{sample} standard deviation. Technically, this is incorrect: the
\textbf{sample} standard deviation should be equal to \(s\) (i.e., the
formula where we divide by \(N\)). These aren't the same thing, either
conceptually or numerically. One is a property of the sample, the other
is an estimated characteristic of the population. However, in almost
every real life application, what we actually care about is the estimate
of the population parameter, and so people always report \(\hat\sigma\)
rather than \(s\).

\begin{tcolorbox}[enhanced jigsaw, title=\textcolor{quarto-callout-note-color}{\faInfo}\hspace{0.5em}{Note}, colframe=quarto-callout-note-color-frame, colbacktitle=quarto-callout-note-color!10!white, bottomtitle=1mm, leftrule=.75mm, rightrule=.15mm, titlerule=0mm, arc=.35mm, colback=white, opacitybacktitle=0.6, toprule=.15mm, toptitle=1mm, bottomrule=.15mm, coltitle=black, breakable, left=2mm, opacityback=0]

Note, whether you should divide by N or N-1 also depends on your
philosophy about what you are doing. For example, if you don't think
that what you are doing is estimating a population parameter, then why
would you divide by N-1? Also, when N is large, it doesn't matter too
much. The difference between a big N, and a big N-1, is just -1.

\end{tcolorbox}

This is the right number to report, of course, it's that people tend to
get a little bit imprecise about terminology when they write it up,
because ``sample standard deviation'' is shorter than ``estimated
population standard deviation''. It's no big deal, and in practice I do
the same thing everyone else does. Nevertheless, I think it's important
to keep the two \textbf{concepts} separate: it's never a good idea to
confuse ``known properties of your sample'' with ``guesses about the
population from which it came''. The moment you start thinking that
\(s\) and \(\hat\sigma\) are the same thing, you start doing exactly
that.

To finish this section off, here's another couple of tables to help keep
things clear:

\begin{longtable}[]{@{}
  >{\raggedright\arraybackslash}p{(\linewidth - 4\tabcolsep) * \real{0.2639}}
  >{\raggedright\arraybackslash}p{(\linewidth - 4\tabcolsep) * \real{0.3333}}
  >{\raggedright\arraybackslash}p{(\linewidth - 4\tabcolsep) * \real{0.4028}}@{}}
\toprule\noalign{}
\begin{minipage}[b]{\linewidth}\raggedright
Symbol
\end{minipage} & \begin{minipage}[b]{\linewidth}\raggedright
What is it?
\end{minipage} & \begin{minipage}[b]{\linewidth}\raggedright
Do we know what it is?
\end{minipage} \\
\midrule\noalign{}
\endhead
\bottomrule\noalign{}
\endlastfoot
\(s^2\) & Sample variance & Yes, calculated from the raw data \\
\(\sigma^2\) & Population variance & Almost never known for sure \\
\(\hat{\sigma}^2\) & Estimate of the population variance & Yes, but not
the same as the sample variance \\
\end{longtable}

\section{Estimating a confidence
interval}\label{estimating-a-confidence-interval}

\begin{quote}
Statistics means never having to say you're certain -- Unknown origin
\end{quote}

Up to this point in this chapter, we've outlined the basics of sampling
theory which statisticians rely on to make guesses about population
parameters on the basis of a sample of data. As this discussion
illustrates, one of the reasons we need all this sampling theory is that
every data set leaves us with some of uncertainty, so our estimates are
never going to be perfectly accurate. The thing that has been missing
from this discussion is an attempt to \textbf{quantify} the amount of
uncertainty in our estimate. It's not enough to be able guess that the
mean IQ of undergraduate psychology students is 115 (yes, I just made
that number up). We also want to be able to say something that expresses
the degree of certainty that we have in our guess. For example, it would
be nice to be able to say that there is a 95\% chance that the true mean
lies between 109 and 121. The name for this is a \textbf{confidence
interval} for the mean.

Armed with an understanding of sampling distributions, constructing a
confidence interval for the mean is actually pretty easy. Here's how it
works. Suppose the true population mean is \(\mu\) and the standard
deviation is \(\sigma\). I've just finished running my study that has
\(N\) participants, and the mean IQ among those participants is
\(\bar{X}\). We know from our discussion of the central limit theorem
that the sampling distribution of the mean is approximately normal. We
also know from our discussion of the normal distribution that there is a
95\% chance that a normally-distributed quantity will fall within two
standard deviations of the true mean. To be more precise, we can use the
\textbf{qnorm()} function to compute the 2.5th and 97.5th percentiles of
the normal distribution

\begin{quote}
qnorm( p = c(.025, .975) ) {[}1{]} -1.959964 1.959964
\end{quote}

Okay, so I lied earlier on. The more correct answer is that a 95\%
chance that a normally-distributed quantity will fall within 1.96
standard deviations of the true mean.

Next, recall that the standard deviation of the sampling distribution is
referred to as the standard error, and the standard error of the mean is
written as SEM. When we put all these pieces together, we learn that
there is a 95\% probability that the sample mean \(\bar{X}\) that we
have actually observed lies within 1.96 standard errors of the
population mean. Oof, that is a lot of mathy talk there. We'll clear it
up, don't worry.

Mathematically, we write this as:
\[\mu - \left( 1.96 \times \mbox{SEM} \right) \ \leq \  \bar{X}\  \leq \  \mu + \left( 1.96 \times \mbox{SEM} \right)\]
where the SEM is equal to \(\sigma / \sqrt{N}\), and we can be 95\%
confident that this is true.

However, that's not answering the question that we're actually
interested in. The equation above tells us what we should expect about
the sample mean, given that we know what the population parameters are.
What we \textbf{want} is to have this work the other way around: we want
to know what we should believe about the population parameters, given
that we have observed a particular sample. However, it's not too
difficult to do this. Using a little high school algebra, a sneaky way
to rewrite our equation is like this:
\[\bar{X} -  \left( 1.96 \times \mbox{SEM} \right) \ \leq \ \mu  \ \leq  \ \bar{X} +  \left( 1.96 \times \mbox{SEM}\right)\]
What this is telling is is that the range of values has a 95\%
probability of containing the population mean \(\mu\). We refer to this
range as a \textbf{95\% confidence interval}, denoted
\(\mbox{CI}_{95}\). In short, as long as \(N\) is sufficiently large --
large enough for us to believe that the sampling distribution of the
mean is normal -- then we can write this as our formula for the 95\%
confidence interval:
\[\mbox{CI}_{95} = \bar{X} \pm \left( 1.96 \times \frac{\sigma}{\sqrt{N}} \right)\]
Of course, there's nothing special about the number 1.96: it just
happens to be the multiplier you need to use if you want a 95\%
confidence interval. If I'd wanted a 70\% confidence interval, I could
have used the \textbf{qnorm()} function to calculate the 15th and 85th
quantiles:

\begin{quote}
qnorm( p = c(.15, .85) ) {[}1{]} -1.036433 1.036433
\end{quote}

and so the formula for \(\mbox{CI}_{70}\) would be the same as the
formula for \(\mbox{CI}_{95}\) except that we'd use 1.04 as our magic
number rather than 1.96.

\subsection{A slight mistake in the
formula}\label{a-slight-mistake-in-the-formula}

As usual, I lied. The formula that I've given above for the 95\%
confidence interval is approximately correct, but I glossed over an
important detail in the discussion. Notice my formula requires you to
use the standard error of the mean, SEM, which in turn requires you to
use the true population standard deviation \(\sigma\).

Yet, before we stressed the fact that we don't actually \textbf{know}
the true population parameters. Because we don't know the true value of
\(\sigma\), we have to use an estimate of the population standard
deviation \(\hat{\sigma}\) instead. This is pretty straightforward to
do, but this has the consequence that we need to use the quantiles of
the \(t\)-distribution rather than the normal distribution to calculate
our magic number; and the answer depends on the sample size. Plus, we
haven't really talked about the \(t\) distribution yet.

When we use the \(t\) distribution instead of the normal distribution,
we get bigger numbers, indicating that we have more uncertainty. And why
do we have that extra uncertainty? Well, because our estimate of the
population standard deviation \(\hat\sigma\) might be wrong! If it's
wrong, it implies that we're a bit less sure about what our sampling
distribution of the mean actually looks like\ldots{} and this
uncertainty ends up getting reflected in a wider confidence interval.

\section{Summary}\label{summary-2}

In this chapter I've covered two main topics. The first half of the
chapter talks about sampling theory, and the second half talks about how
we can use sampling theory to construct estimates of the population
parameters. The section breakdown looks like this:

\begin{itemize}
\item
  Basic ideas about samples, sampling and populations
\item
  Statistical theory of sampling: the law of large numbers, sampling
  distributions and the central limit theorem.
\item
  Estimating means and standard deviations
\item
  confidence intervals
\end{itemize}

As always, there's a lot of topics related to sampling and estimation
that aren't covered in this chapter, but for an introductory psychology
class this is fairly comprehensive I think. For most applied researchers
you won't need much more theory than this. One big question that I
haven't touched on in this chapter is what you do when you don't have a
simple random sample. There is a lot of statistical theory you can draw
on to handle this situation, but it's well beyond the scope of this
book.

\section{Videos}\label{videos-2}

\subsection{Introduction to
Probability}\label{introduction-to-probability}

Jeff has several more videos on probability that you can view on his
\href{https://www.youtube.com/playlist?list=PLKXdxQAT3tCvuex_E1ZnQYaw897ELUSaI}{statistics
playlist}.

\subsection{Chebychev's Theorem}\label{chebychevs-theorem}

\subsection{Z-scores}\label{z-scores-1}

\subsection{Normal Distribution I}\label{normal-distribution-i}

\subsection{Normal Distribution II}\label{normal-distribution-ii}

\bookmarksetup{startatroot}

\chapter{Foundations for inference}\label{foundations-for-inference}

\begin{quote}
Data and data sets are not objective; they are creations of human
design. We give numbers their voice, draw inferences from them, and
define their meaning through our interpretations. ---Katie Crawford
\end{quote}

So far we have been talking about describing data and looking for
possible relationships between things we measure. We began with the
problem of having too many numbers and discussed how they could be
summarized with descriptive statistics, and communicated in graphs. We
also looked at the idea of relationships between things. If one thing
causes change in another thing, then if we measure how one thing goes up
and down we should find that other thing goes up and down, or does
something systematically following the first thing. At the end of the
chapter on correlation, we showed how correlations, which imply a
relationship between two things, are very difficult to interpret. Why?
Because an observed correlation can be caused by a hidden third
variable, or could be a spurious finding ``caused'' by random chance. In
the last chapter, we talked about sampling from distributions, and we
saw how samples can be different because of random error introduced by
the sampling process.

Now we begin our journey into \textbf{inferential statistics}. These are
tools used to make inferences about where our data came from, and to
make inferences about what causes what.

In this chapter we provide some foundational ideas. We will stay mostly
at a conceptual level, and use lots of simulations like we did in the
last chapters. In the remaining chapters we formalize the intuitions
built here to explain how some common inferential statistics work.

\section{Brief review of Experiments}\label{brief-review-of-experiments}

In chapter one we talked a about research methods and experiments.
Experiments are a structured way of collecting data that can permit
inferences about causality. If we wanted to know whether something like
watching cats on YouTube increases happiness we would need an
experiment. We already found out that just finding a bunch of people and
measuring number of hours watching cats, and level of happiness, and
correlating the two will not permit inferences about causation. For one,
the causal flow could be reversed. Maybe being happy causes people to
watch more cat videos. We need an experiment.

An experiment has two parts. A manipulation and a measurement. The
manipulation is under the control of the experimenter. Manipulations are
also called \textbf{independent variables}. For example, we could
manipulate time spent watching cat videos: 1 hour versus 2 hours of cat
videos. The measurement is the data that is collected. We could measure
how happy people are after watching cat videos on a scale from 1 to 100.
Measurements are also called \textbf{dependent variables}. So, in a
basic experiment like the one above, we take measurements of happiness
from people in one of two experimental conditions defined by the
independent variable. Let's say we ran 50 subjects. 25 subjects would be
randomly assigned to watch 1 hour of cat videos, and the other 25
subjects would be randomly assigned to watch 2 hours of cat videos. We
would measure happiness for each subject at the end of the videos. Then
we could look at the data.

What would we want to look at? If watching cat videos caused a change in
happiness, then we would expect the measures of happiness for people
watching 1 hour of cat videos to be different from the measures of
happiness for people watching 2 hours of cat videos. If watching cat
videos does not change happiness, then we would expect no differences in
measures of happiness between conditions. Causal forces cause change,
and the experiment is set up to detect the change.

Now we can state one overarching question, how do we know if the data
changed between conditions? If we can be confident that there was a
change between conditions, we can infer that our manipulation caused a
changed in the measurement. If we cannot be confident there was a
change, then we cannot infer that our manipulation caused a change in
the measurement. We need to build some change detection tools so we can
know a change when we find one.

``Hold on, if we are just looking for a change, wouldn't that be easy to
see by looking at the numbers and seeing if they are different, what's
so hard about that?''. Good question. Now we must take a detour. The
short answer is that there will always be change in the data (remember
variance).

\section{The data came from a
distribution}\label{the-data-came-from-a-distribution}

In the last chapter we discussed samples and distributions, and the idea
that you can take samples from distributions. So, from now on when you
see a bunch of numbers, you should wonder, ``where did these numbers
come from?''. What caused some kinds of numbers to happen more than
other kinds of numbers. The answer to this question requires us to again
veer off into the abstract world of distributions. A
\textbf{distribution} a place where numbers can come from. The
distribution sets the constraints. It determines what numbers are likely
to occur, and what numbers are not likely to occur. Distributions are
abstract ideas. But, they can be made concrete, and we can draw them
with pictures that you have seen already, called histograms.

The next bit might seem slightly repetitive from the previous chapter.
We again look at sampling numbers from a uniform distribution. We show
that individual samples can look quite different from each other. Much
of the beginning part of this chapter will already be familiar to you,
but we take the concepts in a slightly different direction. The
direction is how to make inferences about the role of chance in your
experiment.

\subsection{Uniform distribution}\label{uniform-distribution}

As a reminder from last chapter, Figure~\ref{fig-5unifagain} shows that
the shape of a uniform distribution is completely flat.

\begin{figure}

\centering{

\includegraphics[width=0.75\linewidth,height=\textheight,keepaspectratio]{05-Foundation_Inference_files/figure-pdf/fig-5unifagain-1.pdf}

}

\caption{\label{fig-5unifagain}Uniform distribution showing that the
numbers from 1 to 10 have an equal probability of being sampled}

\end{figure}%

OK, so that doesn't look like much. What is going on here? The y-axis is
labelled \texttt{probability}, and it goes from 0 to 1. The x-axis is
labelled \texttt{Number}, and it goes from one to 10. There is a
horizontal line drawn straight through. This line tells you the
probability of each number from 1 to 10. Notice the line is flat. This
means all of the numbers have the same probability of occurring. More
specifically, there are 10 numbers from 1 to 10 (1,2,3,4,5,6,7,8,9,10),
and they all have an equal chance of occurring. 1/10 = .1, which is the
probability indicated by the horizontal line.

``So what?''. Imagine that this uniform distribution is a number
generating machine. It spits out numbers, but it spits out each number
with the probability indicated by the line. If this distribution was
going to start spitting out numbers, it would spit out 10\% 1s, 10\% 2s,
10\% 3s, and so on, up to 10\% 10s. Wanna see what that would look like?
Let's make it spit out 100 numbers and put them in
Table~\ref{tbl-5100rand}.

\begin{longtable}[]{@{}rrrrrrrrrr@{}}

\caption{\label{tbl-5100rand}100 numbers randomly sampled from a uniform
distribution.}

\tabularnewline

\toprule\noalign{}
\endhead
\bottomrule\noalign{}
\endlastfoot
8 & 6 & 10 & 4 & 6 & 7 & 8 & 4 & 8 & 6 \\
9 & 6 & 9 & 4 & 5 & 2 & 2 & 7 & 3 & 2 \\
7 & 2 & 8 & 7 & 8 & 7 & 8 & 8 & 6 & 4 \\
9 & 3 & 5 & 10 & 4 & 3 & 2 & 4 & 7 & 8 \\
1 & 7 & 1 & 4 & 10 & 10 & 5 & 8 & 4 & 7 \\
4 & 2 & 4 & 9 & 5 & 2 & 2 & 2 & 6 & 6 \\
8 & 8 & 3 & 6 & 3 & 9 & 4 & 4 & 4 & 3 \\
7 & 2 & 5 & 10 & 3 & 7 & 4 & 7 & 4 & 10 \\
3 & 10 & 5 & 7 & 3 & 4 & 7 & 5 & 10 & 2 \\
6 & 6 & 4 & 2 & 2 & 3 & 3 & 8 & 3 & 6 \\

\end{longtable}

We used the uniform distribution to generate these numbers. Officially,
we call this \textbf{sampling} from a \textbf{distribution}. Sampling is
what you do at a grocery store when there is free food. You can keep
taking more. However, if you take all of the samples, then what you have
is called the \textbf{population}. We'll talk more about samples and
populations as we go along.

Because we used the uniform distribution to create numbers, we already
know where our numbers came from. However, we can still pretend for the
moment that someone showed up at your door, showed you these numbers,
and then you wondered where they came from. Can you tell just by looking
at these numbers that they came from a uniform distribution? What would
need to look at? Perhaps you would want to know if all of the numbers
occur with roughly equal frequency, after all they should have right?
That is, if each number had the same chance of occurring, we should see
that each number occurs roughly the same number of times.

We already know what a histogram is, so we can put our sample of 100
numbers into a histogram and see what the counts look like. If all of
the numbers from 1 to 10 occur with equal frequency, then each
individual number should occur about 10 times.
Figure~\ref{fig-5histunif} shows the histogram:

\begin{figure}

\centering{

\includegraphics[width=0.75\linewidth,height=\textheight,keepaspectratio]{05-Foundation_Inference_files/figure-pdf/fig-5histunif-1.pdf}

}

\caption{\label{fig-5histunif}Histogram of 100 numbers randomly sampled
from the uniform distribution containing the integers from 1 to 10}

\end{figure}%

Uh oh, as you can see, not all of the number occurred 10 times each. All
of the bars are not the same height. This shows that randomly sampling
numbers from this distribution does not guarantee that our numbers will
be exactly like the distribution they came from. We can call this
sampling error, or sampling variability.

\subsection{Not all samples are the same, they are usually quite
different}\label{not-all-samples-are-the-same-they-are-usually-quite-different}

Let's look at sampling error more closely. We will sample 20 numbers
from the uniform distribution. We should expect that each number between
1 and 10 occurs about two times each. As before, this expectation can be
visualized in a histogram. To get a better sense of sampling error,
let's repeat the above process ten times. Figure~\ref{fig-5manysamples}
has 10 histograms, each showing what 10 different samples of twenty
numbers looks like:

\begin{figure}

\centering{

\includegraphics[width=1\linewidth,height=\textheight,keepaspectratio]{05-Foundation_Inference_files/figure-pdf/fig-5manysamples-1.pdf}

}

\caption{\label{fig-5manysamples}Histograms for 10 different samples
from the uniform distribution. Each sample contains 20 numbers. The
histograms all look quite different. The differences between the samples
are due to sampling error or random chance.}

\end{figure}%

You might notice right away that none of the histograms are the same.
Even though we are randomly taking 20 numbers from the very same uniform
distribution, each sample of 20 numbers comes out different. This is
sampling variability, or sampling error.

\textbf{?@fig-5expectedUnif} shows an animated version of the process of
repeatedly choosing 20 new random numbers and plotting a histogram. The
horizontal line shows the flat-line shape of the uniform distribution.
The line crosses the y-axis at 2; and, we expect that each number (from
1 to 10) should occur about 2 times each in a sample of 20. However,
each sample bounces around quite a bit, due to random chance.

Looking at the above histograms shows us that figuring out where our
numbers came from can be difficult. In the real world, our measurements
are samples. We usually only have the luxury of getting one sample of
measurements, rather than repeating our own measurements 10 times or
more. If you look at the histograms, you will see that some of them look
like they could have come from the uniform distribution: most of the
bars are near two, and they all fall kind of on a flat line. But, if you
happen to look at a different sample, you might see something that is
very bumpy, with some numbers happening way more than others. This could
suggest to you that those numbers did not come from a uniform
distribution (they're just too bumpy). But let me remind you, all of
these samples came from a uniform distribution, this is what samples
from that distribution look like. This is what chance does to samples,
it makes the individual data points noisy.

\subsection{Large samples are more like the distribution they came
from}\label{large-samples-are-more-like-the-distribution-they-came-from}

Let's refresh the question. Which of the two samples in
Figure~\ref{fig-5whichone} do you think came from a uniform
distribution?

\begin{figure}

\centering{

\includegraphics[width=0.75\linewidth,height=\textheight,keepaspectratio]{05-Foundation_Inference_files/figure-pdf/fig-5whichone-1.pdf}

}

\caption{\label{fig-5whichone}Which of these two samples came from a
uniform distribution?}

\end{figure}%

The answer is that they both did. But, neither of them look like they
did.

Can we improve things, and make it easier to see if a sample came from a
uniform distribution? Yes, we can. All we need to do is increase the
\textbf{sample-size}. We will often use the letter \texttt{n} to refer
to sample-size. N is the number of observations in the sample.

So let's increase the number of observations in each sample from 20 to
100. We will again create 10 samples (each with 100 observations), and
make histograms for each of them. All of these samples will be drawn
from the very same uniform distribution. This, means we should expect
each number from 1 to 10 to occur about 10 times in each sample. The
histograms are shown in Figure~\ref{fig-5unifsamp100}.

\begin{figure}

\centering{

\includegraphics[width=1\linewidth,height=\textheight,keepaspectratio]{05-Foundation_Inference_files/figure-pdf/fig-5unifsamp100-1.pdf}

}

\caption{\label{fig-5unifsamp100}Histograms for different samples from a
uniform distribution. N = 100 for each sample.}

\end{figure}%

Again, most of these histograms don't look very flat, and all of the
bars seem to be going up or down, and they are not exactly at 10 each.
So, we are still dealing with sampling error. It's a pain. It's always
there.

Let's bump up the \(N\) from 100 to 1000 observations per sample. Now we
should expect every number to appear about 100 times each. What happens?

\begin{figure}

\centering{

\includegraphics[width=1\linewidth,height=\textheight,keepaspectratio]{05-Foundation_Inference_files/figure-pdf/fig-5unifsamp1000-1.pdf}

}

\caption{\label{fig-5unifsamp1000}Histograms for different samples from
a uniform distribution. N = 1000 for each sample.}

\end{figure}%

Figure~\ref{fig-5unifsamp1000} shows the histograms are starting to
flatten out. The bars are still not perfectly at 100, because there is
still sampling error (there always will be). But, if you found a
histogram that looked flat and knew that the sample contained many
observations, you might be more confident that those numbers came from a
uniform distribution.

Just for fun let's make the samples really big. Say 100,000 observations
per sample. Here, we should expect that each number occurs about 10,000
times each. What happens?

\begin{figure}

\centering{

\includegraphics[width=1\linewidth,height=\textheight,keepaspectratio]{05-Foundation_Inference_files/figure-pdf/fig-5sampunifALOT-1.pdf}

}

\caption{\label{fig-5sampunifALOT}Histograms for different samples from
a uniform distribution. N = 100,000 for each sample.}

\end{figure}%

Figure~\ref{fig-5sampunifALOT} shows that the histograms for each sample
are starting to look the same. They all have 100,000 observations, and
this gives chance enough opportunity to equally distribute the numbers,
roughly making sure that they all occur very close to the same amount of
times. As you can see, the bars are all very close to 10,000, which is
where they should be if the sample came from a uniform distribution.

\begin{tcolorbox}[enhanced jigsaw, title=\textcolor{quarto-callout-tip-color}{\faLightbulb}\hspace{0.5em}{Pro tip}, colframe=quarto-callout-tip-color-frame, colbacktitle=quarto-callout-tip-color!10!white, bottomtitle=1mm, leftrule=.75mm, rightrule=.15mm, titlerule=0mm, arc=.35mm, colback=white, opacitybacktitle=0.6, toprule=.15mm, toptitle=1mm, bottomrule=.15mm, coltitle=black, breakable, left=2mm, opacityback=0]

The pattern behind a sample will tend to stabilize as sample-size
increases. Small samples will have all sorts of patterns because of
sampling error (chance).

\end{tcolorbox}

Before getting back to the topic of experiments that we started with,
let's ask two more questions. First, which of the two samples in
Figure~\ref{fig-5whichoneB} do you think came from a uniform
distribution? FYI, each of these samples had 20 observations each.

\begin{figure}

\centering{

\includegraphics[width=0.75\linewidth,height=\textheight,keepaspectratio]{05-Foundation_Inference_files/figure-pdf/fig-5whichoneB-1.pdf}

}

\caption{\label{fig-5whichoneB}Which of these samples came from a
uniform distribution?}

\end{figure}%

If you are not confident in the answer, this is because \textbf{sampling
error} (randomness) is fuzzing with the histograms.

Here is the very same question, only this time we will take 1,000
observations for each sample. Which histogram in
Figure~\ref{fig-5whichoneC} do you think came from a uniform
distribution, which one did not?

\begin{figure}

\centering{

\includegraphics[width=0.75\linewidth,height=\textheight,keepaspectratio]{05-Foundation_Inference_files/figure-pdf/fig-5whichoneC-1.pdf}

}

\caption{\label{fig-5whichoneC}Which of these samples came from a
uniform distribution?}

\end{figure}%

Now that we have increased N, we can see the pattern in each sample
becomes more obvious. The histogram for sample 1 has bars near 100, not
perfectly flat, but it resembles a uniform distribution. The histogram
for sample 2 is not flat looking at all.

Congratulations to Us! We have just made some statistical inferences
without using formulas!

``We did?'' Yes, by looking at our two samples we have inferred that
sample 2 did not come from a uniform distribution. We have also inferred
that sample 1 could have come form a uniform distribution. Fantastic.
These are the same kinds of inferences we will be making for the rest of
the course. We will be looking at some numbers, wondering where they
came from, then we will arrange the numbers in such a way so that we can
make inferences about the kind of distribution they came from. That's
it.

\section{Is there a difference?}\label{is-there-a-difference}

Let's get back to experiments. In an experiment we want to know if an
independent variable (our manipulation) causes a change in a dependent
variable (measurement). If this occurs, then we will expect to see some
differences in our measurement as a function of the manipulation.

Consider the light switch example:

\begin{center}\rule{0.5\linewidth}{0.5pt}\end{center}

\textbf{Light Switch Experiment}: You manipulate the switch up
(condition 1 of independent variable), light goes on (measurement). You
manipulate the switch down (condition 2 of independent variable), light
goes off (another measurement). The measurement (light) changes (goes
off and on) as a function of the manipulation (moving switch up or
down).

You can see the change in measurement between the conditions, it is as
obvious as night and day. So, when you conduct a manipulation, and can
see the difference (change) in your measure, you can be pretty confident
that your manipulation is causing the change.

\begin{quote}
note: to be cautious we can say ``something'' about your manipulation is
causing the change, it might not be what you think it is if your
manipulation is very complicated and involves lots of moving parts.
\end{quote}

\begin{center}\rule{0.5\linewidth}{0.5pt}\end{center}

\subsection{Chance can produce
differences}\label{chance-can-produce-differences}

Do you think random chance can produce the appearance of differences,
even when there really aren't any? I hope so. We have already shown that
the process of sampling numbers from a distribution is a chancy process
that produces different samples. Different samples are different, so
yes, chance can produce differences. This can muck up our interpretation
of experiments.

Let's conduct a fictitious experiment where we expect to find no
differences, because we will manipulate something that shouldn't do
anything. Here's the set-up:

You are the experimenter standing in front of a gumball machine. It is
very big, has thousands of gumballs. 50\% of the gumballs are green, and
50\% are red. You want to find out if picking gumballs with your right
hand vs.~your left hand will cause you to pick more green gumballs.
Plus, you will be blindfolded the entire time. The independent variable
is Hand: right hand vs.~left hand. The dependent variable is the
measurement of the color of each gumball.

You run the experiment as follows. 1) put on blind fold. 2) pick 10
gumballs randomly with left hand, set them aside. 3) pick 10 gumballs
randomly with right hand, set them aside. 4) count the number of green
and red gumballs chosen by your left hand, and count the number of green
and red gumballs chosen by your right hand. Hopefully you will agree
that your hands will not be able to tell the difference between the
gumballs. If you don't agree, we will further stipulate the gumballs are
completely identical in every way except their color, so it would be
impossible to tell them apart using your hands. So, what should happen
in this experiment?

``Umm, maybe you get 5 red gum balls and 5 green balls from your left
hand, and also from your right hand?''. Sort of yes, this is what you
would usually get. But, it is not all that you can get. Here is some
data showing what happened from one pretend experiment:

\begin{longtable}[]{@{}lr@{}}
\toprule\noalign{}
hand & gumball \\
\midrule\noalign{}
\endhead
\bottomrule\noalign{}
\endlastfoot
left & 0 \\
left & 0 \\
left & 0 \\
left & 1 \\
left & 0 \\
left & 1 \\
left & 0 \\
left & 0 \\
left & 1 \\
left & 1 \\
right & 0 \\
right & 0 \\
right & 1 \\
right & 0 \\
right & 0 \\
right & 1 \\
right & 0 \\
right & 1 \\
right & 1 \\
right & 0 \\
\end{longtable}

``What am I looking at here''. This is a long-format table. Each row is
one gumball. The first column tells you what hand was used. The second
column tells you what kind of gumball. We will say 1s stand for green
gum balls, and 0s stand for red gumballs. So, did your left hand cause
you to pick more green gumballs than your right hand?

It would be easier to look at the data using a bar graph
(Figure~\ref{fig-5gumballA}). To keep things simple, we only count the
green gumballs (the other gumballs must be red). So, all we need to do
is sum up the 1s. The 0s won't add anything.

\begin{figure}

\centering{

\includegraphics[width=0.75\linewidth,height=\textheight,keepaspectratio]{05-Foundation_Inference_files/figure-pdf/fig-5gumballA-1.pdf}

}

\caption{\label{fig-5gumballA}Counts of green gumballs picked randomly
by each hand.}

\end{figure}%

Oh look, the bars are not the same. One hand picked more green gum balls
than the other. Does this mean that one of your hands secretly knows how
to find green gumballs? No, it's just another case of sampling error,
that thing we call luck or chance. The difference here is caused by
chance, not by the manipulation (which hand you use). \textbf{Major
problem for inference alert}. We run experiments to look for differences
so we can make inferences about whether our manipulations cause change
in our measures. However, this example demonstrates that we can find
differences by chance. How can we know if a difference is real, or just
caused by chance?

\subsection{Differences due to chance can be
simulated}\label{differences-due-to-chance-can-be-simulated}

Remember when we showed that chance can produce correlations. We also
showed that chance is restricted in its ability to produce correlations.
For example, chance more often produces weak correlations than strong
correlations. Remember the window of chance? We found out before that
correlations falling outside the window of chance were very unlikely. We
can do the same thing for differences. Let's find out just what chance
can do in our experiment. Once we know what chance is capable of we will
be in a better position to judge whether our manipulation caused a
difference, or whether it could have been chance.

The first thing to do is pretend you conduct the gumball experiment 10
times in a row. This will produce 10 different sets of results.
Figure~\ref{fig-5gumballsims} shows bar graphs for each replication of
the experiment. Now we can look at whether the left hand chose more
green gumballs than red gumballs.

\begin{figure}

\centering{

\includegraphics[width=1\linewidth,height=\textheight,keepaspectratio]{05-Foundation_Inference_files/figure-pdf/fig-5gumballsims-1.pdf}

}

\caption{\label{fig-5gumballsims}10 simulated replications of picking
gumballs. Each replication gives a slightly different answer. Any
difference between the bars is due to chance, or sampling error. This
shows that chance alone can produce differences, just by the act of
sampling.}

\end{figure}%

These 10 experiments give us a better look at what chance can do. It
should also mesh well with your expectations. If everything is
determined by chance (as we have made it so), then sometimes your left
hand will choose more green balls, sometimes your right hand will choose
more green gumballs, and sometimes they will choose the same amount of
gumballs. Right? Right.

\section{Chance makes some differences more likely than
others}\label{chance-makes-some-differences-more-likely-than-others}

OK, we have seen that chance can produce differences here. But, we still
don't have a good idea about what chance usually does and doesn't do.
For example, if we could find the window of opportunity here, we would
be able find out that chance usually does not produce differences of a
certain large size. If we knew what the size was, then if we ran
experiment and our difference was bigger than what chance can do, we
could be confident that chance did not produce our difference.

Let's think about our measure of green balls in terms of a difference.
For example, in each experiment we counted the green balls for the left
and right hand. What we really want to know is if there is a difference
between them. So, we can calculate the \textbf{difference score}. Let's
decide that the difference score = \# of green gumballs in left hand -
\# of green gumballs in right hand. Figure~\ref{fig-5gumballdiffs}
redraws the 10 bar graphs from above; however, now there is only one bar
for each experiment. This bar represents the difference in number of
green gumballs drawn by the left and right hand.

\begin{figure}

\centering{

\includegraphics[width=0.75\linewidth,height=\textheight,keepaspectratio]{05-Foundation_Inference_files/figure-pdf/fig-5gumballdiffs-1.pdf}

}

\caption{\label{fig-5gumballdiffs}A look at the differences between
number of each kind of gumball for the different replications. The
difference should be zero, but sampling error produces non-zero
differences.}

\end{figure}%

Missing bars mean that there were an equal number of green gumballs
chosen by the left and right hands (difference score is 0). A positive
value means that more green gumballs were chosen by the left than right
hand. A negative value means that more green gumballs were chosen by the
right than left hand. Note that if we decided (and we get to decide) to
calculate the difference in reverse (right hand - left hand), the signs
of the differences scores would flip around.

We are starting to see more of the differences that chance can produce.
The difference scores are mostly between -2 to +2. We could get an even
better impression by running this pretend experiment 100 times instead
of only 10 times. The results are shown in Figure~\ref{fig-5manydiffs}.

\begin{figure}

\centering{

\includegraphics[width=0.75\linewidth,height=\textheight,keepaspectratio]{05-Foundation_Inference_files/figure-pdf/fig-5manydiffs-1.pdf}

}

\caption{\label{fig-5manydiffs}Replicating the experiment 100 times, and
looking at the differences each time. There are mnay kinds of
differences that chance alone can produce.}

\end{figure}%

Ooph, we just ran so many simulated experiments that the x-axis is
unreadable, but it goes from 1 to 100. Each bar represents the
difference of number of green balls chosen randomly by the left or right
hand. Beginning to notice anything? Look at the y-axis, this shows the
size of the difference. Yes, there are lots of bars of different sizes,
this shows us that many kinds of differences do occur by chance.
However, the y-axis is also restricted. It does not go from -10 to +10.
Big differences greater than 5 or -5 don't happen very often.

Now that we have a method for simulating differences due to chance,
let's run 10,000 simulated experiments. But, instead of plotting the
differences in a bar graph for each experiment, how about we look at the
histogram of difference scores. The histogram in
Figure~\ref{fig-5histdiffgumball} provides a clearer picture about which
differences happen most often, and which ones do not. This will be
another window into observing what kinds of differences chance is
capable of producing.

\begin{figure}

\centering{

\includegraphics[width=0.75\linewidth,height=\textheight,keepaspectratio]{05-Foundation_Inference_files/figure-pdf/fig-5histdiffgumball-1.pdf}

}

\caption{\label{fig-5histdiffgumball}A histogram of the differences
obtained by chance over 10,000 replications. The most frequency
difference is 0, which is what we expect by chance. But the differences
can be as large as -10 or +10. Larger differences occur less often by
chance. Chance can't do everything.}

\end{figure}%

Our computer simulation allows us to force chance to operate hundreds of
times, each time it produces a difference. We record the difference,
then at the end of the simulation we plot the histogram of the
differences. The histogram begins to show us the where the differences
came from. Remember the idea that numbers come from a distribution, and
the distribution says how often each number occurs. We are looking at
one of these distributions. It is showing us that chance produces some
differences more often than others. First, chance usually produces 0
differences, that's the biggest bar in the middle. Chance also produces
larger differences, but as the differences get larger (positive or
negative), they occur less frequently. The shape of this histogram is
your chance window, it tells you what chance can do, it tells you what
chance usually does, and what it usually does not do.

You can use this chance window to help you make inferences. If you ran
yourself in the gumball experiment and found that your left hand chose 2
more green gumballs than red gumballs, would you conclude that you left
hand was special, and caused you to choose more green gumballs?
Hopefully not. You could look at the chance window and see that
differences of size +2 do happen fairly often by chance alone. You
should not be surprised if you got a +2 difference. However, what if
your left chose 5 more green gumballs than red gumballs. Well, chance
doesn't do this very often, you might think something is up with your
left hand. If you got a whopping 9 more green gumballs than red
gumballs, you might really start to wonder. This is the kind of thing
that could happen (it's possible), but virtually never happens by
chance. When you get things that almost never happen by chance, you can
be more confident that the difference reflects a causal force that is
not chance.

\section{The Crump Test}\label{the-crump-test}

We are going to be doing a lot of inference throughout the rest of this
course. Pretty much all of it will come down to one question. Did chance
produce the differences in my data? We will be talking about experiments
mostly, and in experiments we want to know if our manipulation caused a
difference in our measurement. But, we measure things that have natural
variability, so every time we measure things we will always find a
difference. We want to know if the difference we found (between our
experimental conditions) could have been produced by chance. If chance
is a very unlikely explanation of our observed difference, we will make
the inference that chance did not produce the difference, and that
something about our experimental manipulation did produce the
difference. This is it (for this textbook).

\begin{tcolorbox}[enhanced jigsaw, title=\textcolor{quarto-callout-note-color}{\faInfo}\hspace{0.5em}{Note}, colframe=quarto-callout-note-color-frame, colbacktitle=quarto-callout-note-color!10!white, bottomtitle=1mm, leftrule=.75mm, rightrule=.15mm, titlerule=0mm, arc=.35mm, colback=white, opacitybacktitle=0.6, toprule=.15mm, toptitle=1mm, bottomrule=.15mm, coltitle=black, breakable, left=2mm, opacityback=0]

Statistics is not only about determining whether chance could have
produced a pattern in the observed data. The same tools we are talking
about here can be generalized to ask whether any kind of distribution
could have produced the differences. This allows comparisons between
different models of the data, to see which one was the most likely,
rather than just rejecting the unlikely ones (e.g., chance). But, we'll
leave those advanced topics for another textbook.

\end{tcolorbox}

This chapter is about building intuitions for making these kinds of
inferences about the role of chance in your data. It's not clear to me
what are the best things to say, to build up your intuitions for how to
do statistical inference. So, this chapter tries different things, some
of them standard, and some of them made up. What you are about to read,
is a made up way of doing statistical inference, without using the
jargon that we normally use to talk about it. The goal is to do things
without formulas, and without probabilities, and just work with some
ideas using simulations to see what happens. We will look at what chance
can do, then we will talk about what needs to happen in your data in
order for you to be confident that chance didn't do it.

\subsection{Intuitive methods}\label{intuitive-methods}

Warning, this is an unofficial statistical test made up by Matt Crump.
It makes sense to him (me), and if it turns out someone else already
made this up, then Crump didn't do his homework, and we will change the
name of this test to it's original author later on. The point of this
test is to show how simple arithmetic operations that you already
understand can be used to create a statistic tool for inference. This
test uses:

\begin{enumerate}
\def\labelenumi{\arabic{enumi}.}
\tightlist
\item
  Sampling numbers randomly from a distribution
\item
  Adding and subtracting
\item
  Division, to find the mean
\item
  Counting
\item
  Graphing and drawing lines
\item
  NO FORMULAS
\end{enumerate}

\subsection{Part 1: Frequency based intuition about
occurrence}\label{part-1-frequency-based-intuition-about-occurrence}

\textbf{Question}: How many times does something need to happen for it
to happen a lot? Or, how many times does something need to happen for it
to happen not very much, or even really not at all? Small enough for you
to not worry about it at all happening to you?

Would you go outside everyday if you thought that you would get hit by
lightning 1 out of 10 times? I wouldn't. You'd probably be hit by
lightning more than once per month, you'd be dead pretty quickly. 1 out
of 10 is a lot (to me, maybe not to you, there's no right answer here).

Would you go outside everyday if you thought that you would get hit by
lightning 1 out of every 100 days? Jeez, that's a tough one. What would
I even do? If I went out everyday I'd probably be dead in a year! Maybe
I would go out 2 or 3 times per year, I'm risky like that, but I'd
probably live longer if I stayed at home forever. It would massively
suck.

Would you go outside everyday if you thought you would get hit by
lightning 1 out of every 1000 days? Well, you'd probably be dead in 3-6
years if you did that. Are you a gambler? Maybe go out once per month,
still sucks.

Would you go outside everyday if you thought lightning would get you 1
out every 10,000 days? 10,000 is a bigger number, harder to think about.
It translates to getting hit about once every 27 years. Ya, I'd probably
go out 150 days per year, and keep my fingers crossed.

Would you go outside everyday if you thought lightning would get you 1
out every 100,000 days? How many years is that? It's about 273 years.
With those odds, I'd probably go out all the time and forget about being
hit by lightning. It doesn't happen very often.

The point of considering these questions is to get a sense for yourself
of what happens a lot, and what doesn't happen a lot, and how you would
make important decisions based on what happens a lot and what doesn't.

\subsection{Part 2: Simulating chance}\label{part-2-simulating-chance}

This next part could happen a bunch of ways, I'll make loads of
assumptions that I won't defend, and I won't claim the Crump test has
problems. I will claim it helps us make an inference about whether
chance could have produced some differences in data. We've already been
introduced to simulating things, so we'll do that again. Here is what we
will do. I am a cognitive psychologist who happens to be measuring X.
Because of prior research in the field, I know that when I measure X, my
samples will tend to have a particular mean and standard deviation.
Let's say the mean is usually 100, and the standard deviation is usually
15. In this case, I don't care about using these numbers as estimates of
the population parameters, I'm just thinking about what my samples
usually look like. What I want to know is how they behave when I sample
them. I want to see what kind of samples happen a lot, and what kind of
samples don't happen a lot. Now, I also live in the real world, and in
the real world when I run experiments to see what changes X, I usually
only have access to some number of participants, who I am very grateful
too, because they participate in my experiments. Let's say I usually can
run 20 subjects in each condition in my experiments. Let's keep the
experiment simple, with two conditions, so I will need 40 total
subjects.

I would like to learn something to help me with inference. One thing I
would like to learn is what the sampling distribution of the sample mean
looks like. This distribution tells me what kinds of mean values happen
a lot, and what kinds don't happen very often. But, I'm actually going
to skip that bit. Because what I'm really interested in is what the
\textbf{sampling distribution of the difference between my sample means}
looks like. After all, I am going to run an experiment with 20 people in
one condition, and 20 people in the other. Then I am going to calculate
the mean for group A, and the mean for group B, and I'm going to look a
the difference. I will probably find a difference, but my question is,
did my manipulation cause this difference, or is this the kind of thing
that happens a lot by chance. If I knew what chance can do, and how
often it produces differences of particular sizes, I could look at the
difference I observed, then look at what chance can do, and then I can
make a decision! If my difference doesn't happen a lot (we'll get to how
much not a lot is in a bit), then I might be willing to believe that my
manipulation caused a difference. If my difference happens all the time
by chance alone, then I wouldn't be inclined to think my manipulation
caused the difference, because it could have been chance.

So, here's what we'll do, even before running the experiment. We'll do a
simulation. We will sample numbers for group A and Group B, then compute
the means for group A and group B, then we will find the difference in
the means between group A and group B. But, we will do one very
important thing. We will pretend that we haven't actually done a
manipulation. If we do this (do nothing, no manipulation that could
cause a difference), then we know that \textbf{only sampling error}
could cause any differences between the mean of group A and group B.
We've eliminated all other causes, only chance is left. By doing this,
we will be able to see exactly what chance can do. More importantly, we
will see the kinds of differences that occur a lot, and the kinds that
don't occur a lot.

Before we do the simulation, we need to answer one question. How much is
a lot? We could pick any number for a lot. I'm going to pick 10,000.
That is a lot. If something happens only 1 times out 10,000, I am
willing to say that is not a lot.

OK, now we have our number, we are going to simulate the possible mean
differences between group A and group B that could arise by chance. We
do this 10,000 times. This gives chance a lot of opportunity to show us
what it does do, and what it does not do.

This is what I did: I sampled 20 numbers into group A, and 20 into group
B. The numbers both came from the same normal distribution, with mean =
100, and standard deviation = 15. Because the samples are coming from
the same distribution, we expect that on average they will be similar
(but we already know that samples differ from one another). Then, I
compute the mean for each sample, and compute the difference between the
means. I save the \textbf{mean difference score}, and end up with 10,000
of them. Then, I draw the histogram in Figure~\ref{fig-5crumptestdiff}.

\begin{figure}

\centering{

\includegraphics[width=0.75\linewidth,height=\textheight,keepaspectratio]{05-Foundation_Inference_files/figure-pdf/fig-5crumptestdiff-1.pdf}

}

\caption{\label{fig-5crumptestdiff}Histogram of mean differences arising
by chance.}

\end{figure}%

\begin{tcolorbox}[enhanced jigsaw, title=\textcolor{quarto-callout-note-color}{\faInfo}\hspace{0.5em}{Note}, colframe=quarto-callout-note-color-frame, colbacktitle=quarto-callout-note-color!10!white, bottomtitle=1mm, leftrule=.75mm, rightrule=.15mm, titlerule=0mm, arc=.35mm, colback=white, opacitybacktitle=0.6, toprule=.15mm, toptitle=1mm, bottomrule=.15mm, coltitle=black, breakable, left=2mm, opacityback=0]

Of course, we might recognize that chance could do a difference greater
than 15. We just didn't give it the opportunity. We only ran the
simulation 10,000 times. If we ran it a million times, maybe a
difference greater than 15 or even 20 would happen a couple times. If we
ran it a bazillion gazillion times, maybe a difference greater than 30
would happen a couple times. If we go out to infinity, then chance might
produce all sorts of bigger differences once in a while. But, we've
already decided that 1/10,000 is not a lot. So things that happen 0 out
of 10,000 times, like differences greater than 15, are considered to be
extremely unlikely.

\end{tcolorbox}

Now we can see what chance can do to the size of our mean difference.
The x-axis shows the size of the mean difference. We took our samples
from the sample distribution, so the difference between them should
usually be 0, and that's what we see in the histogram.

Pause for a second. Why should the mean differences usually be zero,
wasn't the population mean = 100, shouldn't they be around 100? No.~The
mean of group A will tend to be around 100, and the mean of group B will
tend be around 100. So, the difference score will tend to be 100-100 =
0. That is why we expect a mean difference of zero when the samples are
drawn from the same population.

So, differences near zero happen the most, that's good, that's what we
expect. Bigger or smaller differences happen increasingly less often.
Differences greater than 15 or -15 never happen at all. For our
purposes, it looks like chance only produces differences between -15 to
15.

OK, let's ask a couple simple questions. What was the biggest negative
number that occurred in the simulation? We'll use R for this. All of the
10,000 difference scores are stored in a variable I made called
\texttt{difference}. If we want to find the minimum value, we use the
\texttt{min} function. Here's the result.

\begin{Shaded}
\begin{Highlighting}[]
\FunctionTok{min}\NormalTok{(difference)}
\CommentTok{\#\textgreater{} [1] {-}17.57537}
\end{Highlighting}
\end{Shaded}

OK, so what was the biggest positive number that occurred? Let's use the
\texttt{max} function to find out. It finds the biggest (maximum) value
in the variable. FYI, we've just computed the range, the minimum and
maximum numbers in the data. Remember we learned that before. Anyway,
here's the max.

\begin{Shaded}
\begin{Highlighting}[]
\FunctionTok{max}\NormalTok{(difference)}
\CommentTok{\#\textgreater{} [1] 16.90231}
\end{Highlighting}
\end{Shaded}

Both of these extreme values only occurred once. Those values were so
rare we couldn't even see them on the histogram, the bar was so small.
Also, these biggest negative and positive numbers are pretty much the
same size if you ignore their sign, which makes sense because the
distribution looks roughly symmetrical.

So, what can we say about these two numbers for the min and max? We can
say the min happens 1 times out of 10,000. We can say the max happens 1
times out of 10,000. Is that a lot of times? Not to me. It's not a lot.

So, how often does a difference of 30 (much larger larger than the max)
occur out of 10,000. We really can't say, 30s didn't occur in the
simulation. Going with what we got, we say 0 out of 10,000. That's
never.

We're about to move into part three, which involves drawing decision
lines and talking about them. The really important part about part 3 is
this. What would you say if you ran this experiment once, and found a
mean difference of 30? I would say it happens 0 times of out 10,000 by
chance. I would say chance did not produce my difference of 30. That's
what I would say. We're going to expand upon this right now.

\subsection{Part 3: Judgment and
Decision-making}\label{part-3-judgment-and-decision-making}

Remember, we haven't even conducted an experiment. We're just simulating
what could happen if we did conduct an experiment. We made a histogram.
We can see that chance produces some differences more than others, and
that chance never produced really big differences. What should we do
with this information?

What we are going to do is talk about judgment and decision making. What
kind of judgment and decision making? Well, when you finally do run an
experiment, you will get two means for group A and B, and then you will
need to make some judgments, and perhaps even a decision, if you are so
inclined. You will need to judge whether chance (sampling error) could
have produced the difference you observed. If you judge that it did it
not, you might make the decision to tell people that your experimental
manipulation actually works. If you judge that it could have been
chance, you might make a different decision. These are important
decisions for researchers. Their careers can depend on them. Also, their
decisions matter for the public. Nobody wants to hear fake news from the
media about scientific findings.

So, what we are doing is preparing to make those judgments. We are going
to draw up a plan, before we even see the data, for how we will make
judgments and decisions about what we find. This kind of planning is
extremely important, because we discuss in part 4, that your planning
can help you design an even better experiment than the one you might
have been intending to run. This kind of planning can also be used to
interpret other people's results, as a way of double-checking checking
whether you believe those results are plausible.

The thing about judgement and decision making is that reasonable people
disagree about how to do it, unreasonable people really disagree about
it, and statisticians and researchers disagree about how to do it. I
will propose some things that people will disagree with. That's OK,
these things still make sense. And, the disagreeable things point to
important problems that are very real for any ``real'' statistical
inference test.

Let's talk about some objective facts from our simulation of 10,000
things that we definitely know to be true. For example, we can draw some
lines on the graph, and label some different regions. We'll talk about
two kinds of regions.

\begin{enumerate}
\def\labelenumi{\arabic{enumi}.}
\tightlist
\item
  Region of chance. Chance did it. Chance could have done it
\item
  Region of not chance. Chance didn't do it. Chance couldn't have done
  it.
\end{enumerate}

The regions are defined by the minimum value and the maximum value.
Chance never produced a smaller or bigger number. The region inside the
range is what chance did do, and the the region outside the range on
both sides is what chance never did. It looks like
Figure~\ref{fig-5crumpdecision}:

\begin{figure}

\centering{

\includegraphics[width=0.75\linewidth,height=\textheight,keepaspectratio]{05-Foundation_Inference_files/figure-pdf/fig-5crumpdecision-1.pdf}

}

\caption{\label{fig-5crumpdecision}Applying decision boundaries to the
histogrm of mean differences. The boundaries identify what differences
chance did or did not produce in the simulation.}

\end{figure}%

We have just drawn some lines, and shaded some regions, and made one
plan we could use to make decisions. How would the decisions work. Let's
say you ran the experiment and found a mean difference between groups A
and B of 25. Where is 25 in the figure? It's in the green part. What
does the green part say? NOT CHANCE. What does this mean. It means
chance never made a difference of 25. It did that 0 out of 10,000 times.
If we found a difference of 25, perhaps we could confidently conclude
that chance did not cause the difference. If I found a difference of 25
with this kind of data, I'd be pretty confident chance did not cause the
difference; and, I would give myself license to consider that my
experimental manipulation may be causing the difference.

What about a difference of +10? That's in the red part, where chance
lives. Chance could have done a difference of +10 because we can see
that it did do that sometimes. The red part is the window of what chance
did in our simulation. Anything inside the window could have been a
difference caused by chance. If I found a difference of +10, I'd say,
``ya, it coulda been chance.'' I would also be less confident that the
difference was only caused by my experimental manipulation.

Statistical inference could be this easy. The number you get from your
experiment could be in the chance window (then you can't rule out chance
as a cause), or it could be outside the chance window (then you can rule
out chance). Case closed. Let's all go home.

\subsubsection{Grey areas}\label{grey-areas}

So what's the problem? Depending on who you are, and what kinds of risks
you're willing to take, there might not be a problem. But, if you are
just even a little bit risky then there is a problem that makes clear
judgments about the role of chance difficult. We would like to say
chance did or did not cause our difference. But, we're really always in
the position of admitting that it could have sometimes, or wouldn't have
most times. These are wishy washy statements, they are in between yes or
no. That's OK. Grey is a color too, let's give grey some respect.

``What grey areas are you talking about?, I only see red or green, am I
grey blind?''. Let's look at where some grey areas might be. I say might
be, because people disagree about where the grey is. People have
different comfort levels with grey. Figure~\ref{fig-5crumpuncertainty}
shows my opinion on grey areas.

\begin{figure}

\centering{

\includegraphics[width=0.75\linewidth,height=\textheight,keepaspectratio]{05-Foundation_Inference_files/figure-pdf/fig-5crumpuncertainty-1.pdf}

}

\caption{\label{fig-5crumpuncertainty}The question marks refer to an
area where you have some uncertainty. Differences inside the question
mark region do not happen very often by chance. When you find
differences of these sizes, should you reject the idea that chance
caused your difference? You will always have some uncertainty associated
with this decision because it is clear that chance could have caused the
difference. But, chance usually does not produce differences of these
sizes.}

\end{figure}%

I made two grey areas, and they are reddish grey, because we are still
in the chance window. There are question marks (?) in the grey areas.
Why? The question marks reflect some uncertainty that we have about
those particular differences. For example, if you found a difference
that was in a grey area, say a 15. 15 is less than the maximum, which
means chance did create differences of around 15. But, differences of 15
don't happen very often.

What can you conclude or say about this 15 you found? Can you say
without a doubt that chance did not produce the difference? Of course
not, you know that chance could have. Still, it's one of those things
that doesn't happen a lot. That makes chance an unlikely explanation.
Instead of thinking that chance did it, you might be willing to take a
risk and say that your experimental manipulation caused the difference.
You'd be making a bet that it wasn't chance\ldots but, could be a safe
bet, since you know the odds are in your favor.

You might be thinking that your grey areas aren't the same as the ones
I've drawn. Maybe you want to be more conservative, and make them
smaller. Or, maybe you're more risky, and would make them bigger. Or,
maybe you'd add some grey area going in a little bit to the green area
(after all, chance could probably produce some bigger differences
sometimes, and to avoid those you would have to make the grey area go a
bit into the green area).

Another thing to think about is your decision policy. What will you do,
when your observed difference is in your grey area? Will you always make
the same decision about the role of chance? Or, will you sometimes
flip-flop depending on how you feel. Perhaps, you think that there
shouldn't be a strict policy, and that you should accept some level of
uncertainty. The difference you found could be a real one, or it might
not. There's uncertainty, hard to avoid that.

So let's illustrate one more kind of strategy for making decisions. We
just talked about one that had some lines, and some regions. This makes
it seem like a binary choice: we can either rule out, or not rule out
the role of chance. Another perspective is that everything is a
different shade of grey, like in Figure~\ref{fig-5crumpshade}.

\begin{figure}

\centering{

\includegraphics[width=0.75\linewidth,height=\textheight,keepaspectratio]{05-Foundation_Inference_files/figure-pdf/fig-5crumpshade-1.pdf}

}

\caption{\label{fig-5crumpshade}The shading of the blue bars indicates
levels of confidence in whether a difference could have been produced by
chance. Darker bars represent increased confidence that the difference
was not produced by chance. Bars get darker as the mean difference
increases in absolute value.}

\end{figure}%

OK, so I made it shades of blue (because it was easier in R). Now we can
see two decision plans at the same time. Notice that as the bars get
shorter, they also get become a darker stronger blue. The color can be
used as a guide for your confidence. That is, your confidence in the
belief that your manipulation caused the difference rather than chance.
If you found a difference near a really dark bar, those don't happen
often by chance, so you might be really confident that chance didn't do
it. If you find a difference near a slightly lighter blue bar, you might
be slightly less confident. That is all. You run your experiment, you
get your data, then you have some amount of confidence that it wasn't
produced by chance. This way of thinking is elaborated to very
interesting degrees in the Bayesian world of statistics. We don't wade
too much into that, but mention it a little bit here and there. It's
worth knowing it's out there.

\subsubsection{Making decisions and being
wrong}\label{making-decisions-and-being-wrong}

No matter how you plan to make decisions about your data, you will
always be prone to making some mistakes. You might call one finding
real, when in fact it was caused by chance. This is called a
\textbf{type I} error, or a false positive. You might ignore one
finding, calling it chance, when in fact it wasn't chance (even though
it was in the window). This is called a \textbf{type II} error, or a
false negative.

How you make decisions can influence how often you make errors over
time. If you are a researcher, you will run lots of experiments, and you
will make some amount of mistakes over time. If you do something like
the very strict method of only accepting results as real when they are
in the ``no chance'' zone, then you won't make many type I errors.
Pretty much all of your result will be real. But, you'll also make type
II errors, because you will miss things real things that your decision
criteria says are due to chance. The opposite also holds. If you are
willing to be more liberal, and accept results in the grey as real, then
you will make more type I errors, but you won't make as many type II
errors. Under the decision strategy of using these cutoff regions for
decision-making there is a necessary trade-off. The Bayesian view get's
around this a little bit. Bayesians talk about updating their beliefs
and confidence over time. In that view, all you ever have is some level
of confidence about whether something is real, and by running more
experiments you can increase or decrease your level of confidence. This,
in some fashion, avoids some trade-off between type I and type II
errors.

Regardless, there is another way to reduce type I and type II errors,
and to increase your confidence in your results, even before you do the
experiment. It's called ``knowing how to design a good experiment''.

\subsection{Part 4: Experiment Design}\label{part-4-experiment-design}

We've seen what chance can do. Now, let's venture into an experiment. We
make a change between ecosystems A and B, gather the data, assess the
average outcomes, and then observe the variance. Then we keep our
fingers crossed, hoping that the variance is significant enough to be
beyond natural fluctuations. Yes, nature keeps us guessing.

Here's the catch, we aren't always certain about the magnitude of our
environmental interventions. So, even if an intervention induces a
change, pinning down its exact magnitude can be challenging. And that's
the essence of our experiment. Many interventions in Environmental
Science might not cause large-scale shifts. This poses a challenge in
identifying these subtle, yet potentially crucial, environmental
effects. In a hypothetical scenario, introducing a certain pollinator
species might influence plant growth, but to what extent? If the
difference is marginal, differentiating between natural variation and
the effect of our intervention becomes tricky. Let's say our
intervention involves introducing shade in one ecosystem versus none in
the other. While shade can influence plant growth, if the effect is only
marginal, it becomes hard to ascertain if it wasn't just a natural
occurrence. And, it's not straightforward to intensify the shading to
amplify its impact, without risking other unintended consequences.

EXPERIMENT DESIGN TO THE RESCUE! Newsflash, it is often possible to
change how you run your experiment so that it is \textbf{more sensitive}
to smaller effects. How do you think we can do this? Here is a hint.
It's the stuff you learned about the sampling distribution of the sample
mean, and the role of sample-size. What happens to the sampling
distribution of the sample mean when N (sample size)? The distribution
gets narrower and narrower, and starts to look the a single number (the
hypothetical mean of the hypothetical population). That's great. If you
switch to thinking about mean difference scores, like the distribution
we created in this test, what do you think will happen to that
distribution as we increase N? It will will also shrink. As we increase
N to infinity, it will shrink to 0. Which means that, when N is
infinity, chance never produces any differences at all. We can use this.

For example, we could run our experiment with 20 subjects in each group.
Or, we could decide to invest more time and run 40 subjects in each
group, or 80, or 150. When you are the experimenter, you get to decide
the design. These decisions matter big time. Basically, the more
subjects you have, the more sensitive your experiment. With bigger N,
you will be able to reliably detect smaller mean differences, and be
able to confidently conclude that chance did not produce those small
effects.

Check out the histograms in Figure~\ref{fig-5sampleDistNormal}. This is
the same simulation as before, but with four different sample-sizes: 20,
40, 80, 160. We are doubling our sample-size across each simulation just
to see what happens to the width of the chance window.

\begin{figure}

\centering{

\includegraphics[width=0.75\linewidth,height=\textheight,keepaspectratio]{05-Foundation_Inference_files/figure-pdf/fig-5sampleDistNormal-1.pdf}

}

\caption{\label{fig-5sampleDistNormal}The range or width of the
differences produced by chance shrinks as sample-size increases.}

\end{figure}%

There you have it. The \textbf{sampling distribution of the mean
differences} shrinks toward 0 as sample-size increases. This means if
you run an experiment with a larger sample-size, you will be able to
detect smaller mean differences, and be confident they aren't due to
chance. Table~\ref{tbl-5minmax} contains the minimum and maximum values
that chance produced across the four sample-sizes:

\begin{longtable}[]{@{}rrr@{}}

\caption{\label{tbl-5minmax}The smallest and largest mean differences
produced by chance as a function of sample-size.}

\tabularnewline

\toprule\noalign{}
sample\_size & smallest & largest \\
\midrule\noalign{}
\endhead
\bottomrule\noalign{}
\endlastfoot
20 & -19.64839 & 26.03227 \\
40 & -16.30097 & 14.89490 \\
80 & -12.49659 & 11.94689 \\
160 & -10.38068 & 13.25971 \\

\end{longtable}

The table shows the range of chance behavior is very wider for smaller N
and narrower for larger N. Consider what this narrowing means for your
experiment design. For example, one aspect of the design is the choice
of sample size, N, or in a psychology experiment the number of
participants.

If it turns out your manipulation will cause a difference of +11, then
what should you do? Run an experiment with N = 20 people? I hope not. If
you did that, you could get a mean difference of +11 fairly often by
chance. However, if you ran the experiment with 160 people, then you
would definitely be able to say that +11 was not due to chance, it would
be outside the range of what chance can do. You could even consider
running the experiment with 80 subjects. A +11 there wouldn't happen
often by chance, and you'd be cost-effective, spending less time on the
experiment.

The point is: \textbf{the design of the experiment determines the sizes
of the effects it can detect}. If you want to detect a small effect.
Make your sample size bigger. It's really important to say this is not
the only thing you can do. You can also make your cell-sizes bigger. For
example, often times we take several measurements from a single subject.
The more measurements you take (cell-size), the more stable your
estimate of the subject's mean. We discuss these issues more later. You
can also make a stronger manipulation, when possible.

\subsection{Part 5: I have the power}\label{part-5-i-have-the-power}

\begin{quote}
By the power of greyskull, I HAVE THE POWER - He-man
\end{quote}

The last topic in this section is called \textbf{power}. Later we will
define power in terms of some particular ideas about statistical
inference. Here, we will just talk about the big idea. And, we'll show
how to make sure your design has 100\% power. Because, why not. Why run
a design that doesn't have the power?

The big idea behind power is the concept of sensitivity. The concept of
sensitivity assumes that there is something to be sensitive to. That is,
there is some real difference that can be measured. So, the question is,
how sensitive is your experiment? We've already seen that the number of
subjects (sample-size), changes the sensitivity of the design. More
subjects = more sensitivity to smaller effects.

Let's take a look at one more plot. What we will do is simulate a
measure of sensitivity across a whole bunch of sample sizes, from 10 to
300. We'll do this in steps of 10. For each simulation, we'll compute
the mean differences as we have done. But, rather than showing the
histogram, we'll just compute the smallest value and the largest value.
This is a pretty good measure of the outer reach of chance. Then we'll
plot those values as a function of sample size and see what we've got.

\begin{figure}

\centering{

\includegraphics[width=0.75\linewidth,height=\textheight,keepaspectratio]{05-Foundation_Inference_files/figure-pdf/fig-5crumpminmax-1.pdf}

}

\caption{\label{fig-5crumpminmax}A graph of the maximum and minimum mean
differences produced by chance as a function of sample-size. The range
narrows as sample-size increases showing that chance alone produces a
smaller range of mean differences as sample-size increases.}

\end{figure}%

Figure~\ref{fig-5crumpminmax} shows a reasonably precise window of
sensitivity as a function of sample size. For each sample size, we can
see the maximum difference that chance produced and the minimum
difference. In those simulations, chance never produced bigger or
smaller differences. So, each design is sensitive to any difference that
is underneath the bottom line, or above the top line.

Here's another way of putting it. Which of the sample sizes will be
sensitive to a difference of +10 or -10. That is, if a difference of +10
or -10 was observed, then we could very confidently say that the
difference was not due to chance, because according to these
simulations, chance never produced differences that big. To help us see
which ones are sensitive, Figure~\ref{fig-5crumpredline} draws
horizontal lines at -10 and +10.

\begin{figure}

\centering{

\includegraphics[width=0.75\linewidth,height=\textheight,keepaspectratio]{05-Foundation_Inference_files/figure-pdf/fig-5crumpredline-1.pdf}

}

\caption{\label{fig-5crumpredline}The red line represents the size of a
mean difference that a researcher may be interested in detecting. All of
the dots outside (above or below) the red line represent designs with
small sample-sizes. When a difference of 10 occurs for these designs, we
can rule out chance with confidence. The dots between the red lines
represent designs with larger sample-sizes. These designs never produce
differences as large as 10, so when those differences occur, we can be
confident chance did not produce them.}

\end{figure}%

Based on visual guesstimation, the designs with sample-size
\textgreater= 100 are all sensitive to real differences of 10. Designs
with sample-size \textgreater{} 100 all failed to produce extreme
differences outside of the red lines by chance alone. If these designs
were used, and if an effect of 10 or larger was observed, then we could
be confident that chance alone did not produce the effect. Designing
your experiment so that you know it is sensitive to the thing you are
looking for is the big idea behind power.

\subsection{Summary of Crump Test}\label{summary-of-crump-test}

What did we learn from this so-called fake Crump test that nobody uses?
Well, we learned the basics of what we'll be doing moving forward. And,
we did it all without any hard math or formulas. We sampled numbers, we
computed means, we subtracted means between experimental conditions,
then we repeated that process many times and counted up the mean
differences and put them in a histogram. This showed us what chance do
in an experiment. Then, we discussed how to make decisions around these
facts. And, we showed how we can control the role of chance just by
changing things like sample size.

\section{The randomization test (permutation
test)}\label{the-randomization-test-permutation-test}

Welcome to the first official inferential statistic in this textbook. Up
till now we have been building some intuitions for you. Next, we will
get slightly more formal and show you how we can use random chance to
tell us whether our experimental finding was likely due to chance or
not. We do this with something called a randomization test. The ideas
behind the randomization test are the very same ideas behind the rest of
the inferential statistics that we will talk about in later chapters.
And, surprise, we have already talked about all of the major ideas
already. Now, we will just put the ideas together, and give them the
name \textbf{randomization test}.

Here's the big idea. When you run an experiment and collect some data
you get to find out what happened that one time. But, because you ran
the experiment only once, you don't get to find out what \textbf{could
have happened}. The randomization test is a way of finding out what
\textbf{could have happened}. And, once you know that, you can compare
\textbf{what did happen} in your experiment, with \textbf{what could
have happened}.

\subsection{Pretend example does chewing gum improve your
grades?}\label{pretend-example-does-chewing-gum-improve-your-grades}

Let's say you run an experiment to find out if chewing gum causes
students to get better grades on statistics exams. You randomly assign
20 students to the chewing gum condition, and 20 different students to
the no-chewing gum condition. Then, you give everybody statistics tests
and measure their grades. If chewing gum causes better grades, then the
chewing gum group should have higher grades on average than the group
who did not chew gum.

Let's say the data looked like this:

\begin{longtable}[]{@{}lll@{}}
\toprule\noalign{}
student & gum & no\_gum \\
\midrule\noalign{}
\endhead
\bottomrule\noalign{}
\endlastfoot
1 & 98 & 43 \\
2 & 82 & 63 \\
3 & 100 & 41 \\
4 & 99 & 44 \\
5 & 91 & 49 \\
6 & 100 & 62 \\
7 & 76 & 70 \\
8 & 82 & 74 \\
9 & 80 & 54 \\
10 & 87 & 65 \\
11 & 79 & 79 \\
12 & 98 & 43 \\
13 & 85 & 65 \\
14 & 86 & 67 \\
15 & 95 & 70 \\
16 & 82 & 87 \\
17 & 95 & 79 \\
18 & 90 & 78 \\
19 & 88 & 75 \\
20 & 87 & 45 \\
Sums & 1780 & 1253 \\
Means & 89 & 62.65 \\
\end{longtable}

So, did the students chewing gum do better than the students who didn't
chew gum? Look at the mean test performance at the bottom of the table.
The mean for students chewing gum was 89, and the mean for students who
did not chew gum was 62.65. Just looking at the means, it looks like
chewing gum worked!

``STOP THE PRESSES, this is silly''. We already know this is silly
because we are making pretend data. But, even if this was real data, you
might think, ``Chewing gum won't do anything, this difference could have
been caused by chance, I mean, maybe the better students just happened
to be put into the chewing group, so because of that their grades were
higher, chewing gum didn't do anything\ldots{}''. We agree. But, let's
take a closer look. We already know how the data come out. What we want
to know is how they could have come out, what are all the possibilities?

For example, the data would have come out a bit different if we happened
to have put some of the students from the gum group into the no gum
group, and vice versa. Think of all the ways you could have assigned the
40 students into two groups, there are lots of ways. And, the means for
each group would turn out differently depending on how the students are
assigned to each group.

Practically speaking, it's not possible to run the experiment every
possible way, that would take too long. But, we can nevertheless
estimate how all of those experiments might have turned out using
simulation.

Here's the idea. We will take the 40 measurements (exam scores) that we
found for all the students. Then we will randomly take 20 of them and
pretend they were in the gum group, and we'll take the remaining 20 and
pretend they were in the no gum group. Then we can compute the means
again to find out what would have happened. We can keep doing this over
and over again. Every time computing what happened in that version of
the experiment.

\subsubsection{Doing the randomization}\label{doing-the-randomization}

Before we do that, let's show how the randomization part works. We'll
use fewer numbers to make the process easier to look at. Here are the
first 5 exam scores for students in both groups.

\begin{longtable}[]{@{}lll@{}}
\toprule\noalign{}
student & gum & no\_gum \\
\midrule\noalign{}
\endhead
\bottomrule\noalign{}
\endlastfoot
1 & 98 & 43 \\
2 & 82 & 63 \\
3 & 100 & 41 \\
4 & 99 & 44 \\
5 & 91 & 49 \\
Sums & 470 & 240 \\
Means & 94 & 48 \\
\end{longtable}

Things could have turned out differently if some of the subjects in the
gum group were switched with the subjects in the no gum group. Here's
how we can do some random switching. We will do this using R.

\begin{Shaded}
\begin{Highlighting}[]
\NormalTok{all\_scores       }\OtherTok{\textless{}{-}} \FunctionTok{c}\NormalTok{(gum[}\DecValTok{1}\SpecialCharTok{:}\DecValTok{5}\NormalTok{],no\_gum[}\DecValTok{1}\SpecialCharTok{:}\DecValTok{5}\NormalTok{])}
\NormalTok{randomize\_scores }\OtherTok{\textless{}{-}} \FunctionTok{sample}\NormalTok{(all\_scores)}
\NormalTok{new\_gum          }\OtherTok{\textless{}{-}}\NormalTok{ randomize\_scores[}\DecValTok{1}\SpecialCharTok{:}\DecValTok{5}\NormalTok{]}
\NormalTok{new\_no\_gum       }\OtherTok{\textless{}{-}}\NormalTok{ randomize\_scores[}\DecValTok{6}\SpecialCharTok{:}\DecValTok{10}\NormalTok{]}
\FunctionTok{print}\NormalTok{(new\_gum)}
\CommentTok{\#\textgreater{} [1] 98 91 43 63 44}
\FunctionTok{print}\NormalTok{(new\_no\_gum)}
\CommentTok{\#\textgreater{} [1] 100  41  99  49  82}
\end{Highlighting}
\end{Shaded}

We have taken the first 5 numbers from the original data, and put them
all into a variable called \texttt{all\_scores}. Then we use the
\texttt{sample} function in R to shuffle the scores. Finally, we take
the first 5 scores from the shuffled numbers and put them into a new
variable called \texttt{new\_gum}. Then, we put the last five scores
into the variable \texttt{new\_no\_gum}. Then we printed them, so we can
see them.

If we do this a couple of times and put them in a table, we can indeed
see that the means for gum and no gum would be different if the subjects
were shuffled around. Check it out:

\begin{longtable}[]{@{}lllllll@{}}
\toprule\noalign{}
student & gum & no\_gum & gum2 & no\_gum2 & gum3 & no\_gum3 \\
\midrule\noalign{}
\endhead
\bottomrule\noalign{}
\endlastfoot
1 & 98 & 43 & 99 & 63 & 91 & 49 \\
2 & 82 & 63 & 41 & 98 & 98 & 63 \\
3 & 100 & 41 & 91 & 82 & 41 & 43 \\
4 & 99 & 44 & 49 & 43 & 100 & 82 \\
5 & 91 & 49 & 44 & 100 & 44 & 99 \\
Sums & 470 & 240 & 324 & 386 & 374 & 336 \\
Means & 94 & 48 & 64.8 & 77.2 & 74.8 & 67.2 \\
\end{longtable}

\subsubsection{Simulating the mean differences across the different
randomizations}\label{simulating-the-mean-differences-across-the-different-randomizations}

In our pretend experiment we found that the mean for students chewing
gum was 89, and the mean for students who did not chew gum was 62.65.
The mean difference (gum - no gum) was 26.35. This is a pretty big
difference. This is \textbf{what did happen}. But, \textbf{what could
have happened}? If we tried out all of the experiments where different
subjects were switched around, what does the distribution of the
possible mean differences look like? Let's find out. This is what the
randomization test is all about.

When we do our randomization test we will measure the mean difference in
exam scores between the gum group and the no gum group. Every time we
randomize we will save the mean difference.

Let's look at a short animation of what is happening in the
randomization test. \textbf{?@fig-5randtest} shows data from a different
fake experiment, but the principles are the same. We'll return to the
gum no gum experiment after the animation. The animation is showing
three important things. First, the purple dots show the mean scores in
two groups (didn't study vs study). It looks like there is a difference,
as 1 dot is lower than the other. We want to know if chance could
produce a difference this big. At the beginning of the animation, the
light green and red dots show the individual scores from each of 10
subjects in the design (the purple dots are the means of these original
scores). Now, during the randomizations, we randomly shuffle the
original scores between the groups. You can see this happening
throughout the animation, as the green and red dots appear in different
random combinations. The moving yellow dots show you the new means for
each group after the randomization. The differences between the yellow
dots show you the range of differences that chance could produce.

We are engaging in some visual statistical inference. By looking at the
range of motion of the yellow dots, we are watching what kind of
differences chance can produce. In this animation, the purple dots,
representing the original difference, are generally outside of the range
of chance. The yellow dots don't move past the purple dots, as a result
chance is an unlikely explanation of the difference.

If the purple dots were inside the range of the yellow dots, then when
would know that chance is capable of producing the difference we
observed, and that it does so fairly often. As a result, we should not
conclude the manipulation caused the difference, because it could have
easily occurred by chance.

Let's return to the gum example. After we randomize our scores many
times, and computed the new means, and the mean differences, we will
have loads of mean differences to look at, which we can plot in a
histogram. The histogram gives a picture of \textbf{what could have
happened}. Then, we can compare \textbf{what did happen} with
\textbf{what could have happened}.

Here's the histogram of the mean differences from the randomization
test. For this simulation, we randomized the results from the original
experiment 1000 times. This is what could have happened. The blue line
in Figure~\ref{fig-5randhist} shows where the observed difference lies
on the x-axis.

\begin{figure}

\centering{

\includegraphics[width=0.75\linewidth,height=\textheight,keepaspectratio]{05-Foundation_Inference_files/figure-pdf/fig-5randhist-1.pdf}

}

\caption{\label{fig-5randhist}A histogram of simulated mean differences
for a randomization test}

\end{figure}%

What do you think? Could the difference represented by the blue line
have been caused by chance? My answer is probably not. The histogram
shows us the window of chance. The blue line is not inside the window.
This means we can be pretty confident that the difference we observed
was not due to chance.

We are looking at another window of chance. We are seeing a histogram of
the kinds of mean differences that could have occurred in our
experiment, if we had assigned our subjects to the gum and no gum groups
differently. As you can see, the mean differences range from negative to
positive. The most frequent difference is 0. Also, the distribution
appears to be symmetrical about zero, which shows we had roughly same
the chances of getting a positive or negative difference. Also, notice
that as the differences get larger (in the positive or negative
direction, they become less frequent). The blue line shows us
\textbf{the observed difference}, this is the one we found in our fake
experiment. Where is it? It's way out to the right. It is is well
outside the histogram. In other words, when we look at \textbf{what
could have happened}, we see that \textbf{what did happen} doesn't occur
very often.

IMPORTANT: In this case, when we speak of \textbf{what could have
happened}. We are talking about what could have happened \textbf{by
chance}. When we compare what did happen to what chance could have done,
we can get a better idea of whether our result was caused by chance.

\begin{center}\rule{0.5\linewidth}{0.5pt}\end{center}

OK, let's pretend we got a much smaller mean difference when we first
ran the experiment. We can draw new lines (blue and red) to represent a
smaller mean that we might have found.

\begin{figure}

\centering{

\includegraphics[width=0.75\linewidth,height=\textheight,keepaspectratio]{05-Foundation_Inference_files/figure-pdf/fig-5randquestion-1.pdf}

}

\caption{\label{fig-5randquestion}Would you expect a mean difference
represented by the blue line to occur more or less often by chance
compared to the mean difference represented by the red line?}

\end{figure}%

Look at the blue line in Figure~\ref{fig-5randquestion}. If you found a
mean difference of 10, would you be convinced that your difference was
not caused by chance? As you can see, the blue line is inside the chance
window. Notably, differences of +10 don't very often. You might infer
that your difference was not likely to be due to chance (but you might
be a little bit skeptical, because it could have been). How about the
red line? The red line represents a difference of +5. If you found a
difference of +5 here, would you be confident that your difference was
not caused by chance? I wouldn't be. The red line is totally inside the
chance window, this kind of difference happens fairly often. I'd need
some more evidence to consider the claim the some independent variable
actually caused the difference. I'd be much more comfortable assuming
that sampling error probably caused the difference.

\subsection{Take homes so far}\label{take-homes-so-far}

Have you noticed that we haven't used any formulas yet, but we have been
able to accomplish inferential statistics. We will see some formulas as
we progress, but these aren't as the idea behind the formulas.

Inferential statistics is an attempt to solve the problem: \textbf{where
did my data from?}. In the randomization test example, our question was:
\textbf{where did the differences between the means in my data come
from?}. We know that the differences could be produced by chance alone.
We simulated what chance can due using randomization. Then we plotted
what chance can do using a histogram. Then, we used to picture to help
us make an inference. Did our observed difference come from the
distribution, or not? When the observed difference is clearly inside the
chance distribution, then we can infer that our difference \textbf{could
have been produced by chance}. When the observed difference is not
clearly inside the chance distribution, then we can infer that our
difference was \textbf{probably not produced by chance}.

In my opinion, these pictures are very, very helpful. If one of our
goals is to help ourselves summarize a bunch of complicated numbers to
arrive at an inference, then the pictures do a great job. We don't even
need a summary number, we just need to look at the picture and see if
the observed difference is inside or outside of the window. This is what
it is all about. Creating intuitive and meaningful ways to make
inferences from our data. As we move forward, the main thing that we
will do is formalize our process, and talk more about ``standard''
inferential statistics. For example, rather than looking at a picture
(which is a good thing to do), we will create some helpful numbers. For
example, what if you wanted to the probability that your difference
could have been produced by chance? That could be a single number, like
95\%. If there was a 95\% probability that chance can produce the
difference you observed, you might not be very confident that something
like your experimental manipulation was causing the difference. If there
was only 1\% probability that chance could produce your difference, then
you might be more confident that \textbf{chance did not} produce the
difference; and, you might instead be comfortable with the possibility
that your experimental manipulation actually caused the difference. So,
how can we arrive at those numbers? In order to get there we will
introduce you to some more foundational tools for statistical inference.

\section{Videos}\label{videos-3}

\subsection{Null and Alternate
Hypotheses}\label{null-and-alternate-hypotheses}

\subsection{Types of Errors}\label{types-of-errors}

\bookmarksetup{startatroot}

\chapter{Hypothesis Testing}\label{hypothesis-testing}

\section{Hypothesis Testing - The Nuts \&
Bolts}\label{hypothesis-testing---the-nuts-bolts}

Hypothesis testing helps us figure out if what we believe about a whole
group is likely true, just by looking at a small part of it (a sample).

\begin{center}\rule{0.5\linewidth}{0.5pt}\end{center}

\subsection{Clarifying Alpha, P-value, and Confidence
Level}\label{clarifying-alpha-p-value-and-confidence-level}

Before diving deep, let's clear up some terms you'll come across often.

\textbf{Alpha (}\(\alpha\))

Alpha (\(\alpha\)) is the significance level of a statistical test, and
it quantifies the risk of committing a Type I error. A Type I error
happens when we incorrectly reject a true null hypothesis. The standard
value for alpha is often set at 0.05, implying a 5\% chance of making a
Type I error. In other words, we are willing to accept a 5\% risk of
concluding that a difference exists when there is no actual difference.

\textbf{P-value}

The p-value is another crucial concept in hypothesis testing. It
represents the probability of observing the obtained results, or
something more extreme, assuming that the null hypothesis is true. A
small p-value (usually ≤ 0.05) suggests that the observed data is
inconsistent with the null hypothesis, and thus, you have evidence to
reject it.

\textbf{Confidence Level}

The confidence level is related but distinct from alpha and p-value.
While alpha quantifies the risk of a Type I error, the confidence level
indicates how confident we are in our statistical estimates. The
confidence level is calculated as the complement of alpha:

\[
\text{Confidence Level} = 1 - \alpha
\]

For example, if \(\alpha\) is 0.05, the confidence level would be (1 -
0.05 = 0.95) or 95\%. This means we are 95\% confident that our results
fall within a specific range.

\textbf{Bringing It All Together}

\begin{itemize}
\tightlist
\item
  \textbf{Alpha (}\(\alpha\)): Risk of Type I error (usually 5\%)
\item
  \textbf{P-value}: Probability of observed data given the null is true
\item
  \textbf{Confidence Level}: Confidence in the range of our estimates
  (usually 95\%)
\end{itemize}

Grasping how these three terms connect and differ is key to making sense
of the stats we'll discuss.

\begin{center}\rule{0.5\linewidth}{0.5pt}\end{center}

\subsection{The Steps of Hypothesis Testing Applied to an
Example}\label{the-steps-of-hypothesis-testing-applied-to-an-example}

Let's say we want to know if the average pollution in a set of water
samples is above the legal limit. Or if young deer in a region are, on
average, healthy.

\textbf{Step 1}: Define Your Hypotheses: First, we need to define two
hypotheses: the \textbf{research hypothesis} and the \textbf{null
hypothesis}.

\begin{itemize}
\tightlist
\item
  \textbf{Research Hypothesis (H\textsubscript{a})}: This is what we aim
  to support. \textbf{Keep in mind, we can't exactly ``prove''
  H\textsubscript{a} is correct, we can only say that H\textsubscript{0}
  isn't likely}. It can take a few forms based on the question:

  \begin{itemize}
  \tightlist
  \item
    H\textsubscript{a}: average pollution \textgreater{} legal limit
    (pollution is too high)
  \item
    H\textsubscript{a}: average pollution \textless{} legal limit
    (pollution is too low)
  \item
    H\textsubscript{a}: average pollution ≠ legal limit (pollution is
    just different)
  \end{itemize}
\item
  \textbf{Null Hypothesis (H\textsubscript{0})}: This is the default or
  `no change' scenario. It's opposite to the research hypothesis.

  \begin{itemize}
  \tightlist
  \item
    H\textsubscript{0}: average pollution ≤ legal limit (for the first
    H\textsubscript{a})
  \item
    H\textsubscript{0}: average pollution ≥ legal limit (for the second
    H\textsubscript{a})
  \item
    H\textsubscript{0}: average pollution = legal limit (for the third
    H\textsubscript{a})
  \end{itemize}
\end{itemize}

\textbf{Step 2}: Choose Your Test Statistic: Based on the data, we'll
compute a \textbf{test statistic}. This number will help us decide which
hypothesis seems more likely.

\textbf{Step 3}: Determine the Rejection Region: Before running the
test, we decide on a \textbf{rejection region}. If our test statistic
falls in this region, we'll reject the null hypothesis.

\textbf{Step 4}: Check Assumptions: Before drawing conclusions, ensure
that the test's conditions and assumptions are satisfied.

\textbf{Step 5}: Draw Conclusions: Finally, based on the test statistic
and the rejection region, decide whether to reject the null hypothesis.

\begin{center}\rule{0.5\linewidth}{0.5pt}\end{center}

\subsection{Errors in Hypothesis
Testing}\label{errors-in-hypothesis-testing}

Sometimes, even with the best methods, we make incorrect decisions.

\begin{itemize}
\item
  \textbf{Type I Error} (\(\alpha\)): This happens when we mistakenly
  reject the true null hypothesis. Imagine sending an innocent person to
  jail. Typically, \(\alpha\) is set at 0.05 (5\%).
\item
  \textbf{Type II Error} (\(\beta\)): Here, we mistakenly accept a false
  null hypothesis. Think of it as letting a guilty person go free.
\end{itemize}

\begin{longtable}[]{@{}
  >{\raggedright\arraybackslash}p{(\linewidth - 4\tabcolsep) * \real{0.3333}}
  >{\raggedright\arraybackslash}p{(\linewidth - 4\tabcolsep) * \real{0.3333}}
  >{\raggedright\arraybackslash}p{(\linewidth - 4\tabcolsep) * \real{0.3333}}@{}}
\toprule\noalign{}
\begin{minipage}[b]{\linewidth}\raggedright
Decision
\end{minipage} & \begin{minipage}[b]{\linewidth}\raggedright
If the null hypothesis is True
\end{minipage} & \begin{minipage}[b]{\linewidth}\raggedright
If the null hypothesis is False
\end{minipage} \\
\midrule\noalign{}
\endhead
\bottomrule\noalign{}
\endlastfoot
\textbf{Reject H\textsubscript{0}} & Type I error (prob = \(\alpha\)) &
Correct (prob = 1 - \(\beta\)) \\
\textbf{Fail to reject H\textsubscript{0}} & Correct (prob = 1 -
\(\alpha\)) & Type II error (prob = \(\beta\)) \\
\end{longtable}

\begin{quote}
\textbf{Key Takeaway}: As \(\alpha\) gets smaller, \(\beta\) gets
bigger, and vice-versa.
\end{quote}

\begin{center}\rule{0.5\linewidth}{0.5pt}\end{center}

\begin{center}\rule{0.5\linewidth}{0.5pt}\end{center}

\subsection{Deciphering Significance with
P-values}\label{deciphering-significance-with-p-values}

The p-value is like a reality-check. It tells us how weird our results
are if we assume the starting belief (null hypothesis) is spot on.

\begin{itemize}
\item
  \textbf{One-Tailed Test}: The p-value shows the likelihood of
  observing an average as extreme as our sample's if the null hypothesis
  stands.
\item
  \textbf{Two-Tailed Test}: This p-value represents the odds of spotting
  an average as different from the null value as our sample's.
\end{itemize}

\begin{quote}
\textbf{Rule of Thumb}: If the p-value is less than \(\alpha\), we opt
to reject the null hypothesis.
\end{quote}

\section{Graphical Review}\label{graphical-review}

\subsection{Key Players in Hypothesis Testing
Visualization}\label{key-players-in-hypothesis-testing-visualization}

We define and visualize the core components essential to understanding
the graphical representations of hypothesis testing:

\begin{enumerate}
\def\labelenumi{\arabic{enumi}.}
\item
  \textbf{Null Distribution} - The hypothesized parent distribution
  under the assumption that the null hypothesis H\textsubscript{0} is
  true.
\item
  \textbf{Inferred Parent Distribution} - The parent distribution
  inferred from our sample data. This is what we conceptualize as the
  distribution of H\textsubscript{a}.
\item
  \textbf{True Parent Distribution} - The actual distribution from which
  our sample originates.
\item
  \textbf{Sampling Distribution of the Sample Mean} - Represents the
  distribution of sample means if we were to draw multiple samples from
  the parent distribution. This is crucial for making inferences about
  the \textbf{Inferred Parent Distribution}.
\end{enumerate}

\begin{center}\rule{0.5\linewidth}{0.5pt}\end{center}

In the figure below, we've outlined the various elements crucial for
hypothesis testing. Think of this section as a handy guide. Whenever you
come across detailed graphs later in this chapter, you can circle back
here for clarity.

\begin{figure}

\centering{

\includegraphics[width=1\linewidth,height=\textheight,keepaspectratio]{06-Hypothesis_Testing_files/figure-pdf/fig-5-5.keyfig-1.pdf}

}

\caption{\label{fig-5-5.keyfig}Comparison of four distributions
essential for hypothesis testing: Null, True Parent, Inferred Parent,
and Sampling Distribution of the Sample Mean.}

\end{figure}%

\begin{center}\rule{0.5\linewidth}{0.5pt}\end{center}

For a two-tailed alternative, we are interested in the possibility that
a sample comes from a parent distribution that may have a lower or
higher location than the null.

\begin{figure}

\centering{

\includegraphics[width=1\linewidth,height=\textheight,keepaspectratio]{06-Hypothesis_Testing_files/figure-pdf/fig-5-5.twotailed-1.pdf}

}

\caption{\label{fig-5-5.twotailed}Two-tailed distribution}

\end{figure}%

\begin{center}\rule{0.5\linewidth}{0.5pt}\end{center}

In a one-tailed t-test, we're examining if our sample originates from a
parent distribution that's situated either below or above the null
hypothesis. Unlike a two-tailed test, we're only interested in one of
these directions, not both.

\begin{figure}

\centering{

\includegraphics[width=1\linewidth,height=\textheight,keepaspectratio]{06-Hypothesis_Testing_files/figure-pdf/fig-5-5.onetailed-1.pdf}

}

\caption{\label{fig-5-5.onetailed}Side-by-side comparison of one-tailed
t-test scenarios: exploring if our sample comes from a distribution
either below or above the null hypothesis.}

\end{figure}%

\begin{center}\rule{0.5\linewidth}{0.5pt}\end{center}

In a ``perfect'' world in which the null hypothesis is true, the
sample's parent distribution (solid, orange) is exactly the same parent
distribution described by the null hypothesis (solid, blue).

\begin{figure}

\centering{

\includegraphics[width=1\linewidth,height=\textheight,keepaspectratio]{06-Hypothesis_Testing_files/figure-pdf/fig-5-5.null_identical_sample-1.pdf}

}

\caption{\label{fig-5-5.null_identical_sample}True parent distribution
\& null distribution are the same}

\end{figure}%

\begin{center}\rule{0.5\linewidth}{0.5pt}\end{center}

\textbf{We never know the true parent distribution of the sample} -- we
infer it from the sample. Here, the tall dash-dotted line shows the
sampling distribution of the mean, from which we infer the parent
distribution (green3, dashed).

In this even more perfect world, that parent distribution is the same as
the parent distribution described by the null hypothesis and we have
taken a perfectly representative sample, so all 3 curves line up
perfectly on the same mean. The thick, short, flat dark green line is
the confidence interval for the sample mean.

\begin{figure}

\centering{

\includegraphics[width=1\linewidth,height=\textheight,keepaspectratio]{06-Hypothesis_Testing_files/figure-pdf/fig-5-5.perfectly_representative_sample-1.pdf}

}

\caption{\label{fig-5-5.perfectly_representative_sample}Here, we have a
perfectly representative sample}

\end{figure}%

\begin{center}\rule{0.5\linewidth}{0.5pt}\end{center}

In an imperfect but convenient world, the sample is not a perfect
representation of the parent population, but is fairly close. The sample
mean is close to hypothesized mean, and (in the 2-tailed case) the
confidence interval for the sample mean ``catches'' the mean of the null
hypothesis (pink dashed line). A hypothesis test will correctly
determine that there is not a significant difference between the sample
mean and the mean of the null hypothesis.

\begin{figure}

\centering{

\includegraphics[width=1\linewidth,height=\textheight,keepaspectratio]{06-Hypothesis_Testing_files/figure-pdf/fig-5-5.sample_imperfect_convenient-1.pdf}

}

\caption{\label{fig-5-5.sample_imperfect_convenient}Imperfect, but
convenient}

\end{figure}%

\begin{center}\rule{0.5\linewidth}{0.5pt}\end{center}

In an imperfect and inconvenient world, the random sample is, by chance,
sufficiently imperfect that the apparent (inferred) parent distribution
is far from the true parent distribution and (in the 2-tailed case) the
confidence interval for the sample mean no longer ``catches'' the mean
of the null hypothesis. A hypothesis test will now find a significant
difference between the sample mean and the mean of the null hypothesis.
\textbf{This is a type I error}.

\begin{figure}

\centering{

\includegraphics[width=1\linewidth,height=\textheight,keepaspectratio]{06-Hypothesis_Testing_files/figure-pdf/fig-5-5.type1_error-1.pdf}

}

\caption{\label{fig-5-5.type1_error}Type 1 Error}

\end{figure}%

\begin{center}\rule{0.5\linewidth}{0.5pt}\end{center}

In another imperfect and inconvenient world, the sample (dashed dark
green lines) really is drawn from the alternative distribution (the
sample's true parent distribution; orange), but is unrepresentative of
its parent and similar to the null (solid blue line). The confidence
interval (in the 2-tailed case) of the sample ``catches'' the mean of
the null hypothesis although it is far from the mean of the true parent
of the sample. A hypothesis test will find no significant difference
between the sample mean and the mean of the null hypothesis.
\textbf{This is a type II error.}

\begin{figure}

\centering{

\includegraphics[width=1\linewidth,height=\textheight,keepaspectratio]{06-Hypothesis_Testing_files/figure-pdf/fig-5-5.type2_error-1.pdf}

}

\caption{\label{fig-5-5.type2_error}Type 2 Error}

\end{figure}%

\section{Graphical Review of Test Outcomes that are Not in
Error}\label{graphical-review-of-test-outcomes-that-are-not-in-error}

As you review hypothesis testing, it's essential to remember that we
don't \emph{accept} the null hypothesis. The possibility of a Type I
error means our conclusion might be flawed. Instead of accepting the
null hypothesis, we \emph{fail to reject} H\textsubscript{0}. The
scarcity of data with small sample sizes can lead to significant
differences between the sample mean and the null mean (μ\_0). While it's
tempting to gather more data to be more certain, in the meantime, the
best we can do is fail to reject H\textsubscript{0}.

In the figures below, as in the figures above, the blue lines represent
the null parent distribution (defined by the null mean and the sample's
standard deviation).

The green solid lines denote the apparent parent distribution of our
sample:

\begin{itemize}
\item
  \textbf{Solid lighter green line}: Represents the distribution
  described by our sample mean and standard deviation.
\item
  *\textbf{Dashed dark green line}: Shows the sampling distribution of
  the sample mean, described by our sample mean and the standard error
  (SE).
\end{itemize}

\subsection{Graphical Descriptions:}\label{graphical-descriptions}

\begin{enumerate}
\def\labelenumi{\arabic{enumi}.}
\tightlist
\item
  \textbf{Fail to Reject the Null Hypothesis} - \emph{Sample Mean
  Supports the Null Hypothesis}: The means are far apart, but
  \textbf{not in our direction of interest}. For a one-tailed test, only
  data on one side of the rejection region can support the null
  hypothesis. Question to ponder: If we gather more data and obtain the
  same sample mean, could our conclusion change?
\end{enumerate}

\begin{figure}

\centering{

\includegraphics[width=1\linewidth,height=\textheight,keepaspectratio]{06-Hypothesis_Testing_files/figure-pdf/fig-5-5.fail_to_reject_1-1.pdf}

}

\caption{\label{fig-5-5.fail_to_reject_1}Ha:μ \textless{} X}

\end{figure}%

\begin{enumerate}
\def\labelenumi{\arabic{enumi}.}
\setcounter{enumi}{1}
\tightlist
\item
  \textbf{Fail to Reject the Null Hypothesis}: \emph{The sample mean
  supports the alternate hypothesis (it is on the appropriate side of
  the rejection region), but the sample size is too small}. The sample
  mean is only about 1SE from the null mean, making it too close to be
  significant. Hypothetical situation: With more data and the same
  sample mean, could our conclusion differ?
\end{enumerate}

\begin{figure}

\centering{

\includegraphics[width=1\linewidth,height=\textheight,keepaspectratio]{06-Hypothesis_Testing_files/figure-pdf/fig-5-5.fail_to_reject_2-1.pdf}

}

\caption{\label{fig-5-5.fail_to_reject_2}Ha:μ \textless{} X}

\end{figure}%

\begin{enumerate}
\def\labelenumi{\arabic{enumi}.}
\setcounter{enumi}{2}
\tightlist
\item
  \textbf{Reject the Null Hypothesis}: The sample mean is on the
  appropriate side of the rejection region. It's significantly distant
  from the null mean, over 3 SE, which is typically considered
  significant for most standard values of α.
\end{enumerate}

\begin{figure}

\centering{

\includegraphics[width=1\linewidth,height=\textheight,keepaspectratio]{06-Hypothesis_Testing_files/figure-pdf/fig-5-5.reject_1-1.pdf}

}

\caption{\label{fig-5-5.reject_1}Ha:μ \textless{} X}

\end{figure}%

\begin{enumerate}
\def\labelenumi{\arabic{enumi}.}
\setcounter{enumi}{3}
\tightlist
\item
  \textbf{Reject the Null Hypothesis (Two-Tailed Test)}: Reject the null
  hypothesis for the same reasons as the previous example. This case is
  two-tailed, but nothing else has changed.
\end{enumerate}

\begin{figure}

\centering{

\includegraphics[width=1\linewidth,height=\textheight,keepaspectratio]{06-Hypothesis_Testing_files/figure-pdf/fig-5-5.reject_2_two_tailed-1.pdf}

}

\caption{\label{fig-5-5.reject_2_two_tailed}Ha:μ ≠ X}

\end{figure}%

\section{Graphical Review of Sample Size Effect when Test Outcomes are
in
Error}\label{graphical-review-of-sample-size-effect-when-test-outcomes-are-in-error}

It's a given that we never truly grasp the actual parent distribution of
a sample. An unrepresentative sample can lead either to a Type I or a
Type II error. The term \emph{sampling error} is sometimes invoked to
depict such unrepresentative samples, but it's imperative to understand
that the researcher hasn't committed any mistakes.

\subsection{Graphical Descriptions:}\label{graphical-descriptions-1}

\textbf{Type I Error}: Here, the green curves depict the sampling
distribution (dark green) and the apparent parent distribution (lighter
green) of our sample. But in reality, the sample is a product of the
null distribution (blue). Question to ponder: How would the
representation look if we had utilized a smaller sample size?

\begin{figure}

\centering{

\includegraphics[width=1\linewidth,height=\textheight,keepaspectratio]{06-Hypothesis_Testing_files/figure-pdf/fig-5-5.type1_samplesize-1.pdf}

}

\caption{\label{fig-5-5.type1_samplesize}}

\end{figure}%

The main thing that would change with a larger sample size is that the
sampling distribution of sample means becomes much tighter, thus making
the confidence interval smaller. So here, are we more or less likely to
have a type 1 error with the larger sample size?

\begin{figure}

\centering{

\includegraphics[width=1\linewidth,height=\textheight,keepaspectratio]{06-Hypothesis_Testing_files/figure-pdf/fig-5-5.type_1_samplesize_larger-1.pdf}

}

\caption{\label{fig-5-5.type_1_samplesize_larger}}

\end{figure}%

\textbf{Type II Error}: The sample genuinely hails from the solid orange
parent population. However, it was misleading enough (as depicted by the
dashed green line) to seem analogous to the null distribution (blue).
Query to reflect upon: How would this representation transform if the
sample size was substantially larger?

\begin{figure}

\centering{

\includegraphics[width=1\linewidth,height=\textheight,keepaspectratio]{06-Hypothesis_Testing_files/figure-pdf/fig-5-5.type_2_samplesize-1.pdf}

}

\caption{\label{fig-5-5.type_2_samplesize}}

\end{figure}%

The main thing that would change with a larger sample size is that the
sampling distribution of sample means becomes much tighter, thus making
the confidence interval smaller. So here, are we more or less likely to
have a type 2 error with the larger sample size?

\begin{figure}

\centering{

\includegraphics[width=1\linewidth,height=\textheight,keepaspectratio]{06-Hypothesis_Testing_files/figure-pdf/fig-5-5.type_2_samplesize_larger-1.pdf}

}

\caption{\label{fig-5-5.type_2_samplesize_larger}}

\end{figure}%

\subsection{How Significant is `Significant' -- Interpreting
p-values}\label{how-significant-is-significant-interpreting-p-values}

When we use a rejection region to test a hypothesis, we get a yes-or-no
answer. For a two-tailed test, if we ask whether the confidence interval
``captures'' the null mean, we get a yes-or-no answer as well.

\subsubsection{Calculating p-values}\label{calculating-p-values}

We can do better -- we can get an actual probability value. The blue box
for the z-test tells us how to calculate our p-value if our mean is in
the area of interest for the test.

\begin{enumerate}
\def\labelenumi{\arabic{enumi}.}
\item
  \textbf{Calculate Z-value}: First, we calculate how many standard
  errors our mean is from the null mean. This is the z-value for our
  mean in the world of the null hypothesis.

  \begin{itemize}
  \tightlist
  \item
    For (H\textsubscript{0} \textless{} X), we ask about the upper tail
    probability of our mean.
  \item
    For (H\textsubscript{0} \textgreater{} X), we ask about the lower
    tail probability of our mean.
  \item
    For (H\textsubscript{0} = X), we calculate twice the tail
    probability.
  \end{itemize}
\end{enumerate}

\subsubsection{One-Tailed vs Two-Tailed
Tests}\label{one-tailed-vs-two-tailed-tests}

\begin{itemize}
\item
  \textbf{One-Tailed Test}: The p-value tells us the probability of a
  mean at least as much greater than (H\textsubscript{0}) as our mean,
  when the null hypothesis is true or as much less than
  (H\textsubscript{0}).
\item
  \textbf{Two-Tailed Test}: The p-value tells us the probability of a
  mean at least as different from (H\textsubscript{0}) as our mean, when
  the null hypothesis is true.
\end{itemize}

\subsubsection{Rejecting the Null
Hypothesis}\label{rejecting-the-null-hypothesis}

We reject (H\textsubscript{0}) when (p)-value (\textless{} \alpha). At
that point, our data are too unusual when the null hypothesis is true
for us to believe that the null hypothesis is true.

\begin{itemize}
\tightlist
\item
  \textbf{Small p-value}: When (p) is small, our data provide weak
  support for (H\textsubscript{0}), and we are more sure that
  (H\textsubscript{0}) is not true, and that (H\textsubscript{a}) is
  more likely.
\end{itemize}

\begin{center}\rule{0.5\linewidth}{0.5pt}\end{center}

\section{\texorpdfstring{Review of Ways to Test
(H\textsubscript{0})}{Review of Ways to Test (H0)}}\label{review-of-ways-to-test-h0}

\begin{enumerate}
\def\labelenumi{\arabic{enumi}.}
\item
  \textbf{Confidence Interval}: If the alternative is two-tailed, build
  a 1-(\alpha) confidence interval. If the CI catches
  (H\textsubscript{0}) then fail to reject.
\item
  \textbf{Test Statistic}: Calculate the test statistic and compare to
  the rejection region of size (\alpha). If the test statistic is in the
  rejection region, reject (H\textsubscript{0}).
\item
  \textbf{Probability}: Determine the probability of your test
  statistic. If (p \textless{} \alpha) then reject (H\textsubscript{0}).
\end{enumerate}

\textbf{Note}: The first two methods give you a yes-or-no answer. The
third method gives you some additional information.

\begin{center}\rule{0.5\linewidth}{0.5pt}\end{center}

\subsection{Results Statement}\label{results-statement}

Now that we have started doing statistical tests, we have also started
to think about results. A results statement provides an English language
version of what we discovered, as well as the statistical results. For a
z-test, a results sentence might say:

\begin{quote}
The average level of mercury in the ponds within 10 km of the smelter is
significantly higher than the legal limit (z = 2.85, n = 32, p = 0.004).
\end{quote}

The information in the parentheses, for a one-sample test, is, in this
order,

\begin{enumerate}
\def\labelenumi{\arabic{enumi})}
\tightlist
\item
  the value of the test statistic, in this case, (z);
\item
  the sample size or degrees of freedom; and
\item
  the probability of the test statistic when the null hypothesis is
  true.
\end{enumerate}

Most problems that include a test will require a results sentence.

\begin{center}\rule{0.5\linewidth}{0.5pt}\end{center}

\subsection{Beyond the 0.5 cutoff: Effect-size and
power}\label{beyond-the-0.5-cutoff-effect-size-and-power}

You've probably heard me mention that the 0.5 cutoff for statistical
significance is somewhat arbitrary. So, what's the alternative? Enter
effect size and statistical power. These aren't just buzzwords; they're
foundational elements for conducting meaningful environmental research.
Many scientific journals even have guidelines on how to report them.
Ideally, you should be thinking about these factors before you collect
your first data point. Given their importance, it's time we delve into
what these concepts really mean and why they're crucial for research.

\subsection{The importance of knowing what you're
doing}\label{the-importance-of-knowing-what-youre-doing}

Effect size and power analyses are more than just boxes to tick; they're
essential tools in your research toolkit for understanding environmental
data. Rather than using them simply because you were advised to, see
them as integral to designing meaningful studies. These tools help you
filter out statistical ``noise,'' revealing actionable insights that can
address real-world environmental issues. They shouldn't be applied
blindly but should be part of a thoughtful research strategy aimed at
making your data work for you.

\subsection{Chance vs.~Real Effects: The Playground, the Superpower, and
the Impact
Scale}\label{chance-vs.-real-effects-the-playground-the-superpower-and-the-impact-scale}

In environmental research, the goal is often to identify meaningful
changes---like the improvement of air quality due to reduced pollution.
However, researchers sometimes find themselves grappling with
statistical ``noise'' rather than detecting genuine effects. To navigate
this complex landscape, let's use some analogies.

\textbf{The Playground and the Mischievous Kid}

First, consider your sample size as a playground and chance as a
mischievous kid running around in it. The smaller the playground, the
more room this kid has to create chaos, leading to random variations in
your data. On the flip side, a larger playground restricts the kid's
antics, minimizing the influence of chance. So, your first task is to
design your study like an ultimate playground---spacious and
well-planned to keep chance at bay.

\textbf{The Balls: Different Sizes, Different Impacts}

Next, let's focus on the ``stuff'' being thrown around on this
playground. Think of different types of balls---soccer balls, tennis
balls, ping pong balls, and marbles---as representing different effect
sizes:

\begin{itemize}
\item
  \textbf{Soccer Ball (Strong Effect)}: It's big and noticeable. When it
  lands, you know something significant has happened.
\item
  \textbf{Tennis Ball (Medium Effect)}: Still impactful but not as
  game-changing as a soccer ball.
\item
  \textbf{Ping Pong Ball (Small Effect)}: It might bounce around, but
  it's not going to change the landscape.
\item
  \textbf{Marble (Very Small Effect)}: Almost negligible amid the other
  activities.
\end{itemize}

\textbf{The Superhero: Statistical Power}

Here's where statistical power comes into play. It's your research
superhero, capable of discerning whether the changes you're observing
are due to chance, the size of your playground, or the type of ball
being thrown (Effect Size). Imagine it as a keen-eyed playground
supervisor who can tell the difference between a random bounce and a
meaningful impact.

The Takeaway

\begin{enumerate}
\def\labelenumi{\arabic{enumi}.}
\item
  \textbf{Plan Well}: Design your study like you're building the
  ultimate playground---spacious and well-planned to minimize the role
  of chance.
\item
  \textbf{Know Your Ball}: Understand the potential impact (effect size)
  of what you're introducing into your study. This helps you make
  meaningful conclusions.
\item
  \textbf{Power Up}: Conduct a power analysis to ensure your study is
  equipped to distinguish between meaningful impacts and random noise.
\end{enumerate}

By focusing on these three elements---chance, effect size, and
statistical power---you're not just adhering to research best practices;
you're elevating the quality and impact of your work.

\subsection{Effect size: concrete vs.~abstract
notions}\label{effect-size-concrete-vs.-abstract-notions}

Generally speaking, the big concept of effect size, is simply how big
the differences are, that's it. However, the biggness or smallness of
effects quickly becomes a little bit complicated. On the one hand, the
raw difference in the means can be very meaningful. Let's say we are
measuring performance on a final exam, and we are testing whether or not
a miracle drug can make you do better on the test. Let's say taking the
drug makes you do 5\% better on the test, compared to not taking the
drug. You know what 5\% means, that's basically a whole letter grade.
Pretty good. An effect-size of 25\% would be even better, right? Lots of
measures have a concrete quality to them, and we often want to the size
of the effect expressed in terms of the original measure.

Let's talk about concrete measures some more. How about learning a
musical instrument. Let's say it takes 10,000 hours to become an expert
piano, violin, or guitar player. And, let's say you found something
online that says that using their method, you will learn the instrument
in less time than normal. That is a claim about the effect size of their
method. You would want to know how big the effect is right? For example,
the effect-size could be 10 hours. That would mean it would take you
9,980 hours to become an expert (that's a whole 10 hours less). If I
knew the effect-size was so tiny, I wouldn't bother with their new
method. But, if the effect size was say 1,000 hours, that's a pretty big
deal, that's 10\% less (still doesn't seem like much, but saving 1,000
hours seems like a lot).

In environmental science, we often encounter measures that are not as
straightforward as, say, temperature or pH levels. Take biodiversity
indices as an example. These indices can give us a numerical value
representing the variety of life in a particular ecosystem, but
interpreting what these numbers mean can be challenging.

Imagine you're assessing the impact of a reforestation project. Your
biodiversity index might read 3 before the project and 4 after. That's a
difference of only 1 unit, but what does that actually signify? Is it a
significant improvement, or just a minor change? The raw numbers alone
don't provide enough context.

To make these abstract measures more interpretable, we often turn to
standardized metrics, like z-scores. If that 1-unit difference in
biodiversity corresponds to a shift of one standard deviation, that's a
substantial change worth noting. On the other hand, if the shift is only
0.1 in terms of standard deviation, then the 11-unit difference might
not be as impactful as it first seemed. Standardized measures like
Cohen's d can further help us understand the practical significance of
our findings.

\subsection{Cohen's d}\label{cohens-d}

Let's look a few distributions to firm up some ideas about effect-size.
Figure~\ref{fig-5.5effectdists} has four panels. The first panel (0)
represents the null distribution of no differences. This is the idea
that your manipulation (A vs.~B) doesn't do anything at all, as a result
when you measure scores in conditions A and B, you are effectively
sampling scores from the very same overall distribution. The panel shows
the distribution as green for condition B, but the red one for condition
A is identical and drawn underneath (it's invisible). There is 0
difference between these distributions, so it represent a null effect.

\begin{figure}

\centering{

\includegraphics[width=1\linewidth,height=\textheight,keepaspectratio]{06-Hypothesis_Testing_files/figure-pdf/fig-5.5effectdists-1.pdf}

}

\caption{\label{fig-5.5effectdists}Each panel shows hypothetical
distributions for two conditions. As the effect-size increases, the
difference between the distributions become larger.}

\end{figure}%

The remaining panels are hypothetical examples of what a true effect
could look like, when your manipulation actually causes a difference.
For example, if condition A is a control group, and condition B is a
treatment group, we are looking at three cases where the treatment
manipulation causes a positive shift in the mean of distribution. We are
using normal curves with mean =0 and sd =1 for this demonstration, so a
shift of .5 is a shift of half of a standard deviation. A shift of 1 is
a shift of 1 standard deviation, and a shift of 2 is a shift of 2
standard deviations. We could draw many more examples showing even
bigger shifts, or shifts that go in the other direction.

Let's look at another example, but this time we'll use some concrete
measurements. Let's say we are looking at final exam performance, so our
numbers are grade percentages. Let's also say that we know the mean on
the test is 65\%, with a standard deviation of 5\%. Group A could be a
control that just takes the test, Group B could receive some
``educational'' manipulation designed to improve the test score. These
graphs then show us some hypotheses about what the manipulation may or
may not be doing.

\begin{figure}

\centering{

\includegraphics[width=1\linewidth,height=\textheight,keepaspectratio]{06-Hypothesis_Testing_files/figure-pdf/fig-5.5effectdistsB-1.pdf}

}

\caption{\label{fig-5.5effectdistsB}Each panel shows hypothetical
distributions for two conditions. As the effect-size increases, the
difference between the distributions become larger.}

\end{figure}%

The first panel shows that both condition A and B will sample test
scores from the same distribution (mean =65, with 0 effect). The other
panels show shifted mean for condition B (the treatment that is supposed
to increase test performance). So, the treatment could increase the test
performance by 2.5\% (mean 67.5, .5 sd shift), or by 5\% (mean 70, 1 sd
shift), or by 10\% (mean 75\%, 2 sd shift), or by any other amount. In
terms of our previous metaphor, a shift of 2 standard deviations is more
like jack-hammer in terms of size, and a shift of .5 standard deviations
is more like using a pencil. The thing about research, is we often have
no clue about whether our manipulation will produce a big or small
effect, that's why we are conducting the research.

You might have noticed that the letter \(d\) appears in the above
figure. Why is that? Jacob Cohen (Cohen 1988) used the letter \(d\) in
defining the effect-size for this situation, and now everyone calls it
Cohen's \(d\). The formula for Cohen's \(d\) is:

\(d = \frac{\text{mean for condition 1} - \text{mean for condition 2}}{\text{population standard deviation}}\)

If you notice, this is just a kind of z-score. It is a way to
standardize the mean difference in terms of the population standard
deviation.

It is also worth noting again that this measure of effect-size is
entirely hypothetical for most purposes. In general, researchers do not
know the population standard deviation, they can only guess at it, or
estimate it from the sample. The same goes for means, in the formula
these are hypothetical mean differences in two population distributions.
In practice, researchers do not know these values, they guess at them
from their samples.

Before discussing why the concept of effect-size can be useful, we note
that Cohen's \(d\) is useful for understanding abstract measures. For
example, when you don't know what a difference of 10 or 20 means as a
raw score, you can standardize the difference by the sample standard
deviation, then you know roughly how big the effect is in terms of
standard units. If you thought a 20 was big, but it turned out to be
only 1/10th of a standard deviation, then you would know the effect is
actually quite small with respect to the overall variability in the
data.

\section{Power}\label{power}

When there is a true effect out there to measure, you want to make sure
your design is sensitive enough to detect the effect, otherwise what's
the point. We've already talked about the idea that an effect can have
different sizes. The next idea is that your design can be more less
sensitive in its ability to reliabily measure the effect. We have
discussed this general idea many times already in the textbook, for
example we know that we will be more likely to detect ``significant''
effects (when there are real differences) when we increase our
sample-size. Here, we will talk about the idea of design sensitivity in
terms of the concept of power. Interestingly, the concept of power is a
somewhat limited concept, in that it only exists as a concept within
some philosophies of statistics.

\subsection{A digresssion about hypothesis
testing}\label{a-digresssion-about-hypothesis-testing}

In particular, the concept of power falls out of the Neyman-Pearson
concept of null vs.~alternative hypothesis testing. Neyman-Pearson ideas
are by now the most common and widespread, and in the opinion of some of
us, they are also the most widely misunderstood and abused idea.

What we have been mainly doing is talking about hypothesis testing from
the Fisherian (Sir Ronald Fisher, the ANOVA guy) perspective. This is a
basic perspective that can't be easily ignored. It is also quite
limited. The basic idea is this:

\begin{enumerate}
\def\labelenumi{\arabic{enumi}.}
\tightlist
\item
  We know that chance can cause some differences when we measure
  something between experimental conditions.
\item
  We want to rule out the possibility that the difference that we
  observed can not be due to chance
\item
  We construct large N designs that permit us to do this when a real
  effect is observed, such that we can confidently say that big
  differences that we find are so big (well outside the chance window)
  that it is highly implausible that chance alone could have produced.
\item
  The final conclusion is that chance was extremely unlikely to have
  produced the differences. We then infer that something else, like the
  manipulation, must have caused the difference.
\item
  We don't say anything else about the something else.
\item
  We either reject the null distribution as an explanation (that chance
  couldn't have done it), or retain the null (admit that chance could
  have done it, and if it did we couldn't tell the difference between
  what we found and what chance could do)
\end{enumerate}

Neyman and Pearson introduced one more idea to this mix, the idea of an
alternative hypothesis. The alternative hypothesis is the idea that if
there is a true effect, then the data sampled into each condition of the
experiment must have come from two different distributions. Remember,
when there is no effect we assume all of the data cam from the same
distribution (which by definition can't produce true differences in the
long run, because all of the numbers are coming from the same
distribution). The graphs of effect-sizes from before show examples of
these alternative distributions, with samples for condition A coming
from one distribution, and samples from condition B coming from a
shifted distribution with a different mean.

So, under the Neyman-Pearson tradition, when a researcher find a
signifcant effect they do more than one things. First, they reject the
null-hypothesis of no differences, and they accept the alternative
hypothesis that there was differences. This seems like a sensible thing
to do. And, because the researcher is actually interested in the
properties of the real effect, they might be interested in learning more
about the actual alternative hypothesis, that is they might want to know
if their data come from two different distributions that were separated
by some amount\ldots in other words, they would want to know the size of
the effect that they were measuring.

\subsection{Back to power}\label{back-to-power}

We have now discussed enough ideas to formalize the concept of
statistical power. For this concept to exist we need to do a couple
things.

\begin{enumerate}
\def\labelenumi{\arabic{enumi}.}
\tightlist
\item
  Agree to set an alpha criterion. When the p-value for our
  test-statistic is below this value we will call our finding
  statistically significant, and agree to reject the null hypothesis and
  accept the ``alternative'' hypothesis (sidenote, usually it isn't very
  clear which specific alternative hypothesis was accepted)
\item
  In advance of conducting the study, figure out what kinds of
  effect-sizes our design is capable of detecting with particular
  probabilites.
\end{enumerate}

The power of a study is determined by the relationship between

\begin{enumerate}
\def\labelenumi{\arabic{enumi}.}
\tightlist
\item
  The sample-size of the study
\item
  The effect-size of the manipulation
\item
  The alpha value set by the researcher.
\end{enumerate}

To see this in practice let's do a simulation. We will do a t-test on a
between-groups design 10 subjects in each group. Group A will be a
control group with scores sampled from a normal distribution with mean
of 10, and standard deviation of 5. Group B will be a treatment group,
we will say the treatment has an effect-size of Cohen's \(d\) = .5,
that's a standard deviation shift of .5, so the scores with come from a
normal distribution with mean =12.5 and standard deivation of 5.
Remember 1 standard deviation here is 5, so half of a standard deviation
is 2.5.

The following R script runs this simulated experiment 1000 times. We set
the alpha criterion to .05, this means we will reject the null whenever
the \(p\)-value is less than .05. With this specific design, how many
times out of of 1000 do we reject the null, and accept the alternative
hypothesis?

\begin{verbatim}
#> [1] 191
\end{verbatim}

The answer is that we reject the null, and accept the alternative 191
times out of 1000. In other words our experiment succesfully accepts the
alternative hypothesis 19.1 percent of the time, this is known as the
power of the study. Power is the probability that a design will
succesfully detect an effect of a specific size.

Importantly, power is completely abstract idea that is completely
determined by many assumptions including N, effect-size, and alpha. As a
result, it is best not to think of power as a single number, but instead
as a family of numbers.

For example, power is different when we change N. If we increase N, our
samples will more precisely estimate the true distributions that they
came from. Increasing N reduces sampling error, and shrinks the range of
differences that can be produced by chance. Lets' increase our N in this
simulation from 10 to 20 in each group and see what happens.

\begin{verbatim}
#> [1] 345
\end{verbatim}

Now the number of significant experiments i 345 out of 1000, or a power
of 34.5 percent. That's roughly doubled from before. We have made the
design more sensitive to the effect by increasing N.

We can change the power of the design by changing the alpha-value, which
tells us how much evidence we need to reject the null. For example, if
we set the alpha criterion to 0.01, then we will be more conservative,
only rejecting the null when chance can produce the observed difference
1\% of the time. In our example, this will have the effect of reducing
power. Let's keep N at 20, but reduce the alpha to 0.01 and see what
happens:

\begin{verbatim}
#> [1] 140
\end{verbatim}

Now only 140 out of 1000 experiments are significant, that's 14 power.

Finally, the power of the design depends on the actual size of the
effect caused by the manipulation. In our example, we hypothesized that
the effect caused a shift of .5 standard deviations. What if the effect
causes a bigger shift? Say, a shift of 2 standard deviations. Let's keep
N= 20, and alpha \textless{} .01, but change the effect-size to two
standard deviations. When the effect in the real-world is bigger, it
should be easier to measure, so our power will increase.

\begin{verbatim}
#> [1] 1000
\end{verbatim}

Neat, if the effect-size is actually huge (2 standard deviation shift),
then we have power 100 percent to detect the true effect.

\subsection{Power curves}\label{power-curves}

We mentioned that it is best to think of power as a family of numbers,
rather than as a single number. To elaborate on this consider the power
curve below. This is the power curve for a specific design: a between
groups experiments with two levels, that uses an independent samples
t-test to test whether an observed difference is due to chance.
Critically, N is set to 10 in each group, and alpha is set to .05

In Figure~\ref{fig-5.5powercurve} power (as a proportion, not a
percentage) is plotted on the y-axis, and effect-size (Cohen's d) in
standard deviation units is plotted on the x-axis.

\begin{figure}

\centering{

\includegraphics[width=1\linewidth,height=\textheight,keepaspectratio]{06-Hypothesis_Testing_files/figure-pdf/fig-5.5powercurve-1.pdf}

}

\caption{\label{fig-5.5powercurve}This figure shows power as a function
of effect-size (Cohen's d) for a between-subjects independent samples
t-test, with N=10, and alpha criterion 0.05.}

\end{figure}%

A power curve like this one is very helpful to understand the
sensitivity of a particular design. For example, we can see that a
between subjects design with N=10 in both groups, will detect an effect
of d=.5 (half a standard deviation shift) about 20\% of the time, will
detect an effect of d=.8 about 50\% of the time, and will detect an
effect of d=2 about 100\% of the time. All of the percentages reflect
the power of the design, which is the percentage of times the design
would be expected to find a \(p\) \textless{} 0.05.

Let's imagine that based on prior research, the effect you are
interested in measuring is fairly small, d=0.2. If you want to run an
experiment that will detect an effect of this size a large percentage of
the time, how many subjects do you need to have in each group? We know
from the above graph that with N=10, power is very low to detect an
effect of d=0.2. Let's make Figure~\ref{fig-5.5powercurveN} and vary the
number of subjects rather than the size of the effect.

\begin{figure}

\centering{

\includegraphics[width=1\linewidth,height=\textheight,keepaspectratio]{06-Hypothesis_Testing_files/figure-pdf/fig-5.5powercurveN-1.pdf}

}

\caption{\label{fig-5.5powercurveN}This figure shows power as a function
of N for a between-subjects independent samples t-test, with d=0.2, and
alpha criterion 0.05.}

\end{figure}%

The figure plots power to detect an effect of d=0.2, as a function of N.
The green line shows where power = .8, or 80\%. It looks like we would
nee about 380 subjects in each group to measure an effect of d=0.2, with
power = .8. This means that 80\% of our experiments would succesfully
show p \textless{} 0.05. Often times power of 80\% is recommended as a
reasonable level of power, however even when your design has power =
80\%, your experiment will still fail to find an effect (associated with
that level of power) 20\% of the time!

\section{Planning your design}\label{planning-your-design}

Our discussion of effect size and power highlight the importance of the
understanding the statistical limitations of an experimental design. In
particular, we have seen the relationship between:

\begin{enumerate}
\def\labelenumi{\arabic{enumi}.}
\tightlist
\item
  Sample-size
\item
  Effect-size
\item
  Alpha criterion
\item
  Power
\end{enumerate}

As a general rule of thumb, small N designs can only reliably detect
very large effects, whereas large N designs can reliably detect much
smaller effects. As a researcher, it is your responsibility to plan your
design accordingly so that it is capable of reliably detecting the kinds
of effects it is intended to measure.

\section{Some considerations}\label{some-considerations}

\subsection{Low powered studies}\label{low-powered-studies}

Consider the following case. A researcher runs a study to detect an
effect of interest. There is good reason, from prior research, to
believe the effect-size is d=0.5. The researcher uses a design that has
30\% power to detect the effect. They run the experiment and find a
significant p-value, (p\textless.05). They conclude their manipulation
worked, because it was unlikely that their result could have been caused
by chance. How would you interpret the results of a study like this?
Would you agree with thte researchers that the manipulation likely
caused the difference? Would you be skeptical of the result?

The situation above requires thinking about two kinds of probabilities.
On the one hand we know that the result observed by the researchers does
not occur often by chance (p is less than 0.05). At the same time, we
know that the design was underpowered, it only detects results of the
expected size 30\% of the time. We are face with wondering what kind of
luck was driving the difference. The researchers could have gotten
unlucky, and the difference really could be due to chance. In this case,
they would be making a type I error (saying the result is real when it
isn't). If the result was not due to chance, then they would also be
lucky, as their design only detects this effect 30\% of the time.

Perhaps another way to look at this situation is in terms of the
replicability of the result. Replicability refers to whether or not the
findings of the study would be the same if the experiment was repeated.
Because we know that power is low here (only 30\%), we would expect that
most replications of this experiment would not find a significant
effect. Instead, the experiment would be expected to replicate only 30\%
of the time.

\subsection{Large N and small effects}\label{large-n-and-small-effects}

Perhaps you have noticed that there is an intriguing relationship
between N (sample-size) and power and effect-size. As N increases, so
does power to detect an effect of a particular size. Additionally, as N
increases, a design is capable of detecting smaller and smaller effects
with greater and greater power. For example, if N was large enough, we
would have high power to detect very small effects, say d= 0.01, or even
d=0.001. Let's think about what this means.

Imagine a drug company told you that they ran an experiment with 1
billion people to test whether their drug causes a significant change in
headache pain. Let's say they found a significant effect (with power
=100\%), but the effect was very small, it turns out the drug reduces
headache pain by less than 1\%, let's say 0.01\%. For our imaginary
study we will also assume that this effect is very real, and not caused
by chance.

Clearly the design had enough power to detect the effect, and the effect
was there, so the design did detect the effect. However, the issue is
that there is little practical value to this effect. Nobody is going to
by a drug to reduce their headache pain by 0.01\%, even if it was
``scientifcally proven'' to work. This example brings up two issues.
First, increasing N to very large levels will allow designs to detect
almost any effect (even very tiny ones) with very high power. Second,
sometimes effects are meaningless when they are very small, especially
in applied research such as drug studies.

These two issues can lead to interesting suggestions. For example,
someone might claim that large N studies aren't very useful, because
they can always detect really tiny effects that are practically
meaningless. On the other hand, large N studies will also detect larger
effects too, and they will give a better estimate of the ``true'' effect
in the population (because we know that larger samples do a better job
of estimating population parameters). Additionally, although really
small effects are often not interesting in the context of applied
research, they can be very important in theoretical research. For
example, one theory might predict that manipulating X should have no
effect, but another theory might predict that X does have an effect,
even if it is a small one. So, detecting a small effect can have
theoretical implication that can help rule out false theories. Generally
speaking, researchers asking both theoretical and applied questions
should think about and establish guidelines for ``meaningful''
effect-sizes so that they can run designs of appropriate size to detect
effects of ``meaningful size''.

\subsection{Small N and Large effects}\label{small-n-and-large-effects}

All other things being equal would you trust the results from a study
with small N or large N? This isn't a trick question, but sometimes
people tie themselves into a knot trying to answer it. We already know
that large sample-sizes provide better estimates of the distributions
the samples come from. As a result, we can safely conclude that we
should trust the data from large N studies more than small N studies.

At the same time, you might try to convince yourself otherwise. For
example, you know that large N studies can detect very small effects
that are practically and possibly even theoretically meaningless. You
also know that that small N studies are only capable of reliably
detecting very large effects. So, you might reason that a small N study
is better than a large N study because if a small N study detects an
effect, that effect must be big and meaningful; whereas, a large N study
could easily detect an effect that is tiny and meaningless.

This line of thinking needs some improvement. First, just because a
large N study can detect small effects, doesn't mean that it only
detects small effects. If the effect is large, a large N study will
easily detect it. Large N studies have the power to detect a much wider
range of effects, from small to large. Second, just because a small N
study detected an effect, does not mean that the effect is real, or that
the effect is large. For example, small N studies have more variability,
so the estimate of the effect size will have more error. Also, there is
5\% (or alpha rate) chance that the effect was spurious. Interestingly,
there is a pernicious relationship between effect-size and type I error
rate

\subsection{Type I errors are convincing when N is
small}\label{type-i-errors-are-convincing-when-n-is-small}

So what is this pernicious relationship between Type I errors and
effect-size? Mainly, this relationship is pernicious for small N
studies. For example, the following figure illustrates the results of
1000s of simulated experiments, all assuming the null distribution. In
other words, for all of these simulations there is no true effect, as
the numbers are all sampled from an identical distribution (normal
distribution with mean =0, and standard deviation =1). The true
effect-size is 0 in all cases.

We know that under the null, researchers will find p values that are
less 5\% about 5\% of the time, remember that is the definition. So, if
a researcher happened to be in this situation (where there manipulation
did absolutely nothing), they would make a type I error 5\% of the time,
or if they conducted 100 experiments, they would expect to find a
significant result for 5 of them.

Figure~\ref{fig-5.5effectsizeType1} reports the findings from only the
type I errors, where the simulated study did produce p \textless{} 0.05.
For each type I error, we calculated the exact p-value, as well as the
effect-size (cohen's D) (mean difference divided by standard deviation).
We already know that the true effect-size is zero, however take a look
at this graph, and pay close attention to the smaller sample-sizes.

\begin{figure}

\centering{

\includegraphics[width=1\linewidth,height=\textheight,keepaspectratio]{06-Hypothesis_Testing_files/figure-pdf/fig-5.5effectsizeType1-1.pdf}

}

\caption{\label{fig-5.5effectsizeType1}Effect size as a function of
p-values for type 1 Errors under the null, for a paired samples t-test.}

\end{figure}%

For example, look at the red dots, when sample size is 10. Here we see
that the effect-sizes are quite large. When p is near 0.05 the
effect-size is around .8, and it goes up and up as when p gets smaller
and smaller. What does this mean? It means that when you get unlucky
with a small N design, and your manipulation does not work, but you by
chance find a ``significant'' effect, the effect-size measurement will
show you a ``big effect''. This is the pernicious aspect. When you make
a type I error for small N, your data will make you think there is no
way it could be a type I error because the effect is just so big!.
Notice that when N is very large, like 1000, the measure of effect-size
approaches 0 (which is the true effect-size in the simulation shown in
Figure~\ref{fig-5.5cohensD}).

\begin{figure}

\centering{

\includegraphics[width=1\linewidth,height=\textheight,keepaspectratio]{06-Hypothesis_Testing_files/figure-pdf/fig-5.5cohensD-1.pdf}

}

\caption{\label{fig-5.5cohensD}Each panel shows a histogram of a
different sampling statistic.}

\end{figure}%

\bookmarksetup{startatroot}

\chapter{t-tests}\label{t-tests}

Back in the day, William Sealy Gosset got a job working for Guinness
Breweries. They make the famous Irish stout called Guinness. What
happens next went something like this (total fabrication, but mostly on
point).

Guinness wanted all of their beers to be the best beers. No mistakes, no
bad beers. They wanted to improve their quality control so that when
Guinness was poured anywhere in the world, it would always comes out
fantastic: 5 stars out of 5 every time, the best.

Guinness had some beer tasters, who were super-experts. Every time they
tasted a Guinness from the factory that wasn't 5 out of 5, they knew
right away.

But, Guinness had a big problem. They would make a keg of beer, and they
would want to know if every single pint that would come out would be a 5
out of 5. So, the beer tasters drank pint after pint out of the keg,
until it was gone. Some kegs were all 5 out of 5s. Some weren't,
Guinness needed to fix that. But, the biggest problem was that, after
the testing, there \textbf{was no beer left to sell}, the testers drank
it all (remember I'm making this part up to illustrate a point, they
probably still had beer left to sell).

Guinness had a sampling and population problem. They wanted to know that
the entire population of the beers they made were all 5 out of 5 stars.
But, if they sampled the entire population, they would drink all of
their beer, and wouldn't have any left to sell.

Enter William Sealy Gosset. Gosset figured out the solution to the
problem. He asked questions like this:

\begin{enumerate}
\def\labelenumi{\arabic{enumi}.}
\item
  How many samples do I need to take to know the whole population is 5
  out of 5?
\item
  What's the fewest amount of samples I need to take to know the above,
  that would mean Guinness could test fewer beers for quality, sell more
  beers for profit, and make the product testing time shorter.
\end{enumerate}

Gosset solved those questions, and he invented something called the
\emph{Student's t-test}. Gosset was working for Guinness, and could be
fired for releasing trade-secrets that he invented (the t-test). But,
Gosset published the work anyways, under a pseudonym (Student 1908). He
called himself Student, hence Student's t-test. Now you know the rest of
the story.

It turns out this was a very nice thing for Gosset to have done. t-tests
are used all the time, and they are useful, that's why they are used. In
this chapter we learn how they work.

You'll be surprised to learn that what we've already talked about, (the
Crump Test, and the Randomization Test), are both very very similar to
the t-test. So, in general, you have already been thinking about the
things you need to think about to understand t-tests. You're probably
wondering what is this \(t\), what does \(t\) mean? We will tell you.
Before we tell what it means, we first tell you about one more idea.

\section{Check your confidence in your
mean}\label{check-your-confidence-in-your-mean}

We've talked about getting a sample of data. We know we can find the
mean, we know we can find the standard deviation. We know we can look at
the data in a histogram. These are all useful things to do for us to
learn something about the properties of our data.

You might be thinking of the mean and standard deviation as very
different things that we would not put together. The mean is about
central tendency (where most of the data is), and the standard deviation
is about variance (where most of the data isn't). Yes, they are
different things, but we can use them together to create useful new
things.

What if I told you my sample mean was 50, and I told you nothing else
about my sample. Would you be confident that most of the numbers were
near 50? Would you wonder if there was a lot of variability in the
sample, and many of the numbers were very different from 50. You should
wonder all of those things. The mean alone, just by itself, doesn't tell
you anything about well the mean represents all of the numbers in the
sample.

It could be a representative number, when the standard deviation is very
small, and all the numbers are close to 50. It could be a
non-representative number, when the standard deviation is large, and
many of the numbers are not near 50. You need to know the standard
deviation in order to be confident in how well the mean represents the
data.

How can we put the mean and the standard deviation together, to give us
a new number that tells us about confidence in the mean?

We can do this using a ratio:

\(\frac{mean}{\text{standard deviation}}\)

Think about what happens here. We are dividing a number by a number.
Look at what happens:

\(\frac{number}{\text{same number}} = 1\)

\(\frac{number}{\text{smaller number}} = \text{big number}\)

compared to:

\(\frac{number}{\text{bigger number}} = \text{smaller number}\)

Imagine we have a mean of 50, and a truly small standard deviation of 1.
What do we get with our formula?

\(\frac{50}{1} = 50\)

Imagine we have a mean of 50, and a big standard deviation of 100. What
do we get with our formula?

\(\frac{50}{100} = 0.5\)

Notice, when we have a mean paired with a small standard deviation, our
formula gives us a big number, like 50. When we have a mean paired with
a large standard deviation, our formula gives us a small number, like
0.5. These numbers can tell us something about confidence in our mean,
in a general way. We can be 50 confident in our mean in the first case,
and only 0.5 (not at a lot) confident in the second case.

What did we do here? We created a descriptive statistic by dividing the
mean by the standard deviation. And, we have a sense of how to interpret
this number, when it's big we're more confident that the mean represents
all of the numbers, when it's small we are less confident. This is a
useful kind of number, a ratio between what we think about our sample
(the mean), and the variability in our sample (the standard deviation).
Get used to this idea. Almost everything that follows in this textbook
is based on this kind of ratio. We will see that our ratio turns into
different kinds of ``statistics'', and the ratios will look like this in
general:

\(\text{name of statistic} = \frac{\text{measure of what we know}}{\text{measure of what we don't know}}\)

or, to say it using different words:

\(\text{name of statistic} = \frac{\text{measure of effect}}{\text{measure of error}}\)

In fact, this is the general formula for the t-test. Big surprise!

\section{One-sample t-test: A new
t-test}\label{one-sample-t-test-a-new-t-test}

Now we are ready to talk about t-test. We will talk about three of them.
We start with the one-sample t-test.

Commonly, the one-sample t-test is used to estimate the chances that
your sample came from a particular population. Specifically, you might
want to know whether the mean that you found from your sample, could
have come from a particular population having a particular mean.

Straight away, the one-sample t-test becomes a little confusing (and I
haven't even described it yet). Officially, it uses known parameters
from the population, like the mean of the population and the standard
deviation of the population. However, most times you don't know those
parameters of the population! So, you have to estimate them from your
sample. Remember from the chapters on descriptive statistics and
sampling, our sample mean is an unbiased estimate of the population
mean. And, our sample standard deviation (the one where we divide by
n-1) is an unbiased estimate of the population standard deviation. When
Gosset developed the t-test, he recognized that he could use these
estimates from his samples, to make the t-test. Here is the formula for
the one sample t-test, we first use words, and then become more
specific:

\subsection{Formulas for one-sample
t-test}\label{formulas-for-one-sample-t-test}

\(\text{name of statistic} = \frac{\text{measure of effect}}{\text{measure of error}}\)

\(\text{t} = \frac{\text{measure of effect}}{\text{measure of error}}\)

\(\text{t} = \frac{\text{Mean difference}}{\text{standard error}}\)

\(\text{t} = \frac{\bar{X}-u}{S_{\bar{X}}}\)

\(\text{t} = \frac{\text{Sample Mean  - Population Mean}}{\text{Sample Standard Error}}\)

\(\text{Estimated Standard Error} = \text{Standard Error of Sample} = \frac{s}{\sqrt{N}}\)

Where, s is the sample standard deviation.

Some of you may have gone cross-eyed looking at all of this. Remember,
we've seen it before when we divided our mean by the standard deviation
in the first bit. The t-test is just a measure of a sample mean, divided
by the standard error of the sample mean. That is it.

\subsection{What does t represent?}\label{what-does-t-represent}

\(t\) gives us a measure of confidence, just like our previous ratio for
dividing the mean by a standard deviations. The only difference with
\(t\), is that we divide by the standard error of mean (remember, this
is also a standard deviation, it is the standard deviation of the
sampling distribution of the mean)

\begin{tcolorbox}[enhanced jigsaw, title=\textcolor{quarto-callout-note-color}{\faInfo}\hspace{0.5em}{Note}, colframe=quarto-callout-note-color-frame, colbacktitle=quarto-callout-note-color!10!white, bottomtitle=1mm, leftrule=.75mm, rightrule=.15mm, titlerule=0mm, arc=.35mm, colback=white, opacitybacktitle=0.6, toprule=.15mm, toptitle=1mm, bottomrule=.15mm, coltitle=black, breakable, left=2mm, opacityback=0]

What does the t in t-test stand for? Apparently nothing. Gosset
originally labelled it z. And, Fisher later called it t, perhaps because
t comes after s, which is often used for the sample standard deviation.

\end{tcolorbox}

\(t\) is a property of the data that you collect. You compute it with a
sample mean, and a sample standard error (there's one more thing in the
one-sample formula, the population mean, which we get to in a moment).
This is why we call \(t\), a sample-statistic. It's a statistic we
compute from the sample.

What kinds of numbers should we expect to find for these \(ts\)? How
could we figure that out?

Let's start small and work through some examples. Imagine your sample
mean is 5. You want to know if it came from a population that also has a
mean of 5. In this case, what would \(t\) be? It would be zero: we first
subtract the sample mean from the population mean, \(5-5=0\). Because
the numerator is 0, \(t\) will be zero. So, \(t\) = 0, occurs, when
there is no difference.

Let's say you take another sample, do you think the mean will be 5 every
time, probably not. Let's say the mean is 6. So, what can \(t\) be here?
It will be a positive number, because 6-5= +1. But, will \(t\) be +1?
That depends on the standard error of the sample. If the standard error
of the sample is 1, then \(t\) could be 1, because 1/1 = 1.

If the sample standard error is smaller than 1, what happens to \(t\)?
It get's bigger right? For example, 1 divided by 0.5 = 2. If the sample
standard error was 0.5, \(t\) would be 2. And, what could we do with
this information? Well, it be like a measure of confidence. As \(t\)
get's bigger we could be more confident in the mean difference we are
measuring.

Can \(t\) be smaller than 1? Sure, it can. If the sample standard error
is big, say like 2, then \(t\) will be smaller than one (in our case),
e.g., 1/2 = .5. The direction of the difference between the sample mean
and population mean, can also make the \(t\) become negative. What if
our sample mean was 4. Well, then \(t\) will be negative, because the
mean difference in the numerator will be negative, and the number in the
bottom (denominator) will always be positive (remember why, it's the
standard error, computed from the sample standard deviation, which is
always positive because of the squaring that we did.).

So, that is some intuitions about what the kinds of values t can take.
\(t\) can be positive or negative, and big or small.

Let's do one more thing to build our intuitions about what \(t\) can
look like. How about we sample some numbers and then measure the sample
mean \textbf{and} the standard error of the mean, and then plot those
two things against each each. This will show us how a sample mean
typically varies with respect to the standard error of the mean.

In Figure~\ref{fig-7sampleMSEM}, I pulled 1,000 samples of \(N = 10\)
from a normal distribution (mean = 0, sd = 1). Each time I measured the
mean and standard error of the sample. That gave two descriptive
statistics for each sample, letting us plot each sample as dot in a
scatter plot.

\begin{figure}

\centering{

\includegraphics[width=0.75\linewidth,height=\textheight,keepaspectratio]{07-ttests_files/figure-pdf/fig-7sampleMSEM-1.pdf}

}

\caption{\label{fig-7sampleMSEM}A scatter plot with sample mean on the
x-axis, and standard error of the mean on the y-axis}

\end{figure}%

What we get is a cloud of dots. You might notice the cloud has a
circular quality. There's more dots in the middle, and fewer dots as
they radiate out from the middle. The dot cloud shows us the general
range of the sample mean, for example most of the dots are in between -1
and 1. Similarly, the range for the sample standard error is roughly
between .2 and .5. Remember, each dot represents one sample.

We can look at the same data a different way. For example, rather than
using a scatter plot, we can divide the mean for each dot by the
standard error for each dot. Figure~\ref{fig-7thisexample} shows the
result in a histogram.

\begin{figure}

\centering{

\includegraphics[width=0.75\linewidth,height=\textheight,keepaspectratio]{07-ttests_files/figure-pdf/fig-7thisexample-1.pdf}

}

\caption{\label{fig-7thisexample}A histogram of the sample means divided
by the sample standard errors, this is a t-distribution.}

\end{figure}%

Interesting, we can see the histogram is shaped like a normal curve. It
is centered on 0, which is the most common value. As values become more
extreme, they become less common. If you remember, our formula for
\(t\), was the mean divided by the standard error of the mean. That's
what we did here. This histogram is showing you a \(t\)-distribution.

\subsection{Calculating t from data}\label{calculating-t-from-data}

Let's briefly calculate a t-value from a small sample. Let's say we had
10 students do a true/false quiz with 5 questions on it. There's a 50\%
chance of getting each answer correct.

Every student completes the 5 questions, we grade them, and then we find
their performance (mean percent correct). What we want to know is
whether the students were guessing. If they were all guessing, then the
sample mean should be about 50\%, it shouldn't be different from chance,
which is 50\%. Let's look at Table~\ref{tbl-tsmall}.

\begin{longtable}[]{@{}lllll@{}}

\caption{\label{tbl-tsmall}Calculating the t-value for a one-sample
test.}

\tabularnewline

\toprule\noalign{}
students & scores & mean & Difference\_from\_Mean &
Squared\_Deviations \\
\midrule\noalign{}
\endhead
\bottomrule\noalign{}
\endlastfoot
1 & 50 & 61 & -11 & 121 \\
2 & 70 & 61 & 9 & 81 \\
3 & 60 & 61 & -1 & 1 \\
4 & 40 & 61 & -21 & 441 \\
5 & 80 & 61 & 19 & 361 \\
6 & 30 & 61 & -31 & 961 \\
7 & 90 & 61 & 29 & 841 \\
8 & 60 & 61 & -1 & 1 \\
9 & 70 & 61 & 9 & 81 \\
10 & 60 & 61 & -1 & 1 \\
Sums & 610 & 610 & 0 & 2890 \\
Means & 61 & 61 & 0 & 289 \\
& & & sd & 17.92 \\
& & & SEM & 5.67 \\
& & & t & 1.94003527336861 \\

\end{longtable}

You can see the \texttt{scores} column has all of the test scores for
each of the 10 students. We did the things we need to do to compute the
standard deviation.

Remember the sample standard deviation is the square root of the sample
variance, or:

\(\text{sample standard deviation} = \sqrt{\frac{\sum_{i}^{n}({x_{i}-\bar{x})^2}}{N-1}}\)

\(\text{sd} = \sqrt{\frac{2890}{10-1}} = 17.92\)

The standard error of the mean, is the standard deviation divided by the
square root of N

\(\text{SEM} = \frac{s}{\sqrt{N}} = \frac{17.92}{10} = 5.67\)

\(t\) is the difference between our sample mean (61), and our population
mean (50, assuming chance), divided by the standard error of the mean.

\(\text{t} = \frac{\bar{X}-u}{S_{\bar{X}}} = \frac{\bar{X}-u}{SEM} = \frac{61-50}{5.67} = 1.94\)

And, that is you how calculate \(t\), by hand. It's a pain. I was
annoyed doing it this way. In the lab, you learn how to calculate \(t\)
using software, so it will just spit out \(t\). For example in R, all
you have to do is this:

\begin{Shaded}
\begin{Highlighting}[]
\FunctionTok{t.test}\NormalTok{(scores, }\AttributeTok{mu=}\DecValTok{50}\NormalTok{)}
\CommentTok{\#\textgreater{} }
\CommentTok{\#\textgreater{}  One Sample t{-}test}
\CommentTok{\#\textgreater{} }
\CommentTok{\#\textgreater{} data:  scores}
\CommentTok{\#\textgreater{} t = 1.9412, df = 9, p{-}value = 0.08415}
\CommentTok{\#\textgreater{} alternative hypothesis: true mean is not equal to 50}
\CommentTok{\#\textgreater{} 95 percent confidence interval:}
\CommentTok{\#\textgreater{}  48.18111 73.81889}
\CommentTok{\#\textgreater{} sample estimates:}
\CommentTok{\#\textgreater{} mean of x }
\CommentTok{\#\textgreater{}        61}
\end{Highlighting}
\end{Shaded}

\subsection{How does t behave?}\label{how-does-t-behave}

If \(t\) is just a number that we can compute from our sample (it is),
what can we do with it? How can we use \(t\) for statistical inference?

Remember back to the chapter on sampling and distributions, that's where
we discussed the sampling distribution of the sample mean. Remember, we
made a lot of samples, then computed the mean for each sample, then we
plotted a histogram of the sample means. Later, in that same section, we
mentioned that we could generate sampling distributions for any
statistic. For each sample, we could compute the mean, the standard
deviation, the standard error, and now even \(t\), if we wanted to. We
could generate 10,000 samples, and draw four histograms, one for each
sampling distribution for each statistic.

This is exactly what I did, and the results are shown in the four panels
of Figure~\ref{fig-7sampdists} below. I used a sample size of 20, and
drew random observations for each sample from a normal distribution,
with mean = 0, and standard deviation = 1. Let's look at the sampling
distributions for each of the statistics. \(t\) was computed assuming
with the population mean assumed to be 0.

\begin{figure}

\centering{

\includegraphics[width=1\linewidth,height=\textheight,keepaspectratio]{07-ttests_files/figure-pdf/fig-7sampdists-1.pdf}

}

\caption{\label{fig-7sampdists}Sampling distributions for the mean,
standard deviation, standard error of the mean, and \(t\).}

\end{figure}%

We see four sampling distributions. This is how statistical summaries of
these summaries behave. We have used the word chance windows before.
These are four chance windows, measuring different aspects of the
sample. In this case, all of the samples came from the same normal
distribution. Because of sampling error, each sample is not identical.
The means are not identical, the standard deviations are not identical,
sample standard error of the means are not identical, and the \(t\)s of
the samples are not identical. They all have some variation, as shown by
the histograms. This is how samples of size 20 behave.

We can see straight away, that in this case, we are unlikely to get a
sample mean of 2. That's way outside the window. The range for the
sampling distribution of the mean is around -.5 to +.5, and is centered
on 0 (the population mean, would you believe!).

We are unlikely to get sample standard deviations of between .6 and 1.5,
that is a different range, specific to the sample standard deviation.

Same thing with the sample standard error of the mean, the range here is
even smaller, mostly between .1, and .3. You would rarely find a sample
with a standard error of the mean greater than .3. Virtually never would
you find one of say 1 (for this situation).

Now, look at \(t\). It's range is basically between -3 and +3 here. 3s
barely happen at all. You pretty much never see a 5 or -5 in this
situation.

All of these sampling windows are chance windows, and they can all be
used in the same way as we have used similar sampling distributions
before (e.g., Crump Test, and Randomization Test) for statistical
inference. For all of them we would follow the same process:

\begin{enumerate}
\def\labelenumi{\arabic{enumi}.}
\tightlist
\item
  Generate these distributions
\item
  Look at your sample statistics for the data you have (mean, SD, SEM,
  and \(t\))
\item
  Find the likelihood of obtaining that value or greater
\item
  Obtain that probability
\item
  See if you think your sample statistics were probable or improbable.
\end{enumerate}

We'll formalize this in a second. I just want you to know that what you
will be doing is something that you have already done before. For
example, in the Crump test and the Randomization test we focused on the
distribution of mean differences. We could do that again here, but
instead, we will focus on the distribution of \(t\) values. We then
apply the same kinds of decision rules to the \(t\) distribution, as we
did for the other distributions. Below you will see a graph you have
already seen, except this time it is a distribution of \(t\)s, not mean
differences:

Remember, if we obtained a single \(t\) from one sample we collected, we
could consult the chance window in Figure~\ref{fig-7tdecision} below to
find out whether the \(t\) we obtained from the sample was likely or
unlikely to occur by chance.

\begin{figure}

\centering{

\includegraphics[width=0.75\linewidth,height=\textheight,keepaspectratio]{07-ttests_files/figure-pdf/fig-7tdecision-1.pdf}

}

\caption{\label{fig-7tdecision}Applying decision criteria to the
\(t\)-distribution. Histogram of \(t\)s from samples (n=20) drawn from
the same normal distribution (u=0, sd=1)}

\end{figure}%

\subsection{Making a decision}\label{making-a-decision}

From our early example involving the TRUE/FALSE quizzes, we are now
ready to make some kind of decision about what happened there. We found
a mean difference of 11. We found a \(t\) = 1.9411765. The probability
of this \(t\) or larger occurring is \(p\) = 0.0841503. We were testing
the idea that our sample mean of 61 could have come from a normal
distribution with mean = 50. The \(t\) test tells us that the \(t\) for
our sample, or a larger one, would happen with p = 0.0841503. In other
words, chance can do it a kind of small amount of time, but not often.
In English, this means that all of the students could have been
guessing, but it wasn't that likely that were just guessing.

The next \(t\)-test is called a \textbf{paired samples t-test}. And,
spoiler alert, we will find out that a paired samples t-test is actually
a one-sample t-test in disguise (WHAT!), yes it is. If the one-sample
\(t\)-test didn't make sense to you, read the next section.

\section{Paired-samples t-test}\label{paired-samples-t-test}

For me (Crump), many analyses often boil down to a paired samples
t-test. It just happens that many things I do reduce down to a test like
this.

I am a cognitive psychologist, I conduct research about how people do
things like remember, pay attention, and learn skills. There are lots of
Psychologists like me, who do very similar things.

We all often conduct the same kinds of experiments. They go like this,
and they are called \textbf{repeated measures} designs. They are called
\textbf{repeated measures} designs, because we measure how one person
does something more than once, we \textbf{repeat} the measure.

So, I might measure somebody doing something in condition A, and measure
the same person doing something in Condition B, and then I see that same
person does different things in the two conditions. I \textbf{repeatedly
measure} the same person in both conditions. I am interested in whether
the experimental manipulation changes something about how people perform
the task in question.

\subsection{Mehr, Song, and Spelke
(2016)}\label{mehr-song-and-spelke-2016}

We will introduce the paired-samples t-test with an example using real
data, from a real study. Mehr, Song, and Spelke (2016) were interested
in whether singing songs to infants helps infants become more sensitive
to social cues. For example, infants might need to learn to direct their
attention toward people as a part of learning how to interact socially
with people. Perhaps singing songs to infants aids this process of
directing attention. When an infant hears a familiar song, they might
start to pay more attention to the person singing that song, even after
they are done singing the song. The person who sang the song might
become more socially important to the infant. You will learn more about
this study in the lab for this week. This example, prepares you for the
lab activities. Here is a brief summary of what they did.

First, parents were trained to sing a song to their infants. After many
days of singing this song to the infants, a parent came into the lab
with their infant. In the first session, parents sat with their infants
on their knees, so the infant could watch two video presentations. There
were two videos. Each video involved two unfamiliar new people the
infant had never seen before. Each new person in the video (the singers)
sang one song to the infant. One singer sang the ``familiar'' song the
infant had learned from their parents. The other singer sang an
``unfamiliar'' song the infant had not hear before.

There were two really important measurement phases: the baseline phase,
and the test phase.

The baseline phase occurred before the infants saw and heard each singer
sing a song. During the baseline phase, the infants watched a video of
both singers at the same time. The researchers recorded the proportion
of time that the infant looked at each singer. The baseline phase was
conducted to determine whether infants had a preference to look at
either person (who would later sing them a song).

The test phase occurred \textbf{after} infants saw and heard each song,
sung by each singer. During the test phase, each infant had an
opportunity to watch silent videos of both singers. The researchers
measured the proportion of time the infants spent looking at each
person. The question of interest, was whether the infants would spend a
greater proportion of time looking at the singer who sang the familiar
song, compared to the singer who sang the unfamiliar song.

There is more than one way to describe the design of this study. We will
describe it like this. It was a repeated measures design, with one
independent (manipulation) variable called Viewing phase: Baseline
versus Test. There was one dependent variable (the measurement), which
was proportion looking time (to singer who sung familiar song). This was
a repeated measures design because the researchers measured proportion
looking time twice (they repeated the measure), once during baseline
(before infants heard each singer sing a song), and again during test
(after infants head each singer sing a song).

The important question was whether infants would change their looking
time, and look more at the singer who sang the familiar song during the
test phase, than they did during the baseline phase. This is a question
about a change within individual infants. In general, the possible
outcomes for the study are:

\begin{enumerate}
\def\labelenumi{\arabic{enumi}.}
\item
  No change: The difference between looking time toward the singer of
  the familiar song during baseline and test is zero, no difference.
\item
  Positive change: Infants will look longer toward the singer of the
  familiar song during the test phase (after they saw and heard the
  singers), compared to the baseline phase (before they saw and heard
  the singers). This is a positive difference if we use the formula:
  Test Phase Looking time - Baseline phase looking time (to familiar
  song singer).
\item
  Negative change: Infants will look longer toward the singer of the
  unfamiliar song during the test phase (after they saw and heard the
  singers), compared to the baseline phase (before they saw and heard
  the singers). This is a negative difference if we use the same
  formula: Test Phase Looking time - Baseline phase looking time (to
  familiar song singer).
\end{enumerate}

\subsection{The data}\label{the-data}

Let's take a look at the data for the first 5 infants in the study. This
will help us better understand some properties of the data before we
analyze it. We will see that the data is structured in a particular way
that we can take advantage of with a paired samples t-test. Note, we
look at the first 5 infants to show how the computations work. The
results of the paired-samples t-test change when we use all of the data
from the study.

Here is a table of the data:

\begin{longtable}[]{@{}rrr@{}}
\toprule\noalign{}
infant & Baseline & Test \\
\midrule\noalign{}
\endhead
\bottomrule\noalign{}
\endlastfoot
1 & 0.44 & 0.60 \\
2 & 0.41 & 0.68 \\
3 & 0.75 & 0.72 \\
4 & 0.44 & 0.28 \\
5 & 0.47 & 0.50 \\
\end{longtable}

The table shows proportion looking times toward the singer of the
familiar song during the Baseline and Test phases. Notice there are five
different infants, (1 to 5). Each infant is measured twice, once during
the Baseline phase, and once during the Test phase. To repeat from
before, this is a repeated-measures design, because the infants are
measured repeatedly (twice in this case). Or, this kind of design is
also called a \textbf{paired-samples} design. Why? because each
participant comes with a pair of samples (two samples), one for each
level of the design.

Great, so what are we really interested in here? We want to know if the
mean looking time toward the singer of the familiar song for the Test
phase is higher than the Baseline phase. We are comparing the two sample
means against each other and looking for a difference. We already know
that differences could be obtained by chance alone, simply because we
took two sets of samples, and we know that samples can be different. So,
we are interested in knowing whether chance was likely or unlikely to
have produced any difference we might observe.

In other words, we are interested in looking at the difference scores
between the baseline and test phase for each infant. The question here
is, for each infant, did their proportion looking time to the singer of
the familiar song, increase during the test phase as compared to the
baseline phase.

\subsection{The difference scores}\label{the-difference-scores}

Let's add the difference scores to the table of data so it is easier to
see what we are talking about. The first step in creating difference
scores is to decide how you will take the difference, there are two
options:

\begin{enumerate}
\def\labelenumi{\arabic{enumi}.}
\tightlist
\item
  Test phase score - Baseline Phase Score
\item
  Baseline phase score - Test Phase score
\end{enumerate}

Let's use the first formula. Why? Because it will give us positive
differences when the test phase score is higher than the baseline phase
score. This makes a positive score meaningful with respect to the study
design, we know (because we defined it to be this way), that positive
scores will refer to longer proportion looking times (to singer of
familiar song) during the test phase compared to the baseline phase.

\begin{longtable}[]{@{}rrrr@{}}
\toprule\noalign{}
infant & Baseline & Test & differences \\
\midrule\noalign{}
\endhead
\bottomrule\noalign{}
\endlastfoot
1 & 0.44 & 0.60 & 0.16 \\
2 & 0.41 & 0.68 & 0.27 \\
3 & 0.75 & 0.72 & -0.03 \\
4 & 0.44 & 0.28 & -0.16 \\
5 & 0.47 & 0.50 & 0.03 \\
\end{longtable}

There we have it, the difference scores. The first thing we can do here
is look at the difference scores, and ask how many infants showed the
effect of interest. Specifically, how many infants showed a positive
difference score. We can see that three of five infants showed a
positive difference (they looked more at the singer of the familiar song
during the test than baseline phase), and two the infants showed the
opposite effect (negative difference, they looked more at the singer of
the familiar song during baseline than test).

\subsection{The mean difference}\label{the-mean-difference}

As we have been discussing, the effect of interest in this study is the
mean difference between the baseline and test phase proportion looking
times. We can calculate the \textbf{mean difference}, by finding the
\textbf{mean of the difference scores}. Let's do that, in fact, for fun
let's calculate the mean of the baseline scores, the test scores, and
the difference scores.

\begin{longtable}[]{@{}llll@{}}
\toprule\noalign{}
infant & Baseline & Test & differences \\
\midrule\noalign{}
\endhead
\bottomrule\noalign{}
\endlastfoot
1 & 0.44 & 0.6 & 0.16 \\
2 & 0.41 & 0.68 & 0.27 \\
3 & 0.75 & 0.72 & -0.03 \\
4 & 0.44 & 0.28 & -0.16 \\
5 & 0.47 & 0.5 & 0.03 \\
Sums & 2.51 & 2.78 & 0.27 \\
Means & 0.502 & 0.556 & 0.054 \\
\end{longtable}

We can see there was a positive mean difference of 0.054, between the
test and baseline phases.

Can we rush to judgment and conclude that infants are more socially
attracted to individuals who have sung them a familiar song? I would
hope not based on this very small sample. First, the difference in
proportion looking isn't very large, and of course we recognize that
this difference could have been produced by chance.

We will more formally evaluate whether this difference could have been
caused by chance with the paired-samples t-test. But, before we do that,
let's again calculate \(t\) and discuss what \(t\) tells us over and
above what our measure of the mean of the difference scores tells us.

\subsection{Calculate t}\label{calculate-t}

OK, so how do we calculate \(t\) for a paired-samples \(t\)-test?
Surprise, we use the one-sample t-test formula that you already learned
about! Specifically, we use the one-sample \(t\)-test formula on the
difference scores. We have one sample of difference scores (you can see
they are in one column), so we can use the one-sample \(t\)-test on the
difference scores. Specifically, we are interested in comparing whether
the mean of our difference scores came from a distribution with mean
difference = 0. This is a special distribution we refer to as the
\textbf{null distribution}. It is the distribution no differences. Of
course, this \textbf{null distribution} can produce differences due to
to sampling error, but those differences are not caused by any
experimental manipulation, they caused by the random sampling process.

We calculate \(t\) in a moment. Let's now consider again why we want to
calculate \(t\)? Why don't we just stick with the mean difference we
already have?

Remember, the whole concept behind \(t\), is that it gives an indication
of how confident we should be in our mean. Remember, \(t\) involves a
measure of the mean in the numerator, divided by a measure of variation
(standard error of the sample mean) in the denominator. The resulting
\(t\) value is small when the mean difference is small, or when the
variation is large. So small \(t\)-values tell us that we shouldn't be
that confident in the estimate of our mean difference. Large
\(t\)-values occur when the mean difference is large and/or when the
measure of variation is small. So, large \(t\)-values tell us that we
can be more confident in the estimate of our mean difference. Let's find
\(t\) for the mean difference scores. We use the same formulas as we did
last time:

\begin{longtable}[]{@{}
  >{\raggedright\arraybackslash}p{(\linewidth - 10\tabcolsep) * \real{0.1000}}
  >{\raggedright\arraybackslash}p{(\linewidth - 10\tabcolsep) * \real{0.1286}}
  >{\raggedright\arraybackslash}p{(\linewidth - 10\tabcolsep) * \real{0.0857}}
  >{\raggedright\arraybackslash}p{(\linewidth - 10\tabcolsep) * \real{0.1714}}
  >{\raggedright\arraybackslash}p{(\linewidth - 10\tabcolsep) * \real{0.2143}}
  >{\raggedright\arraybackslash}p{(\linewidth - 10\tabcolsep) * \real{0.3000}}@{}}
\toprule\noalign{}
\begin{minipage}[b]{\linewidth}\raggedright
infant
\end{minipage} & \begin{minipage}[b]{\linewidth}\raggedright
Baseline
\end{minipage} & \begin{minipage}[b]{\linewidth}\raggedright
Test
\end{minipage} & \begin{minipage}[b]{\linewidth}\raggedright
differences
\end{minipage} & \begin{minipage}[b]{\linewidth}\raggedright
diff\_from\_mean
\end{minipage} & \begin{minipage}[b]{\linewidth}\raggedright
Squared\_differences
\end{minipage} \\
\midrule\noalign{}
\endhead
\bottomrule\noalign{}
\endlastfoot
1 & 0.44 & 0.6 & 0.16 & 0.106 & 0.011236 \\
2 & 0.41 & 0.68 & 0.27 & 0.216 & 0.046656 \\
3 & 0.75 & 0.72 & -0.03 & -0.084 & 0.00705600000000001 \\
4 & 0.44 & 0.28 & -0.16 & -0.214 & 0.045796 \\
5 & 0.47 & 0.5 & 0.03 & -0.024 & 0.000575999999999999 \\
Sums & 2.51 & 2.78 & 0.27 & 0 & 0.11132 \\
Means & 0.502 & 0.556 & 0.054 & 0 & 0.022264 \\
& & & & sd & 0.167 \\
& & & & SEM & 0.075 \\
& & & & t & 0.72 \\
\end{longtable}

If we did this test using R, we would obtain almost the same numbers
(there is a little bit of rounding in the table).

\begin{verbatim}
#> 
#>  One Sample t-test
#> 
#> data:  differences
#> t = 0.72381, df = 4, p-value = 0.5092
#> alternative hypothesis: true mean is not equal to 0
#> 95 percent confidence interval:
#>  -0.1531384  0.2611384
#> sample estimates:
#> mean of x 
#>     0.054
\end{verbatim}

Here is a quick write up of our t-test results, t(4) = .72, p = .509.

What does all of that tell us? There's a few things we haven't gotten
into much yet. For example, the 4 represents degrees of freedom, which
we discuss later. The important part, the \(t\) value should start to be
a little bit more meaningful. We got a kind of small t-value didn't we.
It's .72. What can we tell from this value? First, it is positive, so we
know the mean difference is positive. The sign of the \(t\)-value is
always the same as the sign of the mean difference (ours was +0.054). We
can also see that the p-value was .509. We've seen p-values before. This
tells us that our \(t\) value or larger, occurs about 50.9\% of the
time\ldots{} Actually it means more than this. And, to understand it, we
need to talk about the concept of two-tailed and one-tailed tests.

\subsection{\texorpdfstring{Interpreting
\(t\)s}{Interpreting ts}}\label{interpreting-ts}

Remember what it is we are doing here. We are evaluating whether our
sample data could have come from a particular kind of distribution. The
null distribution of no differences. This is the distribution of
\(t\)-values that would occur for samples of size 5, with a mean
difference of 0, and a standard error of the sample mean of .075 (this
is the SEM that we calculated from our sample). We can see what this
particular null-distribution looks like in Figure~\ref{fig-7tnull}.

\begin{figure}

\centering{

\includegraphics[width=0.75\linewidth,height=\textheight,keepaspectratio]{07-ttests_files/figure-pdf/fig-7tnull-1.pdf}

}

\caption{\label{fig-7tnull}A distribution of \(t\)-values that can occur
by chance alone, when there is no difference between the sample and a
population}

\end{figure}%

The \(t\)-distribution above shows us the kinds of values \(t\) will
will take by chance alone, when we measure the mean differences for
pairs of 5 samples (like our current). \(t\) is most likely to be zero,
which is good, because we are looking at the distribution of
no-differences, which should most often be 0! But, sometimes, due to
sampling error, we can get \(t\)s that are bigger than 0, either in the
positive or negative direction. Notice the distribution is symmetrical,
a \(t\) from the null-distribution will be positive half of the time,
and negative half of the time, that is what we would expect by chance.

So, what kind of information do we want know when we find a particular
\(t\) value from our sample? We want to know how likely the \(t\) value
like the one we found occurs just by chance. This is actually a subtly
nuanced kind of question. For example, any particular \(t\) value
doesn't have a specific probability of occurring. When we talk about
probabilities, we are talking about ranges of probabilities. Let's
consider some probabilities. We will use the letter \(p\), to talk about
the probabilities of particular \(t\) values.

\begin{enumerate}
\def\labelenumi{\arabic{enumi}.}
\item
  What is the probability that \(t\) is zero or positive or negative?
  The answer is p=1, or 100\%. We will always have a \(t\) value that is
  zero or non-zero\ldots Actually, if we can't compute the t-value, for
  example when the standard deviation is undefined, I guess then we
  would have a non-number. But, assuming we can calculate \(t\), then it
  will always be 0 or positive or negative.
\item
  What is the probability of \(t\) = 0 or greater than 0? The answer is
  p=.5, or 50\%. 50\% of \(t\)-values are 0 or greater.
\item
  What is the of \(t\) = 0 or smaller than 0? The answer is p=.5, or
  50\%. 50\% of \(t\)-values are 0 or smaller.
\end{enumerate}

We can answer all of those questions just by looking at our
t-distribution, and dividing it into two equal regions, the left side
(containing 50\% of the \(t\) values), and the right side containing
50\% of the \(t\)-values).

What if we wanted to take a more fine-grained approach, let's say we
were interested in regions of 10\%. What kinds of \(t\)s occur 10\% of
the time. We would apply lines like the following. Notice, the
likelihood of bigger numbers (positive or negative) gets smaller, so we
have to increase the width of the bars for each of the intervals between
the bars to contain 10\% of the \(t\)-values, it looks like
Figure~\ref{fig-7percentregions}.

\begin{figure}

\centering{

\includegraphics[width=1\linewidth,height=\textheight,keepaspectratio]{07-ttests_files/figure-pdf/fig-7percentregions-1.pdf}

}

\caption{\label{fig-7percentregions}Splitting the t distribution up into
regions each containing 10\% of the \(t\)-values. The width between the
bars narrows as they approach the center of the distribution, where
there are more \(t\)-values.}

\end{figure}%

Consider the probabilities (\(p\)) of \(t\) for the different ranges.

\begin{enumerate}
\def\labelenumi{\arabic{enumi}.}
\tightlist
\item
  \(t\) \textless= -1.5 (\(t\) is less than or equal to -1.5), \(p\) =
  10\%
\item
  -1.5 \textgreater= \(t\) \textless= -0.9 (\(t\) is equal to or between
  -1.5 and -.9), \(p\) = 10\%
\item
  -.9 \textgreater= \(t\) \textless= -0.6 (\(t\) is equal to or between
  -.9 and -.6), \(p\) = 10\%
\item
  \(t\) \textgreater= 1.5 (\(t\) is greater than or equal to 1.5), \(p\)
  = 10\%
\end{enumerate}

Notice, that the \(p\)s are always 10\%. \(t\)s occur in these ranges
with 10\% probability.

\subsection{\texorpdfstring{Getting the p-values for
\(t\)-values}{Getting the p-values for t-values}}\label{getting-the-p-values-for-t-values}

You might be wondering where I am getting some of these values from. For
example, how do I know that 10\% of \(t\) values (for this null
distribution) have a value of approximately 1.5 or greater than 1.5? The
answer is I used R to tell me.

In most statistics textbooks the answer would be: there is a table at
the back of the book where you can look these things up\ldots This
textbook has no such table. We could make one for you. And, we might do
that. But, we didn't do that yet\ldots{}

So, where do these values come from, how can you figure out what they
are? The complicated answer is that we are not going to explain the math
behind finding these values because, 1) the authors (some of us)
admittedly don't know the math well enough to explain it, and 2) it
would sidetrack us to much, 3) you will learn how to get these numbers
in the lab with software, 4) you will learn how to get these numbers in
lab without the math, just by doing a simulation, and 5) you can do it
in R, or excel, or you can use an
\href{http://www.socscistatistics.com/pvalues/tdistribution.aspx}{online
calculator}.

This is all to say that you can find the \(t\)s and their associated
\(p\)s using software. But, the software won't tell you what these
values mean. That's we are doing here. You will also see that software
wants to know a few more things from you, such as the degrees of freedom
for the test, and whether the test is one-tailed or two tailed. We
haven't explained any of these things yet. That's what we are going to
do now. Note, we explain degrees of freedom last. First, we start with a
one-tailed test.

\subsection{One-tailed tests}\label{one-tailed-tests}

A \textbf{one-tailed test} is sometimes also called a directional test.
It is called a directional test, because a researcher might have a
hypothesis in mind suggesting that the difference they observe in their
means is going to have a particular direction, either a positive
difference, or a negative difference.

Typically, a researcher would set an \textbf{alpha criterion}. The alpha
criterion describes a line in the sand for the researcher. Often, the
alpha criterion is set at \(p = .05\). What does this mean?
Figure~\ref{fig-7critT} shows the \(t\)-distribution and the alpha
criterion.

\begin{figure}

\centering{

\includegraphics[width=0.75\linewidth,height=\textheight,keepaspectratio]{07-ttests_files/figure-pdf/fig-7critT-1.pdf}

}

\caption{\label{fig-7critT}The critical value of t for an alpha
criterion of 0.05. 5\% of all ts are at this value or larger}

\end{figure}%

The figure shows that \(t\) values of +2.13 or greater occur 5\% of the
time. Because the t-distribution is symmetrical, we also know that \(t\)
values of -2.13 or smaller also occur 5\% of the time. Both of these
properties are true under the null distribution of no differences. This
means, that when there really are no differences, a researcher can
expect to find \(t\) values of 2.13 or larger 5\% of the time.

Let's review and connect some of the terms:

\begin{enumerate}
\def\labelenumi{\arabic{enumi}.}
\item
  \textbf{alpha criterion}: the criterion set by the researcher to make
  decisions about whether they believe chance did or did not cause the
  difference. The alpha criterion here is set to \(p = .05\).
\item
  \textbf{Critical} \(t\). The critical \(t\) is the \(t\)-value
  associated with the alpha-criterion. In this case for a one-tailed
  test, it is the \(t\) value where 5\% of all \(t\)s are this number or
  greater. In our example, the critical \(t\) is 2.13. 5\% of all \(t\)
  values (with degrees of freedom = 4) are +2.13, or greater than +2.13.
\item
  \textbf{Observed} \(t\). The observed \(t\) is the one that you
  calculated from your sample. In our example about the infants, the
  observed \(t\) was \(t\) (4) = 0.72.
\item
  \textbf{p-value}. The \(p\)-value is the probability of obtaining the
  observed \(t\) value or larger. Now, you could look back at our
  previous example, and find that the \(p\)-value for \(t\) (4) = .72,
  was \(p = .509\) . HOWEVER, this p-value was not calculated for a
  one-directional test\ldots(we talk about what .509 means in the next
  section).
\end{enumerate}

Figure~\ref{fig-7tonedirection} shows what the \(p\)-value for \(t\) (4)
= .72 using a one-directional test would would look like:

\begin{figure}

\centering{

\includegraphics[width=0.75\linewidth,height=\textheight,keepaspectratio]{07-ttests_files/figure-pdf/fig-7tonedirection-1.pdf}

}

\caption{\label{fig-7tonedirection}A case where the observed value of t
is much less than the critical value for a one-directional t-test.}

\end{figure}%

Let's take this one step at a time. We have located the observed \(t\)
of .72 on the graph. We shaded the right region all grey. What we see is
that the grey region represents .256 or 25.6\% of all \(t\) values. In
other words, 25.6\% of \(t\) values are .72 or larger than .72. You
could expect, by chance alone, to a find a \(t\) value of .72 or larger,
25.6\% of the time. That's fairly often. We did find a \(t\) value of
.72. Now that you know this kind of \(t\) value or larger occurs 25.6\%
of the time, would you be confident that the mean difference was not due
to chance? Probably not, given that chance can produce this difference
fairly often.

Following the ``standard'' decision making procedure, we would claim
that our \(t\) value was \textbf{not statistically significant}, because
it was not large enough. If our observed value was larger than the
critical \(t\) (larger than 2.13), defined by our alpha criterion, then
we would claim that our \(t\) value was \textbf{statistically
significant}. This would be equivalent to saying that we believe it is
unlikely that the difference we observed was due to chance. In general,
for any observed \(t\) value, the associated \(p\)-value tells you how
likely a \(t\) of the observed size or larger would be observed. The
\(p\)-value \textbf{always} refers to a \textbf{range} of \(t\)-values,
never to a single \(t\)-value. Researchers use the alpha criterion of
.05, as a matter of convenience and convention. There are other ways to
interpret these values that do not rely on a strict (significant versus
not) dichotomy.

\subsection{Two-tailed tests}\label{two-tailed-tests}

OK, so that was one-tailed tests\ldots{} What are two tailed tests? The
\(p\)-value that we originally calculated from our paired-samples
\(t\)-test was for a 2-tailed test. Often, the default is that the
\(p\)-value is for a two-tailed test.

The two-tailed test, is asking a more general question about whether a
difference is likely to have been produced by chance. The question is:
what is probability of any difference. It is also called a
\textbf{non-directional} test, because here we don't care about the
direction or sign of the difference (positive or negative), we just care
if there is any kind of difference.

The same basic things as before are involved. We define an alpha
criterion (\(\alpha = 0.05\)). And, we say that any observed \(t\) value
that has a probability of \(p\) \textless.05 (\(p\) is less than .05)
will be called \textbf{statistically significant}, and ones that are
more likely (\(p\) \textgreater.05, \(p\) is greater than .05) will be
called null-results, or not statistically significant. The only
difference is how we draw the alpha range. Before it was on the right
side of the \(t\) distribution (we were conducting a one-sided test
remember, so we were only interested in one side).

Figure~\ref{fig-7twotailedt} shows what the most extreme 5\% of the
\(t\)-values are when we ignore their sign (whether they are positive or
negative).

\begin{figure}

\centering{

\includegraphics[width=0.75\linewidth,height=\textheight,keepaspectratio]{07-ttests_files/figure-pdf/fig-7twotailedt-1.pdf}

}

\caption{\label{fig-7twotailedt}Critical values for a two-tailed test.
Each line represents the location where 2.5\% of all \(t\)s are larger
or smaller than critical value. The total for both tails is 5\%}

\end{figure}%

Here is what we are seeing. A distribution of no differences (the null,
which is what we are looking at), will produce \(t\)s that are 2.78 or
greater 2.5\% of the time, and \(t\)s that are -2.78 or smaller 2.5\% of
the time. 2.5\% + 2.5\% is a total of 5\% of the time. We could also say
that \(t\)s larger than +/- 2.78 occur 5\% of the time.

As a result, the critical \(t\) value is (+/-) 2.78 for a two-tailed
test. As you can see, the two-tailed test is blind to the direction or
sign of the difference. Because of this, the critical \(t\) value is
also higher for a two-tailed test, than for the one-tailed test that we
did earlier. Hopefully, now you can see why it is called a two-tailed
test. There are two tails of the distribution, one on the left and
right, both shaded in green.

\subsection{One or two tailed, which
one?}\label{one-or-two-tailed-which-one}

Now that you know there are two kinds of tests, one-tailed, and
two-tailed, which one should you use? There is some conventional wisdom
on this, but also some debate. In the end, it is up to you to be able to
justify your choice and why it is appropriate for you data. That is the
real answer.

The conventional answer is that you use a one-tailed test when you have
a theory or hypothesis that is making a directional prediction (the
theory predicts that the difference will be positive, or negative).
Similarly, use a two-tailed test when you are looking for any
difference, and you don't have a theory that makes a directional
prediction (it just makes the prediction that there will be a
difference, either positive or negative).

Also, people appear to choose one or two-tailed tests based on how risky
they are as researchers. If you always ran one-tailed tests, your
critical \(t\) values for your set alpha criterion would always be
smaller than the critical \(t\)s for a two-tailed test. Over the long
run, you would make more type I errors, because the criterion to detect
an effect is a lower bar for one than two tailed tests.

\begin{quote}
Remember type 1 errors occur when you reject the idea that chance could
have caused your difference. You often never know when you make this
error. It happens anytime that sampling error was the actual cause of
the difference, but a researcher dismisses that possibility and
concludes that their manipulation caused the difference.
\end{quote}

Similarly, if you always ran two-tailed tests, even when you had a
directional prediction, you would make fewer type I errors over the long
run, because the \(t\) for a two-tailed test is higher than the \(t\)
for a one-tailed test. It seems quite common for researchers to use a
more conservative two-tailed test, even when they are making a
directional prediction based on theory. In practice, researchers tend to
adopt a standard for reporting that is common in their field. Whether or
not the practice is justifiable can sometimes be an open question. The
important task for any researcher, or student learning statistics, is to
be able to justify their choice of test.

\subsection{Degrees of freedom}\label{degrees-of-freedom}

Before we finish up with paired-samples \(t\)-tests, we should talk
about degrees of freedom. Our sense is that students don't really
understand degrees of freedom very well. If you are reading this
textbook, you are probably still wondering what is degrees of freedom,
seeing as we haven't really talked about it all.

For the \(t\)-test, there is a formula for degrees of freedom. For the
one-sample and paired sample \(t\)-tests, the formula is:

\(\text{Degrees of Freedom} = \text{df} = n-1\). Where n is the number
of samples in the test.

In our paired \(t\)-test example, there were 5 infants. Therefore,
degrees of freedom = 5-1 = 4.

OK, that's a formula. Who cares about degrees of freedom, what does the
number mean? And why do we report it when we report a \(t\)-test\ldots{}
you've probably noticed the number in parentheses e.g., \(t\)(4)=.72,
the 4 is the \(df\), or degrees of freedom.

Degrees of freedom is both a concept, and a correction. The concept is
that if you estimate a property of the numbers, and you use this
estimate, you will be forcing some constraints on your numbers.

Consider the numbers: 1, 2, 3. The mean of these numbers is 2. Now,
let's say I told you that the mean of three numbers is 2. Then, how many
of these three numbers have freedom? Funny question right. What we mean
is, how many of the three numbers could be any number, or have the
freedom to be any number.

The first two numbers could be any number. But, once those two numbers
are set, the final number (the third number), MUST be a particular
number that makes the mean 2. The first two numbers have freedom. The
third number has no freedom.

To illustrate. Let's freely pick two numbers: 51 and -3. I used my
personal freedom to pick those two numbers. Now, if our three numbers
are 51, -3, and x, and the mean of these three numbers is 2. There is
only one solution, x has to be -42, otherwise the mean won't be 2. This
is one way to think about degrees of freedom. The degrees of freedom for
these three numbers is n-1 = 3-1= 2, because 2 of the numbers can be
free, but the last number has no freedom, it becomes fixed after the
first two are decided.

Now, statisticians often apply degrees of freedom to their calculations,
especially when a second calculation relies on an estimated value. For
example, when we calculate the standard deviation of a sample, we first
calculate the mean of the sample right! By estimating the mean, we are
fixing an aspect of our sample, and so, our sample now has n-1 degrees
of freedom when we calculate the standard deviation (remember for the
sample standard deviation, we divide by n-1\ldots there's that n-1
again.)

\subsubsection{\texorpdfstring{Simulating how degrees of freedom affects
the \(t\)
distribution}{Simulating how degrees of freedom affects the t distribution}}\label{simulating-how-degrees-of-freedom-affects-the-t-distribution}

There are at least two ways to think the degrees of freedom for a
\(t\)-test. For example, if you want to use math to compute aspects of
the \(t\) distribution, then you need the degrees of freedom to plug in
to the formula\ldots{} If you want to see the formulas I'm talking
about, scroll down on the
\href{https://en.wikipedia.org/wiki/Student\%27s_t-distribution}{\(t\)-test
wikipedia page} and look for the probability density or cumulative
distribution functions\ldots We think that is quite scary for most
people, and one reason why degrees of freedom are not well-understood.

If we wanted to simulate the \(t\) distribution we could more easily see
what influence degrees of freedom has on the shape of the distribution.
Remember, \(t\) is a sample statistic, it is something we measure from
the sample. So, we could simulate the process of measuring \(t\) from
many different samples, then plot the histogram of \(t\) to show us the
simulated \(t\) distribution.

\begin{figure}

\centering{

\includegraphics[width=1\linewidth,height=\textheight,keepaspectratio]{07-ttests_files/figure-pdf/fig-7dft-1.pdf}

}

\caption{\label{fig-7dft}The width of the t distribution shrinks as
sample size and degrees of freedom (from 4 to 100) increases.}

\end{figure}%

In Figure~\ref{fig-7dft} notice that the red distribution for
\(df = 4\), is a little bit shorter, and a little bit wider than the
bluey-green distribution for \(df = 100\). As degrees of freedom
increase the \(t\) distribution gets taller (in the middle), and
narrower in the range. It get's more peaky. Can you guess the reason for
this? Remember, we are estimating a sample statistic, and degrees of
freedom is really just a number that refers to the number of subjects
(well minus one). And, we already know that as we increase \(n\), our
sample statistics become better estimates (less variance) of the
distributional parameters they are estimating. So, \(t\) becomes a
better estimate of it's ``true'' value as sample size increase,
resulting in a more narrow distribution of \(t\)s.

There is a slightly different \(t\) distribution for every degrees of
freedom, and the critical regions associated with 5\% of the extreme
values are thus slightly different every time. This is why we report the
degrees of freedom for each t-test, they define the distribution of
\(t\) values for the sample-size in question. Why do we use n-1 and not
n? Well, we calculate \(t\) using the sample standard deviation to
estimate the standard error or the mean, that estimate uses n-1 in the
denominator, so our \(t\) distribution is built assuming n-1. That's
enough for degrees of freedom\ldots{}

\section{The paired samples t-test strikes
back}\label{the-paired-samples-t-test-strikes-back}

You must be wondering if we will ever be finished talking about paired
samples t-tests\ldots{} why are we doing round 2, oh no! Don't worry,
we're just going to 1) remind you about what we were doing with the
infant study, and 2) do a paired samples t-test on the entire data set
and discuss.

Remember, we were wondering if the infants would look longer toward the
singer who sang the familiar song during the test phase compared to the
baseline phase. We showed you data from 5 infants, and walked through
the computations for the \(t\)-test. As a reminder, it looked like this:

\begin{longtable}[]{@{}
  >{\raggedright\arraybackslash}p{(\linewidth - 10\tabcolsep) * \real{0.1000}}
  >{\raggedright\arraybackslash}p{(\linewidth - 10\tabcolsep) * \real{0.1286}}
  >{\raggedright\arraybackslash}p{(\linewidth - 10\tabcolsep) * \real{0.0857}}
  >{\raggedright\arraybackslash}p{(\linewidth - 10\tabcolsep) * \real{0.1714}}
  >{\raggedright\arraybackslash}p{(\linewidth - 10\tabcolsep) * \real{0.2143}}
  >{\raggedright\arraybackslash}p{(\linewidth - 10\tabcolsep) * \real{0.3000}}@{}}
\toprule\noalign{}
\begin{minipage}[b]{\linewidth}\raggedright
infant
\end{minipage} & \begin{minipage}[b]{\linewidth}\raggedright
Baseline
\end{minipage} & \begin{minipage}[b]{\linewidth}\raggedright
Test
\end{minipage} & \begin{minipage}[b]{\linewidth}\raggedright
differences
\end{minipage} & \begin{minipage}[b]{\linewidth}\raggedright
diff\_from\_mean
\end{minipage} & \begin{minipage}[b]{\linewidth}\raggedright
Squared\_differences
\end{minipage} \\
\midrule\noalign{}
\endhead
\bottomrule\noalign{}
\endlastfoot
1 & 0.44 & 0.6 & 0.16 & 0.106 & 0.011236 \\
2 & 0.41 & 0.68 & 0.27 & 0.216 & 0.046656 \\
3 & 0.75 & 0.72 & -0.03 & -0.084 & 0.00705600000000001 \\
4 & 0.44 & 0.28 & -0.16 & -0.214 & 0.045796 \\
5 & 0.47 & 0.5 & 0.03 & -0.024 & 0.000575999999999999 \\
Sums & 2.51 & 2.78 & 0.27 & 0 & 0.11132 \\
Means & 0.502 & 0.556 & 0.054 & 0 & 0.022264 \\
& & & & sd & 0.167 \\
& & & & SEM & 0.075 \\
& & & & t & 0.72 \\
\end{longtable}

\begin{verbatim}
#> 
#>  One Sample t-test
#> 
#> data:  round(differences, digits = 2)
#> t = 0.72381, df = 4, p-value = 0.5092
#> alternative hypothesis: true mean is not equal to 0
#> 95 percent confidence interval:
#>  -0.1531384  0.2611384
#> sample estimates:
#> mean of x 
#>     0.054
\end{verbatim}

Let's write down the finding one more time: The mean difference was
0.054, \(t\)(4) = .72, \(p\) =.509. We can also now confirm, that the
\(p\)-value was from a two-tailed test. So, what does this all really
mean.

We can say that a \(t\) value with an absolute of .72 or larger occurs
50.9\% of the time. More precisely, the distribution of no differences
(the null), will produce a \(t\) value this large or larger 50.9\% of
the time. In other words, chance alone good have easily produced the
\(t\) value from our sample, and the mean difference we observed or
.054, could easily have been a result of chance.

Let's quickly put all of the data in the \(t\)-test, and re-run the test
using all of the infant subjects.

\begin{verbatim}
#> 
#>  One Sample t-test
#> 
#> data:  differences
#> t = 2.4388, df = 31, p-value = 0.02066
#> alternative hypothesis: true mean is not equal to 0
#> 95 percent confidence interval:
#>  0.01192088 0.13370412
#> sample estimates:
#> mean of x 
#> 0.0728125
\end{verbatim}

Now we get a very different answer. We would summarize the results
saying the mean difference was .073, t(31) = 2.44, p = 0.020. How many
total infants were their? Well the degrees of freedom was 31, so there
must have been 32 infants in the study. Now we see a much smaller
\(p\)-value. This was also a two-tailed test, so we that observing a
\(t\) value of 2.4 or greater (absolute value) only occurs 2\% of the
time. In other words, the distribution of no differences will produce
the observed t-value very rarely. So, it is unlikely that the observed
mean difference of .073 was due to chance (it could have been due to
chance, but that is very unlikely). As a result, we can be somewhat
confident in concluding that something about seeing and hearing a
unfamiliar person sing a familiar song, causes an infant to draw their
attention toward the singer, and this potentially benefits social
learning on the part of the infant.

\section{Independent samples t-test: The return of the
t-test?}\label{independent-samples-t-test-the-return-of-the-t-test}

If you've been following the Star Wars references, we are on last movie
(of the original trilogy)\ldots{} the independent t-test. This is were
basically the same story plays out as before, only slightly different.

Remember there are different \(t\)-tests for different kinds of research
designs. When your design is a \textbf{between-subjects} design, you use
an \textbf{independent samples t-test}. Between-subjects design involve
different people or subjects in each experimental condition. If there
are two conditions, and 10 people in each, then there are 20 total
people. And, there are no paired scores, because every single person is
measured once, not twice, no repeated measures. Because there are no
repeated measures we can't look at the difference scores between
conditions one and two. The scores are not paired in any meaningful way,
to it doesn't make sense to subtract them. So what do we do?

The logic of the independent samples t-test is the very same as the
other \(t\)-tests. We calculated the means for each group, then we find
the difference. That goes into the numerator of the t formula. Then we
get an estimate of the variation for the denominator. We divide the mean
difference by the estimate of the variation, and we get \(t\). It's the
same as before.

The only wrinkle here is what goes into the denominator? How should we
calculate the estimate of the variance? It would be nice if we could do
something very straightforward like this, say for an experiment with two
groups A and B:

\(t = \frac{\bar{A}-\bar{B}}{(\frac{SEM_A+SEM_B}{2})}\)

In plain language, this is just:

\begin{enumerate}
\def\labelenumi{\arabic{enumi}.}
\tightlist
\item
  Find the mean difference for the top part
\item
  Compute the SEM (standard error of the mean) for each group, and
  average them together to make a single estimate, pooling over both
  samples.
\end{enumerate}

This would be nice, but unfortunately, it turns out that finding the
average of two standard errors of the mean is not the best way to do it.
This would create a biased estimator of the variation for the
hypothesized distribution of no differences. We won't go into the math
here, but instead of the above formula, we an use a different one that
gives as an \textbf{unbiased estimate of the pooled standard error of
the sample mean}. Our new and improved \(t\) formula would look like
this:

\(t = \frac{\bar{X_A}-\bar{X_B}}{s_p * \sqrt{\frac{1}{n_A} + \frac{1}{n_B}}}\)

and, \(s_p\), which is the pooled sample standard deviation is defined
as, note the \$s\$es in the formula are variances:

\(s_p = \sqrt{\frac{(n_A-1)s_A^2 + (n_B-1)s^2_B}{n_A +n_B -2}}\)

Believe you me, that is so much more formula than I wanted to type out.
Shall we do one independent \(t\)-test example by hand, just to see the
computations? Let's do it\ldots but in a slightly different way than you
expect. I show the steps using R. I made some fake scores for groups A
and B. Then, I followed all of the steps from the formula, but made R do
each of the calculations. This shows you the needed steps by following
the code. At the end, I print the \(t\)-test values I computed ``by
hand'', and then the \(t\)-test value that the R software outputs using
the \(t\)-test function. You should be able to get the same values for
\(t\), if you were brave enough to compute \(t\) by hand.

\begin{Shaded}
\begin{Highlighting}[]

\DocumentationTok{\#\# By "hand" using R r code}
\NormalTok{a }\OtherTok{\textless{}{-}} \FunctionTok{c}\NormalTok{(}\DecValTok{1}\NormalTok{,}\DecValTok{2}\NormalTok{,}\DecValTok{3}\NormalTok{,}\DecValTok{4}\NormalTok{,}\DecValTok{5}\NormalTok{)}
\NormalTok{b }\OtherTok{\textless{}{-}} \FunctionTok{c}\NormalTok{(}\DecValTok{3}\NormalTok{,}\DecValTok{5}\NormalTok{,}\DecValTok{4}\NormalTok{,}\DecValTok{7}\NormalTok{,}\DecValTok{9}\NormalTok{)}

\NormalTok{mean\_difference }\OtherTok{\textless{}{-}} \FunctionTok{mean}\NormalTok{(a)}\SpecialCharTok{{-}}\FunctionTok{mean}\NormalTok{(b) }\CommentTok{\# compute mean difference}

\NormalTok{variance\_a }\OtherTok{\textless{}{-}} \FunctionTok{var}\NormalTok{(a) }\CommentTok{\# compute variance for A}
\NormalTok{variance\_b }\OtherTok{\textless{}{-}} \FunctionTok{var}\NormalTok{(b) }\CommentTok{\# compute variance for B}

\CommentTok{\# Compute top part and bottom part of sp formula}

\NormalTok{sp\_numerator }\OtherTok{\textless{}{-}}\NormalTok{ (}\DecValTok{4}\SpecialCharTok{*}\NormalTok{variance\_a }\SpecialCharTok{+} \DecValTok{4}\SpecialCharTok{*}\NormalTok{ variance\_b) }
\NormalTok{sp\_denominator }\OtherTok{\textless{}{-}} \DecValTok{5}\SpecialCharTok{+}\DecValTok{5{-}2}
\NormalTok{sp }\OtherTok{\textless{}{-}} \FunctionTok{sqrt}\NormalTok{(sp\_numerator}\SpecialCharTok{/}\NormalTok{sp\_denominator) }\CommentTok{\# compute sp}


\CommentTok{\# compute t following formulat}

\NormalTok{t }\OtherTok{\textless{}{-}}\NormalTok{ mean\_difference }\SpecialCharTok{/}\NormalTok{ ( sp }\SpecialCharTok{*} \FunctionTok{sqrt}\NormalTok{( (}\DecValTok{1}\SpecialCharTok{/}\DecValTok{5}\NormalTok{) }\SpecialCharTok{+}\NormalTok{(}\DecValTok{1}\SpecialCharTok{/}\DecValTok{5}\NormalTok{) ) )}

\NormalTok{t }\CommentTok{\# print results}
\CommentTok{\#\textgreater{} [1] {-}2.017991}


\CommentTok{\# using the R function t.test}
\FunctionTok{t.test}\NormalTok{(a,b, }\AttributeTok{paired=}\ConstantTok{FALSE}\NormalTok{, }\AttributeTok{var.equal =} \ConstantTok{TRUE}\NormalTok{)}
\CommentTok{\#\textgreater{} }
\CommentTok{\#\textgreater{}  Two Sample t{-}test}
\CommentTok{\#\textgreater{} }
\CommentTok{\#\textgreater{} data:  a and b}
\CommentTok{\#\textgreater{} t = {-}2.018, df = 8, p{-}value = 0.0783}
\CommentTok{\#\textgreater{} alternative hypothesis: true difference in means is not equal to 0}
\CommentTok{\#\textgreater{} 95 percent confidence interval:}
\CommentTok{\#\textgreater{}  {-}5.5710785  0.3710785}
\CommentTok{\#\textgreater{} sample estimates:}
\CommentTok{\#\textgreater{} mean of x mean of y }
\CommentTok{\#\textgreater{}       3.0       5.6}
\end{Highlighting}
\end{Shaded}

\section{Simulating data for t-tests}\label{simulating-data-for-t-tests}

An ``advanced'' topic for \(t\)-tests is the idea of using R to conduct
simulations for \(t\)-tests.

If you recall, \(t\) is a property of a sample. We calculate \(t\) from
our sample. The \(t\) distribution is the hypothetical behavior of our
sample. That is, if we had taken thousands upon thousands of samples,
and calculated \(t\) for each one, and then looked at the distribution
of those \(t\)'s, we would have the sampling distribution of \(t\)!

It can be very useful to get in the habit of using R to simulate data
under certain conditions, to see how your sample data, and things like
\(t\) behave. Why is this useful? It mainly prepares you with some
intuitions about how sampling error (random chance) can influence your
results, given specific parameters of your design, such as sample-size,
the size of the mean difference you expect to find in your data, and the
amount of variation you might find. These methods can be used formally
to conduct power-analyses. Or more informally for data sense.

\subsection{Simulating a one-sample
t-test}\label{simulating-a-one-sample-t-test}

Here are the steps you might follow to simulate data for a one sample
\(t\)-test.

\begin{enumerate}
\def\labelenumi{\arabic{enumi}.}
\item
  Make some assumptions about what your sample (that you might be
  planning to collect) might look like. For example, you might be
  planning to collect 30 subjects worth of data. The scores of those
  data points might come from a normal distribution (mean = 50, sd =
  10).
\item
  sample simulated numbers from the distribution, then conduct a
  \(t\)-test on the simulated numbers. Save the statistics you want
  (such as \(t\)s and \(p\)s), and then see how things behave.
\end{enumerate}

Let's do this a couple different times. First, let's simulate samples
with N = 30, taken from a normal (mean= 50, sd = 25). We'll do a
simulation with 1000 simulations. For each simulation, we will compare
the sample mean with a population mean of 50. There should be no
difference on average here. Figure~\ref{fig-7nullt} is the null
distribution that we are simulating.

\begin{figure}

\centering{

\includegraphics[width=0.75\linewidth,height=\textheight,keepaspectratio]{07-ttests_files/figure-pdf/fig-7nullt-1.pdf}

}

\caption{\label{fig-7nullt}The distribution of \(t\)-values under the
null. These are the \(t\) values that are produced by chance alone.}

\end{figure}%

\begin{figure}

\centering{

\includegraphics[width=0.75\linewidth,height=\textheight,keepaspectratio]{07-ttests_files/figure-pdf/fig-7flatp-1.pdf}

}

\caption{\label{fig-7flatp}The distribution of \(p\)-values that are
observed is flat under the null.}

\end{figure}%

Neat. We see both a \(t\) distribution, that looks like \(t\)
distribution as it should. And we see the \(p\) distribution. This shows
us how often we get \(t\) values of particular sizes. You may find it
interesting that the \(p\)-distribution is flat under the null, which we
are simulating here. This means that you have the same chances of a
getting a \(t\) with a p-value between 0 and 0.05, as you would for
getting a \(t\) with a p-value between .90 and .95. Those ranges are
both ranges of 5\%, so there are an equal amount of \(t\) values in them
by definition.

Here's another way to do the same simulation in R, using the replicate
function, instead a for loop:

\begin{figure}

\centering{

\includegraphics[width=0.75\linewidth,height=\textheight,keepaspectratio]{07-ttests_files/figure-pdf/fig-7simtsRep-1.pdf}

}

\caption{\label{fig-7simtsRep}Simulating \(t\)s in R.}

\end{figure}%

\begin{figure}

\centering{

\includegraphics[width=0.75\linewidth,height=\textheight,keepaspectratio]{07-ttests_files/figure-pdf/fig-7simpsRep-1.pdf}

}

\caption{\label{fig-7simpsRep}Simulating \(p\)s in R.}

\end{figure}%

\subsection{Simulating a paired samples
t-test}\label{simulating-a-paired-samples-t-test}

The code below is set up to sample 10 scores for condition A and B from
the same normal distribution. The simulation is conducted 1000 times,
and the \(t\)s and \(p\)s are saved and plotted for each.

\begin{Shaded}
\begin{Highlighting}[]

\NormalTok{save\_ps }\OtherTok{\textless{}{-}} \FunctionTok{length}\NormalTok{(}\DecValTok{1000}\NormalTok{)}
\NormalTok{save\_ts }\OtherTok{\textless{}{-}} \FunctionTok{length}\NormalTok{(}\DecValTok{1000}\NormalTok{)}
\ControlFlowTok{for}\NormalTok{ ( i }\ControlFlowTok{in} \DecValTok{1}\SpecialCharTok{:}\DecValTok{1000}\NormalTok{ )\{}
\NormalTok{  condition\_A }\OtherTok{\textless{}{-}} \FunctionTok{rnorm}\NormalTok{(}\DecValTok{10}\NormalTok{,}\DecValTok{10}\NormalTok{,}\DecValTok{5}\NormalTok{)}
\NormalTok{  condition\_B }\OtherTok{\textless{}{-}} \FunctionTok{rnorm}\NormalTok{(}\DecValTok{10}\NormalTok{,}\DecValTok{10}\NormalTok{,}\DecValTok{5}\NormalTok{)}
\NormalTok{  differences }\OtherTok{\textless{}{-}}\NormalTok{ condition\_A }\SpecialCharTok{{-}}\NormalTok{ condition\_B}
\NormalTok{  t\_test }\OtherTok{\textless{}{-}} \FunctionTok{t.test}\NormalTok{(differences, }\AttributeTok{mu=}\DecValTok{0}\NormalTok{)}
\NormalTok{  save\_ps[i] }\OtherTok{\textless{}{-}}\NormalTok{ t\_test}\SpecialCharTok{$}\NormalTok{p.value}
\NormalTok{  save\_ts[i] }\OtherTok{\textless{}{-}}\NormalTok{ t\_test}\SpecialCharTok{$}\NormalTok{statistic}
\NormalTok{\}}
\end{Highlighting}
\end{Shaded}

\begin{figure}

\centering{

\includegraphics[width=0.75\linewidth,height=\textheight,keepaspectratio]{07-ttests_files/figure-pdf/fig-7simps1000-1.pdf}

}

\caption{\label{fig-7simps1000}1000 simulated ts from the null
distribution}

\end{figure}%

\begin{figure}

\centering{

\includegraphics[width=0.75\linewidth,height=\textheight,keepaspectratio]{07-ttests_files/figure-pdf/fig-7simts1000-1.pdf}

}

\caption{\label{fig-7simts1000}1000 simulated ps from the null
distribution}

\end{figure}%

According to the simulation. When there are no differences between the
conditions, and the samples are being pulled from the very same
distribution, you get these two distributions for \(t\) and \(p\). These
again show how the null distribution of no differences behaves.

For any of these simulations, if you rejected the null-hypothesis (that
your difference was only due to chance), you would be making a type I
error. If you set your alpha criteria to \(\alpha = .05\), we can ask
how many type I errors were made in these 1000 simulations. The answer
is:

\begin{Shaded}
\begin{Highlighting}[]
\FunctionTok{length}\NormalTok{(save\_ps[save\_ps}\SpecialCharTok{\textless{}}\NormalTok{.}\DecValTok{05}\NormalTok{])}
\CommentTok{\#\textgreater{} [1] 54}
\FunctionTok{length}\NormalTok{(save\_ps[save\_ps}\SpecialCharTok{\textless{}}\NormalTok{.}\DecValTok{05}\NormalTok{])}\SpecialCharTok{/}\DecValTok{1000}
\CommentTok{\#\textgreater{} [1] 0.054}
\end{Highlighting}
\end{Shaded}

We happened to make 54. The expectation over the long run is 5\% type I
error rates (if your alpha is .05).

What happens if there actually is a difference in the simulated data,
let's set one condition to have a larger mean than the other:

\begin{Shaded}
\begin{Highlighting}[]

\NormalTok{save\_ps }\OtherTok{\textless{}{-}} \FunctionTok{length}\NormalTok{(}\DecValTok{1000}\NormalTok{)}
\NormalTok{save\_ts }\OtherTok{\textless{}{-}} \FunctionTok{length}\NormalTok{(}\DecValTok{1000}\NormalTok{)}
\ControlFlowTok{for}\NormalTok{ ( i }\ControlFlowTok{in} \DecValTok{1}\SpecialCharTok{:}\DecValTok{1000}\NormalTok{ )\{}
\NormalTok{  condition\_A }\OtherTok{\textless{}{-}} \FunctionTok{rnorm}\NormalTok{(}\DecValTok{10}\NormalTok{,}\DecValTok{10}\NormalTok{,}\DecValTok{5}\NormalTok{)}
\NormalTok{  condition\_B }\OtherTok{\textless{}{-}} \FunctionTok{rnorm}\NormalTok{(}\DecValTok{10}\NormalTok{,}\DecValTok{13}\NormalTok{,}\DecValTok{5}\NormalTok{)}
\NormalTok{  differences }\OtherTok{\textless{}{-}}\NormalTok{ condition\_A }\SpecialCharTok{{-}}\NormalTok{ condition\_B}
\NormalTok{  t\_test }\OtherTok{\textless{}{-}} \FunctionTok{t.test}\NormalTok{(differences, }\AttributeTok{mu=}\DecValTok{0}\NormalTok{)}
\NormalTok{  save\_ps[i] }\OtherTok{\textless{}{-}}\NormalTok{ t\_test}\SpecialCharTok{$}\NormalTok{p.value}
\NormalTok{  save\_ts[i] }\OtherTok{\textless{}{-}}\NormalTok{ t\_test}\SpecialCharTok{$}\NormalTok{statistic}
\NormalTok{\}}
\end{Highlighting}
\end{Shaded}

\begin{figure}

\centering{

\includegraphics[width=0.75\linewidth,height=\textheight,keepaspectratio]{07-ttests_files/figure-pdf/fig-7simtruets-1.pdf}

}

\caption{\label{fig-7simtruets}1000 ts when there is a true difference}

\end{figure}%

\begin{figure}

\centering{

\includegraphics[width=0.75\linewidth,height=\textheight,keepaspectratio]{07-ttests_files/figure-pdf/fig-7simtrueps-1.pdf}

}

\caption{\label{fig-7simtrueps}1000 ps when there is a true difference}

\end{figure}%

Now you can see that the \(p\)-value distribution is skewed to the left.
This is because when there is a true effect, you will get p-values that
are less than .05 more often. Or, rather, you get larger \(t\) values
than you normally would if there were no differences.

In this case, we wouldn't be making a type I error if we rejected the
null when p was smaller than .05. How many times would we do that out of
our 1000 experiments?

\begin{Shaded}
\begin{Highlighting}[]
\FunctionTok{length}\NormalTok{(save\_ps[save\_ps}\SpecialCharTok{\textless{}}\NormalTok{.}\DecValTok{05}\NormalTok{])}
\CommentTok{\#\textgreater{} [1] 227}
\FunctionTok{length}\NormalTok{(save\_ps[save\_ps}\SpecialCharTok{\textless{}}\NormalTok{.}\DecValTok{05}\NormalTok{])}\SpecialCharTok{/}\DecValTok{1000}
\CommentTok{\#\textgreater{} [1] 0.227}
\end{Highlighting}
\end{Shaded}

We happened to get 227 simulations where p was less than .05, that's
only 0.227 experiments. If you were the researcher, would you want to
run an experiment that would be successful only 0.227 of the time? I
wouldn't. I would run a better experiment.

How would you run a better simulated experiment? Well, you could
increase \(n\), the number of subjects in the experiment. Let's increase
\(n\) from 10 to 100, and see what happens to the number of
``significant'' simulated experiments.

\begin{Shaded}
\begin{Highlighting}[]

\NormalTok{save\_ps }\OtherTok{\textless{}{-}} \FunctionTok{length}\NormalTok{(}\DecValTok{1000}\NormalTok{)}
\NormalTok{save\_ts }\OtherTok{\textless{}{-}} \FunctionTok{length}\NormalTok{(}\DecValTok{1000}\NormalTok{)}
\ControlFlowTok{for}\NormalTok{ ( i }\ControlFlowTok{in} \DecValTok{1}\SpecialCharTok{:}\DecValTok{1000}\NormalTok{ )\{}
\NormalTok{  condition\_A }\OtherTok{\textless{}{-}} \FunctionTok{rnorm}\NormalTok{(}\DecValTok{100}\NormalTok{,}\DecValTok{10}\NormalTok{,}\DecValTok{5}\NormalTok{)}
\NormalTok{  condition\_B }\OtherTok{\textless{}{-}} \FunctionTok{rnorm}\NormalTok{(}\DecValTok{100}\NormalTok{,}\DecValTok{13}\NormalTok{,}\DecValTok{5}\NormalTok{)}
\NormalTok{  differences }\OtherTok{\textless{}{-}}\NormalTok{ condition\_A }\SpecialCharTok{{-}}\NormalTok{ condition\_B}
\NormalTok{  t\_test }\OtherTok{\textless{}{-}} \FunctionTok{t.test}\NormalTok{(differences, }\AttributeTok{mu=}\DecValTok{0}\NormalTok{)}
\NormalTok{  save\_ps[i] }\OtherTok{\textless{}{-}}\NormalTok{ t\_test}\SpecialCharTok{$}\NormalTok{p.value}
\NormalTok{  save\_ts[i] }\OtherTok{\textless{}{-}}\NormalTok{ t\_test}\SpecialCharTok{$}\NormalTok{statistic}
\NormalTok{\}}
\end{Highlighting}
\end{Shaded}

\begin{figure}

\centering{

\includegraphics[width=0.75\linewidth,height=\textheight,keepaspectratio]{07-ttests_files/figure-pdf/fig-7simtsB-1.pdf}

}

\caption{\label{fig-7simtsB}1000 ts for n =100, when there is a true
effect}

\end{figure}%

\begin{verbatim}
#> [1] 987
#> [1] 0.987
\end{verbatim}

\begin{figure}

\centering{

\includegraphics[width=0.75\linewidth,height=\textheight,keepaspectratio]{07-ttests_files/figure-pdf/fig-7simpsB-1.pdf}

}

\caption{\label{fig-7simpsB}1000 ps for n =100, when there is a true
effect}

\end{figure}%

Cool, now almost all of the experiments show a \(p\)-value of less than
.05 (using a two-tailed test, that's the default in R). See, you could
use this simulation process to determine how many subjects you need to
reliably find your effect.

\subsection{Simulating an independent samples
t.test}\label{simulating-an-independent-samples-t.test}

Just change the t.test function like so\ldots{} this is for the null,
assuming no difference between groups.

\begin{Shaded}
\begin{Highlighting}[]

\NormalTok{save\_ps }\OtherTok{\textless{}{-}} \FunctionTok{length}\NormalTok{(}\DecValTok{1000}\NormalTok{)}
\NormalTok{save\_ts }\OtherTok{\textless{}{-}} \FunctionTok{length}\NormalTok{(}\DecValTok{1000}\NormalTok{)}
\ControlFlowTok{for}\NormalTok{ ( i }\ControlFlowTok{in} \DecValTok{1}\SpecialCharTok{:}\DecValTok{1000}\NormalTok{ )\{}
\NormalTok{  group\_A }\OtherTok{\textless{}{-}} \FunctionTok{rnorm}\NormalTok{(}\DecValTok{10}\NormalTok{,}\DecValTok{10}\NormalTok{,}\DecValTok{5}\NormalTok{)}
\NormalTok{  group\_B }\OtherTok{\textless{}{-}} \FunctionTok{rnorm}\NormalTok{(}\DecValTok{10}\NormalTok{,}\DecValTok{10}\NormalTok{,}\DecValTok{5}\NormalTok{)}
\NormalTok{  t\_test }\OtherTok{\textless{}{-}} \FunctionTok{t.test}\NormalTok{(group\_A, group\_B, }\AttributeTok{paired=}\ConstantTok{FALSE}\NormalTok{, }\AttributeTok{var.equal=}\ConstantTok{TRUE}\NormalTok{)}
\NormalTok{  save\_ps[i] }\OtherTok{\textless{}{-}}\NormalTok{ t\_test}\SpecialCharTok{$}\NormalTok{p.value}
\NormalTok{  save\_ts[i] }\OtherTok{\textless{}{-}}\NormalTok{ t\_test}\SpecialCharTok{$}\NormalTok{statistic}
\NormalTok{\}}
\end{Highlighting}
\end{Shaded}

\begin{figure}

\centering{

\includegraphics[width=0.75\linewidth,height=\textheight,keepaspectratio]{07-ttests_files/figure-pdf/fig-7simtsC-1.pdf}

}

\caption{\label{fig-7simtsC}1000 ts for n =100, when there is a true
effect}

\end{figure}%

\begin{verbatim}
#> [1] 40
#> [1] 0.04
\end{verbatim}

\begin{figure}

\centering{

\includegraphics[width=0.75\linewidth,height=\textheight,keepaspectratio]{07-ttests_files/figure-pdf/fig-7simpsC-1.pdf}

}

\caption{\label{fig-7simpsC}1000 ps for n =100, when there is a true
effect}

\end{figure}%

\section{Videos}\label{videos-4}

\subsection{One or Two tailed tests}\label{one-or-two-tailed-tests}

\bookmarksetup{startatroot}

\chapter{ANOVA}\label{anova}

A fun bit of stats history (Salsburg 2001). Sir Ronald Fisher invented
the ANOVA, which we learn about in this section. He wanted to publish
his new test in the journal Biometrika. The editor at the time was Karl
Pearson (remember Pearson's \(r\) for correlation?). Pearson and Fisher
were apparently not on good terms, they didn't like each other. Pearson
refused to publish Fisher's new test. So, Fisher eventually published
his work in the Journal of Agricultural Science. Funnily enough, the
feud continued onto the next generation. Years after Fisher published
his ANOVA, Karl Pearson's son Egon Pearson, and Jersey Neyman revamped
Fisher's ideas, and re-cast them into what is commonly known as null
vs.~alternative hypothesis testing. Fisher didn't like this very much.

We present the ANOVA in the Fisherian sense, and at the end describe the
Neyman-Pearson approach that invokes the concept of null vs.~alternative
hypotheses.

\section{ANOVA is Analysis of
Variance}\label{anova-is-analysis-of-variance}

ANOVA stands for Analysis Of Variance. It is a widely used technique for
assessing the likelihood that differences found between means in sample
data could be produced by chance. You might be thinking, well don't we
have \(t\)-tests for that? Why do we need the ANOVA, what do we get
that's new that we didn't have before?

What's new with the ANOVA, is the ability to test a wider range of means
beyond just two. In all of the \(t\)-test examples we were always
comparing two things. For example, we might ask whether the difference
between two sample means could have been produced by chance. What if our
experiment had more than two conditions or groups? We would have more
than 2 means. We would have one mean for each group or condition. That
could be a lot depending on the experiment. How would we compare all of
those means? What should we do, run a lot of \(t\)-tests, comparing
every possible combination of means? Actually, you could do that. Or,
you could do an ANOVA.

In practice, we will combine both the ANOVA test and \(t\)-tests when
analyzing data with many sample means (from more than two groups or
conditions). Just like the \(t\)-test, there are different kinds of
ANOVAs for different research designs. There is one for between-subjects
designs, and a slightly different one for repeated measures designs. We
talk about both, beginning with the ANOVA for between-subjects designs.

\section{One-factor ANOVA}\label{one-factor-anova}

The one-factor ANOVA is sometimes also called a between-subjects ANOVA,
an independent factor ANOVA, or a one-way ANOVA (which is a bit of a
misnomer as we discuss later). The critical ingredient for a one-factor,
between-subjects ANOVA, is that you have one independent variable, with
at least two-levels. When you have one IV with two levels, you can run a
\(t\)-test. You can also run an ANOVA. Interestingly, they give you
almost the exact same results. You will get a \(p\)-value from both
tests that is identical (they are really doing the same thing under the
hood). The \(t\)-test gives a \(t\)-value as the important sample
statistic. The ANOVA gives you the \(F\)-value (for Fisher, the inventor
of the test) as the important sample statistic. It turns out that
\(t^2\) equals \(F\), when there are only two groups in the design. They
are the same test. Side-note, it turns out they are all related to
Pearson's r too (but we haven't written about this relationship yet in
this textbook).

Remember that \(t\) is computed directly from the data. It's like a mean
and standard error that we measure from the sample. In fact it's the
mean difference divided by the standard error of the sample. It's just
another descriptive statistic isn't it.

The same thing is true about \(F\). \(F\) is computed directly from the
data. In fact, the idea behind \(F\) is the same basic idea that goes
into making \(t\). Here is the general idea behind the formula, it is
again a ratio of the effect we are measuring (in the numerator), and the
variation associated with the effect (in the denominator).

\(\text{name of statistic} = \frac{\text{measure of effect}}{\text{measure of error}}\)

\(\text{F} = \frac{\text{measure of effect}}{\text{measure of error}}\)

The difference with \(F\), is that we use variances to describe both the
measure of the effect and the measure of error. So, \(F\) is a ratio of
two variances.

Remember what we said about how these ratios work. When the variance
associated with the effect is the same size as the variance associated
with sampling error, we will get two of the same numbers, this will
result in an \(F\)-value of 1. When the variance due to the effect is
larger than the variance associated with sampling error, then \(F\) will
be greater than 1. When the variance associated with the effect is
smaller than the variance associated with sampling error, \(F\) will be
less than one.

Let's rewrite in plainer English. We are talking about two concepts that
we would like to measure from our data. 1) A measure of what we can
explain, and 2) a measure of error, or stuff about our data we can't
explain. So, the \(F\) formula looks like this:

\(\text{F} = \frac{\text{Can Explain}}{\text{Can't Explain}}\)

When we can explain as much as we can't explain, \(F\) = 1. This isn't
that great of a situation for us to be in. It means we have a lot of
uncertainty. When we can explain much more than we can't we are doing a
good job, \(F\) will be greater than 1. When we can explain less than
what we can't, we really can't explain very much, \(F\) will be less
than 1. That's the concept behind making \(F\).

If you saw an \(F\) in the wild, and it was .6. Then you would
automatically know the researchers couldn't explain much of their data.
If you saw an \(F\) of 5, then you would know the researchers could
explain 5 times more than the couldn't, that's pretty good. And the
point of this is to give you an intuition about the meaning of an
\(F\)-value, even before you know how to compute it.

\subsection{\texorpdfstring{Computing the
\(F\)-value}{Computing the F-value}}\label{computing-the-f-value}

Fisher's ANOVA is very elegant in my opinion. It starts us off with a
big problem we always have with data. We have a lot of numbers, and
there is a lot of variation in the numbers, what to do? Wouldn't it be
nice to split up the variation into to kinds, or sources. If we could
know what parts of the variation were being caused by our experimental
manipulation, and what parts were being caused by sampling error, we
would be making really good progress. We would be able to know if our
experimental manipulation was causing more change in the data than
sampling error, or chance alone. If we could measure those two parts of
the total variation, we could make a ratio, and then we would have an
\(F\) value. This is what the ANOVA does. It splits the total variation
in the data into two parts. The formula is:

Total Variation = Variation due to Manipulation + Variation due to
sampling error

This is a nice idea, but it is also vague. We haven't specified our
measure of variation. What should we use?

Remember the sums of squares that we used to make the variance and the
standard deviation? That's what we'll use. Let's take another look at
the formula, using sums of squares for the measure of variation:

\(SS_\text{total} = SS_\text{Effect} + SS_\text{Error}\)

\subsection{SS Total}\label{ss-total}

The total sums of squares, or \(SS\text{Total}\) is a way of thinking
about all of the variation in a set of data. It's pretty straightforward
to measure. No tricky business. All we do is find the difference between
each score and the grand mean, then we square the differences and add
them all up.

Let's imagine we had some data in three groups, A, B, and C. For
example, we might have 3 scores in each group. The data could look like
this:

\begin{longtable}[]{@{}llll@{}}
\toprule\noalign{}
groups & scores & diff & diff\_squared \\
\midrule\noalign{}
\endhead
\bottomrule\noalign{}
\endlastfoot
A & 20 & 13 & 169 \\
A & 11 & 4 & 16 \\
A & 2 & -5 & 25 \\
B & 6 & -1 & 1 \\
B & 2 & -5 & 25 \\
B & 7 & 0 & 0 \\
C & 2 & -5 & 25 \\
C & 11 & 4 & 16 \\
C & 2 & -5 & 25 \\
Sums & 63 & 0 & 302 \\
Means & 7 & 0 & 33.5555555555556 \\
\end{longtable}

The data is organized in long format, so that each row is a single
score. There are three scores for the A, B, and C groups. The mean of
all of the scores is called the \textbf{Grand Mean}. It's calculated in
the table, the Grand Mean = 7.

We also calculated all of the difference scores \textbf{from the Grand
Mean}. The difference scores are in the column titled \texttt{diff}.
Next, we squared the difference scores, and those are in the next column
called \texttt{diff\_squared}.

Remember, the difference scores are a way of measuring variation. They
represent how far each number is from the Grand Mean. If the Grand Mean
represents our best guess at summarizing the data, the difference scores
represent the error between the guess and each actual data point. The
only problem with the difference scores is that they sum to zero
(because the mean is the balancing point in the data). So, it is
convenient to square the difference scores, this turns all of them into
positive numbers. The size of the squared difference scores still
represents error between the mean and each score. And, the squaring
operation exacerbates the differences as the error grows larger
(squaring a big number makes a really big number, squaring a small
number still makes a smallish number).

OK fine! We have the squared deviations from the grand mean, we know
that they represent the error between the grand mean and each score.
What next? SUM THEM UP!

When you add up all of the individual squared deviations (difference
scores) you get the sums of squares. That's why it's called the sums of
squares (SS).

Now, we have the first part of our answer:

\(SS_\text{total} = SS_\text{Effect} + SS_\text{Error}\)

\(SS_\text{total} = 302\) and

\(302 = SS_\text{Effect} + SS_\text{Error}\)

What next? If you think back to what you learned about algebra, and
solving for X, you might notice that we don't really need to find the
answers to both missing parts of the equation. We only need one, and we
can solve for the other. For example, if we found \(SS_\text{Effect}\),
then we could solve for \(SS_\text{Error}\).

\subsection{SS Effect}\label{ss-effect}

\(SS_\text{Total}\) gave us a number representing all of the change in
our data, how all the scores are different from the grand mean.

What we want to do next is estimate how much of the total change in the
data might be due to the experimental manipulation. For example, if we
ran an experiment that causes causes change in the measurement, then the
means for each group will be different from other. As a result, the
manipulation forces change onto the numbers, and this will naturally
mean that some part of the total variation in the numbers is caused by
the manipulation.

The way to isolate the variation due to the manipulation (also called
effect) is to look at the means in each group, and calculate the
difference scores between each group mean and the grand mean, and then
sum the squared deviations to find \(SS_\text{Effect}\).

Consider this table, showing the calculations for \(SS_\text{Effect}\).

\begin{longtable}[]{@{}lllll@{}}
\toprule\noalign{}
groups & scores & means & diff & diff\_squared \\
\midrule\noalign{}
\endhead
\bottomrule\noalign{}
\endlastfoot
A & 20 & 11 & 4 & 16 \\
A & 11 & 11 & 4 & 16 \\
A & 2 & 11 & 4 & 16 \\
B & 6 & 5 & -2 & 4 \\
B & 2 & 5 & -2 & 4 \\
B & 7 & 5 & -2 & 4 \\
C & 2 & 5 & -2 & 4 \\
C & 11 & 5 & -2 & 4 \\
C & 2 & 5 & -2 & 4 \\
Sums & 63 & 63 & 0 & 72 \\
Means & 7 & 7 & 0 & 8 \\
\end{longtable}

Notice we created a new column called \texttt{means}. For example, the
mean for group A was 11. You can see there are three 11s, one for each
observation in row A. The means for group B and C happen to both be 5.
So, the rest of the numbers in the means column are 5s.

What we are doing here is thinking of each score in the data from the
viewpoint of the group means. The group means are our best attempt to
summarize the data in those groups. From the point of view of the mean,
all of the numbers are treated as the same. The mean doesn't know how
far off it is from each score, it just knows that all of the scores are
centered on the mean.

\begin{quote}
Let's pretend you are the mean for group A. That means you are an 11.
Someone asks you ``hey, what's the score for the first data point in
group A?''. Because you are the mean, you say, I know that, it's 11.
``What about the second score?''\ldots it's 11\ldots{} they're all 11,
so far as I can tell\ldots{}``Am I missing something\ldots{}'', asked
the mean.
\end{quote}

Now that we have converted each score to it's mean value we can find the
differences between each mean score and the grand mean, then square
them, then sum them up. We did that, and found that the
\(SS_\text{Effect} = 72\).

\(SS_\text{Effect}\) represents the amount of variation that is caused
by differences between the means. I also refer to this as the amount of
variation that the researcher can explain (by the means, which represent
differences between groups or conditions that were manipulated by the
researcher).

Notice also that \(SS_\text{Effect} = 72\), and that 72 is smaller than
\(SS_\text{total} = 302\). That is very important. \(SS_\text{Effect}\)
by definition can never be larger than \(SS_\text{total}\).

\subsection{SS Error}\label{ss-error}

Great, we made it to SS Error. We already found SS Total, and SS Effect,
so now we can solve for SS Error just like this:

\(SS_\text{total} = SS_\text{Effect} + SS_\text{Error}\)

switching around:

\$ SS\_\text{Error} = SS\_\text{total} - SS\_\text{Effect} \$

\$ SS\_\text{Error} = 302 - 72 = 230 \$

We could stop here and show you the rest of the ANOVA, we're almost
there. But, the next step might not make sense unless we show you how to
calculate \(SS_\text{Error}\) directly from the data, rather than just
solving for it. We should do this just to double-check our work anyway.

\begin{longtable}[]{@{}lllll@{}}
\toprule\noalign{}
groups & scores & means & diff & diff\_squared \\
\midrule\noalign{}
\endhead
\bottomrule\noalign{}
\endlastfoot
A & 20 & 11 & -9 & 81 \\
A & 11 & 11 & 0 & 0 \\
A & 2 & 11 & 9 & 81 \\
B & 6 & 5 & -1 & 1 \\
B & 2 & 5 & 3 & 9 \\
B & 7 & 5 & -2 & 4 \\
C & 2 & 5 & 3 & 9 \\
C & 11 & 5 & -6 & 36 \\
C & 2 & 5 & 3 & 9 \\
Sums & 63 & 63 & 0 & 230 \\
Means & 7 & 7 & 0 & 25.5555555555556 \\
\end{longtable}

Alright, we did almost the same thing as we did to find
\(SS_\text{Effect}\). Can you spot the difference? This time for each
score we first found the group mean, then we found the error in the
group mean estimate for each score. In other words, the values in the
\(diff\) column are the differences between each score and it's group
mean. The values in the \texttt{diff\_squared} column are the squared
deviations. When we sum up the squared deviations, we get another Sums
of Squares, this time it's the \(SS_\text{Error}\). This is an
appropriate name, because these deviations are the ones that the group
means can't explain!

\subsection{Degrees of freedom}\label{degrees-of-freedom-1}

Degrees of freedom come into play again with ANOVA. This time, their
purpose is a little bit more clear. \(Df\)s can be fairly simple when we
are doing a relatively simple ANOVA like this one, but they can become
complicated when designs get more complicated.

Let's talk about the degrees of freedom for the \(SS_\text{Effect}\) and
\(SS_\text{Error}\).

The formula for the degrees of freedom for \(SS_\text{Effect}\) is

\(df_\text{Effect} = \text{Groups} -1\), where Groups is the number of
groups in the design.

In our example, there are 3 groups, so the df is 3-1 = 2. You can think
of the df for the effect this way. When we estimate the grand mean (the
overall mean), we are taking away a degree of freedom for the group
means. Two of the group means can be anything they want (they have
complete freedom), but in order for all three to be consistent with the
Grand Mean, the last group mean has to be fixed.

The formula for the degrees of freedom for \(SS_\text{Error}\) is

\(df_\text{Error} = \text{scores} - \text{groups}\), or the number of
scores minus the number of groups. We have 9 scores and 3 groups, so our
\(df\) for the error term is 9-3 = 6. Remember, when we computed the
difference score between each score and its group mean, we had to
compute three means (one for each group) to do that. So, that reduces
the degrees of freedom by 3. 6 of the difference scores could be
anything they want, but the last 3 have to be fixed to match the means
from the groups.

\subsection{Mean Squared Error}\label{mean-squared-error}

OK, so we have the degrees of freedom. What's next? There are two steps
left. First we divide the \(SS\)es by their respective degrees of
freedom to create something new called Mean Squared Error. Let's talk
about why we do this.

First of all, remember we are trying to accomplish this goal:

\(\text{F} = \frac{\text{measure of effect}}{\text{measure of error}}\)

We want to build a ratio that divides a measure of an effect by a
measure of error. Perhaps you noticed that we already have a measure of
an effect and error! How about the \(SS_\text{Effect}\) and
\(SS_\text{Error}\). They both represent the variation due to the
effect, and the leftover variation that is unexplained. Why don't we
just do this?

\(\frac{SS_\text{Effect}}{SS_\text{Error}}\)

Well, of course you could do that. What would happen is you can get some
really big and small numbers for your inferential statistic. And, the
kind of number you would get wouldn't be readily interpretable like a
\(t\) value or a \(z\) score.

The solution is to \textbf{normalize} the \(SS\) terms. Don't worry,
normalize is just a fancy word for taking the average, or finding the
mean. Remember, the SS terms are all sums. And, each sum represents a
different number of underlying properties.

For example, the SS\_\text{Effect} represents the sum of variation for
three means in our study. We might ask the question, well, what is the
average amount of variation for each mean\ldots You might think to
divide SS\_\text{Effect} by 3, because there are three means, but
because we are estimating this property, we divide by the degrees of
freedom instead (\# groups - 1 = 3-1 = 2). Now we have created something
new, it's called the \(MSE_\text{Effect}\).

\(MSE_\text{Effect} = \frac{SS_\text{Effect}}{df_\text{Effect}}\)

\(MSE_\text{Effect} = \frac{72}{2} = 36\)

This might look alien and seem a bit complicated. But, it's just another
mean. It's the mean of the sums of squares for the effect. If this
reminds you of the formula for the variance, good memory. The
\(SME_\text{Effect}\) is a measure variance for the change in the data
due to changes in the means (which are tied to the experimental
conditions).

The \(SS_\text{Error}\) represents the sum of variation for nine scores
in our study. That's a lot more scores, so the \(SS_\text{Error}\) is
often way bigger than than \(SS_\text{Effect}\). If we left our SSes
this way and divided them, we would almost always get numbers less than
one, because the \(SS_\text{Error}\) is so big. What we need to do is
bring it down to the average size. So, we might want to divide our
\(SS_\text{Error}\) by 9, after all there were nine scores. However,
because we are estimating this property, we divide by the degrees of
freedom instead (scores-groups) = 9-3 = 6). Now we have created
something new, it's called the \(MSE_\text{Error}\).

\(MSE_\text{Error} = \frac{SS_\text{Error}}{df_\text{Error}}\)

\(MSE_\text{Error} = \frac{230}{6} = 38.33\)

\subsection{Calculate F}\label{calculate-f}

Now that we have done all of the hard work, calculating \(F\) is easy:

\(\text{F} = \frac{\text{measure of effect}}{\text{measure of error}}\)

\(\text{F} = \frac{MSE_\text{Effect}}{MSE_\text{Error}}\)

\(\text{F} = \frac{36}{38.33} = .939\)

Done!

\subsection{The ANOVA TABLE}\label{the-anova-table}

You might suspect we aren't totally done here. We've walked through the
steps of computing \(F\). Remember, \(F\) is a sample statistic, we
computed \(F\) directly from the data. There were a whole bunch of
pieces we needed, the dfs, the SSes, the MSEs, and then finally the F.

All of these little pieces are conveniently organized by ANOVA tables.
ANOVA tables look like this:

\begin{longtable}[]{@{}lrrrrr@{}}
\toprule\noalign{}
& Df & Sum Sq & Mean Sq & F value & Pr(\textgreater F) \\
\midrule\noalign{}
\endhead
\bottomrule\noalign{}
\endlastfoot
groups & 2 & 72 & 36.00000 & 0.9391304 & 0.4417359 \\
Residuals & 6 & 230 & 38.33333 & NA & NA \\
\end{longtable}

You are looking at the print-out of an ANOVA summary table from R.
Notice, it had columns for \(Df\), \(SS\) (Sum Sq), \(MSE\) (Mean Sq),
\(F\), and a \(p\)-value. There are two rows. The \texttt{groups} row is
for the Effect (what our means can explain). The \texttt{Residuals} row
is for the Error (what our means can't explain). Different programs give
slightly different labels, but they are all attempting to present the
same information in the ANOVA table. There isn't anything special about
the ANOVA table, it's just a way of organizing all the pieces. Notice,
the MSE for the effect (36) is placed above the MSE for the error
(38.333), and this seems natural because we divide 36/38.33 in or to get
the \(F\)-value!

\section{What does F mean?}\label{what-does-f-mean}

We've just noted that the ANOVA has a bunch of numbers that we
calculated straight from the data. All except one, the \(p\)-value. We
did not calculate the \(p\)-value from the data. Where did it come from,
what does it mean? How do we use this for statistical inference. Just so
you don't get too worried, the \(p\)-value for the ANOVA has the very
same general meaning as the \(p\)-value for the \(t\)-test, or the
\(p\)-value for any sample statistic. It tells us that the probability
that we would observe our test statistic or larger, under the
distribution of no differences (the null).

As we keep saying, \(F\) is a sample statistic. Can you guess what we do
with sample statistics in this textbook? We did it for the Crump Test,
the Randomization Test, and the \(t\)-test\ldots{} We make fake data, we
simulate it, we compute the sample statistic we are interested in, then
we see how it behaves over many replications or simulations.

Let's do that for \(F\). This will help you understand what \(F\) really
is, and how it behaves. We are going to created the sampling
distribution of \(F\). Once we have that you will be able to see where
the \(p\)-values come from. It's the same basic process that we followed
for the \(t\) tests, except we are measuring \(F\) instead of \(t\).

Here is the set-up, we are going to run an experiment with three levels.
In our imaginary experiment we are going to test whether a new magic
pill can make you smarter. The independent variable is the number of
magic pills you take: 1, 2, or 3. We will measure your smartness using a
smartness test. We will assume the smartness test has some known
properties, the mean score on the test is 100, with a standard deviation
of 10 (and the distribution is normal).

The only catch is that our magic pill does NOTHING AT ALL. The fake
people in our fake experiment will all take sugar pills that do
absolutely nothing to their smartness. Why would we want to simulate
such a bunch of nonsense? The answer is that this kind of simulation is
critical for making inferences about chance if you were to conduct a
real experiment.

Here are some more details for the experiment. Each group will have 10
different subjects, so there will be a total of 30 subjects. We are
going to run this experiment 10,000 times. Each time drawing numbers
randomly from the very same normal distribution. We are going to
calculate \(F\) from our sample data every time, and then we are going
to draw the histogram of \(F\)-values. Figure~\ref{fig-8fnull} shows the
sampling distribution of \(F\) for our situation.

\begin{figure}

\centering{

\includegraphics[width=1\linewidth,height=\textheight,keepaspectratio]{08-ANOVA_files/figure-pdf/fig-8fnull-1.pdf}

}

\caption{\label{fig-8fnull}A simulation of 10,000 experiments from a
null distribution where there is no differences. The histogram shows
10,000 \(F\)-values, one for each simulation. These are values that F
can take in this situation. All of these \(F\)-values were produced by
random sampling error.}

\end{figure}%

Let's note a couple things about the \(F\) distribution. 1) The smallest
value is 0, and there are no negative values. Does this make sense?
\(F\) can never be negative because it is the ratio of two variances,
and variances are always positive because of the squaring operation. So,
yes, it makes sense that the sampling distribution of \(F\) is always 0
or greater. 2) it does not look normal. No it does not. \(F\) can have
many different looking shapes, depending on the degrees of freedom in
the numerator and denominator. However, these aspects are too important
for now.

Remember, before we talked about some intuitive ideas for understanding
\(F\), based on the idea that \(F\) is a ratio of what we can explain
(variance due to mean differences), divided by what we can't explain
(the error variance). When the error variance is higher than the effect
variance, then we will always get an \(F\)-value less than one. You can
see that we often got \(F\)-values less than one in the simulation. This
is sensible, after all we were simulating samples coming from the very
same distribution. On average there should be no differences between the
means. So, on average the part of the total variance that is explained
by the means should be less than one, or around one, because it should
be roughly the same as the amount of error variance (remember, we are
simulating no differences).

At the same time, we do see that some \(F\)-values are larger than 1.
There are little bars that we can see going all the way up to about 5.
If you were to get an \(F\)-value of 5, you might automatically think,
that's a pretty big \(F\)-value. Indeed it kind of is, it means that you
can explain 5 times more of variance than you can't explain. That seems
like a lot. You can also see that larger \(F\)-values don't occur very
often. As a final reminder, what you are looking at is how the
\(F\)-statistic (measured from each of 10,000 simulated experiments)
behaves when the only thing that can cause differences in the means is
random sampling error. Just by chance sometimes the means will be
different. You are looking at another chance window. These are the
\(F\)s that chance can produce.

\subsection{Making Decisions}\label{making-decisions}

We can use the sampling distribution of \(F\) (for the null) to make
decisions about the role of chance in a real experiment. For example, we
could do the following.

\begin{enumerate}
\def\labelenumi{\arabic{enumi}.}
\tightlist
\item
  Set an alpha criterion of \(p\) = 0.05
\item
  Find out the critical value for \(F\), for our particular situation
  (with our \(df\)s for the numerator and denominator).
\end{enumerate}

Let's do that. I've drawn the line for the critical value onto the
histogram in Figure~\ref{fig-8criticalF}:

\begin{figure}

\centering{

\includegraphics[width=1\linewidth,height=\textheight,keepaspectratio]{08-ANOVA_files/figure-pdf/fig-8criticalF-1.pdf}

}

\caption{\label{fig-8criticalF}The critical value for \(F\) where 5\% of
all \(F\)-values lie beyond this point}

\end{figure}%

Alright, now we can see that only 5\% of all \(F\)-values from from this
sampling distribution will be 3.35 or larger. We can use this
information.

How would we use it? Imagine we ran a real version of this experiment.
And, we really used some pills that just might change smartness. If we
ran the exact same design, with 30 people in total (10 in each group),
we could set an \(F\) criterion of 3.35 for determining whether any of
our results reflected a causal change in smartness due to the pills, and
not due to random chance. For example, if we found an \(F\)-value of
3.34, which happens, just less than 5\% of the time, we might conclude
that random sampling error did not produce the differences between our
means. Instead, we might be more confident that the pills actually did
something, after all an \(F\)-value of 3.34 doesn't happen very often,
it is unlikely (only 5 times out of 100) to occur by chance.

\subsection{Fs and means}\label{fs-and-means}

Up to here we have been building your intuition for understanding \(F\).
We went through the calculation of \(F\) from sample data. We went
through the process of simulating thousands of \(F\)s to show you the
null distribution. We have not talked so much about what researchers
really care about\ldots The MEANS! The actual results from the
experiment. Were the means different? that's often what people want to
know. So, now we will talk about the means, and \(F\), together.

Notice, if I told you I ran an experiment with three groups, testing
whether some manipulation changes the behavior of the groups, and I told
you that I found a big \(F\)!, say an \(F\) of 6!. And, that the \(F\)
of 6 had a \(p\)-value of .001. What would you know based on that
information alone? You would only know that Fs of 6 don't happen very
often by chance. In fact they only happen 0.1\% of the time, that's
hardly at all. If someone told me those values, I would believe that the
results they found in their experiment were not likely due to chance.
However, I still would not know what the results of the experiment were!
Nobody told us what the means were in the different groups, we don't
know what happened!

IMPORTANT: even though we don't know what the means were, we do know
something about them, whenever we get \(F\)-values and \(p\)-values like
that (big \(F\)s, and very small associated \(p\)s)\ldots{} Can you
guess what we know? I'll tell you. We automatically know that there
\textbf{must have been some differences between the means}. If there was
no differences between the means, then the variance explained by the
means (the numerator for \(F\)) would not be very large. So, we know
that there must be some differences, we just don't know what they are.
Of course, if we had the data, all we would need to do is look at the
means for the groups (the ANOVA table doesn't report this, we need to do
it as a separate step).

\subsubsection{ANOVA is an omnibus test}\label{anova-is-an-omnibus-test}

This property of the ANOVA is why the ANOVA is sometimes called the
\textbf{omnibus test}. Omnibus is a fun word, it sounds like a bus I'd
like to ride. The meaning of omnibus, according to the dictionary, is
``comprising several items''. The ANOVA is, in a way, one omnibus test,
comprising several little tests.

For example, if you had three groups, A, B, and C. You get could
differences between

\begin{enumerate}
\def\labelenumi{\arabic{enumi}.}
\tightlist
\item
  A and B
\item
  B and C
\item
  A and C
\end{enumerate}

That's three possible differences you could get. You could run separate
\(t\)-tests, to test whether each of those differences you might have
found could have been produced by chance. Or, you could run an ANOVA,
like what we have been doing, to ask one more general question about the
differences. Here is one way to think about what the omnibus test is
testing:

Hypothesis of no differences anywhere: \$ A = B = C \$

Any differences anywhere:

\begin{enumerate}
\def\labelenumi{\alph{enumi}.}
\tightlist
\item
  \$ A \neq B = C \$
\item
  \$ A = B \neq C \$
\item
  \$ A \neq C = B \$
\end{enumerate}

The \(\neq\) symbol means ``does not equal'', it's an equal sign with a
cross through it (no equals allowed!).

How do we put all of this together. Generally, when we get a small
\(F\)-value, with a large \(p\)-value, we will not reject the hypothesis
of no differences. We will say that we do not have evidence that the
means of the three groups are in any way different, and the differences
that are there could easily have been produced by chance. When we get a
large F with a small \(p\)-value (one that is below our alpha
criterion), we will generally reject the hypothesis of no differences.
We would then assume that at least one group mean is not equal to one of
the others. That is the omnibus test. Rejecting the null in this way is
rejecting the idea there are no differences. But, the \(F\) test still
does not tell you which of the possible group differences are the ones
that are different.

\subsubsection{Looking at a bunch of group
means}\label{looking-at-a-bunch-of-group-means}

We just ran 10,000 experiments and we didn't even once look at the group
means for any of the experiments. Different patterns of group means
under the null are shown in Figure~\ref{fig-8manyDiffs} for a subset of
10 random simulations.

\begin{figure}

\centering{

\includegraphics[width=1\linewidth,height=\textheight,keepaspectratio]{08-ANOVA_files/figure-pdf/fig-8manyDiffs-1.pdf}

}

\caption{\label{fig-8manyDiffs}Different patterns of group means under
the null (all scores for each group sampled from the same
distribution).}

\end{figure}%

Whoa, that's a lot to look at. What is going on here? Each little box
represents the outcome of a simulated experiment. The dots are the means
for each group (whether subjects took 1 , 2, or 3 magic pills). The
y-axis shows the mean smartness for each group. The error bars are
standard errors of the mean.

You can see that each of the 10 experiments turn out different.
Remember, we sampled 10 numbers for each group from the \textbf{same}
normal distribution with mean = 100, and sd = 10. So, we know that the
\textbf{correct} means for each sample should actually be 100 every
single time. However, they are not 100 every single time because
of?\ldots{}\textbf{sampling error} (Our good friend that we talk about
all the time).

For most of the simulations the error bars are all overlapping, this
suggests visually that the means are not different. However, some of
them look like they are not overlapping so much, and this would suggest
that they are different. This is the siren song of chance (sirens lured
sailors to their deaths at sea\ldots beware of the siren call of
chance). If we concluded that any of these sets of means had a true
difference, we would be committing a type I error. Because we made the
simulation, we know that none of these means are actually different.
But, when you are running a real experiment, you don't get to know this
for sure.

\subsubsection{Looking at bar graphs}\label{looking-at-bar-graphs}

Let's look at the exact same graph as above, but this time use bars to
visually illustrate the means, instead of dots. We'll re-do our
simulation of 10 experiments, so the pattern will be a little bit
different:

\begin{figure}

\centering{

\includegraphics[width=1\linewidth,height=\textheight,keepaspectratio]{08-ANOVA_files/figure-pdf/fig-8manyDiffsBar-1.pdf}

}

\caption{\label{fig-8manyDiffsBar}Different patterns of group means
under the null (all scores for each group sampled from the same
distribution).}

\end{figure}%

In Figure~\ref{fig-8manyDiffsBar} the heights of the bars display the
means for each pill group. The pattern across simulations is generally
the same. Some of the fake experiments look like there might be
differences, and some of them don't.

\subsubsection{\texorpdfstring{What mean differences look like when
\(F\) is less than
1}{What mean differences look like when F is less than 1}}\label{what-mean-differences-look-like-when-f-is-less-than-1}

We are now giving you some visual experience looking at what means look
like from a particular experiment. This is for your stats intuition.
We're trying to improve your data senses.

What we are going to do now is similar to what we did before. Except
this time we are going to look at 10 simulated experiments, where all of
the \(F\)-values were less than 1. All of these \(F\)-values would also
be associated with fairly large \(p\)-values. When F is less than one,
we would not reject the hypothesis of no differences. So, when we look
at patterns of means when F is less than 1, we should see mostly the
same means, and no big differences.

\begin{figure}

\centering{

\includegraphics[width=1\linewidth,height=\textheight,keepaspectratio]{08-ANOVA_files/figure-pdf/fig-8flesshthanone-1.pdf}

}

\caption{\label{fig-8flesshthanone}Different patterns of group means
under the null (sampled from same distribution) when F is less than 1.}

\end{figure}%

In Figure~\ref{fig-8flesshthanone} the numbers in the panels now tell us
which simulations actually produced \(F\)s of less than 1.

We see here that all the bars aren't perfectly flat, that's OK. What's
more important is that for each panel, the error bars for each mean are
totally overlapping with all the other error bars. We can see visually
that our estimate of the mean for each sample is about the same for all
of the bars. That's good, we wouldn't make any type I errors here.

\subsubsection{What mean differences look like when F \textgreater{}
3.35}\label{what-mean-differences-look-like-when-f-3.35}

Earlier we found that the critical value for \(F\) in our situation was
3.35, this was the location on the \(F\) distribution where only 5\% of
\(F\)s were 3.35 or greater. We would reject the hypothesis of no
differences whenever \(F\) was greater than 3.35. In this case, whenever
we did that, we would be making a type I error. That is because we are
simulating the distribution of no differences (remember all of our
sample means are coming from the exact same distribution). So, now we
can take a look at what type I errors look like. In other words, we can
run some simulations and look at the pattern in the means, only when
\(F\) happens to be 3.35 or greater (this only happens 5\% of the time,
so we might have to let the computer simulate for a while). Let's see
what that looks like:

\begin{figure}

\centering{

\includegraphics[width=1\linewidth,height=\textheight,keepaspectratio]{08-ANOVA_files/figure-pdf/fig-8sigdiffs-1.pdf}

}

\caption{\label{fig-8sigdiffs}Different patterns of group means under
the null when F is above critical value (these are all type I Errors).}

\end{figure}%

The numbers in the panels now tell us which simulations actually
produced \(F\)s that were greater than 3.35

What do you notice about the pattern of means inside each panel of
Figure~\ref{fig-8sigdiffs}? Now, every the panels show at least one mean
that is different from the others. Specifically, the error bars for one
mean do not overlap with the error bars for one or another mean. This is
what mistakes looks like. These are all type I errors. They are
insidious. When they happen to you by chance, the data really does
appear to show a strong pattern, your \(F\)-value is large, and your
\(p\)-value is small! It is easy to be convinced by a type I error (it's
the siren song of chance).

\section{ANOVA on Real Data}\label{anova-on-real-data}

We've covered many fundamentals about the ANOVA, how to calculate the
necessary values to obtain an \(F\)-statistic, and how to interpret the
\(F\)-statistic along with it's associate \(p\)-value once we have one.
In general, you will be conducting ANOVAs and playing with \(F\)s and
\(p\)s using software that will automatically spit out the numbers for
you. It's important that you understand what the numbers mean, that's
why we've spent time on the concepts. We also recommend that you try to
compute an ANOVA by hand at least once. It builds character, and let's
you know that you know what you are doing with the numbers.

But, we've probably also lost the real thread of all this. The core
thread is that when we run an experiment we use our inferential
statistics, like ANOVA, to help us determine whether the differences we
found are likely due to chance or not. In general, we like to find out
that the differences that we find are not due to chance, but instead to
due to our manipulation.

So, we return to the application of the ANOVA to a real data set with a
real question. This is the same one that you will be learning about in
the lab. We give you a brief overview here so you know what to expect.

\subsection{Tetris and bad memories}\label{tetris-and-bad-memories}

Yup, you read that right. The research you will learn about tests
whether playing Tetris after watching a scary movie can help prevent you
from having bad memories from the movie (James et al. 2015). Sometimes
in life people have intrusive memories, and they think about things
they'd rather not have to think about. This research looks at one method
that could reduce the frequency of intrusive memories.

Here's what they did. Subjects watched a scary movie, then at the end of
the week they reported how many intrusive memories about the movie they
had. The mean number of intrusive memories was the measurement (the
dependent variable). This was a between-subjects experiment with four
groups. Each group of subjects received a different treatment following
the scary movie. The question was whether any of these treatments would
reduce the number of intrusive memories. All of these treatments
occurred after watching the scary movie:

\begin{enumerate}
\def\labelenumi{\arabic{enumi}.}
\tightlist
\item
  No-task control: These participants completed a 10-minute music filler
  task after watching the scary movie.
\item
  Reactivation + Tetris: These participants were shown a series of
  images from the trauma film to reactivate the traumatic memories
  (i.e., reactivation task). Then, participants played the video game
  Tetris for 12 minutes.
\item
  Tetris Only: These participants played Tetris for 12 minutes, but did
  not complete the reactivation task.
\item
  Reactivation Only: These participants completed the reactivation task,
  but did not play Tetris.
\end{enumerate}

For reasons we elaborate on in the lab, the researchers hypothesized
that the \texttt{Reactivation+Tetris} group would have fewer intrusive
memories over the week than the other groups.

Let's look at the findings. Note you will learn how to do all of these
steps in the lab. For now, we just show the findings and the ANOVA
table. Then we walk through how to interpret it.

\begin{figure}

\centering{

\includegraphics[width=1\linewidth,height=\textheight,keepaspectratio]{08-ANOVA_files/figure-pdf/fig-8tetrisData-1.pdf}

}

\caption{\label{fig-8tetrisData}Mean number of intrusive memories per
week as a function of experimental treatments.}

\end{figure}%

OOooh, look at that. We did something fancy.
Figure~\ref{fig-8tetrisData} shows the data from the four groups. The
height of each bar shows the mean intrusive memories for the week. The
dots show the individual scores for each subject in each group (useful
to to the spread of the data). The error bars show the standard errors
of the mean.

What can we see here? Right away it looks like there is some support for
the research hypothesis. The green bar, for the Reactivation + Tetris
group had the lowest mean number of intrusive memories. Also, the error
bar is not overlapping with any of the other error bars. This implies
that the mean for the Reactivation + Tetris group is different from the
means for the other groups. And, this difference is probably not very
likely by chance.

We can now conduct the ANOVA on the data to ask the omnibus question. If
we get a an \(F\)-value with an associated \(p\)-value of less than .05
(the alpha criterion set by the authors), then we can reject the
hypothesis of no differences. Let's see what happens:

\begin{longtable}[]{@{}lrrrrr@{}}
\toprule\noalign{}
& Df & Sum Sq & Mean Sq & F value & Pr(\textgreater F) \\
\midrule\noalign{}
\endhead
\bottomrule\noalign{}
\endlastfoot
Condition & 3 & 114.8194 & 38.27315 & 3.794762 & 0.0140858 \\
Residuals & 68 & 685.8333 & 10.08578 & NA & NA \\
\end{longtable}

We see the ANOVA table, it's up there. We could report the results from
the ANOVA table like this:

\begin{quote}
There was a significant main effect of treatment condition, F(3, 68) =
3.79, MSE = 10.08, p=0.014.
\end{quote}

We called this a significant effect because the \(p\)-value was less
than 0.05. In other words, the \(F\)-value of 3.79 only happens 1.4\% of
the time when the null is true. Or, the differences we observed in the
means only occur by random chance (sampling error) 1.4\% of the time.
Because chance rarely produces this kind of result, the researchers made
the inference that chance DID NOT produce their differences, instead,
they were inclined to conclude that the Reactivation + Tetris treatment
really did cause a reduction in intrusive memories. That's pretty neat.

\subsection{Comparing means after the
ANOVA}\label{comparing-means-after-the-anova}

Remember that the ANOVA is an omnibus test, it just tells us whether we
can reject the idea that all of the means are the same. The F-test
(synonym for ANOVA) that we just conducted suggested we could reject the
hypothesis of no differences. As we discussed before, that must mean
that there are some differences in the pattern of means.

Generally after conducting an ANOVA, researchers will conduct follow-up
tests to compare differences between specific means. We will talk more
about this practice throughout the textbook. There are many recommended
practices for follow-up tests, and there is a lot of debate about what
you should do. We are not going to wade into this debate right now.
Instead we are going to point out that \textbf{you need to do something}
to compare the means of interest after you conduct the ANOVA, because
the ANOVA is just the beginning\ldots It usually doesn't tell you want
you want to know. You might wonder why bother conducting the ANOVA in
the first place\ldots Not a terrible question at all. A good question.
You will see as we talk about more complicated designs, why ANOVAs are
so useful. In the present example, they are just a common first step.
There are required next steps, such as what we do next.

How can you compare the difference between two means, from a
between-subjects design, to determine whether or not the difference you
observed is likely or unlikely to be produced by chance? We covered this
one already, it's the independent \(t\)-test. We'll do a couple
\(t\)-tests, showing the process.

\subsubsection{Control
vs.~Reactivation+Tetris}\label{control-vs.-reactivationtetris}

What we really want to know is if Reactivation+Tetris caused fewer
intrusive memories\ldots but compared to what? Well, if it did
something, the Reactivation+Tetris group should have a smaller mean than
the Control group. So, let's do that comparison:

\begin{verbatim}
#> 
#>  Two Sample t-test
#> 
#> data:  Days_One_to_Seven_Number_of_Intrusions by Condition
#> t = 2.9893, df = 34, p-value = 0.005167
#> alternative hypothesis: true difference in means between group Control and group Reactivation+Tetris is not equal to 0
#> 95 percent confidence interval:
#>  1.031592 5.412852
#> sample estimates:
#>             mean in group Control mean in group Reactivation+Tetris 
#>                          5.111111                          1.888889
\end{verbatim}

We found that there was a significant difference between the control
group (M=5.11) and Reactivation + Tetris group (M=1.89), t(34) = 2.99,
p=0.005.

Above you just saw an example of reporting another \(t\)-test. This
sentences does an OK job of telling the reader everything they want to
know. It has the means for each group, and the important bits from the
\(t\)-test.

More important, as we suspected the difference between the control and
Reactivation + Tetris group was likely not due to chance.

\subsubsection{Control vs.~Tetris\_only}\label{control-vs.-tetris_only}

Now we can really start wondering what caused the difference. Was it
just playing Tetris? Does just playing Tetris reduce the number of
intrusive memories during the week? Let's compare that to control:

\begin{verbatim}
#> 
#>  Two Sample t-test
#> 
#> data:  Days_One_to_Seven_Number_of_Intrusions by Condition
#> t = 1.0129, df = 34, p-value = 0.3183
#> alternative hypothesis: true difference in means between group Control and group Tetris_only is not equal to 0
#> 95 percent confidence interval:
#>  -1.230036  3.674480
#> sample estimates:
#>     mean in group Control mean in group Tetris_only 
#>                  5.111111                  3.888889
\end{verbatim}

Here we did not find a significant difference. We found that no
significant difference between the control group (M=5.11) and Tetris
Only group (M=3.89), t(34) = 2.99, p=0.318.

So, it seems that not all of the differences between our means are large
enough to be called statistically significant. In particular, the
difference here, or larger, happens by chance 31.8\% of the time.

You could go on doing more comparisons, between all of the different
pairs of means. Each time conducting a \(t\)-test, and each time saying
something more specific about the patterns across the means than you get
to say with the omnibus test provided by the ANOVA.

Usually, it is the pattern of differences across the means that you as a
researcher are primarily interested in understanding. Your theories will
make predictions about how the pattern turns out (e.g., which specific
means should be higher or lower and by how much). So, the practice of
doing comparisons after an ANOVA is really important for establishing
the patterns in the means.

\section{ANOVA Summary}\label{anova-summary}

We have just finished a rather long introduction to the ANOVA, and the
\(F\)-test. The next couple of chapters continue to explore properties
of the ANOVA for different kinds of experimental designs. In general,
the process to follow for all of the more complicated designs is very
similar to what we did here, which boils down to two steps:

\begin{enumerate}
\def\labelenumi{\arabic{enumi})}
\tightlist
\item
  conduct the ANOVA on the data
\item
  conduct follow-up tests, looking at differences between particular
  means
\end{enumerate}

So what's next\ldots the ANOVA for repeated measures designs. See you in
the next chapter.

\bookmarksetup{startatroot}

\chapter{Repeated Measures ANOVA}\label{repeated-measures-anova}

This chapter introduces you to \textbf{repeated measures ANOVA}.
Repeated measures ANOVAs are very common in Psychology, because
psychologists often use repeated measures designs, and repeated measures
ANOVAs are the appropriate test for making inferences about repeated
measures designs.

Remember the paired sample \(t\)-test? We used that test to compare two
means from a repeated measures design. Remember what a repeated measures
design is? It's also called a within-subjects design. These designs
involve measuring the same subject more than once. Specifically, at
least once for every experimental condition. In the paired \(t\)-test
example, we discussed a simple experiment with only two experimental
conditions. There, each subject would contribute a measurement to level
one and level two of the design.

However, paired-samples \(t\)-tests are limited to comparing two means.
What if you had a design that had more than two experimental conditions?
For example, perhaps your experiment had 3 levels for the independent
variable, and each subject contributed data to each of the three levels?

This is starting to sounds like an ANOVA problem. ANOVAs are capable of
evaluating whether there is a difference between any number of means,
two or greater. So, we can use an ANOVA for our repeated measures design
with three levels for the independent variable.

Great! So, what makes a repeated measures ANOVA different from the ANOVA
we just talked about?

\section{Repeated measures design}\label{repeated-measures-design}

Let's use the exact same toy example from the previous chapter, but
let's convert it to a repeated measures design.

Last time, we imagined we had some data in three groups, A, B, and C,
such as in Table~\ref{tbl-8rmdata1}:

\begin{longtable}[]{@{}lr@{}}

\caption{\label{tbl-8rmdata1}Example data for three groups.}

\tabularnewline

\toprule\noalign{}
groups & scores \\
\midrule\noalign{}
\endhead
\bottomrule\noalign{}
\endlastfoot
A & 20 \\
A & 11 \\
A & 2 \\
B & 6 \\
B & 2 \\
B & 7 \\
C & 2 \\
C & 11 \\
C & 2 \\

\end{longtable}

The above table represents a between-subject design where each score
involves a unique subject.

Let's change things up a tiny bit, and imagine we only had 3 subjects in
total in the experiment. And, that each subject contributed data to the
three levels of the independent variable, A, B, and C. Before we called
the IV \texttt{groups}, because there were different groups of subjects.
Let's change that to \texttt{conditions}, because now the same group of
subjects participates in all three conditions. Table~\ref{tbl-8rmdata2}
shows a within-subjects (repeated measures) version of this experiment:

\begin{longtable}[]{@{}rlr@{}}

\caption{\label{tbl-8rmdata2}Example data for a repeated measures design
with three conditions, where each subject contributes data in each
condition.}

\tabularnewline

\toprule\noalign{}
subjects & conditions & scores \\
\midrule\noalign{}
\endhead
\bottomrule\noalign{}
\endlastfoot
1 & A & 20 \\
2 & A & 11 \\
3 & A & 2 \\
1 & B & 6 \\
2 & B & 2 \\
3 & B & 7 \\
1 & C & 2 \\
2 & C & 11 \\
3 & C & 2 \\

\end{longtable}

\section{Partitioning the Sums of
Squares}\label{partitioning-the-sums-of-squares}

Time to introduce a new name for an idea you learned about last chapter,
it's called \textbf{partitioning the sums of squares}. Sometimes an
obscure new name can be helpful for your understanding of what is going
on. ANOVAs are all about partitioning the sums of squares. We already
did some partitioning in the last chapter. What do we mean by
partitioning?

Imagine you had a big empty house with no rooms in it. What would happen
if you partitioned the house? What would you be doing? One way to
partition the house is to split it up into different rooms. You can do
this by adding new walls and making little rooms everywhere. That's what
partitioning means, to split up.

The act of partitioning, or splitting up, is the core idea of ANOVA. To
use the house analogy. Our total sums of squares (SS Total) is our big
empty house. We want to split it up into little rooms. Before we
partitioned SS Total using this formula:

\(SS_\text{TOTAL} = SS_\text{Effect} + SS_\text{Error}\)

Remember, the \(SS_\text{Effect}\) was the variance we could attribute
to the means of the different groups, and \(SS_\text{Error}\) was the
leftover variance that we couldn't explain. \(SS_\text{Effect}\) and
\(SS_\text{Error}\) are the partitions of \(SS_\text{TOTAL}\), they are
the little rooms.

In the between-subjects case above, we got to split \(SS_\text{TOTAL}\)
into two parts. What is most interesting about the repeated-measures
design, is that we get to split \(SS_\text{TOTAL}\) into three parts,
there's one more partition. Can you guess what the new partition is?
Hint: whenever we have a new way to calculate means in our design, we
can always create a partition for those new means. What are the new
means in the repeated measures design?

Here is the new idea for partitioning \(SS_\text{TOTAL}\) in a
repeated-measures design:

\(SS_\text{TOTAL} = SS_\text{Effect} + SS_\text{Subjects} +SS_\text{Error}\)

We've added \(SS_\text{Subjects}\) as the new idea in the formula.
What's the idea here? Well, because each subject was measured in each
condition, we have a new set of means. These are the means for each
subject, collapsed across the conditions. For example, subject 1 has a
mean (mean of their scores in conditions A, B, and C); subject 2 has a
mean (mean of their scores in conditions A, B, and C); and subject 3 has
a mean (mean of their scores in conditions A, B, and C). There are three
subject means, one for each subject, collapsed across the conditions.
And, we can now estimate the portion of the total variance that is
explained by these subject means.

We just showed you a ``formula'' to split up \(SS_\text{TOTAL}\) into
three parts, but we called the formula an idea. We did that because the
way we wrote the formula is a little bit misleading, and we need to
clear something up. Before we clear the thing up, we will confuse you
just a little bit. Be prepared to be confused a little bit.

First, we need to introduce you to some more terms. It turns out that
different authors use different words to describe parts of the ANOVA.
This can be really confusing. For example, we described the SS formula
for a between subjects design like this:

\(SS_\text{TOTAL} = SS_\text{Effect} + SS_\text{Error}\)

However, the very same formula is often written differently, using the
words between and within in place of effect and error, it looks like
this:

\(SS_\text{TOTAL} = SS_\text{Between} + SS_\text{Within}\)

Whoa, hold on a minute. Haven't we switched back to talking about a
\textbf{between-subjects} ANOVA. YES! Then why are we using the word
\textbf{within}, what does that mean? YES! We think this is very
confusing for people. Here the word \textbf{within} has a special
meaning. It \textbf{does not} refer to a within-subjects design. Let's
explain. First, \(SS_\text{Between}\) (which we have been calling
\(SS_\text{Effect}\)) refers to variation \textbf{between} the group
means, that's why it is called \(SS_\text{Between}\). Second, and most
important, \(SS_\text{Within}\) (which we have been calling
\(SS_\text{Error}\)), refers to the leftover variation within each group
mean. Specifically, it is the variation between each group mean and each
score in the group. ``AAGGH, you've just used the word between to
describe within group variation!''. Yes! We feel your pain. Remember,
for each group mean, every score is probably off a little bit from the
mean. So, the scores within each group have some variation. This is the
within group variation, and it is why the leftover error that we can't
explain is often called \(SS_\text{Within}\).

OK. So why did we introduce this new confusing way of talking about
things? Why can't we just use \(SS_\text{Error}\) to talk about this
instead of \(SS_\text{Within}\), which you might (we do) find confusing.
We're getting there, but perhaps Figure~\ref{fig-9splitSS} will help
out.

\begin{figure}

\centering{

\includegraphics[width=1\linewidth,height=\textheight,keepaspectratio]{imgs/figures/SS_ANOVA.png}

}

\caption{\label{fig-9splitSS}Illustration showing how the total sums of
squares are partitioned differently for a between versus
repeated-measures design}

\end{figure}%

The figure lines up the partitioning of the Sums of Squares for both
between-subjects and repeated-measures designs. In both designs,
\(SS_\text{Total}\) is first split up into two pieces
\(SS_\text{Effect (between-groups)}\) and
\(SS_\text{Error (within-groups)}\). At this point, both ANOVAs are the
same. In the repeated measures case we split the
\(SS_\text{Error (within-groups)}\) into two more littler parts, which
we call \(SS_\text{Subjects (error variation about the subject mean)}\)
and \(SS_\text{Error (left-over variation we can't explain)}\).

So, when we earlier wrote the formula to split up SS in the
repeated-measures design, we were kind of careless in defining what we
actually meant by \(SS_\text{Error}\), this was a little too vague:

\(SS_\text{TOTAL} = SS_\text{Effect} + SS_\text{Subjects} +SS_\text{Error}\)

The critical feature of the repeated-measures ANOVA, is that the
\(SS_\text{Error}\) that we will later use to compute the MSE in the
denominator for the \(F\)-value, is smaller in a repeated-measures
design, compared to a between subjects design. This is because the
\(SS_\text{Error (within-groups)}\) is split into two parts,
\(SS_\text{Subjects (error variation about the subject mean)}\) and
\(SS_\text{Error (left-over variation we can't explain)}\).

To make this more clear, consider Figure~\ref{fig-9moresplit}:

\begin{figure}

\centering{

\includegraphics[width=1\linewidth,height=\textheight,keepaspectratio]{imgs/figures/SS_RMANOVA.png}

}

\caption{\label{fig-9moresplit}Close-up showing that the Error term is
split into two parts in the repeated measures design}

\end{figure}%

As we point out, the \(SS_\text{Error (left-over)}\) in the green circle
will be a smaller number than the \(SS_\text{Error (within-group)}\).
That's because we are able to subtract out the \(SS_\text{Subjects}\)
part of the \(SS_\text{Error (within-group)}\). As we will see shortly,
this can have the effect of producing larger F-values when using a
repeated-measures design compared to a between-subjects design.

\section{Calculating the RM ANOVA}\label{calculating-the-rm-anova}

Now that you are familiar with the concept of an ANOVA table (remember
the table from last chapter where we reported all of the parts to
calculate the \(F\)-value?), we can take a look at the things we need to
find out to make the ANOVA table. Figure~\ref{fig-9rmtable} presents an
abstract for the repeated-measures ANOVA table. It shows us all the
thing we need to calculate to get the \(F\)-value for our data.

\begin{figure}

\centering{

\includegraphics[width=1\linewidth,height=\textheight,keepaspectratio]{imgs/figures/RMANOVA_table.png}

}

\caption{\label{fig-9rmtable}Equations for computing the ANOVA table for
a repeated measures design}

\end{figure}%

So, what we need to do is calculate all the \(SS\)es that we did before
for the between-subjects ANOVA. That means the next three steps are
identical to the ones you did before. In fact, I will just basically
copy the next three steps to find \(SS_\text{TOTAL}\),
\(SS_\text{Effect}\) , and \(SS_\text{Error (within-conditions)}\).
After that we will talk about splitting up
\(SS_\text{Error (within-conditions)}\) into two parts, this is the new
thing for this chapter. Here we go!

\subsection{SS Total}\label{ss-total-1}

The total sums of squares, or \(SS\text{Total}\) measures the total
variation in a set of data. All we do is find the difference between
each score and the grand mean, then we square the differences and add
them all up.

\begin{longtable}[]{@{}lllll@{}}
\toprule\noalign{}
subjects & conditions & scores & diff & diff\_squared \\
\midrule\noalign{}
\endhead
\bottomrule\noalign{}
\endlastfoot
1 & A & 20 & 13 & 169 \\
2 & A & 11 & 4 & 16 \\
3 & A & 2 & -5 & 25 \\
1 & B & 6 & -1 & 1 \\
2 & B & 2 & -5 & 25 \\
3 & B & 7 & 0 & 0 \\
1 & C & 2 & -5 & 25 \\
2 & C & 11 & 4 & 16 \\
3 & C & 2 & -5 & 25 \\
Sums & & 63 & 0 & 302 \\
Means & & 7 & 0 & 33.5555555555556 \\
\end{longtable}

The mean of all of the scores is called the \textbf{Grand Mean}. It's
calculated in the table, the Grand Mean = 7.

We also calculated all of the difference scores \textbf{from the Grand
Mean}. The difference scores are in the column titled \texttt{diff}.
Next, we squared the difference scores, and those are in the next column
called \texttt{diff\_squared}.

When you add up all of the individual squared deviations (difference
scores) you get the sums of squares. That's why it's called the sums of
squares (SS).

Now, we have the first part of our answer:

\(SS_\text{total} = SS_\text{Effect} + SS_\text{Error}\)

\(SS_\text{total} = 302\) and

\(302 = SS_\text{Effect} + SS_\text{Error}\)

\subsection{SS Effect}\label{ss-effect-1}

\(SS_\text{Total}\) gave us a number representing all of the change in
our data, how they all are different from the grand mean.

What we want to do next is estimate how much of the total change in the
data might be due to the experimental manipulation. For example, if we
ran an experiment that causes causes change in the measurement, then the
means for each group will be different from other, and the scores in
each group will be different from each. As a result, the manipulation
forces change onto the numbers, and this will naturally mean that some
part of the total variation in the numbers is caused by the
manipulation.

The way to isolate the variation due to the manipulation (also called
effect) is to look at the means in each group, and the calculate the
difference scores between each group mean and the grand mean, and then
the squared deviations to find the sum for \(SS_\text{Effect}\).

Consider this table, showing the calculations for \(SS_\text{Effect}\).

\begin{longtable}[]{@{}llllll@{}}
\toprule\noalign{}
subjects & conditions & scores & means & diff & diff\_squared \\
\midrule\noalign{}
\endhead
\bottomrule\noalign{}
\endlastfoot
1 & A & 20 & 11 & 4 & 16 \\
2 & A & 11 & 11 & 4 & 16 \\
3 & A & 2 & 11 & 4 & 16 \\
1 & B & 6 & 5 & -2 & 4 \\
2 & B & 2 & 5 & -2 & 4 \\
3 & B & 7 & 5 & -2 & 4 \\
1 & C & 2 & 5 & -2 & 4 \\
2 & C & 11 & 5 & -2 & 4 \\
3 & C & 2 & 5 & -2 & 4 \\
Sums & & 63 & 63 & 0 & 72 \\
Means & & 7 & 7 & 0 & 8 \\
\end{longtable}

Notice we created a new column called \texttt{means}, these are the
means for each condition, A, B, and C.

\(SS_\text{Effect}\) represents the amount of variation that is caused
by differences between the means. The \texttt{diff} column is the
difference between each condition mean and the grand mean, so for the
first row, we have 11-7 = 4, and so on.

We found that \(SS_\text{Effect} = 72\), this is the same as the ANOVA
from the previous chapter

\subsection{SS Error
(within-conditions)}\label{ss-error-within-conditions}

Great, we made it to SS Error. We already found SS Total, and SS Effect,
so now we can solve for SS Error just like this:

\(SS_\text{total} = SS_\text{Effect} + SS_\text{Error (within-conditions)}\)

switching around:

\$ SS\_\text{Error} = SS\_\text{total} - SS\_\text{Effect} \$

\$ SS\_\text{Error (within conditions)} = 302 - 72 = 230 \$

Or, we could compute \(SS_\text{Error (within conditions)}\) directly
from the data as we did last time:

\begin{longtable}[]{@{}llllll@{}}
\toprule\noalign{}
subjects & conditions & scores & means & diff & diff\_squared \\
\midrule\noalign{}
\endhead
\bottomrule\noalign{}
\endlastfoot
1 & A & 20 & 11 & -9 & 81 \\
2 & A & 11 & 11 & 0 & 0 \\
3 & A & 2 & 11 & 9 & 81 \\
1 & B & 6 & 5 & -1 & 1 \\
2 & B & 2 & 5 & 3 & 9 \\
3 & B & 7 & 5 & -2 & 4 \\
1 & C & 2 & 5 & 3 & 9 \\
2 & C & 11 & 5 & -6 & 36 \\
3 & C & 2 & 5 & 3 & 9 \\
Sums & & 63 & 63 & 0 & 230 \\
Means & & 7 & 7 & 0 & 25.5555555555556 \\
\end{longtable}

When we compute \(SS_\text{Error (within conditions)}\) directly, we
find the difference between each score and the condition mean for that
score. This gives us the remaining error variation around the condition
mean, that the condition mean does not explain.

\subsection{SS Subjects}\label{ss-subjects}

Now we are ready to calculate new partition, called
\(SS_\text{Subjects}\). We first find the means for each subject. For
subject 1, this is the mean of their scores across Conditions A, B, and
C. The mean for subject 1 is 9.33 (repeating). Notice there is going to
be some rounding error here, that's OK for now.

The \texttt{means} column now shows all of the subject means. We then
find the difference between each subject mean and the grand mean. These
deviations are shown in the \texttt{diff} column. Then we square the
deviations, and sum them up.

\begin{longtable}[]{@{}
  >{\raggedright\arraybackslash}p{(\linewidth - 10\tabcolsep) * \real{0.1154}}
  >{\raggedright\arraybackslash}p{(\linewidth - 10\tabcolsep) * \real{0.1410}}
  >{\raggedright\arraybackslash}p{(\linewidth - 10\tabcolsep) * \real{0.0897}}
  >{\raggedright\arraybackslash}p{(\linewidth - 10\tabcolsep) * \real{0.2179}}
  >{\raggedright\arraybackslash}p{(\linewidth - 10\tabcolsep) * \real{0.2692}}
  >{\raggedright\arraybackslash}p{(\linewidth - 10\tabcolsep) * \real{0.1667}}@{}}
\toprule\noalign{}
\begin{minipage}[b]{\linewidth}\raggedright
subjects
\end{minipage} & \begin{minipage}[b]{\linewidth}\raggedright
conditions
\end{minipage} & \begin{minipage}[b]{\linewidth}\raggedright
scores
\end{minipage} & \begin{minipage}[b]{\linewidth}\raggedright
means
\end{minipage} & \begin{minipage}[b]{\linewidth}\raggedright
diff
\end{minipage} & \begin{minipage}[b]{\linewidth}\raggedright
diff\_squared
\end{minipage} \\
\midrule\noalign{}
\endhead
\bottomrule\noalign{}
\endlastfoot
1 & A & 20 & 9.33 & 2.33 & 5.4289 \\
2 & A & 11 & 8 & 1 & 1 \\
3 & A & 2 & 3.66 & -3.34 & 11.1556 \\
1 & B & 6 & 9.33 & 2.33 & 5.4289 \\
2 & B & 2 & 8 & 1 & 1 \\
3 & B & 7 & 3.66 & -3.34 & 11.1556 \\
1 & C & 2 & 9.33 & 2.33 & 5.4289 \\
2 & C & 11 & 8 & 1 & 1 \\
3 & C & 2 & 3.66 & -3.34 & 11.1556 \\
Sums & & 63 & 62.97 & -0.0299999999999994 & 52.7535 \\
Means & & 7 & 6.99666666666667 & -0.00333333333333326 & 5.8615 \\
\end{longtable}

We found that the sum of the squared deviations \(SS_\text{Subjects}\) =
52.75. Note again, this has some small rounding error because some of
the subject means had repeating decimal places, and did not divide
evenly.

We can see the effect of the rounding error if we look at the sum and
mean in the \texttt{diff} column. We know these should be both zero,
because the Grand mean is the balancing point in the data. The sum and
mean are both very close to zero, but they are not zero because of
rounding error.

\subsection{SS Error (left-over)}\label{ss-error-left-over}

Now we can do the last thing. Remember we wanted to split up the
\(SS_\text{Error (within conditions)}\) into two parts,
\(SS_\text{Subjects}\) and \(SS_\text{Error (left-over)}\). Because we
have already calculate \(SS_\text{Error (within conditions)}\) and
\(SS_\text{Subjects}\), we can solve for
\(SS_\text{Error (left-over)}\):

\(SS_\text{Error (left-over)} = SS_\text{Error (within conditions)} - SS_\text{Subjects}\)

\(SS_\text{Error (left-over)} = SS_\text{Error (within conditions)} - SS_\text{Subjects} = 230 - 52.75 = 177.25\)

\subsection{Check our work}\label{check-our-work}

Before we continue to compute the MSEs and F-value for our data, let's
quickly check our work. For example, we could have R compute the
repeated measures ANOVA for us, and then we could look at the ANOVA
table and see if we are on the right track so far.

\begin{longtable}[]{@{}lrrrrr@{}}

\caption{\label{tbl-8rmANOVA1}Example ANOVA table table reporting the
degrees of freedom, sums of squares, mean squares, \(F\) value and
associated \(p\) value.}

\tabularnewline

\toprule\noalign{}
& Df & Sum Sq & Mean Sq & F value & Pr(\textgreater F) \\
\midrule\noalign{}
\endhead
\bottomrule\noalign{}
\endlastfoot
Residuals & 2 & 52.66667 & 26.33333 & NA & NA \\
conditions & 2 & 72.00000 & 36.00000 & 0.8120301 & 0.505848 \\
Residuals & 4 & 177.33333 & 44.33333 & NA & NA \\

\end{longtable}

Table~\ref{tbl-8rmANOVA1} looks good. We found the \(SS_\text{Effect}\)
to be 72, and the SS for the conditions (same thing) in the table is
also 72. We found the \(SS_\text{Subjects}\) to be 52.75, and the SS for
the first residual (same thing) in the table is also 53.66 repeating.
That's close, and our number is off because of rounding error. Finally,
we found the \(SS_\text{Error (left-over)}\) to be 177.25, and the SS
for the bottom residuals in the table (same thing) in the table is
177.33 repeating, again close but slightly off due to rounding error.

We have finished our job of computing the sums of squares that we need
in order to do the next steps, which include computing the MSEs for the
effect and the error term. Once we do that, we can find the F-value,
which is the ratio of the two MSEs.

Before we do that, you may have noticed that we solved for
\(SS_\text{Error (left-over)}\), rather than directly computing it from
the data. In this chapter we are not going to show you the steps for
doing this. We are not trying to hide anything from, instead it turns
out these steps are related to another important idea in ANOVA. We
discuss this idea, which is called an \textbf{interaction} in the next
chapter, when we discuss \textbf{factorial} designs (designs with more
than one independent variable).

\subsection{Compute the MSEs}\label{compute-the-mses}

Calculating the MSEs (mean squared error) that we need for the
\(F\)-value involves the same general steps as last time. We divide each
SS by the degrees of freedom for the SS.

The degrees of freedom for \(SS_\text{Effect}\) are the same as before,
the number of conditions - 1. We have three conditions, so the df is 2.
Now we can compute the \(MSE_\text{Effect}\).

\(MSE_\text{Effect} = \frac{SS_\text{Effect}}{df} = \frac{72}{2} = 36\)

The degrees of freedom for \(SS_\text{Error (left-over)}\) are different
than before, they are the (number of subjects - 1) multiplied by the
(number of conditions -1). We have 3 subjects and three conditions, so
\((3-1) * (3-1) = 2*2 =4\). You might be wondering why we are
multiplying these numbers. Hold that thought for now and wait until the
next chapter. Regardless, now we can compute the
\(MSE_\text{Error (left-over)}\).

\(MSE_\text{Error (left-over)} = \frac{SS_\text{Error (left-over)}}{df} = \frac{177.33}{4}= 44.33\)

\subsection{Compute F}\label{compute-f}

We just found the two MSEs that we need to compute \(F\). We went
through all of this to compute \(F\) for our data, so let's do it:

\(F = \frac{MSE_\text{Effect}}{MSE_\text{Error (left-over)}} = \frac{36}{44.33}= 0.812\)

And, there we have it!

\subsection{p-value}\label{p-value}

We already conducted the repeated-measures ANOVA using R and reported
the ANOVA. Here it is again. The table shows the \(p\)-value associated
with our \(F\)-value.

\begin{longtable}[]{@{}lrrrrr@{}}
\toprule\noalign{}
& Df & Sum Sq & Mean Sq & F value & Pr(\textgreater F) \\
\midrule\noalign{}
\endhead
\bottomrule\noalign{}
\endlastfoot
Residuals & 2 & 52.66667 & 26.33333 & NA & NA \\
conditions & 2 & 72.00000 & 36.00000 & 0.8120301 & 0.505848 \\
Residuals & 4 & 177.33333 & 44.33333 & NA & NA \\
\end{longtable}

We might write up the results of our experiment and say that the main
effect condition was not significant, F(2,4) = 0.812, MSE = 44.33, p =
0.505.

What does this statement mean? Remember, that the \(p\)-value represents
the probability of getting the \(F\) value we observed or larger under
the null (assuming that the samples come from the same distribution, the
assumption of no differences). So, we know that an \(F\)-value of 0.812
or larger happens fairly often by chance (when there are no real
differences), in fact it happens 50.5\% of the time. As a result, we do
not reject the idea that any differences in the means we have observed
could have been produced by chance.

\section{Things worth knowing}\label{things-worth-knowing}

Repeated Measures ANOVAs have some special properties that are worth
knowing about. The main special property is that the error term used to
for the \(F\)-value (the MSE in the denominator) will always be smaller
than the error term used for the \(F\)-value the ANOVA for a
between-subjects design. We discussed this earlier. It is smaller,
because we subtract out the error associated with the subject means.

This can have the consequence of generally making \(F\)-values in
repeated measures designs larger than \(F\)-values in between-subjects
designs. When the number in the bottom of the \(F\) formula is generally
smaller, it will generally make the resulting ratio a larger number.
That's what happens when you make the number in the bottom smaller.

Because big \(F\) values usually let us reject the idea that differences
in our means are due to chance, the repeated-measures ANOVA becomes a
more sensitive test of the differences (its \(F\)-values are usually
larger).

At the same time, there is a trade-off here. The repeated measures ANOVA
uses different degrees of freedom for the error term, and these are
typically a smaller number of degrees of freedom. So, the
\(F\)-distributions for the repeated measures and between-subjects
designs are actually different \(F\)-distributions, because they have
different degrees of freedom.

\subsection{Repeated vs between-subjects
ANOVA}\label{repeated-vs-between-subjects-anova}

Let's do a couple simulations to see some the differences between the
ANOVA for a repeated measures design, and the ANOVA for a
between-subjects design.

We will do the following.

\begin{enumerate}
\def\labelenumi{\arabic{enumi}.}
\tightlist
\item
  Simulate a design with three conditions, A, B, and C
\item
  sample 10 scores into each condition from the same normal distribution
  (mean = 100, SD = 10)
\item
  We will include a subject factor for the repeated-measures version.
  Here there are 10 subjects, each contributing three scores, one each
  condition
\item
  For the between-subjects design there are 30 different subjects, each
  contributing one score in the condition they were assigned to (really
  the group).
\end{enumerate}

We run 1000 simulated experiments for each design. We calculate the
\(F\) for each experiment, for both the between and repeated measures
designs. Figure~\ref{fig-9critcompare} has the sampling distributions of
\(F\) for both designs.

\begin{figure}

\centering{

\includegraphics[width=1\linewidth,height=\textheight,keepaspectratio]{09-RMANOVA_files/figure-pdf/fig-9critcompare-1.pdf}

}

\caption{\label{fig-9critcompare}Comparing critical F values for a
between and repeated measures design}

\end{figure}%

These two \(F\) sampling distributions look pretty similar. However,
they are subtly different. The between \(F\) distribution has degrees of
freedom 2, and 27, for the numerator and denominator. There are 3
conditions, so \$df\$1 = 3-1 = 2. There are 30 subjects, so \$df\$2 =
30-3 =27. The critical value, assuming an alpha of 0.05 is 3.35. This
means \(F\) is 3.35 or larger 5\% of the time under the null.

The repeated-measures \(F\) distribution has degrees of freedom 2, and
18, for the numerator and denominator. There are 3 conditions, so
\$df\$1 = 3-1 = 2. There are 10 subjects, so \$df\$2 = (10-1)\emph{(3-1)
= 9}2 = 18. The critical value, assuming an alpha of 0.05 is 3.55. This
means \(F\) is 3.55 or larger 5\% of the time under the null.

The critical value for the repeated measures version is slightly higher.
This is because when \$df\$2 (the denominator) is smaller, the
\(F\)-distribution spreads out to the right a little bit. When it is
skewed like this, we get some bigger \(F\)s a greater proportion of the
time.

So, in order to detect a real difference, you need an \(F\) of 3.35 or
greater in a between-subjects design, or an \(F\) of 3.55 or greater for
a repeated-measures design. The catch here is that when there is a real
difference between the means, you will detect it more often with the
repeated-measures design, even though you need a larger \(F\) (to pass
the higher critical \(F\)-value for the repeated measures design).

\subsection{repeated measures designs are more
sensitive}\label{repeated-measures-designs-are-more-sensitive}

To illustrate why repeated-measures designs are more sensitive, we will
conduct another set of simulations.

We will do something slightly different this time. We will make sure
that the scores for condition A, are always a little bit higher than the
other scores. In other words, we will program in a real true difference.
Specifically, the scores for condition will be sampled from a normal
distribution with mean = 105, and SD = 10. This mean is 5 larger than
the means for the other two conditions (still set to 100).

With a real difference in the means, we should now reject the hypothesis
of no differences more often. We should find \(F\) values larger than
the critical value more often. And, we should find \(p\)-values for each
experiment that are smaller than .05 more often, those should occur more
than 5\% of the time.

To look at this we conduct 1000 experiments for each design, we conduct
the ANOVA, then we save the \(p\)-value we obtained for each experiment.
This is like asking how many times will we find a \(p\)-value less than
0.05, when there is a real difference (in this case an average of 5)
between some of the means. Figure~\ref{fig-9fpvaluedist} contains
histograms of the \(p\)-values:

\begin{figure}

\centering{

\includegraphics[width=1\linewidth,height=\textheight,keepaspectratio]{09-RMANOVA_files/figure-pdf/fig-9fpvaluedist-1.pdf}

}

\caption{\label{fig-9fpvaluedist}\(p\)-value distributions for a between
and within-subjects ANOVA}

\end{figure}%

Here we have two distributions of observed p-values for the simulations.
The red line shows the location of 0.05. Overall, we can see that for
both designs, we got a full range of \(p\)-values from 0 to 1. This
means that many times we would not have rejected the hypothesis of no
differences (even though we know there is a small difference). We would
have rejected the null every time the \(p\)-value was less than 0.05.

For the between subject design, there were 590 experiments with a \(p\)
less than 0.05, or 0.59 of experiments were ``significant'', with
alpha=.05.

For the within subject design, there were 562 experiments with a \(p\)
less than 0.05, or 0.562 of experiments were ``significant'', with
alpha=.05.

OK, well, you still might not be impressed. In this case, the
between-subjects design detected the true effect slightly more often
than the repeated measures design. Both them were right around 55\% of
the time. Based on this, we could say the two designs are pretty
comparable in their sensitivity, or ability to detect a true difference
when there is one.

However, remember that the between-subjects design uses 30 subjects, and
the repeated measures design only uses 10. We had to make a big
investment to get our 30 subjects. And, we're kind of unfairly comparing
the between design (which is more sensitive because it has more
subjects) with the repeated measures design that has fewer subjects.

What do you think would happen if we ran 30 subjects in the repeated
measures design? Let's find out. Figure~\ref{fig-9pdistwithin} re-plots
the above, but this time only for the repeated measures design. We
increase \(N\) from 10 to 30.

\begin{figure}

\centering{

\includegraphics[width=1\linewidth,height=\textheight,keepaspectratio]{09-RMANOVA_files/figure-pdf/fig-9pdistwithin-1.pdf}

}

\caption{\label{fig-9pdistwithin}\(p\)-value distribution for
within-subjects design with \(n = 30\)}

\end{figure}%

Wowsers! Look at that. When we ran 30 subjects in the repeated measures
design almost all of the \(p\)-values were less than .05. There were 983
experiments with a \(p\) less than 0.05, or 0.983 of experiments were
``significant'', with alpha=.05. That's huge! If we ran the repeated
measures design, we would almost always detect the true difference when
it is there. This is why the repeated measures design can be more
sensitive than the between-subjects design.

\section{Real Data}\label{real-data}

Let's look at some real data from a published experiment that uses a
repeated measures design. This is the same example that you will be
using in the lab for repeated measures ANOVA. The data happen to be
taken from a recent study conducted by Lawrence Behmer and myself, at
Brooklyn College (Behmer and Crump 2017).

We were interested in how people perform sequences of actions. One
question is whether people learn individual parts of actions, or the
whole larger pattern of a sequence of actions. We looked at these issues
in a computer keyboard typing task. One of our questions was whether we
would replicate some well known findings about how people type words and
letters.

From prior work we knew that people type words way faster than than
random letters, but if you made the random letters a little bit more
English-like, then people type those letter strings a little bit faster,
but not as slow as random string.

In the study, 38 participants sat in front of a computer and typed 5
letter strings one at a time. Sometimes the 5 letter made a word (Normal
condition, TRUCK), sometimes they were completely random (Random
Condition, JWYFG), and sometimes they followed patterns like you find in
English (Bigram Condition, QUEND), but were not actual words. So, the
independent variable for the typing material had three levels. We
measured every single keystroke that participants made. This gave us a
few different dependent measures. Let's take a look a the reaction
times. This is how long it took for participants to start typing the
first letter in the string.

\begin{figure}

\centering{

\includegraphics[width=1\linewidth,height=\textheight,keepaspectratio]{09-RMANOVA_files/figure-pdf/fig-9behmcrump-1.pdf}

}

\caption{\label{fig-9behmcrump}Results from Behmer \& Crump (2017)}

\end{figure}%

OK, I made a figure showing the mean reaction times for the different
typing material conditions. You will notice that there are two sets of
lines. That's because there was another manipulation I didn't tell you
about. In one block of trials participants got to look at the keyboard
while they typed, but in the other condition we covered up the keyboard
so people had to type without looking. Finally, the error bars are
standard error of the means.

\begin{tcolorbox}[enhanced jigsaw, title=\textcolor{quarto-callout-note-color}{\faInfo}\hspace{0.5em}{Note}, colframe=quarto-callout-note-color-frame, colbacktitle=quarto-callout-note-color!10!white, bottomtitle=1mm, leftrule=.75mm, rightrule=.15mm, titlerule=0mm, arc=.35mm, colback=white, opacitybacktitle=0.6, toprule=.15mm, toptitle=1mm, bottomrule=.15mm, coltitle=black, breakable, left=2mm, opacityback=0]

Note, the use of error bars for repeated-measures designs is not very
straightforward. In fact the standard error of the means that we have
added here are not very meaningful for judging whether the differences
between the means are likely not due to chance. They would be if this
was a between-subjects design. We will update this textbook with a
longer discussion of this issue, for now we will just live with these
error bars.

\end{tcolorbox}

For the purpose of this example, we will say, it sure looks like the
previous finding replicated. For example, people started typing Normal
words faster than Bigram strings (English-like), and they started typing
random letters the most slowly of all. Just like prior research had
found.

Let's focus only on the block of trials where participants were allowed
to look at the keyboard while they typed, that's the red line, for the
``visible keyboard'' block. We can see the means look different. Let's
next ask, what is the likelihood that chance (random sampling error)
could have produced these mean differences. To do that we run a
repeated-measures ANOVA in R. Here is the ANOVA table.

\begin{longtable}[]{@{}lrrrrr@{}}
\toprule\noalign{}
& Df & Sum Sq & Mean Sq & F value & Pr(\textgreater F) \\
\midrule\noalign{}
\endhead
\bottomrule\noalign{}
\endlastfoot
Residuals & 37 & 2452611.9 & 66286.808 & NA & NA \\
Stimulus & 2 & 1424914.0 & 712457.010 & 235.7342 & 0 \\
Residuals1 & 74 & 223649.4 & 3022.289 & NA & NA \\
\end{longtable}

Alright, we might report the results like this. There was a significant
main effect of Stimulus type, F(2, 74) = 235.73, MSE = 3022.289, p
\textless{} 0.001.

Notice a couple things. First, this is a huge \(F\)-value. It's 253!
Notice also that the p-value is listed as 0. That doesn't mean there is
zero chance of getting an F-value this big under the null. This is a
rounding error. The true p-value is 0.00000000000000\ldots{} The zeros
keep going for a while. This means there is only a vanishingly small
probability that these differences could have been produced by sampling
error. So, we reject the idea that the differences between our means
could be explained by chance. Instead, we are pretty confident, based on
this evidence and and previous work showing the same thing, that our
experimental manipulation caused the difference. In other words, people
really do type normal words faster than random letters, and they type
English-like strings somewhere in the middle in terms of speed.

\section{Summary}\label{summary-3}

In this chapter you were introduced to the repeated-measures ANOVA. This
analysis is appropriate for within-subjects or repeated measures
designs. The main difference between the independent factor ANOVA and
the repeated measures ANOVA, is the ability to partial out variance due
to the individual subject means. This can often result in the
repeated-measures ANOVA being more sensitive to true effects than the
between-subjects ANOVA.

\bookmarksetup{startatroot}

\chapter{Factorial ANOVA}\label{factorial-anova}

In environmental science, things are rarely simple. Factors like
climate, soil type, and water availability all work together, affecting
ecosystems in ways that aren't always straightforward. That's where
factorial designs come in handy. They help us understand how different
factors interact with each other.

So far, we've mostly looked at situations where we change one thing at a
time and see what happens. But in real-world ecosystems or when studying
climate interactions, there are usually many things changing at once.
Factorial designs allow us to study these complex situations more
effectively. They help us see not just what each factor does on its own,
but also how they influence each other when they're all at play
together.

Imagine an investigation into the growth rates of a particular plant
species. The dependent variable here would be the growth rate, while the
independent variables could range from soil pH to sunlight exposure. In
environmental factorial designs, each independent variable, with their
respective levels, weaves into a tapestry of possible outcomes, allowing
us to observe not just isolated effects, but the symphony of
interactions.

Let's explore some factorial design examples tailored to our
environmental inquiries:

\begin{enumerate}
\def\labelenumi{\arabic{enumi}.}
\tightlist
\item
  1 IV (two levels)
\end{enumerate}

A t-test would suffice here, as we are dealing with just two levels of
our independent variable.

\begin{enumerate}
\def\labelenumi{\alph{enumi}.}
\item
  Soil pH (Acidic vs.~Neutral): How does soil pH affect plant growth? We
  have one IV (soil pH), with two levels (Acidic vs.~Neutral).
\item
  Sunlight Exposure (Full Sun vs.~Partial Shade): Does the amount of
  sunlight influence plant growth rates? Here, there's one IV (sunlight
  exposure), with two levels (Full Sun vs.~Partial Shade).
\end{enumerate}

\begin{enumerate}
\def\labelenumi{\arabic{enumi}.}
\setcounter{enumi}{1}
\tightlist
\item
  1 IV (three levels):
\end{enumerate}

An ANOVA is appropriate here, given the variable has more than two
levels.

\begin{enumerate}
\def\labelenumi{\alph{enumi}.}
\item
  Water Availability (Low, Medium, High): How does varying water
  availability impact plant growth? Our single IV (water availability)
  has three levels (Low, Medium, High).
\item
  Fertilizer Type (No fertilizer, Organic, Synthetic): What's the effect
  of different fertilizer types on plant growth? One IV (fertilizer
  type) has three levels.
\end{enumerate}

\begin{enumerate}
\def\labelenumi{\arabic{enumi}.}
\setcounter{enumi}{2}
\tightlist
\item
  \textbf{2 IVs: IV1 (two levels), IV2 (two levels)}
\end{enumerate}

We would employ a factorial ANOVA for this design, which we haven't
discussed yet.

\begin{enumerate}
\def\labelenumi{\alph{enumi}.}
\tightlist
\item
  IV1 (Soil pH: Acidic vs.~Neutral); IV2 (Water Availability: Low
  vs.~High): How do soil pH and water availability together affect plant
  growth? Here, we have two IVs---soil pH and water availability. Each
  IV has two levels, leading to a \textbf{2x2} Factorial Design. This
  type of design is fully crossed; each level of one IV is paired with
  each level of the other IV to observe the combined effect on growth
  rates.
\end{enumerate}

OK, let's stop here for the moment. The first two designs both had one
IV. The third design shows an example of a design with 2 IVs (soil pH
and water availability), each with two levels. This is called a
\textbf{2x2 Factorial Design}. It is called a \textbf{factorial} design,
because the levels of each independent variable are fully crossed. This
means that first each level of one IV, the levels of the other IV are
also manipulated.

\section{Factorial basics}\label{factorial-basics}

\subsection{2x2 Designs}\label{x2-designs}

We've just started talking about a \textbf{2x2 Factorial design}. We
said this means the IVs are crossed. To illustrate this, take a look at
Figure~\ref{fig-10designs}. We show an abstract version and a concrete
version using soil pH and water availability as the two IVs, each with
two levels in the design:

\begin{figure}

\centering{

\includegraphics[width=0.75\linewidth,height=\textheight,keepaspectratio]{imgs/figures/2x2Design.png}

}

\caption{\label{fig-10designs}Structure of 2x2 factorial designs}

\end{figure}%

\begin{itemize}
\tightlist
\item
  IV1: Soil Type (Clay vs.~Loam)
\item
  IV2: Drought Condition (Yes vs.~No)
\end{itemize}

Each combination of soil type and drought condition creates a unique
environment for the plants, leading to four distinct scenarios to
measure growth rates. So, we have 2 IVs, each with 2 levels, for a total
of 4 conditions. This is why we call it a 2x2 design. 2x2 = 4. The
notation tells us how to calculate the total number of conditions.

\subsection{Factorial Notation}\label{factorial-notation}

Anytime \textbf{all of the levels} of each IV in a design are fully
crossed, so that they all occur for each level of every other IV, we can
say the design is a \textbf{fully factorial} design.

Our notation system succinctly encapsulates the structure of factorial
designs. Each IV gets a number representing its levels. Here are a few
examples:

\begin{itemize}
\item
  2x2: Two IVs, each with two levels, yielding four unique conditions.
\item
  2x3: Two IVs, with the first having two levels and the second three,
  resulting in six conditions.
\item
  3x2: Similar to the 2x3, but the first IV has three levels, and the
  second has two, also giving us six conditions.
\item
  4x4: Two IVs, each with four levels, resulting in sixteen conditions.
\end{itemize}

Expanding on our designs, a 2x3 factorial design could investigate the
impact of two soil types across three different levels of water
availability, equating to six unique conditions to analyze plant growth.

\section{Purpose of Factorial
Designs}\label{purpose-of-factorial-designs}

Factorial designs let us ask nuanced questions about ecological
phenomena. By manipulating multiple variables simultaneously, we gain
insights into how different environmental factors might interact to
influence a specific outcome, like plant growth or species distribution.

\subsection{Factorials manipulate an effect of
interest}\label{factorials-manipulate-an-effect-of-interest}

Factorial designs enable researchers to sift through multiple layers of
influence to grasp the broader picture. Complicated? Certainly, but it's
complexity that mirrors the real world where multiple factors often come
into play simultaneously.

Let's make sense of this by contemplating a multifaceted environmental
study. Imagine we're environmental scientists, eager to measure the
effects of various pollutants on aquatic life. Here's what we could do:

\begin{enumerate}
\def\labelenumi{\arabic{enumi}.}
\item
  \textbf{Choose a bioindicator}: Our subjects are the varied aquatic
  macroinvertebrates, whose species diversity is a critical measure of
  water health---our dependent variable.
\item
  \textbf{Identify impactful variables}: We've identified the usual
  suspect -- excess Nutrient-loaded agricultural runoff (Pollutant A),
  notorious for depleting oxygen and threatening our aquatic buddies.
  We'll add this to our macroinvertebrates' environment and observe the
  ripple effects on their diversity, using a pristine control (No
  Pollutant) for comparison.
\item
  \textbf{Measure the impact}: We'll examine the richness and variety of
  macroinvertebrate species when exposed to each type of pollutant.
\item
  \textbf{Detect the variation}: Variations in macroinvertebrate
  diversity will illuminate the `Pollution effect'. The impact of
  pollutaiton on macroinvertebrate diversity.
\end{enumerate}

The proof is in the visuals. We aim to contrast macroinvertebrate
diversity in the shadow of Pollutant A with a pollutant-free scenario.
Figure~\ref{fig-10biodiversityeffect} shows how the data might look.

\begin{figure}

\centering{

\includegraphics[width=1\linewidth,height=\textheight,keepaspectratio]{10-FactorialANOVA_files/figure-pdf/fig-10biodiversityeffect-1.pdf}

}

\caption{\label{fig-10biodiversityeffect}Example data illustrating the
influence of different pollution types on aquatic macroinvertebrate
species diversity}

\end{figure}%

Behold! The macroinvertebrate lineup takes a hit from both pollutants,
but they really don't like the industrial waste after-party. In general,
it is very common to use the word \textbf{effect} to refer to the
differences caused by the manipulation. This is what we could call the
``Pollution effect''.

\subsection{Manipulating the Pollution
effect}\label{manipulating-the-pollution-effect}

This is where factorial designs come in to play. We've already
pinpointed a `Pollution effect'---the change in macroinvertebrate
biodiversity when pollutants enter their habitat. This effect serves as
our guide, leading us to ask: Where does pollution strike the hardest?
What conditions exacerbate its harmful impact?

One possible lever for controlling this Pollution effect could be the
timing of fertilizer application, reducing runoff by strategic
scheduling post-rainfall. Cover cropping might also act as a shield,
protecting our waters from nutrient excess. But now, we introduce a new
player: flow rate. Will high flow rate reduce impacts of pollution,
since the pollution will scoot on out of there? Or, paradoxically, could
it exacerbate the issue, hastening the spread of pollutants that might
otherwise degrade if left undisturbed?

Our question evolves: Does flow rate influence the Pollution Effect? We
hypothesize that a higher flow rate might reduce the observable
Pollution Effect compared to a more stagnant, low flow environment. If
our theory holds water, the results could flow out like the data in
Figure~\ref{fig-10flow}.

\begin{figure}

\centering{

\includegraphics[width=1\linewidth,height=\textheight,keepaspectratio]{10-FactorialANOVA_files/figure-pdf/fig-10flow-1.pdf}

}

\caption{\label{fig-10flow}Example data showing how the Pollution effect
could be modulated by a Flow manipulation. Pollution plotted on the
x-axis, makes it more difficult to compare the changes in the pollution
effect between flow rate conditions}

\end{figure}%

In this graph, we maintain consistency by placing the Pollution
conditions on the x-axis.The bars represent the average
macroinvertebrate diversity for each combination of flow and pollution
conditions, with colors distinguishing the flow rates. But, it's not as
helpful as it could be. We can try to interpret this graph, but
\textbf{?@fig-10flowB} plots the same data in a way that makes it easier
to see what we are talking about.

\begin{figure}

\centering{

\includegraphics[width=1\linewidth,height=\textheight,keepaspectratio]{10-FactorialANOVA_files/figure-pdf/fig-10FlowB-1.pdf}

}

\caption{\label{fig-10FlowB}Example data showing how the pollution
effect could be modulated by a flow rate manipulation. Flow condition
plotted on the x-axis, makes it easier to compare the changes in the
pollution effect between Flow conditions}

\end{figure}%

Here, the x-axis represents the Flow Rate, with the color of the bars
indicating the Pollution condition. This graph layout simplifies the
comparison of the Pollution effect within each Flow Rate condition,
making it easier to discern the interaction between these two factors.

\textbf{Low-Flow condition}: In the low flow condition, our
macroinvertebrates were observed in both polluted and unpolluted waters.
This mirrors the baseline scenario of our study. Consistently with our
predictions, the graph would likely show a significant difference: a
higher species count in unpolluted waters compared to polluted ones.
Thus, we'd observe a Pollution effect, with a specific difference in
species count illustrating the impact of pollution under low flow
conditions.

\textbf{Flow condition}: In the high flow condition, the
macroinvertebrates also faced both polluted and unpolluted environments.
However, the dynamic nature of high flow was hypothesized to influence
their response to pollution. The expectation was that the swift current
might diminish the observable Pollution effect by dispersing pollutants
more effectively. If the graph supports our hypothesis, we'd see a
smaller difference in species count between the polluted and unpolluted
conditions under high flow, indicating a lessened Pollution effect.

Should our research validate these predictions, we could infer that flow
rate does indeed modulate the Pollution effect. In a low flow
environment, the Pollution effect might be pronounced, as observed by
the larger difference in species counts. Conversely, in a high flow
scenario, the effect diminishes, with the difference in biodiversity
less stark. Therefore, the manipulation of flow rate could potentially
alter the Pollution effect by a quantifiable margin, underscored by the
variation in species counts between the two flow conditions.v

This is our description of why factorial designs are so useful. They
allow researchers to find out what kinds of manipulations can cause
changes in the effects they measure. In our environmental study, for
instance, we've measured the Pollution effect on macroinvertebrate
biodiversity. Then, we introduced the variable of flow rate to see if
and how it might alter this effect. If our goal is to untangle the web
of ecological dynamics, we'd need to delve into the mechanisms by which
flow rate could influence pollutant dispersion and, subsequently,
biodiversity. We have the initial evidence suggesting that flow rate can
indeed sway the Pollution effect. The next step is to craft hypotheses
on the nature of this influence---perhaps proposing that faster flows
dilute pollutants more effectively, thus mitigating their negative
impact on biodiversity. These hypotheses then become the basis for
further experiments, designed to test their validity and expand our
understanding of these complex environmental interactions.

\section{Graphing the means}\label{graphing-the-means}

In our example above we showed you two bar graphs of the very same means
for our 2x2 design. Even though the graphs plot identical means, they
look different, so they are more or less easy to interpret by looking at
them. Results from 2x2 designs are also often plotted with line graphs.
Those look different too. There are four different graphs in
Figure~\ref{fig-10visualdiffs}, using bars and lines to plot the very
same means from before. We are showing you this so that you realize
\textbf{how you graph your data matters} because it makes it more or
less easy for people to understand the results. Also, how the data is
plotted matters for what you need to look at to interpret the results.

\begin{figure}

\centering{

\includegraphics[width=1\linewidth,height=\textheight,keepaspectratio]{10-FactorialANOVA_files/figure-pdf/fig-10visualdiffs-1.pdf}

}

\caption{\label{fig-10visualdiffs}same example means plotted using bar
graphs or line graphs, and with Pollution or Flow on the x-axis}

\end{figure}%

\section{Knowing what you want to find
out}\label{knowing-what-you-want-to-find-out}

When you conduct a design with more than one IV, you get more means to
look at. As a result, there are more kinds of questions that you can ask
of the data. Sometimes it turns out that the questions that you can ask,
are not the ones that you want to ask, or have an interest in asking.
Because you ran the design with more than one IV, you have the
opportunity to ask these kinds of extra questions.

What kinds of extra questions? Let's keep going with our Pollution
effect experiment. We have the first IV where we manipulated Pollution.
So, we could find the overall means in spot-the difference for the
Pollution vs.~no-Pollution conditions (that's two means). The second IV
was Flow. We could find the overall means in spot-the-difference
performance for the Flow vs.~no-Flow conditions (that's two more means).
We could do what we already did, and look at the means for each
combination, that is the mean for Pollution/Flow, Pollution/no-Flow,
no-Pollution/Flow, and no-Pollution/no-Flow (that's four more means, if
you're counting).

There's even more. We could look at the mean Pollution effect (the
difference between pollution and no pollution) for the low flow
condition, and the mean Pollution effect for the high flow condition
(that's two more).

Figure~\ref{fig-10alleffects} shows multiple ways of looking at the
means across four panels.

\begin{figure}

\centering{

\includegraphics[width=1\linewidth,height=\textheight,keepaspectratio]{10-FactorialANOVA_files/figure-pdf/fig-10alleffects-1.pdf}

}

\caption{\label{fig-10alleffects}Each panel shows the mean for different
effects in the design}

\end{figure}%

The top left panel of our graph offers a snapshot of macroinvertebrate
populations under different environmental scenarios. To assess the
impact of Pollution, we look at the differences in species counts
between the unpolluted (aqua bar) and polluted (red bar) conditions
within both High Flow and Low Flow contexts. This visual comparison
sheds light on how Pollution affects species diversity across varying
flow rates.

The top right panel, however, does not delve into the effects of Flow.
Rather, it showcases the general impact of Pollution on species
diversity, known as the main effect of Pollution. This is a direct
comparison of the number of species found in unpolluted versus polluted
waters, without the flow rate being a factor.

The bottom left panel similarly isolates another variable, focusing
solely on the main effect of Flow. It contrasts species diversity in
High Flow versus Low Flow conditions, without the influence of Pollution
being considered.

It's in the bottom right panel that we explore the interaction between
Flow and Pollution. Here, the y-axis quantifies the Pollution effect,
reflecting the change in species diversity from unpolluted to polluted
conditions. In the Low-Flow scenario, there's a substantial drop (a
difference of 20, from 35 to 15) when Pollution is introduced. In
contrast, under High Flow, the introduction of Pollution results in a
smaller decrease (a difference of -5, from 30 to 25). The disparity
between these two outcomes (20 in Low Flow versus -5 in High Flow)
indicates that Flow significantly moderates the effect of Pollution.
This differential effect, where the influence of one factor (Pollution)
varies according to the level of another (Flow), is what we define as an
\textbf{interaction}.

\begin{tcolorbox}[enhanced jigsaw, title=\textcolor{quarto-callout-tip-color}{\faLightbulb}\hspace{0.5em}{Pro tip}, colframe=quarto-callout-tip-color-frame, colbacktitle=quarto-callout-tip-color!10!white, bottomtitle=1mm, leftrule=.75mm, rightrule=.15mm, titlerule=0mm, arc=.35mm, colback=white, opacitybacktitle=0.6, toprule=.15mm, toptitle=1mm, bottomrule=.15mm, coltitle=black, breakable, left=2mm, opacityback=0]

Before you start your environmental detective work, know what you're
looking for. What's your question? What clues (means) are relevant? Keep
your eyes on the prize, and don't get lost in the sea of data!

\end{tcolorbox}

\section{Simplified Analysis of 2x2 Repeated Measures
Design}\label{simplified-analysis-of-2x2-repeated-measures-design}

In discussions of factorial designs, the concept of Factorial ANOVAs
often comes up. These are complex statistical tests that we'll explore
shortly. However, before we delve into Factorial ANOVAs, let's look at
how to analyze a 2x2 repeated measures design using paired-samples
t-tests. This approach is less common but yields results comparable to
those from an ANOVA.

Understanding ANOVA can be challenging, and it gets even trickier with
factorial designs. To ease into this complexity, we'll start with
t-tests to demonstrate the principles behind Factorial ANOVA. Conducting
t-tests requires precise comparisons, which will help you grasp what
you're seeking to understand from factorial designs. Once you're clear
on your research questions, you can apply ANOVA to uncover the answers
and know exactly what to look for in the results. This step-by-step
approach makes learning a more enjoyable journey!

First, let's define two key terms: \textbf{main effects} and
\textbf{interactions}. In any factorial design, you have the chance to
analyze these elements. The number of \textbf{main effects} and
\textbf{interactions} you can investigate depends on the number of
independent variables (IVs) in your design.

\subsection{Main Effects}\label{main-effects}

A main effect represents the average difference associated with a single
independent variable (IV). There's one main effect for each IV. In a 2x2
design, which includes two IVs, there are two main effects. For
instance, we might have one main effect for Pollution and another for
Flow. A significant main effect suggests that the differences observed
are not likely due to random chance.

In a 2x2x2 design, you'd evaluate three main effects, corresponding to
each IV. A 3x3x3 design, despite having more levels, still involves
three IVs, so you'd again have three main effects.

\subsection{Interaction}\label{interaction}

The concept of interaction can be perplexing in factorial designs.
Simply put, an interaction occurs when the effect of one IV on the
outcome is influenced by another IV. For example, we observed that the
presence of Flow affected the magnitude of the Pollution effect. The
Pollution effect was more pronounced without Flow and less so with Flow,
indicating an interaction.

Another way to think about interactions is to consider them as the
difference between differences. If the Pollution effect (the difference
in outcomes between polluted and unpolluted conditions) changes when we
introduce different Flow conditions, we're observing an interaction.
This concept can seem complex, but don't worry---we'll go through more
examples to clarify it.

The number of possible interactions in a design is tied to the number of
IVs. A 2x2 design has one interaction: the combined effect of the two
IVs. This single interaction examines whether the impact of one IV
varies across the levels of the other IV. In designs with more than two
IVs, the potential for interactions increases. For example, a design
with three IVs (A, B, and C) would have three 2-way interactions (AB,
AC, and BC) and one 3-way interaction (ABC).

\subsection{Looking at the data}\label{looking-at-the-data-1}

Understanding our analysis begins with a clear view of the data.
Consider our hypothetical study on attention, with five experimental
tanks that each experience all conditions, making this a fully
repeated-measures design. The dataset might look something like this:

\begin{table}
\centering
\begin{tabular}{r|r|r|r|r}
\hline
\multicolumn{1}{c|}{ } & \multicolumn{1}{c|}{No Pollution} & \multicolumn{1}{c|}{Pollution} & \multicolumn{1}{c|}{No Pollution} & \multicolumn{1}{c}{Pollution} \\
\cline{2-2} \cline{3-3} \cline{4-4} \cline{5-5}
\multicolumn{1}{c|}{ } & \multicolumn{2}{c|}{Low Flow} & \multicolumn{2}{c}{High Flow} \\
\cline{2-3} \cline{4-5}
subject & A & B & C & D\\
\hline
1 & 10 & 5 & 12 & 9\\
\hline
2 & 8 & 4 & 13 & 8\\
\hline
3 & 11 & 3 & 14 & 10\\
\hline
4 & 9 & 4 & 11 & 11\\
\hline
5 & 10 & 2 & 13 & 12\\
\hline
\end{tabular}
\end{table}

\subsection{Main effect of Pollution}\label{main-effect-of-pollution}

To assess the main effect of Pollution, we compare the mean scores
across the no-Pollution and Pollution conditions, regardless of the Flow
conditions.

\begin{table}
\centering
\begin{tabular}{l|l|l|l|l|l|l|l}
\hline
\multicolumn{1}{c|}{ } & \multicolumn{2}{c|}{No Pollution} & \multicolumn{2}{c|}{Pollution} & \multicolumn{3}{c}{ } \\
\cline{2-3} \cline{4-5}
subject & A & B & C & D & Mean\_No\_Pollution & Mean\_Pollution & Pollution\_Effect\\
\hline
1 & 10 & 5 & 12 & 9 & 11 & 7 & 4\\
\hline
2 & 8 & 4 & 13 & 8 & 10.5 & 6 & 4.5\\
\hline
3 & 11 & 3 & 14 & 10 & 12.5 & 6.5 & 6\\
\hline
4 & 9 & 4 & 11 & 11 & 10 & 7.5 & 2.5\\
\hline
5 & 10 & 2 & 13 & 12 & 11.5 & 7 & 4.5\\
\hline
Means &  &  &  &  & 11.1 & 6.8 & 4.3\\
\hline
\end{tabular}
\end{table}

The mean score in the no-Pollution condition is 11.1, while in the
Pollution condition, it is 6.8. The main effect of Pollution, therefore,
is 4.3, which represents the difference between these two means.

To determine if this main effect of Pollution is statistically
significant and not due to chance, we can perform a paired samples
t-test comparing the mean no-Pollution scores to the mean Pollution
scores for each subject (tank). Alternatively, a one-sample t-test on
the Pollution effect scores, testing against a mean difference of zero,
would yield the same conclusion.

For the paired samples t-test, we compare the mean scores of the
no-Pollution condition to the Pollution condition for each tank. We use
the Mean\_No\_Pollution and Mean\_Pollution columns from our data frame:

\begin{verbatim}
#> 
#>  Welch Two Sample t-test
#> 
#> data:  as.numeric(fake_data$Mean_No_Pollution[1:5]) and as.numeric(fake_data$Mean_Pollution[1:5], paired = TRUE)
#> t = 8.6, df = 6.502, p-value = 8.726e-05
#> alternative hypothesis: true difference in means is not equal to 0
#> 95 percent confidence interval:
#>  3.099079 5.500921
#> sample estimates:
#> mean of x mean of y 
#>      11.1       6.8
\end{verbatim}

For the one-sample t-test, we test whether the mean difference
(Pollution effect) is significantly different from zero:

\begin{verbatim}
#> 
#>  One Sample t-test
#> 
#> data:  as.numeric(fake_data$Pollution_Effect[1:5])
#> t = 7.6615, df = 4, p-value = 0.00156
#> alternative hypothesis: true mean is not equal to 0
#> 95 percent confidence interval:
#>  2.741724 5.858276
#> sample estimates:
#> mean of x 
#>       4.3
\end{verbatim}

If we were to write-up our results for the main effect of Pollution we
could say something like this:

The main effect of Pollution was significant, \(t\)(4) = 7.66, \(p\) =
0.001. The mean biodiversity was higher in the no-Pollution condition (M
= 11.1) than the Pollution condition (M = 6.8).

\subsection{Main effect of Flow}\label{main-effect-of-flow}

The main effect of Flow compares the overall means for all scores in the
no-Flow and Flow conditions, collapsing over the Flow conditions.

The yellow columns show the no-Flow scores for each subject (tank). The
blue columns show the Flow scores for each subject.

The overall means for for each subject, for the two Flow conditions are
shown to the right. For example, subject 1 had a 10 and 5 in the no-Flow
condition, so their mean is 7.5.

We are interested in the main effect of Flow. This is the difference
between the AB column (average of subject scores in the no-Flow
condition) and the CD column (average of the subject scores in the Flow
condition). These differences for each subject are shown in the last
green column. The overall means, averaging over subjects are in the
bottom green row.

\begin{table}
\centering
\begin{tabular}{l|>{}l|>{}l|>{}l|>{}l|l|l|>{}l}
\hline
\multicolumn{1}{c|}{ } & \multicolumn{4}{c|}{ All Conditions} & \multicolumn{3}{c}{ } \\
\cline{2-5}
\multicolumn{1}{c|}{ } & \multicolumn{2}{c|}{Low Flow} & \multicolumn{2}{c|}{High Flow} & \multicolumn{2}{c|}{ Flow Means } & \multicolumn{1}{c}{Flow Effect } \\
\cline{2-3} \cline{4-5} \cline{6-7} \cline{8-8}
\multicolumn{1}{c|}{ } & \multicolumn{1}{c|}{No Pollution} & \multicolumn{1}{c|}{Pollution} & \multicolumn{1}{c|}{No Pollution} & \multicolumn{1}{c|}{Pollution} & \multicolumn{1}{c|}{Low Flow} & \multicolumn{1}{c|}{High Flow} & \multicolumn{1}{c}{Difference} \\
\cline{2-2} \cline{3-3} \cline{4-4} \cline{5-5} \cline{6-6} \cline{7-7} \cline{8-8}
subject & A & B & C & D & AB & CD & CD.minus.AB\\
\hline
1 & \cellcolor{yellow}{10} & \cellcolor{yellow}{5} & \cellcolor{lightgray}{12} & \cellcolor{lightgray}{9} & 7.5 & 10.5 & \cellcolor{lightgray}{3}\\
\hline
2 & \cellcolor{yellow}{8} & \cellcolor{yellow}{4} & \cellcolor{lightgray}{13} & \cellcolor{lightgray}{8} & 6 & 10.5 & \cellcolor{lightgray}{4.5}\\
\hline
3 & \cellcolor{yellow}{11} & \cellcolor{yellow}{3} & \cellcolor{lightgray}{14} & \cellcolor{lightgray}{10} & 7 & 12 & \cellcolor{lightgray}{5}\\
\hline
4 & \cellcolor{yellow}{9} & \cellcolor{yellow}{4} & \cellcolor{lightgray}{11} & \cellcolor{lightgray}{11} & 6.5 & 11 & \cellcolor{lightgray}{4.5}\\
\hline
5 & \cellcolor{yellow}{10} & \cellcolor{yellow}{2} & \cellcolor{lightgray}{13} & \cellcolor{lightgray}{12} & 6 & 12.5 & \cellcolor{lightgray}{6.5}\\
\hline
\cellcolor{lightgray}{Means} & \cellcolor{lightgray}{} & \cellcolor{lightgray}{} & \cellcolor{lightgray}{} & \cellcolor{lightgray}{} & \cellcolor{lightgray}{6.6} & \cellcolor{lightgray}{11.3} & \cellcolor{lightgray}{4.7}\\
\hline
\end{tabular}
\end{table}

Just looking at the means, we can see there was a main effect of Flow.
The mean number of species was 11.3 in the Flow condition, and 6.6 in
the no-Flow condition. So, the size of the main effect of Flow was 4.7.

Is a difference of this size likely o unlikely due to chance? We could
conduct a paired-samples \(t\)-test on the AB vs.~CD means, or a
one-sample \(t\)-test on the difference scores. They both give the same
answer:

Here's the paired samples version:

\begin{verbatim}
#> 
#>  Paired t-test
#> 
#> data:  CD and AB
#> t = 8.3742, df = 4, p-value = 0.001112
#> alternative hypothesis: true mean difference is not equal to 0
#> 95 percent confidence interval:
#>  3.141724 6.258276
#> sample estimates:
#> mean difference 
#>             4.7
\end{verbatim}

Here's the one sample version:

\begin{verbatim}
#> 
#>  One Sample t-test
#> 
#> data:  CD - AB
#> t = 8.3742, df = 4, p-value = 0.001112
#> alternative hypothesis: true mean is not equal to 0
#> 95 percent confidence interval:
#>  3.141724 6.258276
#> sample estimates:
#> mean of x 
#>       4.7
\end{verbatim}

If we were to write-up our results for the main effect of Flow we could
say something like this:

The main effect of Flow was significant, t(4) = 8.37, p = 0.001. The
mean number of species was higher in the Flow condition (M = 11.3) than
the no-Flow condition (M = 6.6).

\subsection{Interaction between Pollution and
Flow}\label{interaction-between-pollution-and-flow}

Now we are ready to look at the interaction. Remember, the whole point
of this fake study was what? Can you remember?

Here's a reminder. We wanted to know if giving Flows versus not would
change the size of the Pollution effect.

Notice, neither the main effect of Pollution, or the main effect of
Flow, which we just went through the process of computing, answers this
question.

In order to answer the question we need to do two things. First, compute
Pollution effect for each subject when they were in the no-Flow
condition. Second, compute the Pollution effect for each subject when
they were in the Flow condition.

Then, we can compare the two Pollution effects and see if they are
different. The comparison between the two Pollution effects is what we
call the \textbf{interaction effect}. Remember, this is a difference
between two difference scores. We first get the difference scores for
the Pollution effects in the no-Flow and Flow conditions. Then we find
the difference scores between the two Pollution effects. This difference
of differences is the interaction effect (green column in the table)

\begin{table}
\centering
\begin{tabular}{l|>{}l|>{}l|>{}l|>{}l|l|l|>{}l}
\hline
\multicolumn{1}{c|}{ } & \multicolumn{4}{c|}{ All Conditions} & \multicolumn{3}{c}{ } \\
\cline{2-5}
\multicolumn{1}{c|}{ } & \multicolumn{2}{c|}{Low Flow} & \multicolumn{2}{c|}{High Flow} & \multicolumn{2}{c|}{ Pollution Effects } & \multicolumn{1}{c}{Interaction Effect } \\
\cline{2-3} \cline{4-5} \cline{6-7} \cline{8-8}
\multicolumn{1}{c|}{ } & \multicolumn{1}{c|}{No Pollution} & \multicolumn{1}{c|}{Pollution} & \multicolumn{1}{c|}{No Pollution} & \multicolumn{1}{c|}{Pollution} & \multicolumn{1}{c|}{Low Flow} & \multicolumn{1}{c|}{High Flow} & \multicolumn{1}{c}{Difference} \\
\cline{2-2} \cline{3-3} \cline{4-4} \cline{5-5} \cline{6-6} \cline{7-7} \cline{8-8}
subject & A & B & C & D & A-B & C-D & (A-B)-(C-D)\\
\hline
1 & \cellcolor{yellow}{10} & \cellcolor{lightgray}{5} & \cellcolor{yellow}{12} & \cellcolor{lightgray}{9} & 5 & 3 & \cellcolor{lightgray}{2}\\
\hline
2 & \cellcolor{yellow}{8} & \cellcolor{lightgray}{4} & \cellcolor{yellow}{13} & \cellcolor{lightgray}{8} & 4 & 5 & \cellcolor{lightgray}{-1}\\
\hline
3 & \cellcolor{yellow}{11} & \cellcolor{lightgray}{3} & \cellcolor{yellow}{14} & \cellcolor{lightgray}{10} & 8 & 4 & \cellcolor{lightgray}{4}\\
\hline
4 & \cellcolor{yellow}{9} & \cellcolor{lightgray}{4} & \cellcolor{yellow}{11} & \cellcolor{lightgray}{11} & 5 & 0 & \cellcolor{lightgray}{5}\\
\hline
5 & \cellcolor{yellow}{10} & \cellcolor{lightgray}{2} & \cellcolor{yellow}{13} & \cellcolor{lightgray}{12} & 8 & 1 & \cellcolor{lightgray}{7}\\
\hline
\cellcolor{lightgray}{Means} & \cellcolor{lightgray}{} & \cellcolor{lightgray}{} & \cellcolor{lightgray}{} & \cellcolor{lightgray}{} & \cellcolor{lightgray}{6} & \cellcolor{lightgray}{2.6} & \cellcolor{lightgray}{3.4}\\
\hline
\end{tabular}
\end{table}

The mean Pollution effects in the no-Flow (6) and Flow (2.6) conditions
were different. This difference is the interaction effect. The size of
the interaction effect was 3.4.

How can we test whether the interaction effect was likely or unlikely
due to chance? We could run another paired-sample \(t\)-test between the
two Pollution effect measures for each subject, or a one sample
\(t\)-test on the green column (representing the difference between the
differences). Both of these \(t\)-tests will give the same results:

Here's the paired samples version:

\begin{verbatim}
#> 
#>  Paired t-test
#> 
#> data:  A_B and C_D
#> t = 2.493, df = 4, p-value = 0.06727
#> alternative hypothesis: true mean difference is not equal to 0
#> 95 percent confidence interval:
#>  -0.3865663  7.1865663
#> sample estimates:
#> mean difference 
#>             3.4
\end{verbatim}

Here's the one sample version:

\begin{verbatim}
#> 
#>  One Sample t-test
#> 
#> data:  A_B - C_D
#> t = 2.493, df = 4, p-value = 0.06727
#> alternative hypothesis: true mean is not equal to 0
#> 95 percent confidence interval:
#>  -0.3865663  7.1865663
#> sample estimates:
#> mean of x 
#>       3.4
\end{verbatim}

Oh look, the interaction was not significant. At least, if we had set
our alpha criterion to 0.05, it would not have met that criteria. We
could write up the results like this. The two-way interaction between
between Pollution and Flow was not significant, \(t\)(4) = 2.493, \(p\)
= 0.067.

Often times when a result is ``not significant'' according to the alpha
criteria, the pattern among the means is not described further. One
reason for this practice is that the researcher is treating the means as
if they are not different (because there was an above alpha probability
that the observed differences were due to chance). If they are not
different, then there is no pattern to report.

There are differences in opinion among reasonable and expert
statisticians on what should or should not be reported. Let's say we
wanted to report the observed mean differences, we would write something
like this:

The two-way interaction between between Pollution and Flow was not
significant, t(4) = 2.493, p = 0.067. The mean Pollution effect in the
no-Flow condition was 6 and the mean Pollution effect in the Flow
condition was 2.6.

\subsection{Writing it all up}\label{writing-it-all-up}

We have completed an analysis of a 2x2 repeated measures design using
paired-samples \(t\)-tests. Here is what a full write-up of the results
could look like.

The main effect of Pollution was significant, \(t\)(4) = 7.66, \(p\) =
0.001. The mean number of species was higher in the no-Pollution
condition (M = 11.1) than the Pollution condition (M = 6.8).

The main effect of Flow was significant, \(t\)(4) = 8.37, \(p\) = 0.001.
The mean number of species was higher in the Flow condition (M = 11.3)
than the no-Flow condition (M = 6.6).

The two-way interaction between between Pollution and Flow was not
significant, \(t\)(4) = 2.493, \(p\) = 0.067. The mean Pollution effect
in the no-Flow condition was 6 and the mean Pollution effect in the Flow
condition was 2.6.

\textbf{Interim Summary}. We went through this exercise to show you how
to break up the data into individual comparisons of interest. Generally
speaking, a 2x2 repeated measures design would not be analyzed with
three paired-samples \(t\)-test. This is because it is more convenient
to use the repeated measures ANOVA for this task. We will do this in a
moment to show you that they give the same results. And, by the same
results, what we will show is that the \(p\)-values for each main
effect, and the interaction, are the same. The ANOVA will give us
\(F\)-values rather than \(t\) values. It turns out that in this
situation, the \(F\)-values are related to the \(t\) values. In fact,
\(t^2 = F\).

\subsection{2x2 Repeated Measures
ANOVA}\label{x2-repeated-measures-anova}

We just showed how a 2x2 repeated measures design can be analyzed using
paired-sampled \(t\)-tests. We broke up the analysis into three parts.
The main effect for Pollution, the main effect for Flow, and the 2-way
interaction between Pollution and Flow. We claimed the results of the
paired-samples \(t\)-test analysis would mirror what we would find if we
conducted the analysis using an ANOVA. Let's show that the results are
the same. Here are the results from the 2x2 repeated-measures ANOVA,
using the \texttt{aov} function in R.

\begin{tabular}{l|r|r|r|r|r}
\hline
  & Df & Sum Sq & Mean Sq & F value & Pr(>F)\\
\hline
Residuals & 4 & 3.70 & 0.925 & NA & NA\\
\hline
Pollution & 1 & 92.45 & 92.450 & 58.698413 & 0.0015600\\
\hline
Residuals1 & 4 & 6.30 & 1.575 & NA & NA\\
\hline
Flow & 1 & 110.45 & 110.450 & 70.126984 & 0.0011122\\
\hline
Residuals2 & 4 & 6.30 & 1.575 & NA & NA\\
\hline
Pollution:Flow & 1 & 14.45 & 14.450 & 6.215054 & 0.0672681\\
\hline
Residuals & 4 & 9.30 & 2.325 & NA & NA\\
\hline
\end{tabular}

Let's compare these results with the paired-samples \(t\)-tests.

\textbf{Main effect of Pollution}: Using the paired samples \(t\)-test,
we found \(t\)(4) =7.6615, \(p\)=0.00156. Using the ANOVA we found,
\(F\)(1,4) = 58.69, \(p\)=0.00156. See, the \(p\)-values are the same,
and \(t^2 = 7.6615^2 = 58.69 = F\).

\textbf{Main effect of Flow}: Using the paired samples \(t\)-test, we
found \(t\)(4) =8.3742, \(p\)=0.001112. Using the ANOVA we found,
\(F\)(1,4) = 70.126, \(p\)=0.001112. See, the \(p\)-values are the same,
and \(t^2 = 8.3742^2 = 70.12 = F\).

\textbf{Interaction effect}: Using the paired samples \(t\)-test, we
found \(t\)(4) =2.493, \(p\)=0.06727. Using the ANOVA we found,
\(F\)(1,4) = 6.215, \(p\)=0.06727. See, the \(p\)-values are the same,
and \(t^2 = 2.493^2 = 6.215 = F\).

There you have it. The results from a 2x2 repeated measures ANOVA are
the same as you would get if you used paired-samples \(t\)-tests for the
main effects and interactions.

\section{2x2 Between-subjects ANOVA}\label{x2-between-subjects-anova}

You must be wondering how to calculate a 2x2 ANOVA. We haven't discussed
this yet. We've only shown you that you don't have to do it when the
design is a 2x2 repeated measures design (note this is a special case).

We are now going to work through some examples of calculating the ANOVA
table for 2x2 designs. We will start with the between-subjects ANOVA for
2x2 designs. We do essentially the same thing that we did before (in the
other ANOVAs), and the only new thing is to show how to compute the
interaction effect.

Remember the logic of the ANOVA is to partition the variance into
different parts. The SS formula for the between-subjects 2x2 ANOVA looks
like this:

\(SS_\text{Total} = SS_\text{Effect IV1} + SS_\text{Effect IV2} + SS_\text{Effect IV1xIV2} + SS_\text{Error}\)

In the following sections we use tables to show the calculation of each
SS. We use the same example as before with the exception that \textbf{we
are turning this into a between-subjects design}. There are now 5
different subjects in each condition, for a total of 20 subjects. As a
result, we remove the subjects column.

\subsection{SS Total}\label{ss-total-2}

We calculate the grand mean (mean of all of the score). Then, we
calculate the differences between each score and the grand mean. We
square the difference scores, and sum them up. That is
\(SS_\text{Total}\), reported in the bottom yellow row.

\begin{table}
\centering
\begin{tabular}{l|l|l|l|l|l|l|l|l|>{}l|>{}l|>{}l|>{}l}
\hline
\multicolumn{1}{c|}{ } & \multicolumn{4}{c|}{All Conditions} & \multicolumn{4}{c|}{Difference from Grand Mean} & \multicolumn{4}{c}{Squared Differences} \\
\cline{2-5} \cline{6-9} \cline{10-13}
\multicolumn{1}{c|}{ } & \multicolumn{2}{c|}{Low Flow} & \multicolumn{2}{c|}{High Flow} & \multicolumn{2}{c|}{Low Flow} & \multicolumn{2}{c|}{High Flow} & \multicolumn{2}{c|}{Low Flow} & \multicolumn{2}{c}{High Flow} \\
\cline{2-3} \cline{4-5} \cline{6-7} \cline{8-9} \cline{10-11} \cline{12-13}
\multicolumn{1}{c|}{ } & \multicolumn{1}{c|}{No Pollution} & \multicolumn{1}{c|}{Pollution} & \multicolumn{1}{c|}{No Pollution} & \multicolumn{1}{c|}{Pollution} & \multicolumn{1}{c|}{No Pollution} & \multicolumn{1}{c|}{Pollution} & \multicolumn{1}{c|}{No Pollution} & \multicolumn{1}{c|}{Pollution} & \multicolumn{1}{c|}{No Pollution} & \multicolumn{1}{c|}{Pollution} & \multicolumn{1}{c|}{No Pollution} & \multicolumn{1}{c}{Pollution} \\
\cline{2-2} \cline{3-3} \cline{4-4} \cline{5-5} \cline{6-6} \cline{7-7} \cline{8-8} \cline{9-9} \cline{10-10} \cline{11-11} \cline{12-12} \cline{13-13}
  & A & B & C & D & A-GrandM & B-GrandM & C-GrandM & D-GrandM & (A-GrandM)\textasciicircum{}2 & (B-GrandM)\textasciicircum{}2 & (C-GrandM)\textasciicircum{}2 & (D-GrandM)\textasciicircum{}2\\
\hline
 & 10 & 5 & 12 & 9 & 1.05 & -3.95 & 3.05 & 0.05 & \cellcolor{yellow}{1.1025} & \cellcolor{yellow}{15.6025} & \cellcolor{yellow}{9.3025} & \cellcolor{yellow}{0.0025}\\
\hline
 & 8 & 4 & 13 & 8 & -0.95 & -4.95 & 4.05 & -0.95 & \cellcolor{yellow}{0.9025} & \cellcolor{yellow}{24.5025} & \cellcolor{yellow}{16.4025} & \cellcolor{yellow}{0.9025}\\
\hline
 & 11 & 3 & 14 & 10 & 2.05 & -5.95 & 5.05 & 1.05 & \cellcolor{yellow}{4.2025} & \cellcolor{yellow}{35.4025} & \cellcolor{yellow}{25.5025} & \cellcolor{yellow}{1.1025}\\
\hline
 & 9 & 4 & 11 & 11 & 0.05 & -4.95 & 2.05 & 2.05 & \cellcolor{yellow}{0.0025} & \cellcolor{yellow}{24.5025} & \cellcolor{yellow}{4.2025} & \cellcolor{yellow}{4.2025}\\
\hline
 & 10 & 2 & 13 & 12 & 1.05 & -6.95 & 4.05 & 3.05 & \cellcolor{yellow}{1.1025} & \cellcolor{yellow}{48.3025} & \cellcolor{yellow}{16.4025} & \cellcolor{yellow}{9.3025}\\
\hline
\cellcolor{lightgray}{Means} & \cellcolor{lightgray}{9.6} & \cellcolor{lightgray}{3.6} & \cellcolor{lightgray}{12.6} & \cellcolor{lightgray}{10} & \cellcolor{lightgray}{} & \cellcolor{lightgray}{} & \cellcolor{lightgray}{} & \cellcolor{lightgray}{} & \cellcolor{yellow}{} & \cellcolor{yellow}{} & \cellcolor{yellow}{} & \cellcolor{yellow}{}\\
\hline
\cellcolor{lightgray}{Grand Mean} & \cellcolor{lightgray}{8.95} & \cellcolor{lightgray}{} & \cellcolor{lightgray}{} & \cellcolor{lightgray}{} & \cellcolor{lightgray}{} & \cellcolor{lightgray}{} & \cellcolor{lightgray}{} & \cellcolor{lightgray}{} & \cellcolor{yellow}{} & \cellcolor{yellow}{} & \cellcolor{yellow}{} & \cellcolor{yellow}{}\\
\hline
\cellcolor{yellow}{sums} & \cellcolor{yellow}{} & \cellcolor{yellow}{} & \cellcolor{yellow}{} & \cellcolor{yellow}{} & \cellcolor{yellow}{} & \cellcolor{yellow}{} & \cellcolor{yellow}{} & \cellcolor{yellow}{Sums} & \cellcolor{yellow}{7.3125} & \cellcolor{yellow}{148.3125} & \cellcolor{yellow}{71.8125} & \cellcolor{yellow}{15.5125}\\
\hline
\cellcolor{yellow}{SS Total} & \cellcolor{yellow}{} & \cellcolor{yellow}{} & \cellcolor{yellow}{} & \cellcolor{yellow}{} & \cellcolor{yellow}{} & \cellcolor{yellow}{} & \cellcolor{yellow}{} & \cellcolor{yellow}{SS Total} & \cellcolor{yellow}{242.95} & \cellcolor{yellow}{} & \cellcolor{yellow}{} & \cellcolor{yellow}{}\\
\hline
\end{tabular}
\end{table}

\subsection{SS Pollution}\label{ss-pollution}

We need to compute the SS for the main effect for Pollution. We
calculate the grand mean (mean of all of the scores). Then, we calculate
the means for the two Pollution conditions. Then we treat each score as
if it was the mean for it's respective Pollution condition. We find the
differences between each Pollution condition mean and the grand mean.
Then we square the differences and sum them up. That is
\(SS_\text{Pollution}\), reported in the bottom yellow row.

\begin{table}
\centering
\begin{tabular}{l|l|l|l|l|l|l|l|l|>{}l|>{}l|>{}l|>{}l}
\hline
\multicolumn{1}{c|}{ } & \multicolumn{4}{c|}{All Conditions} & \multicolumn{4}{c|}{Pollution Mean - GM} & \multicolumn{4}{c}{Squared Differences} \\
\cline{2-5} \cline{6-9} \cline{10-13}
\multicolumn{1}{c|}{ } & \multicolumn{2}{c|}{Low Flow} & \multicolumn{2}{c|}{High Flow} & \multicolumn{2}{c|}{Low Flow} & \multicolumn{2}{c|}{High Flow} & \multicolumn{2}{c|}{Low Flow} & \multicolumn{2}{c}{High Flow} \\
\cline{2-3} \cline{4-5} \cline{6-7} \cline{8-9} \cline{10-11} \cline{12-13}
\multicolumn{1}{c|}{ } & \multicolumn{1}{c|}{No Pollution} & \multicolumn{1}{c|}{Pollution} & \multicolumn{1}{c|}{No Pollution} & \multicolumn{1}{c|}{Pollution} & \multicolumn{1}{c|}{No Pollution} & \multicolumn{1}{c|}{Pollution} & \multicolumn{1}{c|}{No Pollution} & \multicolumn{1}{c|}{Pollution} & \multicolumn{1}{c|}{No Pollution} & \multicolumn{1}{c|}{Pollution} & \multicolumn{1}{c|}{No Pollution} & \multicolumn{1}{c}{Pollution} \\
\cline{2-2} \cline{3-3} \cline{4-4} \cline{5-5} \cline{6-6} \cline{7-7} \cline{8-8} \cline{9-9} \cline{10-10} \cline{11-11} \cline{12-12} \cline{13-13}
  & A & B & C & D & NDM-GM A & DM-GM B & NDM-GM C & DM-GM D & (NDM-GM )\textasciicircum{}2 A & (DM-GM)\textasciicircum{}2 B & (NDM-GM)\textasciicircum{}2 C & (DM-GM)\textasciicircum{}2 D\\
\hline
 & 10 & 5 & 12 & 9 & 2.15 & -2.15 & 2.15 & -2.15 & \cellcolor{yellow}{4.6225} & \cellcolor{yellow}{4.6225} & \cellcolor{yellow}{4.6225} & \cellcolor{yellow}{4.6225}\\
\hline
 & 8 & 4 & 13 & 8 & 2.15 & -2.15 & 2.15 & -2.15 & \cellcolor{yellow}{4.6225} & \cellcolor{yellow}{4.6225} & \cellcolor{yellow}{4.6225} & \cellcolor{yellow}{4.6225}\\
\hline
 & 11 & 3 & 14 & 10 & 2.15 & -2.15 & 2.15 & -2.15 & \cellcolor{yellow}{4.6225} & \cellcolor{yellow}{4.6225} & \cellcolor{yellow}{4.6225} & \cellcolor{yellow}{4.6225}\\
\hline
 & 9 & 4 & 11 & 11 & 2.15 & -2.15 & 2.15 & -2.15 & \cellcolor{yellow}{4.6225} & \cellcolor{yellow}{4.6225} & \cellcolor{yellow}{4.6225} & \cellcolor{yellow}{4.6225}\\
\hline
 & 10 & 2 & 13 & 12 & 2.15 & -2.15 & 2.15 & -2.15 & \cellcolor{yellow}{4.6225} & \cellcolor{yellow}{4.6225} & \cellcolor{yellow}{4.6225} & \cellcolor{yellow}{4.6225}\\
\hline
\cellcolor{lightgray}{Means} & \cellcolor{lightgray}{9.6} & \cellcolor{lightgray}{3.6} & \cellcolor{lightgray}{12.6} & \cellcolor{lightgray}{10} & \cellcolor{lightgray}{} & \cellcolor{lightgray}{} & \cellcolor{lightgray}{} & \cellcolor{lightgray}{} & \cellcolor{yellow}{} & \cellcolor{yellow}{} & \cellcolor{yellow}{} & \cellcolor{yellow}{}\\
\hline
\cellcolor{lightgray}{Grand Mean} & \cellcolor{lightgray}{8.95} & \cellcolor{lightgray}{No Pollution} & \cellcolor{lightgray}{11.1} & \cellcolor{lightgray}{Pollution} & \cellcolor{lightgray}{6.8} & \cellcolor{lightgray}{} & \cellcolor{lightgray}{} & \cellcolor{lightgray}{} & \cellcolor{yellow}{} & \cellcolor{yellow}{} & \cellcolor{yellow}{} & \cellcolor{yellow}{}\\
\hline
\cellcolor{yellow}{sums} & \cellcolor{yellow}{} & \cellcolor{yellow}{} & \cellcolor{yellow}{} & \cellcolor{yellow}{} & \cellcolor{yellow}{} & \cellcolor{yellow}{} & \cellcolor{yellow}{} & \cellcolor{yellow}{Sums} & \cellcolor{yellow}{23.1125} & \cellcolor{yellow}{23.1125} & \cellcolor{yellow}{23.1125} & \cellcolor{yellow}{23.1125}\\
\hline
\cellcolor{yellow}{SS Pollution} & \cellcolor{yellow}{} & \cellcolor{yellow}{} & \cellcolor{yellow}{} & \cellcolor{yellow}{} & \cellcolor{yellow}{} & \cellcolor{yellow}{} & \cellcolor{yellow}{} & \cellcolor{yellow}{SS Pollution} & \cellcolor{yellow}{92.45} & \cellcolor{yellow}{} & \cellcolor{yellow}{} & \cellcolor{yellow}{}\\
\hline
\end{tabular}
\end{table}

These tables are a lot to look at! Notice here, that we first found the
grand mean (8.95). Then we found the mean for all the scores in the
no-Pollution condition (columns A and C), that was 11.1. All of the
difference scores for the no-Pollution condition are 11.1-8.95 = 2.15.
We also found the mean for the scores in the Pollution condition
(columns B and D), that was 6.8. So, all of the difference scores are
6.8-8.95 = -2.15. Remember, means are the balancing point in the data,
this is why the difference scores are +2.15 and -2.15. The grand mean
8.95 is in between the two condition means (11.1 and 6.8), by a
difference of 2.15.

\subsection{SS Flow}\label{ss-flow}

We need to compute the SS for the main effect for Flow. We calculate the
grand mean (mean of all of the scores). Then, we calculate the means for
the two Flow conditions. Then we treat each score as if it was the mean
for it's respective Flow condition. We find the differences between each
Flow condition mean and the grand mean. Then we square the differences
and sum them up. That is \(SS_\text{Flow}\), reported in the bottom
yellow row.

\begin{table}
\centering
\begin{tabular}{l|l|l|l|l|l|l|l|l|>{}l|>{}l|>{}l|>{}l}
\hline
\multicolumn{1}{c|}{ } & \multicolumn{4}{c|}{All Conditions} & \multicolumn{4}{c|}{Flow Mean - GM} & \multicolumn{4}{c}{Squared Differences} \\
\cline{2-5} \cline{6-9} \cline{10-13}
\multicolumn{1}{c|}{ } & \multicolumn{2}{c|}{Low Flow} & \multicolumn{2}{c|}{High Flow} & \multicolumn{2}{c|}{Low Flow} & \multicolumn{2}{c|}{High Flow} & \multicolumn{2}{c|}{Low Flow} & \multicolumn{2}{c}{High Flow} \\
\cline{2-3} \cline{4-5} \cline{6-7} \cline{8-9} \cline{10-11} \cline{12-13}
\multicolumn{1}{c|}{ } & \multicolumn{1}{c|}{No Pollution} & \multicolumn{1}{c|}{Pollution} & \multicolumn{1}{c|}{No Pollution} & \multicolumn{1}{c|}{Pollution} & \multicolumn{1}{c|}{No Pollution} & \multicolumn{1}{c|}{Pollution} & \multicolumn{1}{c|}{No Pollution} & \multicolumn{1}{c|}{Pollution} & \multicolumn{1}{c|}{No Pollution} & \multicolumn{1}{c|}{Pollution} & \multicolumn{1}{c|}{No Pollution} & \multicolumn{1}{c}{Pollution} \\
\cline{2-2} \cline{3-3} \cline{4-4} \cline{5-5} \cline{6-6} \cline{7-7} \cline{8-8} \cline{9-9} \cline{10-10} \cline{11-11} \cline{12-12} \cline{13-13}
  & A & B & C & D & NRM-GM A & NRM-GM B & RM-GM C & RM-GM D & (NRM-GM )\textasciicircum{}2 A & (NRM-GM)\textasciicircum{}2 B & (RM-GM)\textasciicircum{}2 C & (RM-GM)\textasciicircum{}2 D\\
\hline
 & 10 & 5 & 12 & 9 & -2.35 & -2.35 & 2.35 & 2.35 & \cellcolor{yellow}{5.5225} & \cellcolor{yellow}{5.5225} & \cellcolor{yellow}{5.5225} & \cellcolor{yellow}{5.5225}\\
\hline
 & 8 & 4 & 13 & 8 & -2.35 & -2.35 & 2.35 & 2.35 & \cellcolor{yellow}{5.5225} & \cellcolor{yellow}{5.5225} & \cellcolor{yellow}{5.5225} & \cellcolor{yellow}{5.5225}\\
\hline
 & 11 & 3 & 14 & 10 & -2.35 & -2.35 & 2.35 & 2.35 & \cellcolor{yellow}{5.5225} & \cellcolor{yellow}{5.5225} & \cellcolor{yellow}{5.5225} & \cellcolor{yellow}{5.5225}\\
\hline
 & 9 & 4 & 11 & 11 & -2.35 & -2.35 & 2.35 & 2.35 & \cellcolor{yellow}{5.5225} & \cellcolor{yellow}{5.5225} & \cellcolor{yellow}{5.5225} & \cellcolor{yellow}{5.5225}\\
\hline
 & 10 & 2 & 13 & 12 & -2.35 & -2.35 & 2.35 & 2.35 & \cellcolor{yellow}{5.5225} & \cellcolor{yellow}{5.5225} & \cellcolor{yellow}{5.5225} & \cellcolor{yellow}{5.5225}\\
\hline
\cellcolor{lightgray}{Means} & \cellcolor{lightgray}{9.6} & \cellcolor{lightgray}{3.6} & \cellcolor{lightgray}{12.6} & \cellcolor{lightgray}{10} & \cellcolor{lightgray}{} & \cellcolor{lightgray}{} & \cellcolor{lightgray}{} & \cellcolor{lightgray}{} & \cellcolor{yellow}{} & \cellcolor{yellow}{} & \cellcolor{yellow}{} & \cellcolor{yellow}{}\\
\hline
\cellcolor{lightgray}{Grand Mean} & \cellcolor{lightgray}{8.95} & \cellcolor{lightgray}{Low Flow} & \cellcolor{lightgray}{6.6} & \cellcolor{lightgray}{High Flow} & \cellcolor{lightgray}{11.3} & \cellcolor{lightgray}{} & \cellcolor{lightgray}{} & \cellcolor{lightgray}{} & \cellcolor{yellow}{} & \cellcolor{yellow}{} & \cellcolor{yellow}{} & \cellcolor{yellow}{}\\
\hline
\cellcolor{yellow}{sums} & \cellcolor{yellow}{} & \cellcolor{yellow}{} & \cellcolor{yellow}{} & \cellcolor{yellow}{} & \cellcolor{yellow}{} & \cellcolor{yellow}{} & \cellcolor{yellow}{} & \cellcolor{yellow}{Sums} & \cellcolor{yellow}{27.6125} & \cellcolor{yellow}{27.6125} & \cellcolor{yellow}{27.6125} & \cellcolor{yellow}{27.6125}\\
\hline
\cellcolor{yellow}{SS Flow} & \cellcolor{yellow}{} & \cellcolor{yellow}{} & \cellcolor{yellow}{} & \cellcolor{yellow}{} & \cellcolor{yellow}{} & \cellcolor{yellow}{} & \cellcolor{yellow}{} & \cellcolor{yellow}{SS Flow} & \cellcolor{yellow}{110.45} & \cellcolor{yellow}{} & \cellcolor{yellow}{} & \cellcolor{yellow}{}\\
\hline
\end{tabular}
\end{table}

Now we treat each no-Flow score as the mean for the no-Flow condition
(6.6), and subtract it from the grand mean (8.95), to get -2.35. Then,
we treat each Flow score as the mean for the Flow condition (11.3), and
subtract it from the grand mean (8.95), to get +2.35. Then we square the
differences and sum them up.

\subsection{SS Pollution by Flow}\label{ss-pollution-by-flow}

We need to compute the SS for the interaction effect between Pollution
and Flow. This is the new thing that we do in an ANOVA with more than
one IV. How do we calculate the variation explained by the interaction?

The heart of the question is something like this. Do the individual
means for each of the four conditions do something a little bit
different than the group means for both of the independent variables?

For example, consider the overall mean for all of the scores in the Low
Flow group, we found that to be 6.6 Now, was the mean for each no-Flow
group in the whole design a 6.6? For example, in the no-Pollution group,
was the mean for column A (the no-Flow condition in that group) also
6.6? The answer is no, it was 9.6. How about the Pollution group? Was
the mean for the Flow condition in the Pollution group (column B) 6.6?
No, it was 3.6. The mean of 9.6 and 3.6 is 6.6. If there was no hint of
an interaction, we would expect that the means for the Flow condition in
both levels of the Pollution group would be the same, they would both be
6.6. However, when there is an interaction, the means for the Flow group
will depend on the levels of the group from another IV. In this case, it
looks like there is an interaction because the means are different from
6.6, they are 9.6 and 3.6 for the no-Pollution and Pollution conditions.
This is extra-variance that is not explained by the mean for the Flow
condition. We want to capture this extra variance and sum it up. Then we
will have measure of the portion of the variance that is due to the
interaction between the Flow and Pollution conditions.

What we will do is this. We will find the four condition means. Then we
will see how much additional variation they explain beyond the group
means for Flow and Pollution. To do this we treat each score as the
condition mean for that score. Then we subtract the mean for the
Pollution group, and the mean for the Flow group, and then we add the
grand mean. This gives us the unique variation that is due to the
interaction. We could also say that we are subtracting each condition
mean from the grand mean, and then adding back in the Pollution mean and
the Flow mean, that would amount to the same thing, and perhaps make
more sense.

Here is a formula to describe the process for each score:

\(\bar{X}_\text{condition} -\bar{X}_\text{IV1} - \bar{X}_\text{IV2} + \bar{X}_\text{Grand Mean}\)

Or we could write it this way:

\(\bar{X}_\text{condition} - \bar{X}_\text{Grand Mean} + \bar{X}_\text{IV1} + \bar{X}_\text{IV2}\)

When you look at the following table, we apply this formula to the
calculation of each of the differences scores. We then square the
difference scores, and sum them up to get \(SS_\text{Interaction}\),
which is reported in the bottom yellow row.

\begin{table}
\centering
\begin{tabular}{l|l|l|l|l|l|l|l|l|>{}l|>{}l|>{}l|>{}l}
\hline
\multicolumn{1}{c|}{ } & \multicolumn{4}{c|}{All Conditions} & \multicolumn{4}{c|}{Interaction Differences} & \multicolumn{4}{c}{Squared Differences} \\
\cline{2-5} \cline{6-9} \cline{10-13}
\multicolumn{1}{c|}{ } & \multicolumn{2}{c|}{Low Flow} & \multicolumn{2}{c|}{High Flow} & \multicolumn{2}{c|}{Low Flow} & \multicolumn{2}{c|}{High Flow} & \multicolumn{2}{c|}{Low Flow} & \multicolumn{2}{c}{High Flow} \\
\cline{2-3} \cline{4-5} \cline{6-7} \cline{8-9} \cline{10-11} \cline{12-13}
\multicolumn{1}{c|}{ } & \multicolumn{1}{c|}{No Pollution} & \multicolumn{1}{c|}{Pollution} & \multicolumn{1}{c|}{No Pollution} & \multicolumn{1}{c|}{Pollution} & \multicolumn{1}{c|}{No Pollution} & \multicolumn{1}{c|}{Pollution} & \multicolumn{1}{c|}{No Pollution} & \multicolumn{1}{c|}{Pollution} & \multicolumn{1}{c|}{No Pollution} & \multicolumn{1}{c|}{Pollution} & \multicolumn{1}{c|}{No Pollution} & \multicolumn{1}{c}{Pollution} \\
\cline{2-2} \cline{3-3} \cline{4-4} \cline{5-5} \cline{6-6} \cline{7-7} \cline{8-8} \cline{9-9} \cline{10-10} \cline{11-11} \cline{12-12} \cline{13-13}
  & A & B & C & D & A-NP-LF+GM & B-P-LF+GM & C-NP-HF+GM & D-P-HF+GM & (A-NP-LF+GM)\textasciicircum{}2 A & (B-B-LF+GM)\textasciicircum{}2 B & (C-NP-HF+GM)\textasciicircum{}2 C & (D-P-HF+GM)\textasciicircum{}2 D\\
\hline
 & 10 & 5 & 12 & 9 & 0.85 & -0.85 & -0.85 & 0.85 & \cellcolor{yellow}{0.7225} & \cellcolor{yellow}{0.7225} & \cellcolor{yellow}{0.7225} & \cellcolor{yellow}{0.7225}\\
\hline
 & 8 & 4 & 13 & 8 & 0.85 & -0.85 & -0.85 & 0.85 & \cellcolor{yellow}{0.7225} & \cellcolor{yellow}{0.7225} & \cellcolor{yellow}{0.7225} & \cellcolor{yellow}{0.7225}\\
\hline
 & 11 & 3 & 14 & 10 & 0.85 & -0.85 & -0.85 & 0.85 & \cellcolor{yellow}{0.7225} & \cellcolor{yellow}{0.7225} & \cellcolor{yellow}{0.7225} & \cellcolor{yellow}{0.7225}\\
\hline
 & 9 & 4 & 11 & 11 & 0.85 & -0.85 & -0.85 & 0.85 & \cellcolor{yellow}{0.7225} & \cellcolor{yellow}{0.7225} & \cellcolor{yellow}{0.7225} & \cellcolor{yellow}{0.7225}\\
\hline
 & 10 & 2 & 13 & 12 & 0.85 & -0.85 & -0.85 & 0.85 & \cellcolor{yellow}{0.7225} & \cellcolor{yellow}{0.7225} & \cellcolor{yellow}{0.7225} & \cellcolor{yellow}{0.7225}\\
\hline
\cellcolor{lightgray}{Means} & \cellcolor{lightgray}{9.6} & \cellcolor{lightgray}{3.6} & \cellcolor{lightgray}{12.6} & \cellcolor{lightgray}{10} & \cellcolor{lightgray}{} & \cellcolor{lightgray}{} & \cellcolor{lightgray}{} & \cellcolor{lightgray}{} & \cellcolor{yellow}{} & \cellcolor{yellow}{} & \cellcolor{yellow}{} & \cellcolor{yellow}{}\\
\hline
\cellcolor{lightgray}{Grand Mean} & \cellcolor{lightgray}{8.95} & \cellcolor{lightgray}{} & \cellcolor{lightgray}{} & \cellcolor{lightgray}{} & \cellcolor{lightgray}{} & \cellcolor{lightgray}{} & \cellcolor{lightgray}{} & \cellcolor{lightgray}{} & \cellcolor{yellow}{} & \cellcolor{yellow}{} & \cellcolor{yellow}{} & \cellcolor{yellow}{}\\
\hline
\cellcolor{yellow}{sums} & \cellcolor{yellow}{} & \cellcolor{yellow}{} & \cellcolor{yellow}{} & \cellcolor{yellow}{} & \cellcolor{yellow}{} & \cellcolor{yellow}{} & \cellcolor{yellow}{} & \cellcolor{yellow}{Sums} & \cellcolor{yellow}{3.6125} & \cellcolor{yellow}{3.6125} & \cellcolor{yellow}{3.6125} & \cellcolor{yellow}{3.6125}\\
\hline
\cellcolor{yellow}{SS Interaction} & \cellcolor{yellow}{} & \cellcolor{yellow}{} & \cellcolor{yellow}{} & \cellcolor{yellow}{} & \cellcolor{yellow}{} & \cellcolor{yellow}{} & \cellcolor{yellow}{} & \cellcolor{yellow}{SS Interaction} & \cellcolor{yellow}{14.45} & \cellcolor{yellow}{} & \cellcolor{yellow}{} & \cellcolor{yellow}{}\\
\hline
\end{tabular}
\end{table}

\subsection{SS Error}\label{ss-error-1}

The last thing we need to find is the SS Error. We can solve for that
because we found everything else in this formula:

\(SS_\text{Total} = SS_\text{Effect IV1} + SS_\text{Effect IV2} + SS_\text{Effect IV1xIV2} + SS_\text{Error}\)

Even though this textbook meant to explain things in a step by step way,
we guess you are tired from watching us work out the 2x2 ANOVA by hand.
You and me both, making these tables was a lot of work. We have already
shown you how to compute the SS for error before, so we will not do the
full example here. Instead, we solve for SS Error using the numbers we
have already obtained.

\$SS\_\text{Error} = SS\_\text{Total}- SS\_\text{Effect IV1} -
SS\_\text{Effect IV2} - SS\_\text{Effect IV1xIV2} \$

\$SS\_\text{Error} = 242.95 - 92.45 - 110.45 - 14.45 = 25.6 \$

\subsection{Check your work}\label{check-your-work}

We are going to skip the part where we divide the SSes by their dfs to
find the MSEs so that we can compute the three \(F\)-values. Instead, if
we have done the calculations of the \(SS\)es correctly, they should be
same as what we would get if we used R to calculate the \(SS\)es. Let's
make R do the work, and then compare to check our work.

\begin{tabular}{l|r|r|r|r|r}
\hline
  & Df & Sum Sq & Mean Sq & F value & Pr(>F)\\
\hline
Pollution & 1 & 92.45 & 92.45 & 57.78125 & 0.0000011\\
\hline
Flow & 1 & 110.45 & 110.45 & 69.03125 & 0.0000003\\
\hline
Pollution:Flow & 1 & 14.45 & 14.45 & 9.03125 & 0.0083879\\
\hline
Residuals & 16 & 25.60 & 1.60 & NA & NA\\
\hline
\end{tabular}

A quick look through the column \texttt{Sum\ Sq} shows that we did our
work by hand correctly. Congratulations to us! Note, this is not the
same results as we had before with the repeated measures ANOVA. We
conducted a between-subjects design, so we did not get to further
partition the SS error into a part due to subject variation and a
left-over part. We also gained degrees of freedom in the error term. It
turns out with this specific set of data, we find p-values of less than
0.05 for all effects (main effects and the interaction, which was not
less than 0.05 using the same data, but treating it as a
repeated-measures design)

\section{Fireside chat}\label{fireside-chat}

Sometimes it's good to get together around a fire and have a chat. Let's
pretend we're sitting around a fire.

It's been a long day. A long couple of weeks and months since we started
this course on statistics. We just went through the most complicated
things we have done so far. This is a long chapter. What should we do
next?

Here's a couple of options. We could work through, by hand, more and
more ANOVAs. Do you want to do that? I don't, making these tables isn't
too bad, but it takes a lot of time. It's really good to see everything
that we do laid bare in the table form a few times. We've done that
already. It's really good for you to attempt to calculate an ANOVA by
hand at least once in your life. It builds character. It helps you know
that you know what you are doing, and what the ANOVA is doing. We can't
make you do this, we can only make the suggestion. If we keep doing
these by hand, it is not good for us, and it is not you doing them by
hand. So, what are the other options.

The other options are to work at a slightly higher level. We will
discuss some research designs, and the ANOVAs that are appropriate for
their analysis. We will conduct the ANOVAs using R, and print out the
ANOVA tables. This is what you do in the lab, and what most researchers
do. They use software most of the time to make the computer do the work.
Because of this, it is most important that you know what the software is
doing. You can make mistakes when telling software what to do, so you
need to be able to check the software's work so you know when the
software is giving you wrong answers. All of these skills are built up
over time through the process of analyzing different data sets. So, for
the remainder of our discussion on ANOVAs we stick to that higher level.
No more monster tables of SSes. You are welcome.

\section{Factorial summary}\label{factorial-summary}

We have introduced you to factorial designs, which are simply designs
with more than one IV. The special property of factorial designs is that
all of the levels of each IV need to be crossed with the other IVs.

We showed you how to analyse a repeated measures 2x2 design with paired
samples-tests, and what an ANOVA table would look like if you did this
in R. We also went through, by hand, the task of calculating an ANOVA
table for a 2x2 between subjects design.

The main point we want you take away is that factorial designs are
extremely useful for determining things that cause effects to change.
Generally a researcher measures an effect of interest (their IV 1).
Then, they want to know what makes that effect get bigger or smaller.
They want to exert experimental control over their effect. For example,
they might have a theory that says doing X should make the effect
bigger, but doing Y should make it smaller. They can test these theories
using factorial designs, and manipulating X or Y as a second independent
variable.

In a factorial design each IV will have it's own main effect. Sometimes
the main effect themselves are what the researcher is interested in
measures. But more often, it is the interaction effect that is most
relevant. The interaction can test whether the effect of IV1 changes
between the levels of IV2. When it does, researchers can infer that
their second manipulation (IV2) causes change in their effect of
interest. These changes are then documented and used to test underlying
causal theories about the effects of interest.

\bookmarksetup{startatroot}

\chapter{More On Factorial Designs}\label{more-on-factorial-designs}

Portions adapted from the Factorial ANOVA chapter, contributors Keryn
Bain, Rachel Blakey, Stephanie Brodie, Corey Callaghan, Will Cornwell,
Kingsley Griffin, Matt Holland, James Lavender, Andrew Letten, Shinichi
Nakagawa, Shaun Nielsen, Alistair Poore, Gordana Popovic, Fiona Robinson
and Jakub Stoklosa. ``Environmental Computing''
\url{https://environmentalcomputing.net/}

\hfill\break

In this chapter, we're diving deeper into factorial designs, a
cornerstone of understanding complex data in environmental science.
You're already familiar with the idea of having more than one
independent variable (IV) in your experiments. These IVs can be
structured in various ways: all between-subjects, all within-subjects
(like repeated measures), or a mix of both. ANOVA is our trusty tool to
analyze these designs, giving us insights into each IV's main effect and
their interactions.

\section{Looking at main effects and
interactions}\label{looking-at-main-effects-and-interactions}

Factorial designs are very common in environmental research. You'll
often come across studies showing results from these designs. It's
crucial to be comfortable interpreting these results. The key skill here
is to recognize patterns of main effects and interactions in data
graphs. This can get tricky with more than two IVs, each having multiple
levels.

\subsection{2x2 designs}\label{x2-designs-1}

Let's explore 2x2 designs. Here, you can expect two main effects and one
interaction. You'll compare means for each main effect and interaction.
There are eight possible outcomes in such a design:

\begin{enumerate}
\def\labelenumi{\arabic{enumi}.}
\tightlist
\item
  no IV1 main effect, no IV2 main effect, no interaction
\item
  IV1 main effect, no IV2 main effect, no interaction
\item
  IV1 main effect, no IV2 main effect, interaction
\item
  IV1 main effect, IV2 main effect, no interaction
\item
  IV1 main effect, IV2 main effect, interaction
\item
  no IV1 main effect, IV2 main effect, no interaction
\item
  no IV1 main effect, IV2 main effect, interaction
\item
  no IV1 main effect, no IV2 main effect, interaction
\end{enumerate}

OK, so if you run a 2x2, any of these 8 general patterns could occur in
your data. That's a lot to keep track of isn't it? As you develop your
skills in examining graphs that plot means, you should be able to look
at the graph and visually guesstimate if there is, or is not, a main
effect or interaction. You will need you inferential statistics to tell
you for sure, but it is worth knowing how to know see the patterns.

Let's visualize these outcomes using R. We'll create bar and line graphs
to illustrate these patterns. Bar graphs are great for seeing
differences in means directly, while line graphs help us spot
interactions -- look for crossing lines as a hint of interaction.
Figure~\ref{fig-11bar22} shows the possible patterns of main effects and
interactions in bar graph form. Here is a legend for the labels in the
panels.

\begin{itemize}
\tightlist
\item
  1 = there was a main effect for IV1.
\item
  \textasciitilde1 = there was \textbf{not} a main effect for IV1
\item
  2 = there was a main effect for IV2
\item
  \textasciitilde2 = there was \textbf{not} a main effect of IV2
\item
  1x2 = there was an interaction
\item
  \textasciitilde1x2 = there was \textbf{not} an interaction
\end{itemize}

\begin{figure}

\centering{

\includegraphics[width=1\linewidth,height=\textheight,keepaspectratio]{11-MixedANOVA_files/figure-pdf/fig-11bar22-1.pdf}

}

\caption{\label{fig-11bar22}8 Example patterns for means for each of the
possible kinds of general outcomes in a 2x2 design.}

\end{figure}%

Figure~\ref{fig-11lines22} shows the same eight patterns in line graph
form:

\begin{figure}

\centering{

\includegraphics[width=1\linewidth,height=\textheight,keepaspectratio]{11-MixedANOVA_files/figure-pdf/fig-11lines22-1.pdf}

}

\caption{\label{fig-11lines22}Line graphs showing 8 possible general
outcomes for a 2x2 design.}

\end{figure}%

In line graphs, interactions are more apparent. Parallel lines suggest
no interaction, while crossing lines indicate potential interactions.
The position of points relative to each other helps identify main
effects. Things get complicated fast. When designing experiments, aim
for the minimum number of independent variables (IVs) and levels needed
to answer your research question. This approach makes interpreting your
data more straightforward and your conclusions clearer. Whenever you see
that someone ran a 4x3x7x2 design, your head should spin. It's just too
complicated.

\section{Interpreting main effects and
interactions}\label{interpreting-main-effects-and-interactions}

Understanding main effects and interactions is essential for accurately
interpreting research data, especially in complex fields like
environmental science.

A \textbf{main effect} refers to the consistent impact of an independent
variable (IV) on a dependent variable (DV). For example, in
environmental studies, consider the effect of a specific fertilizer (IV)
on plant growth (DV). If using this fertilizer consistently results in
increased growth compared to not using it, we observe a clear main
effect. This effect remains true regardless of other variables such as
soil type or weather conditions.

Often, it is convenient to think of main effects as a consistent
influence of one manipulation. However, the picture changes when we
introduce an interaction. An interaction occurs when the effect of one
IV depends on another IV. By definition, an interactino means that some
main effect is \textbf{not} behaving consistently across different
situations. For instance, the impact of our fertilizer might vary
depending on the level of sunlight or the type of soil, indicating an
interaction between these factors and the fertilizer. This interaction
disrupts the consistency of the main effect, suggesting that the effect
of the fertilizer is not uniform across all conditions.

Researchers often phrase their findings to highlight this complexity:
``We found a main effect of X, \textbf{BUT}, this main effect was
qualified by an interaction between X and Y.'' The use of ``BUT'' here
is crucial. It signals that the main effect cannot be fully understood
without considering the interaction. The interaction indicates that the
influence of the IV changes under different conditions, making it
essential to consider these variables together for a complete
understanding.

In environmental science, this becomes particularly relevant when
studying ecosystems or climate interactions, where multiple variables
interplay in complex ways. The interpretation of main effects and
interactions in such contexts is not just about identifying individual
effects but understanding how these effects change in different
environmental settings.

Here are two generalized examples to help you make sense of these
issues:

\subsection{A consistent main effect and an
interaction}\label{a-consistent-main-effect-and-an-interaction}

\begin{figure}

\centering{

\includegraphics[width=1\linewidth,height=\textheight,keepaspectratio]{11-MixedANOVA_files/figure-pdf/fig-11mainint-1.pdf}

}

\caption{\label{fig-11mainint}Example means showing a generally
consistent main effect along with an interaction}

\end{figure}%

Figure~\ref{fig-11mainint} shows a main effect and interaction. There is
a main effect of IV2: the level 1 means (red points and line) are both
lower than the level 2 means (aqua points and line). There is also an
interaction. The size of the difference between the red and aqua points
in the A condition (left) is bigger than the size of the difference in
the B condition.

\textbf{How would we interpret this}? We could say there WAS a main
effect of IV2, BUT it was qualified by an IV1 x IV2 interaction.

\textbf{What's the qualification}? The size of the IV2 effect changed as
a function of the levels of IV1. It was big for level A, and small for
level B of IV1.

\textbf{What does the qualification mean for the main effect}? Well,
first it means the main effect can be changed by the other IV. That's
important to know. Does it also mean that the main effect is not a real
main effect because there was an interaction? Not really, there is a
generally consistent effect of IV2. The green points are above the red
points in all cases. Whatever IV2 is doing, it seems to work in at least
a couple situations, even if the other IV also causes some change to the
influence.

\subsection{An inconsistent main effect and an
interaction}\label{an-inconsistent-main-effect-and-an-interaction}

\begin{figure}

\centering{

\includegraphics[width=1\linewidth,height=\textheight,keepaspectratio]{11-MixedANOVA_files/figure-pdf/fig-11mainintInc-1.pdf}

}

\caption{\label{fig-11mainintInc}Example data showing how an interaction
exists, and a main effect does not, even though the means for the main
effect may show a difference}

\end{figure}%

Figure~\ref{fig-11mainintInc} shows another 2x2 design. You should see
an interaction here straight away. The difference between the aqua and
red points in condition A (left two dots) is huge, and there is 0
difference between them in condition B. Is there an interaction? Yes!

Are there any main effects here? With data like this, sometimes an ANOVA
will suggest that you do have significant main effects. For example,
what is the mean difference between level 1 and 2 of IV2? That is the
average of the green points ( (10+5)/2 = 15/2= 7.5 ) compared to the
average of the red points (5). There will be a difference of 2.5 for the
main effect (7.5 vs.~5).

Starting to see the issue here? From the perspective of the main effect
(which collapses over everything and ignores the interaction), there is
an overall effect of 2.5. In other words, level 2 adds 2.5 in general
compared to level 1. However, we can see from the graph that IV2 does
not do anything in general. It does not add 2.5s everywhere. It adds 5
in condition A, and nothing in condition B. It only does one thing in
one condition.

What is happening here is that a ``main effect'' is produced by the
process of averaging over a clear interaction.

\textbf{How would we interpret this}? We might have to say there was a
main effect of IV2, BUT we would definitely say it was qualified by an
IV1 x IV2 interaction.

\textbf{What's the qualification}? The size of the IV2 effect completely
changes as a function of the levels of IV1. It was big for level A, and
nonexistent for level B of IV1.

\textbf{What does the qualification mean for the main effect}? In this
case, we might doubt whether there is a main effect of IV2 at all. It
could turn out that IV2 does not have a general influence over the DV
all of the time, it may only do something in very specific
circumstances, in combination with the presence of other factors.

\section{Mixed Designs}\label{mixed-designs}

In this book, we've explored various research designs, emphasizing that
they can take different forms. These designs can be categorized as
either between-subjects, where different subjects are in each group, or
within-subjects, where the same subjects participate in all conditions.
When you combine these approaches in a single study, you create what's
known as a mixed design.

A mixed design occurs when one of your independent variables (IVs) is
treated as a between-subjects factor, while another is treated as a
within-subjects factor. This blend offers a unique approach to examining
how different variables interact and affect the outcome.

In environmental science research, mixed designs are particularly useful
for studying complex interactions between variables that vary both
within and between subjects. For instance, consider a study examining
the impact of a new agricultural technique (IV1) on crop yield (DV).
This technique could be applied to different plots of land
(between-subjects factor), while also measuring the impact across
different seasons (within-subjects factor). Such a design allows
researchers to understand not only the overall effectiveness of the
technique but also how its impact varies seasonally.

The key to successfully navigating mixed designs lies in understanding
how to calculate the appropriate statistical measures. Specifically, the
F-values for each effect in an ANOVA (Analysis of Variance) are
constructed using different error terms, depending on whether the IV is
a within-subjects or between-subjects factor. While it's possible to run
an ANOVA with any combination of between and within-subjects IVs, the
complexity increases with the number of variables and their
categorizations.

As this is an introductory text, we won't delve into the detailed
formulas for constructing ANOVA tables with mixed designs. More advanced
textbooks offer comprehensive discussions on this topic, and many
resources are available online for those interested in deeper
exploration.

\section{More complicated designs}\label{more-complicated-designs}

Up until now we have focused on the simplest case for factorial designs,
the 2x2 design, with two IVs, each with 2 levels. It is worth spending
some time looking at a few more complicated designs and how to interpret
them.

\subsection{3x2 design}\label{x2-design}

In a 3x2 design there are two IVs. IV1 has three levels, and IV2 has two
levels. Typically, there would be one DV. Let's apply this to an
environmental science scenario.First, let's make the design concrete.

Imagine a study examining the impact of different irrigation methods
(IV1: drip irrigation vs.~sprinkler irrigation) on crop yield (DV)
across three types of soil (IV2: sandy, loamy, clayey). The main effects
would be the overall impact of irrigation method and soil type on crop
yield, while the interaction would explore how these effects vary
together.

For instance, drip irrigation might consistently produce higher yields
than sprinkler irrigation, showing a main effect of IV1. Soil type might
also independently affect yield, with loamy soil perhaps leading to the
highest yields, followed by clayey and sandy soils, indicating a main
effect of IV2. An interaction would occur if, for example, the advantage
of drip irrigation over sprinkler irrigation is more pronounced in sandy
soil compared to clayey soil. Note that these examples are hypothetical
to illustrate the concept.

The factorial ANOVA will test:

\begin{itemize}
\tightlist
\item
  whether there are any differences in crop yield among the three levels
  of soil type
\item
  whether there are any differences in crop yield among the two levels
  of irrigation
\item
  whether there is any interaction between irrigation type and soil type
\end{itemize}

You have three null hypotheses:

\begin{itemize}
\tightlist
\item
  There is no difference between the means for each level of soil type:
\end{itemize}

H\textsubscript{0}: \(\mu_{Clay} = \mu_{Loam} = \mu_{Sand}\)\\

\begin{itemize}
\tightlist
\item
  There is no difference between the means for each level of irrigation:
\end{itemize}

H\textsubscript{0}: \(\mu_{Drip} = \mu_{Sprinkler}\)\\

\begin{itemize}
\tightlist
\item
  There is no interaction between the factors.
\end{itemize}

Remember, this is far better than running two separate single factor
ANOVAs that contrast irrigation effects for each level of soil type
because you have more statistical power (higher degrees of freedom) for
the tests of interest, and you get a formal test of the interaction
between factors which is often scientifically interesting.

We might expect data like shown in Figure~\ref{fig-1123design}:

\begin{figure}

\centering{

\includegraphics[width=1\linewidth,height=\textheight,keepaspectratio]{11-MixedANOVA_files/figure-pdf/fig-1123design-1.pdf}

}

\caption{\label{fig-1123design}Example means for a 3x2 factorial design
in environmental science}

\end{figure}%

The figure shows some pretend means in all conditions. Let's talk about
the main effects and interaction.

\begin{enumerate}
\def\labelenumi{\arabic{enumi}.}
\item
  \textbf{Main Effect of Irrigation Method}: The main effect of the
  irrigation method is evident. Drip irrigation (represented by red
  line) generally leads to higher crop yields compared to sprinkler
  irrigation (represented by aqua line).
\item
  \textbf{Main Effect of Soil Type}: The main effect of soil type is
  clearly present. Clayey soils show the highest yield, followed by
  loamy soils, then sandy soils
\item
  \textbf{Interaction Between Irrigation Method and Soil Type}: Is there
  an interaction? Yes, there is. Remember, an interaction occurs when
  the effect of one IV depends on the levels of an another. The
  advantage of drip irrigation over sprinkler irrigation is more
  pronounced in sandy soil compared to clayey soil. So, the size of the
  irrigation effect (drip vs.~sprinkler) changes with the type of soil.
  There is evidence in the means for an interaction. You would have to
  conduct an inferential test on the interaction term to see if these
  differences were likely or unlikely to be due to sampling error.
\end{enumerate}

If there was no interaction and no main effect of soil type, we would
see something like the pattern in Figure~\ref{fig-1123one}.

\begin{figure}

\centering{

\includegraphics[width=1\linewidth,height=\textheight,keepaspectratio]{11-MixedANOVA_files/figure-pdf/fig-1123one-1.pdf}

}

\caption{\label{fig-1123one}Example means for a 3x2 design in
environmental science with only one main effect}

\end{figure}%

What would you say about the interaction if you saw the pattern in
Figure~\ref{fig-1123int}?

\begin{figure}

\centering{

\includegraphics[width=1\linewidth,height=\textheight,keepaspectratio]{11-MixedANOVA_files/figure-pdf/fig-1123int-1.pdf}

}

\caption{\label{fig-1123int}Example means for a 3x2 design in
environmental science showing a different interaction pattern}

\end{figure}%

The correct answer is that there is evidence in the means for an
interaction. Remember, we are measuring the irrigation effect (effect of
drip vs.~sprinkler) three times. The irrigation effect is the same for
clayey and loamy soils, but it is much smaller for sandy soils. The size
of the irrigation effect depends on the levels of the soil type IV, so
here again there is an interaction.

\subsection{2x2x2 designs}\label{x2x2-designs}

Let's take it up a notch and look at a 2x2x2 design. In a 2x2x2 design,
there are three independent variables (IVs), each with two levels. This
design allows for the examination of three main effects, three two-way
interactions, and one three-way interaction.

We'll add another independent variable to our example from before: crop
type (wheat vs.~corn) as our third IV. So overall, in this 2x2x2 design,
we'll consider three independent variables (IVs): irrigation method
(IV1: drip vs.~sprinkler), soil type (IV2: sandy vs.~clayey), and crop
type (IV3: wheat vs.~corn). The dependent variable (DV) is still crop
yield. This design helps us understand not just individual effects but
also how these factors interact in various combinations.

\begin{figure}

\centering{

\includegraphics[width=1\linewidth,height=\textheight,keepaspectratio]{11-MixedANOVA_files/figure-pdf/fig-11222-1.pdf}

}

\caption{\label{fig-11222}Example means from a 2x2x2 design in
environmental science with no three-way interaction.}

\end{figure}%

n Figure~\ref{fig-11222}, we have two panels: one for corn and one for
wheat. You can think of the 2x2x2 as two 2x2 designs, one for each crop
type. The key takeaway? Both wheat and corn show similar patterns,
indicating a 2x2 interaction between irrigation method and soil type. We
observe main effects for irrigation and soil type, but no main effect
for crop type, and importantly, no three-way interaction.

But what exactly is a three-way interaction? It occurs when the pattern
of a 2x2 interaction differs across the levels of the third variable.
Let's visualize this with Figure~\ref{fig-11222int}.

\begin{figure}

\centering{

\includegraphics[width=1\linewidth,height=\textheight,keepaspectratio]{11-MixedANOVA_files/figure-pdf/fig-11222int-1.pdf}

}

\caption{\label{fig-11222int}Example means from a 2x2x2 design in
environmental science with a three-way interaction.}

\end{figure}%

We are looking at a 3-way interaction between irrigation type, crop
type, and soil type. What is going on here?

For corn crop yields, we see that there is a smaller irrigation effect
in clayey soils, but the effect of irrigation gets bigger in sandy
soils. A pattern like this might make sense, sandy soils don't retain
much water so the irrigation method might matter more.

The wheat crop yields show a different pattern. Here, the irrigation
effect is large in clayey soils and smaller in sandy soils. This
difference in patterns between corn and wheat yields indicates a
three-way interaction among irrigation type, soil type, and crop type.

In other words, the 2x2 interaction for the corn is \textbf{different}
from the 2x2 interaction for the wheat. This can be conceptualized as an
interaction between the two interactions, and as a result there is a
three-way interaction, called a 2x2x2 interaction.

A general pattern here. Imagine you had a 2x2x2x2 design. That would
have a 4-way interaction. What would that mean? It would mean that the
pattern of the 2x2x2 interaction changes across the levels of the 4th
IV. If two three-way interactions are different, then there is a
four-way interaction.This becomes very complicated very quickly, another
reminder of why simplicity in design is desirable.

\subsection{Understanding and Interpreting Interactions in Environmental
Science}\label{understanding-and-interpreting-interactions-in-environmental-science}

So, you've got a handle on what interactions are and what they might
look like. But there's still a key question hanging in the air:
\textbf{Why do interactions matter?}

\subsubsection{Interpreting the Results}\label{interpreting-the-results}

Remember our example exploring the impact of different irrigation
methods on crop yield across various soil types? The data in
Figure~\ref{fig-1123design} revealed something interesting: drip
irrigation significantly boosts crop yield in sandy soil, but this
effect diminishes in loamy and clayey soils. This is what we call an
interaction effect: the impact of the irrigation method (drip
vs.~sprinkler) on crop yield varies depending on the soil type.

This interaction is important. It tells us that the effectiveness of an
irrigation method is not uniform across all soil types. It suggests that
environmental factors (like soil type) can influence how well an
intervention (like irrigation method) works.

\subsubsection{Practical Implications in Environmental
Science}\label{practical-implications-in-environmental-science}

Understanding these interactions has real-world implications for
environmental management and policy-making. Consider these examples:

\begin{enumerate}
\def\labelenumi{\arabic{enumi}.}
\item
  \textbf{Resource Allocation}: By understanding how different soil
  types interact with various irrigation methods, farmers can tailor
  their agricultural practices more precisely. For instance, in areas
  with clayey soil, which retains water well, less frequent irrigation
  might be more suitable, reducing water usage and preserving natural
  resources.
\item
  \textbf{Climate Adaptation Strategies}: Understanding these
  interactions could play a role in developing climate adaptation
  strategies. For regions facing increased rainfall variability due to
  climate change, selecting the right combination of soil management and
  irrigation techniques can help in maintaining crop yields despite
  changing weather patterns.
\item
  \textbf{Policy Formulation}: Insights from these interactions can
  guide the creation of more nuanced agricultural policies. For example,
  providing subsidies or incentives for adopting certain irrigation
  methods in specific soil types could optimize crop yield and promote
  sustainability.
\end{enumerate}

Think about it: are there other environmental factors where
understanding interactions could be crucial for effective management and
policy-making?

\bookmarksetup{startatroot}

\chapter{Analysis of Covariance
(ANCOVA)}\label{analysis-of-covariance-ancova}

\section{General Linear Models (GLM)}\label{general-linear-models-glm}

General Linear Models (GLMs) are a class of models that encompass
various types of statistical analyses, including ANOVA, and regression.
They're flexible enough to handle different types of data and
relationships:

In our journey through statistical modeling, we've encountered three
primary parametric models, each suited for different types of data
scenarios:

\begin{itemize}
\tightlist
\item
  \textbf{No Groups and No Relationships (H0):}

  \begin{itemize}
  \tightlist
  \item
    This scenario often emerges when our ANOVA or regression analysis
    yields non-significant results.
  \item
    \textbf{What We Report:} The grand mean and overall variance or
    standard deviation. Alternatively, we can use a confidence interval
    to encapsulate this information.
  \end{itemize}
\item
  \textbf{Two or More Categories with Significantly Different Means
  (t-test, ANOVA):}

  \begin{itemize}
  \tightlist
  \item
    Here, we delve into data where group means are distinct and
    significant.
  \item
    \textbf{What We Report:} Group means. In classical ANOVA, we might
    report a common variance or standard deviation, or opt for
    group-specific measures (potentially through confidence intervals),
    ensuring to note that variances are not significantly different.
    With Welch's t-test, we focus on group-specific variance or standard
    deviation.
  \end{itemize}
\item
  \textbf{Two Continuous Variables with a Significant Linear or
  Monotonic Relationship (Regression):}

  \begin{itemize}
  \tightlist
  \item
    This model applies when there's a significant linear relationship
    between two continuous variables.
  \item
    \textbf{What We Report:} The equation of the regression line, along
    with its confidence limits. We can also select specific x-values and
    provide confidence intervals for the predicted y-values at those
    points.
  \end{itemize}
\end{itemize}

\subsection{Model Statements:}\label{model-statements}

\begin{itemize}
\tightlist
\item
  \textbf{No Groups/Relationships:} \(y \sim N(\mu, \sigma^2)\)
\item
  \textbf{Categories with Different Means:}
  \(y_{i,j} = \mu + \tau_i + \epsilon\) (t-test allows for unequal
  variance: \(y_{i,j} = \mu + \tau_i + \epsilon_i\))
\item
  \textbf{Linear Relationship:} \(y = \beta_0 + \beta_1x + \epsilon\)
\end{itemize}

\subsection{Flexibility and Applicability of
GLMs}\label{flexibility-and-applicability-of-glms}

Ms enable us to model more complex relationships beyond the basic
categorical X with continuous Y (as in t-tests and ANOVA) or continuous
X with continuous Y (as in regression). They can handle diverse data
types and model relationships that aren't strictly linear, making them
crucial for studying environmental systems.

So far, in our exploration of parametric tests, we have primarily
focused on two types of relationships:

\begin{itemize}
\tightlist
\item
  \textbf{A categorical X and a continuous Y:} This includes tests like
  the t-test and ANOVA.
\item
  \textbf{A continuous X and a continuous Y:} This is typically analyzed
  using linear regression.
\end{itemize}

General Linear Models (GLMs) significantly expand our analytically
capabilities. They are not limited to just these basic scenarios but
offer a more versatile toolkit. GLMs are particularly adept at handling:

\begin{itemize}
\tightlist
\item
  \textbf{More Complex Relationships:} They can model scenarios where
  relationships between variables are not strictly linear.
\item
  \textbf{Diverse Data Types:} GLMs are suitable for various data types,
  including count data, which is often encountered in environmental
  studies.
\end{itemize}

\subsection{Understanding ANOVA as a linear
model:}\label{understanding-anova-as-a-linear-model}

ANOVA, or Analysis of Variance, is traditionally viewed as a technique
to compare means across multiple groups. However, at its core, ANOVA is
a linear model. It's a special case where the predictors are
categorical, not continuous. This distinction is important but subtle.
Let's delve into an example that illustrates this concept clearly.

This example should be familiar from lecture. Imagine a researcher is
exploring the effects of metal contamination on the species richness of
sessile marine invertebrates. They're particularly interested in the
impact of copper and the orientation of the substrate on which these
organisms live. To investigate this, they conduct a factorial
experiment, measuring species richness across different levels of copper
enrichment (None, Low, High) and substrate orientation (Vertical,
Horizontal). This setup allows us to not only consider each factor
separately but also examine their potential interaction.

The ANOVA framework provides three null hypotheses to test:

\begin{enumerate}
\def\labelenumi{\arabic{enumi}.}
\item
  There are no differences in species richness across the copper levels.
\item
  There are no differences in species richness between substrate
  orientations.
\item
  There is no interaction effect between copper levels and substrate
  orientation.
\end{enumerate}

These hypotheses can be represented in a linear model as follows:

\(y_{ijk} = \beta_0 + \beta_{Copper_i} + \beta_{Orientation_j} + \beta_{Copper \times Orientation_{ij}} + \varepsilon_{ijk}\)

Here, \(\beta_0\) is the grand mean (intercept), \(\beta_{Copper_i}\)
and \(\beta_{Orientation_j}\) are the main effects of the copper levels
and orientation, respectively, and
\(\beta_{Copper \times Orientation_{ij}}\) represents the interaction
between these factors. The \(\varepsilon_{ijk}\) term captures the
residual variance, or the random deviation of each observation from the
model prediction. The significance of each beta coefficient is tested to
determine if it makes a meaningful contribution to the model. A
non-significant beta suggests that the corresponding factor or
interaction does not have a distinct effect on the outcome, and
therefore, might be omitted from the model for parsimony.

When conducting an ANOVA, we're essentially fitting a linear model with
categorical predictors. These predictors are represented by beta
coefficients in our model, which show the unique contribution of each
factor (like copper levels or substrate orientation) to the outcome
variable (such as species richness). The critical question we ask is
whether each beta coefficient is significant---does it make a meaningful
difference to our model?

In statistical terms, we assess this by examining the F-statistic. The
F-statistic is a ratio that compares the amount of variance explained by
a particular factor to the variance not explained by the model
(within-group variance or error). A higher F-statistic indicates that
the factor explains a significant portion of the variability in the
outcome variable, while a lower F-statistic suggests that the factor
does not contribute much to our understanding of the outcome variable.

\(F = \frac{MS_{Treatment}}{MS_{Error}}\).

In an ANOVA with two factors, we calculate the F-statistic for each
factor and their interaction. We don't need to show the actual
calculation here---what matters is that the F-statistic tells us if the
factor makes a significant contribution. If it does, we keep the beta
coefficient in our model. If not, we may consider omitting it for a
simpler model.

It turns out that ANOVAs are just a type of linear model in which the
predictor variable is categorical. In practice, we can perform ANOVA
using the \texttt{lm()} function in R, treating the categorical
predictors as factors. This allows us to use the familiar beta notation
and interpret ANOVA as a linear regression model with categorical
predictors.

\section{Moving Beyond Regression and ANOVA to
ANCOVA}\label{moving-beyond-regression-and-anova-to-ancova}

As we've explored statistical modeling, we've understood the power of
both regression and ANOVA. Now, let's merge these concepts to broaden
our analytical horizon. ANCOVA, or Analysis of Covariance, allows us to
examine relationships across different categories, enhancing our ability
to compare and understand varying trends within our data.

Consider these questions that may arise in environmental research:

\begin{itemize}
\item
  Are dbh (Diameter at Breast Height) and height related similarly for
  tulip poplars and oaks?
\item
  Are biomass and BTUs (British Thermal Units) related similarly for
  corn stover and Miscanthus?
\item
  Does the exposure to PFAS correlate with the lifetime incidence of
  cancer uniformily across low- and high-income American populations?
\end{itemize}

This specific subsection of general linear models is known as
\textbf{Analysis of Covariance -- ANCOVA}. Each of these questions
challenges us to compare relationships across categories, which is
precisely where ANCOVA shines.

\begin{center}\rule{0.5\linewidth}{0.5pt}\end{center}

\subsection{Understanding ANCOVA}\label{understanding-ancova}

With ANCOVA, if we identify a significant difference between two linear
relationships, our model will represent two distinct lines. The
challenge then becomes integrating these two lines into a single
equation.

\subsubsection{Conceptualizing the
Transformation}\label{conceptualizing-the-transformation}

Consider the following equations:

\begin{itemize}
\tightlist
\item
  \texttt{y\ =\ 2x\ +\ 3}
\item
  \texttt{y\ =\ 3x\ -\ 4}
\end{itemize}

The question we pose is: What modifications are required to transform
the first equation into the second? This involves determining the
adjustments needed in terms of x (the slope) and the constant term.
Understanding this transformation is key to grasping how ANCOVA allows
us to compare different linear relationships within a single model
framework.

\section{Introducing the Indicator or Dummy
Variable}\label{introducing-the-indicator-or-dummy-variable}

In General Linear Models (GLMs), indicator or dummy variables allow us
to include categorical variables in models traditionally designed for
continuous variables.

\subsection{Understanding Indicator
Variables}\label{understanding-indicator-variables}

Indicator variables are used to encode categories. For \texttt{n}
categories, you need \texttt{n-1} indicator variables.

Consider a scenario with 5 species of wombat. We can code these species
using 4 indicator variables:

\begin{longtable}[]{@{}lllll@{}}
\toprule\noalign{}
Ind1 & Ind2 & Ind3 & Ind4 & Species \\
\midrule\noalign{}
\endhead
\bottomrule\noalign{}
\endlastfoot
1 & 0 & 0 & 0 & Species 1 \\
0 & 1 & 0 & 0 & Species 2 \\
0 & 0 & 1 & 0 & Species 3 \\
0 & 0 & 0 & 1 & Species 4 \\
0 & 0 & 0 & 0 & Species 5 \\
\end{longtable}

In the simplest case of two categories, only one indicator variable is
needed:

\begin{itemize}
\tightlist
\item
  0 = Wombat Species 1
\item
  1 = Wombat Species 2
\end{itemize}

\subsubsection{Flipping the Switch: Indicator Variables in
Action}\label{flipping-the-switch-indicator-variables-in-action}

In the context of GLMs, ``turning on'' an indicator variable means
assigning it a value of 1. This action activates certain terms in the
equation that are multiplied by the indicator variable, thereby altering
the model's output.

When an indicator variable (\texttt{xc}) is set to 1, it effectively
activates any terms in the equation that are multiplied by \texttt{xc}.
This can change the slope and/or intercept of the regression line,
depending on how \texttt{xc} is used in the equation.

For example, in the equation \texttt{y\ =\ 2xl\ +\ 3\ +\ 1xcxl\ -\ 7xc}:

\begin{itemize}
\item
  When \texttt{xc} = 0 (turned off), the equation simplifies to
  \texttt{y\ =\ 2xl\ +\ 3}. Here, the terms \texttt{1xcxl} and
  \texttt{-7xc} are deactivated because they are multiplied by
  \texttt{xc}, which is 0.
\item
  When \texttt{xc} = 1 (turned on), the equation becomes
  \texttt{y\ =\ (2xl\ +\ 3)\ +\ (1xl\ -\ 7)}. In this case, the terms
  \texttt{1xcxl} and \texttt{-7xc} are activated, altering the slope and
  intercept of the line.
\end{itemize}

\subsubsection{Visualizing the Effect}\label{visualizing-the-effect}

The R plots below demonstrate the change in the regression line when the
indicator variable is toggled between being active (\texttt{xc\ =\ 1})
and inactive (\texttt{xc\ =\ 0}).

\textbf{When \texttt{xc} = 0:}

\emph{Equation Simplified:} \texttt{y\ =\ 2xl\ +\ 3}

\textbf{When \texttt{xc} = 1:}

\emph{Equation Modified:} \texttt{y\ =\ (2xl\ +\ 3)\ +\ (1xl\ -\ 7)}

The plot on the left shows the regression line when the indicator
variable is inactive. The equation simplifies, reflecting a scenario
where the categorical variable does not influence the outcome.

The plot on the right illustrates the regression line when the indicator
variable is active. The equation now includes additional terms,
showcasing how the presence of the categorical variable changes the
relationship between \texttt{xl} and \texttt{y}.

\includegraphics[width=0.75\linewidth,height=\textheight,keepaspectratio]{12-ANCOVA_files/figure-pdf/unnamed-chunk-3-1.pdf}

Using indicator variables, we can create models in which regression
lines change completely depending on which category we're modeling. If
we need to tweak both the intercept and the slope, then we will need two
new regression parameters -- \(\beta_2\) and \(\beta_3\).

\subsection{Interaction Term in GLMs}\label{interaction-term-in-glms}

Just like in ANOVA, interaction terms in GLMs tell us whether one
variable's effect on the outcome changes when another variable comes
into play.

\begin{itemize}
\tightlist
\item
  \textbf{Simple Definition:} In statistics, an interaction term helps
  us understand if the effect of one factor (like temperature) on an
  outcome (like plant growth) changes when another factor (like
  rainfall) is also considered. It's like asking, ``Does the
  relationship between temperature and plant growth change when we also
  consider how much it rains?''
\end{itemize}

\subsubsection{The Math Behind The
Interactions}\label{the-math-behind-the-interactions}

\begin{itemize}
\tightlist
\item
  \textbf{Model Equation Explained:}
\end{itemize}

Let's break down a typical equation:

\[
y = \beta_0 + \beta_1x_l + \beta_2x_c + \beta_3x_lx_c + \epsilon
\]

Here, \(x_lx_c\) is the interaction term, and \(\beta _3\) is its
coefficient.

\begin{itemize}
\tightlist
\item
  \(y\) is what we're trying to predict (like plant growth).
\item
  \(\beta_0\) is the starting point of our prediction when all other
  factors are zero.
\item
  \(\beta_1x_l\) shows how our prediction changes with changes in a
  continuous variable (like temperature).
\item
  \(\beta_2x_c\) shows the change with a categorical variable (like type
  of plant).
\item
  \(\beta_3x_lx_c\) is the key player here. It shows how the effect of
  our continuous variable (temperature) changes across different
  categories (types of plants).
\item
  \(\epsilon\) is the error term, accounting for variations we can't
  explain with our model.
\item
  \(\beta_1x_l\) and \(\beta_2x_c\) represent the main effects of the
  continuous and categorical variables, respectively.
\end{itemize}

\subsubsection{When is the Interaction Term
Significant?}\label{when-is-the-interaction-term-significant}

\begin{itemize}
\tightlist
\item
  \textbf{Significance of} \(\beta_3x_lx_c\) :
\end{itemize}

If \(\beta_3x_lx_c\) (our interaction term) is significant, it means the
relationship between our continuous variable (like temperature) and our
outcome (plant growth) is different for different categories (like types
of plants). If \(\beta_3x_lx_c\) is not significant, it suggests that
the effect of our continuous variable is consistent across categories,
and we might not need this term in our model.

\subsubsection{What are the options for our
model?}\label{what-are-the-options-for-our-model}

\begin{enumerate}
\def\labelenumi{\arabic{enumi}.}
\tightlist
\item
  \textbf{Full Model with Interaction
  (}\(\beta_0, \beta_1, \beta_2, \beta_3\)):

  \begin{itemize}
  \tightlist
  \item
    When both the categorical (\(x_c\)) and continuous (\(x_l\))
    variables are significant, and there is a significant interaction
    (\(\beta_3\)), the model unfolds into two distinct linear equations
    for each category of\(x_c\). This indicates that the relationship
    between the continuous variable and the outcome differs depending on
    the category of the categorical variable.
  \item
    Equations:

    \begin{itemize}
    \tightlist
    \item
      For \(x_c = 0\): \(y = \beta_0 + \beta_1x_l\)
    \item
      For \(x_c = 1\):
      \(y = (\beta_0 + \beta_2) + (\beta_1 + \beta_3)x_l\)
    \end{itemize}
  \end{itemize}
\item
  **Model without Interaction ((\beta\_0, \beta\_1, \beta\_2)):**

  \begin{itemize}
  \tightlist
  \item
    If both the categorical and continuous variables are significant,
    but there is no interaction, the model simplifies to parallel lines
    for each category of \$x\_c \$,, indicating that the slope (effect
    of \(x_l\), is consistent across categories.
  \item
    Equations:

    \begin{itemize}
    \tightlist
    \item
      For \(x_c = 0\): \(y = \beta_0 + \beta_1x_l\)
    \item
      For \(x_c = 1\): \(y = (\beta_0 + \beta_2) + \beta_1x_l\)
    \end{itemize}
  \end{itemize}
\item
  \textbf{Simple Linear Regression (}\(\beta_0, \beta_1\)):

  \begin{itemize}
  \tightlist
  \item
    If only the continuous variable (\(x_l\)) is significant, the model
    reduces to a simple linear regression, indicating a linear
    relationship between\(x_l\) and \(y\), regardless of the category
    of\(x_c\).
  \item
    Equation: \(y = \beta_0 + \beta_1x_l\)
  \end{itemize}
\item
  \textbf{Two Different Means (T-test Equivalent)
  (}\(\beta_0, \beta_2\)):

  \begin{itemize}
  \tightlist
  \item
    If the continuous variable is not significant but the categorical
    variable is, the model effectively becomes a comparison of means
    between two groups (similar to a t-test).
  \item
    Equations:

    \begin{itemize}
    \tightlist
    \item
      For \(x_c = 0\): \(y = \beta_0\)
    \item
      For \(x_c = 1\): \(y = \beta_0 + \beta_2\)
    \end{itemize}
  \end{itemize}
\item
  \textbf{Single Mean (One Sample Data) (}\(\beta_0\)):

  \begin{itemize}
  \tightlist
  \item
    If neither variable is significant, the model reduces to a single
    mean, indicating no effect of either the continuous or categorical
    variable.
  \item
    Equation: \(y = \beta_0\)
  \end{itemize}
\end{enumerate}

\subsection{Terminology in GLMs}\label{terminology-in-glms}

Understanding the terminology used in General Linear Models (GLMs) is
important for grasping the concepts and effectively communicating your
findings. Here are some key terms:

\begin{itemize}
\item
  \textbf{Fixed Factor:} This refers to a categorical variable with
  specific, predefined categories. For example, if you're studying the
  effect of different seasons (spring, summer, autumn, winter) on plant
  growth, `season' is a fixed factor because it represents specific,
  distinct categories of interest.
\item
  \textbf{Random Factor:} This is a categorical variable where the
  categories represent a random sample from a larger population. For
  instance, if you're examining the output quality from different
  machines in a factory, and these machines are randomly selected from a
  larger set, then `machine' is a random factor.
\item
  \textbf{Covariate:} This is your continuous variable that varies along
  with the dependent variable (Y). In environmental studies, this could
  be something like temperature, rainfall, or pollution levels, which
  you suspect might influence your outcome of interest (like species
  distribution or plant growth).
\end{itemize}

\begin{center}\rule{0.5\linewidth}{0.5pt}\end{center}

\section{So, What Exactly is ANCOVA?}\label{so-what-exactly-is-ancova}

Analysis of Covariance (ANCOVA) is a statistical method that combines
the principles of ANOVA and regression. It's designed to compare
categorical independent variables---such as different treatments or
groups---while controlling for the influence of continuous variables.
These continuous variables, known as covariates, are typically not the
primary focus but could affect the outcome.

For ANCOVA, you need:

\begin{enumerate}
\def\labelenumi{\arabic{enumi})}
\tightlist
\item
  A continuous dependent variable (the main effect you're studying)
\item
  At least one continuous independent variable (covariate)
\item
  At least one categorical independent variable (which can be either a
  fixed or random factor)
\end{enumerate}

\subsection{The Null Hypothesis in
ANCOVA}\label{the-null-hypothesis-in-ancova}

\textbf{Demystifying the Concept}: The null hypothesis in ANCOVA
suggests that once we adjust for the covariates, the categorical
independent variables do not significantly affect the dependent
variable. Essentially, we're testing whether the apparent differences
between groups are genuine or just statistical noise.

\textbf{Mathematically Speaking}: In a simple ANCOVA model, the null
hypothesis posits that the adjusted means for each level of the
categorical variable are equivalent, once we account for the covariate:

\(H_0: \mu_1 = \mu_2 = \ldots = \mu_t\)

\(H_A: \mu_i \neq \mu_j^* \text{ for some } i \neq j\)

This translates to: ``After controlling for the covariate, the different
levels of the categorical independent variable do not lead to different
outcomes.''

\subsection{Testing the Null
Hypothesis}\label{testing-the-null-hypothesis}

\textbf{Conducting the Test}: When we perform ANCOVA, we examine the
null hypothesis through an omnibus F-test. If we find sufficient
evidence to reject the null hypothesis, we infer that significant
differences exist among the adjusted means of the dependent variable
across the categorical groups, even after considering the covariate.

\textbf{Interpreting the Findings}: Rejecting the null hypothesis
indicates that the categorical independent variable has a significant
impact on the dependent variable, aside from the covariate's effect.
It's crucial to remember, though, that statistical significance doesn't
automatically translate to practical significance. We must always
consider the results within the specific context and aims of the study.

\section{Illustrating ANCOVA with an
Example}\label{illustrating-ancova-with-an-example}

\subsection{Setting the Stage}\label{setting-the-stage}

Imagine we're environmental scientists studying the impact of air
pollution---a continuous variable---on bird species distribution, our
dependent variable. We want to see if this relationship differs between
urban and rural areas---our categorical variable. For this purpose,
we've created a synthetic dataset to model the scenario.

\subsection{Crafting the Model}\label{crafting-the-model}

With our data in hand, we'd construct an ANCOVA model to discern whether
location type influences bird distribution, beyond what can be explained
by pollution levels alone. This would involve coding our categorical
variable (urban vs.~rural) into a dummy variable and including air
pollution as a covariate in the model. I've done this for you already by
creating the synthetic dataset.

\begin{itemize}
\tightlist
\item
  \textbf{Dataset Overview:} Our dataset comprises 200 observations,
  capturing bird species density in various urban and rural locations,
  along with corresponding levels of air pollution.
\end{itemize}

\subsection{Why We're Running the
ANCOVA}\label{why-were-running-the-ancova}

\begin{itemize}
\item
  \textbf{Research Question:} Does the effect of air pollution on bird
  species density differ between urban and rural areas?
\item
  \textbf{Expectation:} We predict that air pollution will have a more
  pronounced negative effect on bird species density in urban areas than
  in rural areas.
\end{itemize}

\subsection{Null and Alternative Hypotheses in our example
ANCOVA}\label{null-and-alternative-hypotheses-in-our-example-ancova}

\begin{enumerate}
\def\labelenumi{\arabic{enumi}.}
\tightlist
\item
  \textbf{Omnibus Null Hypothesis} (\(H_0\)): The null hypothesis posits
  that once we adjust for air pollution, the mean bird species density
  does not differ between urban and rural areas.
\end{enumerate}

\(H_0: \mu^*_{\text{Urban}} = \mu^*_{\text{Rural}}\)

\(H_A: \mu^*_{\text{Urban}} \neq \mu^*_{\text{Rural}}\)

Where \(\mu^*_{\text{Urban}}\) and \(\mu^*_{\text{Rural}}\) represent
the adjusted means of bird species density for urban and rural areas,
respectively..

\begin{enumerate}
\def\labelenumi{\arabic{enumi}.}
\setcounter{enumi}{1}
\item
  \textbf{Null Hypothesis for Slope} (\(H_{01}\)): This hypothesis
  examines if the effect of air pollution on bird species density is
  consistent between urban and rural areas.

  \begin{itemize}
  \tightlist
  \item
    \(H_{01}\): There is no interaction between air pollution and area
    type; the slopes are parallel.
  \item
    \(H_{A1}\): There is an interaction; the effect of air pollution on
    bird species density differs between urban and rural areas.
  \end{itemize}
\end{enumerate}

\(H_{01}: \beta_{\text{interaction}} = 0\)

\(H_{A1}: \beta_{\text{interaction}} \neq 0\)

\begin{enumerate}
\def\labelenumi{\arabic{enumi}.}
\setcounter{enumi}{2}
\item
  \textbf{Null Hypothesis for Intercept} (\(H_{02}\)): If we find no
  significant interaction, we then consider the intercepts. This
  hypothesis tests whether the baseline bird species density, at zero
  air pollution, differs between urban and rural areas.

  \begin{itemize}
  \item
    \(H_{02}\): The intercepts are the same, indicating no difference in
    bird species density between urban and rural areas at zero air
    pollution.
  \item
    \(H_{A2}\): The intercepts differ, suggesting an inherent difference
    in bird species density between urban and rural areas, independent
    of air pollution.
  \end{itemize}
\end{enumerate}

\(H_{02}: \beta_{0\text{,Urban}} = \beta_{0\text{,Rural}}\)

\(H_{A2}: \beta_{0\text{,Urban}} \neq \beta_{0\text{,Rural}}\)

Rejecting \(H_{02}\) would imply that there is an inherent difference in
bird species density due to area type, even before considering the air
pollution levels.

These hypotheses guide our ANCOVA model and analysis, aiming to isolate
the effects of urbanization from the influence of air pollution on avian
populations.

\subsection{Analysis}\label{analysis}

\subsubsection{Explore the data}\label{explore-the-data}

\begin{verbatim}
#>   AreaType AirPollution BirdDensity
#> 1    Urban        148.9          25
#> 2    Urban         39.7          39
#> 3    Urban         37.6          24
#> 4    Rural         58.8          30
#> 5    Urban         54.9          18
#> 6    Rural         38.2          35
\end{verbatim}

Let's create a summary of our current dataset

\begin{Shaded}
\begin{Highlighting}[]
\FunctionTok{summary}\NormalTok{(data)}
\CommentTok{\#\textgreater{}    AreaType          AirPollution     BirdDensity   }
\CommentTok{\#\textgreater{}  Length:200         Min.   : 25.80   Min.   : 6.00  }
\CommentTok{\#\textgreater{}  Class :character   1st Qu.: 54.45   1st Qu.:18.00  }
\CommentTok{\#\textgreater{}  Mode  :character   Median : 83.60   Median :25.00  }
\CommentTok{\#\textgreater{}                     Mean   : 86.14   Mean   :24.83  }
\CommentTok{\#\textgreater{}                     3rd Qu.:117.72   3rd Qu.:32.00  }
\CommentTok{\#\textgreater{}                     Max.   :149.90   Max.   :43.00}
\end{Highlighting}
\end{Shaded}

This will give us a quick statistical summary of our variables.

Now, let's visualize the distribution of bird density in different
areas.

\begin{Shaded}
\begin{Highlighting}[]

\CommentTok{\# Boxplot for bird density by area type}
\FunctionTok{boxplot}\NormalTok{(BirdDensity }\SpecialCharTok{\textasciitilde{}}\NormalTok{ AreaType, }\AttributeTok{data =}\NormalTok{ data, }
        \AttributeTok{main =} \StringTok{"Bird Density by Area Type"}\NormalTok{, }
        \AttributeTok{xlab =} \StringTok{"Area Type"}\NormalTok{, }\AttributeTok{ylab =} \StringTok{"Bird Density"}\NormalTok{, }
        \AttributeTok{col =} \StringTok{"aquamarine"}\NormalTok{, }\AttributeTok{border =} \StringTok{"black"}\NormalTok{)}
\end{Highlighting}
\end{Shaded}

\includegraphics[width=0.75\linewidth,height=\textheight,keepaspectratio]{12-ANCOVA_files/figure-pdf/unnamed-chunk-6-1.pdf}

And the same for air pollution:

\begin{Shaded}
\begin{Highlighting}[]
\FunctionTok{boxplot}\NormalTok{(AirPollution }\SpecialCharTok{\textasciitilde{}}\NormalTok{area\_type, }\AttributeTok{data=}\NormalTok{data, }\AttributeTok{main=}\StringTok{"Air Pollution by Area"}\NormalTok{, }\AttributeTok{xlab=}\StringTok{"Area Type"}\NormalTok{, }\AttributeTok{ylab=} \StringTok{"Air Pollution"}\NormalTok{, }\AttributeTok{col=}\StringTok{"aquamarine"}\NormalTok{, }\AttributeTok{border=}\StringTok{"black"}\NormalTok{)}
\end{Highlighting}
\end{Shaded}

\includegraphics[width=0.75\linewidth,height=\textheight,keepaspectratio]{12-ANCOVA_files/figure-pdf/unnamed-chunk-7-1.pdf}

\subsection{Results of the ANCOVA
Model}\label{results-of-the-ancova-model}

The anova results

\begin{Shaded}
\begin{Highlighting}[]
\NormalTok{ancova }\OtherTok{\textless{}{-}} \FunctionTok{aov}\NormalTok{(BirdDensity }\SpecialCharTok{\textasciitilde{}}\NormalTok{ AirPollution}\SpecialCharTok{*}\NormalTok{AreaType, }\AttributeTok{data=}\NormalTok{data)}
\FunctionTok{summary}\NormalTok{(ancova)}
\CommentTok{\#\textgreater{}                        Df Sum Sq Mean Sq F value  Pr(\textgreater{}F)    }
\CommentTok{\#\textgreater{} AirPollution            1      6     5.9   0.086  0.7690    }
\CommentTok{\#\textgreater{} AreaType                1   1688  1688.0  24.788 1.4e{-}06 ***}
\CommentTok{\#\textgreater{} AirPollution:AreaType   1    291   291.3   4.277  0.0399 *  }
\CommentTok{\#\textgreater{} Residuals             196  13347    68.1                    }
\CommentTok{\#\textgreater{} {-}{-}{-}}
\CommentTok{\#\textgreater{} Signif. codes:  0 \textquotesingle{}***\textquotesingle{} 0.001 \textquotesingle{}**\textquotesingle{} 0.01 \textquotesingle{}*\textquotesingle{} 0.05 \textquotesingle{}.\textquotesingle{} 0.1 \textquotesingle{} \textquotesingle{} 1}
\end{Highlighting}
\end{Shaded}

The linear model results

\begin{Shaded}
\begin{Highlighting}[]
\NormalTok{ancova\_lm }\OtherTok{\textless{}{-}} \FunctionTok{lm}\NormalTok{(BirdDensity }\SpecialCharTok{\textasciitilde{}}\NormalTok{ AirPollution}\SpecialCharTok{*}\NormalTok{AreaType, }\AttributeTok{data=}\NormalTok{data)}
\FunctionTok{summary}\NormalTok{(ancova\_lm)}
\CommentTok{\#\textgreater{} }
\CommentTok{\#\textgreater{} Call:}
\CommentTok{\#\textgreater{} lm(formula = BirdDensity \textasciitilde{} AirPollution * AreaType, data = data)}
\CommentTok{\#\textgreater{} }
\CommentTok{\#\textgreater{} Residuals:}
\CommentTok{\#\textgreater{}      Min       1Q   Median       3Q      Max }
\CommentTok{\#\textgreater{} {-}17.9260  {-}6.2694   0.4038   5.9023  19.2857 }
\CommentTok{\#\textgreater{} }
\CommentTok{\#\textgreater{} Coefficients:}
\CommentTok{\#\textgreater{}                             Estimate Std. Error t value Pr(\textgreater{}|t|)    }
\CommentTok{\#\textgreater{} (Intercept)                 31.54648    2.28553  13.803  \textless{} 2e{-}16 ***}
\CommentTok{\#\textgreater{} AirPollution                {-}0.04333    0.02476  {-}1.750 0.081641 .  }
\CommentTok{\#\textgreater{} AreaTypeUrban              {-}11.57676    3.02164  {-}3.831 0.000172 ***}
\CommentTok{\#\textgreater{} AirPollution:AreaTypeUrban   0.06695    0.03237   2.068 0.039943 *  }
\CommentTok{\#\textgreater{} {-}{-}{-}}
\CommentTok{\#\textgreater{} Signif. codes:  0 \textquotesingle{}***\textquotesingle{} 0.001 \textquotesingle{}**\textquotesingle{} 0.01 \textquotesingle{}*\textquotesingle{} 0.05 \textquotesingle{}.\textquotesingle{} 0.1 \textquotesingle{} \textquotesingle{} 1}
\CommentTok{\#\textgreater{} }
\CommentTok{\#\textgreater{} Residual standard error: 8.252 on 196 degrees of freedom}
\CommentTok{\#\textgreater{} Multiple R{-}squared:  0.1295, Adjusted R{-}squared:  0.1162 }
\CommentTok{\#\textgreater{} F{-}statistic: 9.717 on 3 and 196 DF,  p{-}value: 5.23e{-}06}
\end{Highlighting}
\end{Shaded}

\subsubsection{Main Effects:}\label{main-effects-1}

\begin{itemize}
\item
  \textbf{Air Pollution}: The ANCOVA model does not find a significant
  main effect of air pollution on bird density (p = 0.7690 from ANOVA),
  indicating no consistent impact across the areas. However, the linear
  model's marginal p-value (p = 0.081641) suggests a trend that might
  have been significant with a larger sample size or reduced
  variability.
\item
  \textbf{Area Type}: There is a significant main effect of AreaType on
  bird density (p \textless{} 0.001), indicating that there are
  considerable differences in bird density between rural and urban
  areas, with urban areas showing lower bird density.
\end{itemize}

\subsubsection{Interpreting the Beta
Coefficients:}\label{interpreting-the-beta-coefficients}

\begin{itemize}
\item
  \textbf{Intercept}: Reflecting the expected bird density in rural
  areas at zero air pollution, the intercept is quite high, indicating a
  relatively healthy bird population in the absence of pollution.
\item
  \textbf{Area Type}: The beta coefficient for our area type factor is
  significantly different from zero (p \textless{} 0.001), suggesting a
  distinct difference in bird density between urban and rural areas.
  However, this effect is captured more precisely in the interaction
  term.
\item
  \textbf{Interaction Term}: The significant beta for the interaction
  term confirms our hypothesis that air pollution's effect on bird
  density is not uniform across urban and rural areas.
\end{itemize}

\subsubsection{Visual Representation:}\label{visual-representation}

The scatter plot with regression lines will demonstrate these
relationships visually, showing the differing trends for urban and rural
areas suggested by the significant interaction term.

\includegraphics[width=0.75\linewidth,height=\textheight,keepaspectratio]{12-ANCOVA_files/figure-pdf/unnamed-chunk-10-1.pdf}

\subsubsection{Interpretation:}\label{interpretation}

The significant interaction term indicates that the simple main effect
of air pollution is not adequate to describe its impact on bird density.
In urban areas, there seems to be a slightly positive or less negative
relationship compared to rural areas, which could reflect a range of
urban-specific factors affecting bird populations differently than in
rural areas.

\subsection{Contextualizing the
Findings:}\label{contextualizing-the-findings}

The lack of a significant main effect for air pollution in the presence
of a significant interaction suggests complex underlying ecological
dynamics. These findings call for nuanced environmental management
strategies that address the specific challenges and conditions of urban
and rural habitats to support bird conservation effectively. By
considering the interaction term's significance, we gain a more accurate
understanding of the ecological effects of air pollution, which is
essential for developing targeted conservation policies.

\section{Chapter Summary}\label{chapter-summary}

In this chapter, we explored the fundamental principles and applications
of Analysis of Covariance (ANCOVA), a powerful statistical tool that
extends beyond the capabilities of ANOVA by incorporating covariates. We
began by understanding the conceptual framework of ANOVA as a linear
model, setting the stage for the more complex ANCOVA analysis. Through
our discussions, we emphasized the importance of the null and
alternative hypotheses in ANCOVA, using practical examples like the
study of bird species density in different environments to illustrate
these concepts. The chapter highlighted how ANCOVA adjusts for the
effects of additional variables, allowing us to more accurately isolate
and understand the impact of our factors of interest. By integrating
real-world scenarios and focusing on clear, practical applications, we
aimed to demystify the process of hypothesis testing in ANCOVA, making
it accessible and relevant to environmental science.

\bookmarksetup{startatroot}

\chapter{Thinking about answering questions with
data}\label{thinking-about-answering-questions-with-data}

You might be happy that this is the last chapter (so far) of this
textbook. At this point we are in the last weeks of our introductory
statistics course. It's called ``introductory'' for a reason. There's
just too much out there to cover in one short semester. In this chapter
we acknowledge some of the things we haven't yet covered, and treat them
as things that you should think about. If there is one take home message
that we want to get across to you, it's that when you ask questions with
data, you should be able to \textbf{justify} how you answer those
questions.

\section{Effect-size and power}\label{effect-size-and-power}

If you already know something about statistics while you were reading
this book, you might have noticed that we neglected to discuss the topic
of effect-size, and we barely talked about statistical power. We will
talk a little bit about these things here.

First, it is worth pointing out that over the years, at least in
Psychology, many societies and journals have made recommendations about
how researchers should report their statistical analyses. Among the
recommendations is that measures of ``effect size'' should be reported.
Similarly, many journals now require that researchers report an ``a
priori'' power-analysis (the recommendation is this should be done
before the data is collected). Because these recommendations are so
prevalent, it is worth discussing what these ideas refer to. At the same
time, the meaning of effect-size and power somewhat depend on your
``philosophical'' bent, and these two ideas can become completely
meaningless depending on how you think of statistics. For these
complicating reasons we have suspended our discussion of the topic until
now.

The question or practice of using measures of effect size and conducting
power-analyses are also good examples of the more general need to think
about about what you are doing. If you are going to report effect size,
and conduct power analyses, these activities should not be done blindly
because someone else recommends that you do them, these activities and
other suitable ones should be done as a part of justifying what you are
doing. It is a part of thinking about how to make your data answer
questions for you.

\subsection{Chance vs.~real effects}\label{chance-vs.-real-effects}

Let's rehash something we've said over and over again. First,
researchers are interested in whether their manipulation causes a change
in their measurement. If it does, they can become confident that they
have uncovered a causal force (the manipulation). However, we know that
differences in the measure between experimental conditions can arise by
chance alone, just by sampling error. In fact, we can create pictures
that show us the window of chance for a given statistic, these tells us
roughly the range and likelihoods of getting various differences just by
chance. With these windows in hand, we can then determine whether the
differences we found in some data that we collected were likely or
unlikely to be due to chance. We also learned that sample-size plays a
big role in the shape of the chance window. Small samples give chance a
large opportunity make big differences. Large samples give chance a
small opportunity to make big differences. The general lesson up to this
point has been, design an experiment with a large enough sample to
detect the effect of interest. If your design isn't well formed, you
could easily be measuring noise, and your differences could be caused by
sampling error. Generally speaking, this is still a very good lesson:
better designs produce better data; and you can't fix a broken design
with statistics.

There is clearly another thing that can determine whether or not your
differences are due to chance. That is the effect itself. If the
manipulation does cause a change, then there is an effect, and that
effect is a real one. Effects refer to differences in the measurement
between experimental conditions. The thing about effects is that they
can be big or small, they have a size.

For example, you can think of a manipulation in terms of the size of its
hammer. A strong manipulation is like a jack-hammer: it is loud, it
produces a big effect, it creates huge differences. A medium
manipulation is like regular hammer: it works, you can hear it, it
drives a nail into wood, but it doesn't destroy concrete like a
jack-hammer, it produces a reliable effect. A small manipulation is like
tapping something with a pencil: it does something, you can barely hear
it, and only in a quiet room, it doesn't do a good job of driving a nail
into wood, and it does nothing to concrete, it produces tiny, unreliable
effects. Finally, a really small effect would be hammering something
with a feather, it leaves almost no mark and does nothing that is
obviously perceptiple to nails or pavement. The lesson is, if you want
to break up concrete, use a jack-hammer; or, if you want to measure your
effect, make your manipulation stronger (like a jack-hammer) so it
produces a bigger difference.

\subsection{Effect size: concrete vs.~abstract
notions}\label{effect-size-concrete-vs.-abstract-notions-1}

Generally speaking, the big concept of effect size, is simply how big
the differences are, that's it. However, the biggness or smallness of
effects quickly becomes a little bit complicated. On the one hand, the
raw difference in the means can be very meaningful. Let's saw we are
measuring performance on a final exam, and we are testing whether or not
a miracle drug can make you do better on the test. Let's say taking the
drug makes you do 5\% better on the test, compared to not taking the
drug. You know what 5\% means, that's basically a whole letter grade.
Pretty good. An effect-size of 25\% would be even better right! Lot's of
measures have a concrete quality to them, and we often want to the size
of the effect expressed in terms of the original measure.

Let's talk about concrete measures some more. How about learning a
musical instrument. Let's say it takes 10,000 hours to become an expert
piano, violin, or guitar player. And, let's say you found something
online that says that using their method, you will learn the instrument
in less time than normal. That is a claim about the effect size of their
method. You would want to know how big the effect is right? For example,
the effect-size could be 10 hours. That would mean it would take you
9,980 hours to become an expert (that's a whole 10 hours less). If I
knew the effect-size was so tiny, I wouldn't bother with their new
method. But, if the effect size was say 1,000 hours, that's a pretty big
deal, that's 10\% less (still doesn't seem like much, but saving 1,000
hours seems like a lot).

Just as often as we have concrete measures that are readily
interpretable, Psychology often produces measures that are extremely
difficult to interpret. For example, questionnaire measures often have
no concrete meaning, and only an abstract statistical meaning. If you
wanted to know whether a manipulation caused people to more or less
happy, and you used to questionnaire to measure happiness, you might
find that people were 50 happy in condition 1, and 60 happy in condition
2, that's a difference of 10 happy units. But how much is 10? Is that a
big or small difference? It's not immediately obvious. What is the
solution here? A common solution is to provide a standardized measure of
the difference, like a z-score. For example, if a difference of 10
reflected a shift of one standard deviation that would be useful to
know, and that would be a sizeable shift. If the difference was only a
.1 shift in terms of standard deviation, then the difference of 10
wouldn't be very large. We elaborate on this idea next in describing
cohen's d.

\subsection{Cohen's d}\label{cohens-d-1}

Let's look a few distributions to firm up some ideas about effect-size.
Figure~\ref{fig-13effectdists} has four panels. The first panel (0)
represents the null distribution of no differences. This is the idea
that your manipulation (A vs.~B) doesn't do anything at all, as a result
when you measure scores in conditions A and B, you are effectively
sampling scores from the very same overall distribution. The panel shows
the distribution as green for condition B, but the red one for condition
A is identical and drawn underneath (it's invisible). There is 0
difference between these distributions, so it represent a null effect.

\begin{figure}

\centering{

\includegraphics[width=1\linewidth,height=\textheight,keepaspectratio]{13-Thinking_files/figure-pdf/fig-13effectdists-1.pdf}

}

\caption{\label{fig-13effectdists}Each panel shows hypothetical
distributions for two conditions. As the effect-size increases, the
difference between the distributions become larger.}

\end{figure}%

The remaining panels are hypothetical examples of what a true effect
could look like, when your manipulation actually causes a difference.
For example, if condition A is a control group, and condition B is a
treatment group, we are looking at three cases where the treatment
manipulation causes a positive shift in the mean of distribution. We are
using normal curves with mean =0 and sd =1 for this demonstration, so a
shift of .5 is a shift of half of a standard deviation. A shift of 1 is
a shift of 1 standard deviation, and a shift of 2 is a shift of 2
standard deviations. We could draw many more examples showing even
bigger shifts, or shifts that go in the other direction.

Let's look at another example, but this time we'll use some concrete
measurements. Let's say we are looking at final exam performance, so our
numbers are grade percentages. Let's also say that we know the mean on
the test is 65\%, with a standard deviation of 5\%. Group A could be a
control that just takes the test, Group B could receive some
``educational'' manipulation designed to improve the test score. These
graphs then show us some hypotheses about what the manipulation may or
may not be doing.

\begin{figure}

\centering{

\includegraphics[width=1\linewidth,height=\textheight,keepaspectratio]{13-Thinking_files/figure-pdf/fig-13effectdistsB-1.pdf}

}

\caption{\label{fig-13effectdistsB}Each panel shows hypothetical
distributions for two conditions. As the effect-size increases, the
difference between the distributions become larger.}

\end{figure}%

The first panel shows that both condition A and B will sample test
scores from the same distribution (mean =65, with 0 effect). The other
panels show shifted mean for condition B (the treatment that is supposed
to increase test performance). So, the treatment could increase the test
performance by 2.5\% (mean 67.5, .5 sd shift), or by 5\% (mean 70, 1 sd
shift), or by 10\% (mean 75\%, 2 sd shift), or by any other amount. In
terms of our previous metaphor, a shift of 2 standard deviations is more
like jack-hammer in terms of size, and a shift of .5 standard deviations
is more like using a pencil. The thing about research, is we often have
no clue about whether our manipulation will produce a big or small
effect, that's why we are conducting the research.

You might have noticed that the letter \(d\) appears in the above
figure. Why is that? Jacob Cohen (\textbf{cohen1988?}) used the letter
\(d\) in defining the effect-size for this situation, and now everyone
calls it Cohen's \(d\). The formula for Cohen's \(d\) is:

\(d = \frac{\text{mean for condition 1} - \text{mean for condition 2}}{\text{population standard deviation}}\)

If you notice, this is just a kind of z-score. It is a way to
standardize the mean difference in terms of the population standard
deviation.

It is also worth noting again that this measure of effect-size is
entirely hypothetical for most purposes. In general, researchers do not
know the population standard deviation, they can only guess at it, or
estimate it from the sample. The same goes for means, in the formula
these are hypothetical mean differences in two population distributions.
In practice, researchers do not know these values, they guess at them
from their samples.

Before discussing why the concept of effect-size can be useful, we note
that Cohen's \(d\) is useful for understanding abstract measures. For
example, when you don't know what a difference of 10 or 20 means as a
raw score, you can standardize the difference by the sample standard
deviation, then you know roughly how big the effect is in terms of
standard units. If you thought a 20 was big, but it turned out to be
only 1/10th of a standard deviation, then you would know the effect is
actually quite small with respect to the overall variability in the
data.

\section{Power}\label{power-1}

When there is a true effect out there to measure, you want to make sure
your design is sensitive enough to detect the effect, otherwise what's
the point. We've already talked about the idea that an effect can have
different sizes. The next idea is that your design can be more less
sensitive in its ability to reliabily measure the effect. We have
discussed this general idea many times already in the textbook, for
example we know that we will be more likely to detect ``significant''
effects (when there are real differences) when we increase our
sample-size. Here, we will talk about the idea of design sensitivity in
terms of the concept of power. Interestingly, the concept of power is a
somewhat limited concept, in that it only exists as a concept within
some philosophies of statistics.

\subsection{A digresssion about hypothesis
testing}\label{a-digresssion-about-hypothesis-testing-1}

In particular, the concept of power falls out of the Neyman-Pearson
concept of null vs.~alternative hypothesis testing. Up to this point, we
have largely avoided this terminology. This is perhaps a disservice in
that the Neyman-Pearson ideas are by now the most common and widespread,
and in the opinion of some of us, they are also the most widely
misunderstood and abused idea, which is why we have avoided these ideas
until now.

What we have been mainly doing is talking about hypothesis testing from
the Fisherian (Sir Ronald Fisher, the ANOVA guy) perspective. This is a
basic perspective that we think can't be easily ignored. It is also
quite limited. The basic idea is this:

\begin{enumerate}
\def\labelenumi{\arabic{enumi}.}
\tightlist
\item
  We know that chance can cause some differences when we measure
  something between experimental conditions.
\item
  We want to rule out the possibility that the difference that we
  observed can not be due to chance
\item
  We construct large N designs that permit us to do this when a real
  effect is observed, such that we can confidently say that big
  differences that we find are so big (well outside the chance window)
  that it is highly implausible that chance alone could have produced.
\item
  The final conclusion is that chance was extremely unlikely to have
  produced the differences. We then infer that something else, like the
  manipulation, must have caused the difference.
\item
  We don't say anything else about the something else.
\item
  We either reject the null distribution as an explanation (that chance
  couldn't have done it), or retain the null (admit that chance could
  have done it, and if it did we couldn't tell the difference between
  what we found and what chance could do)
\end{enumerate}

Neyman and Pearson introduced one more idea to this mix, the idea of an
alternative hypothesis. The alternative hypothesis is the idea that if
there is a true effect, then the data sampled into each condition of the
experiment must have come from two different distributions. Remember,
when there is no effect we assume all of the data cam from the same
distribution (which by definition can't produce true differences in the
long run, because all of the numbers are coming from the same
distribution). The graphs of effect-sizes from before show examples of
these alternative distributions, with samples for condition A coming
from one distribution, and samples from condition B coming from a
shifted distribution with a different mean.

So, under the Neyman-Pearson tradition, when a researcher find a
signifcant effect they do more than one things. First, they reject the
null-hypothesis of no differences, and they accept the alternative
hypothesis that there was differences. This seems like a sensible thing
to do. And, because the researcher is actually interested in the
properties of the real effect, they might be interested in learning more
about the actual alternative hypothesis, that is they might want to know
if their data come from two different distributions that were separated
by some amount\ldots in other words, they would want to know the size of
the effect that they were measuring.

\subsection{Back to power}\label{back-to-power-1}

We have now discussed enough ideas to formalize the concept of
statistical power. For this concept to exist we need to do a couple
things.

\begin{enumerate}
\def\labelenumi{\arabic{enumi}.}
\tightlist
\item
  Agree to set an alpha criterion. When the p-value for our
  test-statistic is below this value we will call our finding
  statistically significant, and agree to reject the null hypothesis and
  accept the ``alternative'' hypothesis (sidenote, usually it isn't very
  clear which specific alternative hypothesis was accepted)
\item
  In advance of conducting the study, figure out what kinds of
  effect-sizes our design is capable of detecting with particular
  probabilites.
\end{enumerate}

The power of a study is determined by the relationship between

\begin{enumerate}
\def\labelenumi{\arabic{enumi}.}
\tightlist
\item
  The sample-size of the study
\item
  The effect-size of the manipulation
\item
  The alpha value set by the researcher.
\end{enumerate}

To see this in practice let's do a simulation. We will do a t-test on a
between-groups design 10 subjects in each group. Group A will be a
control group with scores sampled from a normal distribution with mean
of 10, and standard deviation of 5. Group B will be a treatment group,
we will say the treatment has an effect-size of Cohen's \(d\) = .5,
that's a standard deviation shift of .5, so the scores with come from a
normal distribution with mean =12.5 and standard deivation of 5.
Remember 1 standard deviation here is 5, so half of a standard deviation
is 2.5.

The following R script runs this simulated experiment 1000 times. We set
the alpha criterion to .05, this means we will reject the null whenever
the \(p\)-value is less than .05. With this specific design, how many
times out of of 1000 do we reject the null, and accept the alternative
hypothesis?

\begin{verbatim}
#> [1] 192
\end{verbatim}

The answer is that we reject the null, and accept the alternative 192
times out of 1000. In other words our experiment succesfully accepts the
alternative hypothesis 19.2 percent of the time, this is known as the
power of the study. Power is the probability that a design will
succesfully detect an effect of a specific size.

Importantly, power is completely abstract idea that is completely
determined by many assumptions including N, effect-size, and alpha. As a
result, it is best not to think of power as a single number, but instead
as a family of numbers.

For example, power is different when we change N. If we increase N, our
samples will more precisely estimate the true distributions that they
came from. Increasing N reduces sampling error, and shrinks the range of
differences that can be produced by chance. Lets' increase our N in this
simulation from 10 to 20 in each group and see what happens.

\begin{verbatim}
#> [1] 330
\end{verbatim}

Now the number of significant experiments i 330 out of 1000, or a power
of 33 percent. That's roughly doubled from before. We have made the
design more sensitive to the effect by increasing N.

We can change the power of the design by changing the alpha-value, which
tells us how much evidence we need to reject the null. For example, if
we set the alpha criterion to 0.01, then we will be more conservative,
only rejecting the null when chance can produce the observed difference
1\% of the time. In our example, this will have the effect of reducing
power. Let's keep N at 20, but reduce the alpha to 0.01 and see what
happens:

\begin{verbatim}
#> [1] 153
\end{verbatim}

Now only 153 out of 1000 experiments are significant, that's 15.3 power.

Finally, the power of the design depends on the actual size of the
effect caused by the manipulation. In our example, we hypothesized that
the effect caused a shift of .5 standard deviations. What if the effect
causes a bigger shift? Say, a shift of 2 standard deviations. Let's keep
N= 20, and alpha \textless{} .01, but change the effect-size to two
standard deviations. When the effect in the real-world is bigger, it
should be easier to measure, so our power will increase.

\begin{verbatim}
#> [1] 1000
\end{verbatim}

Neat, if the effect-size is actually huge (2 standard deviation shift),
then we have power 100 percent to detect the true effect.

\subsection{Power curves}\label{power-curves-1}

We mentioned that it is best to think of power as a family of numbers,
rather than as a single number. To elaborate on this consider the power
curve below. This is the power curve for a specific design: a between
groups experiments with two levels, that uses an independent samples
t-test to test whether an observed difference is due to chance.
Critically, N is set to 10 in each group, and alpha is set to .05

In Figure~\ref{fig-13powercurve} power (as a proportion, not a
percentage) is plotted on the y-axis, and effect-size (Cohen's d) in
standard deviation units is plotted on the x-axis.

\begin{figure}

\centering{

\includegraphics[width=1\linewidth,height=\textheight,keepaspectratio]{13-Thinking_files/figure-pdf/fig-13powercurve-1.pdf}

}

\caption{\label{fig-13powercurve}This figure shows power as a function
of effect-size (Cohen's d) for a between-subjects independent samples
t-test, with N=10, and alpha criterion 0.05.}

\end{figure}%

A power curve like this one is very helpful to understand the
sensitivity of a particular design. For example, we can see that a
between subjects design with N=10 in both groups, will detect an effect
of d=.5 (half a standard deviation shift) about 20\% of the time, will
detect an effect of d=.8 about 50\% of the time, and will detect an
effect of d=2 about 100\% of the time. All of the percentages reflect
the power of the design, which is the percentage of times the design
would be expected to find a \(p\) \textless{} 0.05.

Let's imagine that based on prior research, the effect you are
interested in measuring is fairly small, d=0.2. If you want to run an
experiment that will detect an effect of this size a large percentage of
the time, how many subjects do you need to have in each group? We know
from the above graph that with N=10, power is very low to detect an
effect of d=0.2. Let's make Figure~\ref{fig-13powercurveN} and vary the
number of subjects rather than the size of the effect.

\begin{figure}

\centering{

\includegraphics[width=1\linewidth,height=\textheight,keepaspectratio]{13-Thinking_files/figure-pdf/fig-13powercurveN-1.pdf}

}

\caption{\label{fig-13powercurveN}This figure shows power as a function
of N for a between-subjects independent samples t-test, with d=0.2, and
alpha criterion 0.05.}

\end{figure}%

The figure plots power to detect an effect of d=0.2, as a function of N.
The green line shows where power = .8, or 80\%. It looks like we would
nee about 380 subjects in each group to measure an effect of d=0.2, with
power = .8. This means that 80\% of our experiments would succesfully
show p \textless{} 0.05. Often times power of 80\% is recommended as a
reasonable level of power, however even when your design has power =
80\%, your experiment will still fail to find an effect (associated with
that level of power) 20\% of the time!

\section{Planning your design}\label{planning-your-design-1}

Our discussion of effect size and power highlight the importance of the
understanding the statistical limitations of an experimental design. In
particular, we have seen the relationship between:

\begin{enumerate}
\def\labelenumi{\arabic{enumi}.}
\tightlist
\item
  Sample-size
\item
  Effect-size
\item
  Alpha criterion
\item
  Power
\end{enumerate}

As a general rule of thumb, small N designs can only reliably detect
very large effects, whereas large N designs can reliably detect much
smaller effects. As a researcher, it is your responsibility to plan your
design accordingly so that it is capable of reliably detecting the kinds
of effects it is intended to measure.

\section{Some considerations}\label{some-considerations-1}

\subsection{Low powered studies}\label{low-powered-studies-1}

Consider the following case. A researcher runs a study to detect an
effect of interest. There is good reason, from prior research, to
believe the effect-size is d=0.5. The researcher uses a design that has
30\% power to detect the effect. They run the experiment and find a
significant p-value, (p\textless.05). They conclude their manipulation
worked, because it was unlikely that their result could have been caused
by chance. How would you interpret the results of a study like this?
Would you agree with thte researchers that the manipulation likely
caused the difference? Would you be skeptical of the result?

The situation above requires thinking about two kinds of probabilities.
On the one hand we know that the result observed by the researchers does
not occur often by chance (p is less than 0.05). At the same time, we
know that the design was underpowered, it only detects results of the
expected size 30\% of the time. We are face with wondering what kind of
luck was driving the difference. The researchers could have gotten
unlucky, and the difference really could be due to chance. In this case,
they would be making a type I error (saying the result is real when it
isn't). If the result was not due to chance, then they would also be
lucky, as their design only detects this effect 30\% of the time.

Perhaps another way to look at this situation is in terms of the
replicability of the result. Replicability refers to whether or not the
findings of the study would be the same if the experiment was repeated.
Because we know that power is low here (only 30\%), we would expect that
most replications of this experiment would not find a significant
effect. Instead, the experiment would be expected to replicate only 30\%
of the time.

\subsection{Large N and small
effects}\label{large-n-and-small-effects-1}

Perhaps you have noticed that there is an intriguiing relationship
between N (sample-size) and power and effect-size. As N increases, so
does power to detect an effect of a particular size. Additionally, as N
increases, a design is capable of detecting smaller and smaller effects
with greater and greater power. For example, if N was large enough, we
would have high power to detect very small effects, say d= 0.01, or even
d=0.001. Let's think about what this means.

Imagine a drug company told you that they ran an experiment with 1
billion people to test whether their drug causes a significant change in
headache pain. Let's say they found a significant effect (with power
=100\%), but the effect was very small, it turns out the drug reduces
headache pain by less than 1\%, let's say 0.01\%. For our imaginary
study we will also assume that this effect is very real, and not caused
by chance.

Clearly the design had enough power to detect the effect, and the effect
was there, so the design did detect the effect. However, the issue is
that there is little practical value to this effect. Nobody is going to
by a drug to reduce their headache pain by 0.01\%, even if it was
``scientifcally proven'' to work. This example brings up two issues.
First, increasing N to very large levels will allow designs to detect
almost any effect (even very tiny ones) with very high power. Second,
sometimes effects are meaningless when they are very small, especially
in applied research such as drug studies.

These two issues can lead to interesting suggestions. For example,
someone might claim that large N studies aren't very useful, because
they can always detect really tiny effects that are practically
meaningless. On the other hand, large N studies will also detect larger
effects too, and they will give a better estimate of the ``true'' effect
in the population (because we know that larger samples do a better job
of estimating population parameters). Additionally, although really
small effects are often not interesting in the context of applied
research, they can be very important in theoretical research. For
example, one theory might predict that manipulating X should have no
effect, but another theory might predict that X does have an effect,
even if it is a small one. So, detecting a small effect can have
theoretical implication that can help rule out false theories. Generally
speaking, researchers asking both theoretical and applied questions
should think about and establish guidelines for ``meaningful''
effect-sizes so that they can run designs of appropriate size to detect
effects of ``meaningful size''.

\subsection{Small N and Large
effects}\label{small-n-and-large-effects-1}

All other things being equal would you trust the results from a study
with small N or large N? This isn't a trick question, but sometimes
people tie themselves into a knot trying to answer it. We already know
that large sample-sizes provide better estimates of the distributions
the samples come from. As a result, we can safely conclude that we
should trust the data from large N studies more than small N studies.

At the same time, you might try to convince yourself otherwise. For
example, you know that large N studies can detect very small effects
that are practically and possibly even theoretically meaningless. You
also know that that small N studies are only capable of reliably
detecting very large effects. So, you might reason that a small N study
is better than a large N study because if a small N study detects an
effect, that effect must be big and meaningful; whereas, a large N study
could easily detect an effect that is tiny and meaningless.

This line of thinking needs some improvement. First, just because a
large N study can detect small effects, doesn't mean that it only
detects small effects. If the effect is large, a large N study will
easily detect it. Large N studies have the power to detect a much wider
range of effects, from small to large. Second, just because a small N
study detected an effect, does not mean that the effect is real, or that
the effect is large. For example, small N studies have more variability,
so the estimate of the effect size will have more error. Also, there is
5\% (or alpha rate) chance that the effect was spurious. Interestingly,
there is a pernicious relationship between effect-size and type I error
rate

\subsection{Type I errors are convincing when N is
small}\label{type-i-errors-are-convincing-when-n-is-small-1}

So what is this pernicious relationship between Type I errors and
effect-size? Mainly, this relationship is pernicious for small N
studies. For example, the following figure illustrates the results of
1000s of simulated experiments, all assuming the null distribution. In
other words, for all of these simulations there is no true effect, as
the numbers are all sampled from an identical distribution (normal
distribution with mean =0, and standard deviation =1). The true
effect-size is 0 in all cases.

We know that under the null, researchers will find p values that are
less 5\% about 5\% of the time, remember that is the definition. So, if
a researcher happened to be in this situation (where there manipulation
did absolutely nothing), they would make a type I error 5\% of the time,
or if they conducted 100 experiments, they would expect to find a
significant result for 5 of them.

Figure~\ref{fig-13effectsizeType1} reports the findings from only the
type I errors, where the simulated study did produce p \textless{} 0.05.
For each type I error, we calculated the exact p-value, as well as the
effect-size (cohen's D) (mean difference divided by standard deviation).
We already know that the true effect-size is zero, however take a look
at this graph, and pay close attention to the smaller sample-sizes.

\begin{figure}

\centering{

\includegraphics[width=1\linewidth,height=\textheight,keepaspectratio]{13-Thinking_files/figure-pdf/fig-13effectsizeType1-1.pdf}

}

\caption{\label{fig-13effectsizeType1}Effect size as a function of
p-values for type 1 Errors under the null, for a paired samples t-test.}

\end{figure}%

For example, look at the red dots, when sample size is 10. Here we see
that the effect-sizes are quite large. When p is near 0.05 the
effect-size is around .8, and it goes up and up as when p gets smaller
and smaller. What does this mean? It means that when you get unlucky
with a small N design, and your manipulation does not work, but you by
chance find a ``significant'' effect, the effect-size measurement will
show you a ``big effect''. This is the pernicious aspect. When you make
a type I error for small N, your data will make you think there is no
way it could be a type I error because the effect is just so big!.
Notice that when N is very large, like 1000, the measure of effect-size
approaches 0 (which is the true effect-size in the simulation shown in
Figure~\ref{fig-13cohensD}).

\begin{figure}

\centering{

\includegraphics[width=1\linewidth,height=\textheight,keepaspectratio]{13-Thinking_files/figure-pdf/fig-13cohensD-1.pdf}

}

\caption{\label{fig-13cohensD}Each panel shows a histogram of a
different sampling statistic.}

\end{figure}%

\bookmarksetup{startatroot}

\chapter{GIFs}\label{gifs}

This is the place where I put the stats gifs as I make them. The gifs
can downloaded from this page, or they can be downloaded from this
folder on the github repo for this book
\url{https://github.com/CrumpLab/statistics/tree/master/gifs}. Please
feel free to use them however you wish. The source code for compiling
the gifs in R is shown alongside each gif. The animations are made
possible by the \textbf{gganimate} package.

\textbf{This is a work in progress, subject to change and addition}

\section{Correlation GIFs}\label{correlation-gifs}

Note regression lines and confidence bands can be added using
\texttt{geom\_smooth(method=lm,\ se=T)}

\subsection{N=10, both variables drawn from a uniform
distribution}\label{n10-both-variables-drawn-from-a-uniform-distribution}

\begin{Shaded}
\begin{Highlighting}[]
\NormalTok{all\_df}\OtherTok{\textless{}{-}}\FunctionTok{data.frame}\NormalTok{()}
\ControlFlowTok{for}\NormalTok{(sim }\ControlFlowTok{in} \DecValTok{1}\SpecialCharTok{:}\DecValTok{10}\NormalTok{)\{}
\NormalTok{  North\_pole }\OtherTok{\textless{}{-}} \FunctionTok{runif}\NormalTok{(}\DecValTok{10}\NormalTok{,}\DecValTok{1}\NormalTok{,}\DecValTok{10}\NormalTok{)}
\NormalTok{  South\_pole }\OtherTok{\textless{}{-}} \FunctionTok{runif}\NormalTok{(}\DecValTok{10}\NormalTok{,}\DecValTok{1}\NormalTok{,}\DecValTok{10}\NormalTok{)}
\NormalTok{  t\_df}\OtherTok{\textless{}{-}}\FunctionTok{data.frame}\NormalTok{(}\AttributeTok{simulation=}\FunctionTok{rep}\NormalTok{(sim,}\DecValTok{10}\NormalTok{),}
\NormalTok{                                  North\_pole,}
\NormalTok{                                  South\_pole)}
\NormalTok{  all\_df}\OtherTok{\textless{}{-}}\FunctionTok{rbind}\NormalTok{(all\_df,t\_df)}
\NormalTok{\}}


\FunctionTok{ggplot}\NormalTok{(all\_df,}\FunctionTok{aes}\NormalTok{(}\AttributeTok{x=}\NormalTok{North\_pole,}\AttributeTok{y=}\NormalTok{South\_pole))}\SpecialCharTok{+}
  \FunctionTok{geom\_point}\NormalTok{()}\SpecialCharTok{+}
  \FunctionTok{geom\_smooth}\NormalTok{(}\AttributeTok{method=}\NormalTok{lm, }\AttributeTok{se=}\ConstantTok{FALSE}\NormalTok{)}\SpecialCharTok{+}
  \FunctionTok{theme\_classic}\NormalTok{()}\SpecialCharTok{+}
  \FunctionTok{transition\_states}\NormalTok{(}
\NormalTok{    simulation,}
    \AttributeTok{transition\_length =} \DecValTok{2}\NormalTok{,}
    \AttributeTok{state\_length =} \DecValTok{1}
\NormalTok{  )}\SpecialCharTok{+}\FunctionTok{enter\_fade}\NormalTok{() }\SpecialCharTok{+} 
  \FunctionTok{exit\_shrink}\NormalTok{() }\SpecialCharTok{+}
  \FunctionTok{ease\_aes}\NormalTok{(}\StringTok{\textquotesingle{}sine{-}in{-}out\textquotesingle{}}\NormalTok{)}
\end{Highlighting}
\end{Shaded}

\subsection{Correlation between random deviates from uniform
distribution across four sample
sizes}\label{correlation-between-random-deviates-from-uniform-distribution-across-four-sample-sizes}

N= 10,50,100,1000 All values sampled from a uniform distribution

\begin{Shaded}
\begin{Highlighting}[]
\NormalTok{all\_df}\OtherTok{\textless{}{-}}\FunctionTok{data.frame}\NormalTok{()}
\ControlFlowTok{for}\NormalTok{(sim }\ControlFlowTok{in} \DecValTok{1}\SpecialCharTok{:}\DecValTok{10}\NormalTok{)\{}
  \ControlFlowTok{for}\NormalTok{(n }\ControlFlowTok{in} \FunctionTok{c}\NormalTok{(}\DecValTok{10}\NormalTok{,}\DecValTok{50}\NormalTok{,}\DecValTok{100}\NormalTok{,}\DecValTok{1000}\NormalTok{))\{}
\NormalTok{  North\_pole }\OtherTok{\textless{}{-}} \FunctionTok{runif}\NormalTok{(n,}\DecValTok{1}\NormalTok{,}\DecValTok{10}\NormalTok{)}
\NormalTok{  South\_pole }\OtherTok{\textless{}{-}} \FunctionTok{runif}\NormalTok{(n,}\DecValTok{1}\NormalTok{,}\DecValTok{10}\NormalTok{)}
\NormalTok{  t\_df}\OtherTok{\textless{}{-}}\FunctionTok{data.frame}\NormalTok{(}\AttributeTok{nsize=}\FunctionTok{rep}\NormalTok{(n,n),}
                   \AttributeTok{simulation=}\FunctionTok{rep}\NormalTok{(sim,n),}
\NormalTok{                                  North\_pole,}
\NormalTok{                                  South\_pole)}
\NormalTok{  all\_df}\OtherTok{\textless{}{-}}\FunctionTok{rbind}\NormalTok{(all\_df,t\_df)}
\NormalTok{  \}}
\NormalTok{\}}


\FunctionTok{ggplot}\NormalTok{(all\_df,}\FunctionTok{aes}\NormalTok{(}\AttributeTok{x=}\NormalTok{North\_pole,}\AttributeTok{y=}\NormalTok{South\_pole))}\SpecialCharTok{+}
  \FunctionTok{geom\_point}\NormalTok{()}\SpecialCharTok{+}
  \FunctionTok{geom\_smooth}\NormalTok{(}\AttributeTok{method=}\NormalTok{lm, }\AttributeTok{se=}\ConstantTok{FALSE}\NormalTok{)}\SpecialCharTok{+}
  \FunctionTok{theme\_classic}\NormalTok{()}\SpecialCharTok{+}
  \FunctionTok{facet\_wrap}\NormalTok{(}\SpecialCharTok{\textasciitilde{}}\NormalTok{nsize)}\SpecialCharTok{+}
  \FunctionTok{transition\_states}\NormalTok{(}
\NormalTok{    simulation,}
    \AttributeTok{transition\_length =} \DecValTok{2}\NormalTok{,}
    \AttributeTok{state\_length =} \DecValTok{1}
\NormalTok{  )}\SpecialCharTok{+}\FunctionTok{enter\_fade}\NormalTok{() }\SpecialCharTok{+} 
  \FunctionTok{exit\_shrink}\NormalTok{() }\SpecialCharTok{+}
  \FunctionTok{ease\_aes}\NormalTok{(}\StringTok{\textquotesingle{}sine{-}in{-}out\textquotesingle{}}\NormalTok{)}
\end{Highlighting}
\end{Shaded}

\subsection{Correlation between random deviates from normal distribution
across four sample
sizes}\label{correlation-between-random-deviates-from-normal-distribution-across-four-sample-sizes}

N= 10,50,100,1000 All values sampled from the same normal distribution
(mean=0, sd=1)

\begin{Shaded}
\begin{Highlighting}[]
\NormalTok{all\_df}\OtherTok{\textless{}{-}}\FunctionTok{data.frame}\NormalTok{()}
\ControlFlowTok{for}\NormalTok{(sim }\ControlFlowTok{in} \DecValTok{1}\SpecialCharTok{:}\DecValTok{10}\NormalTok{)\{}
  \ControlFlowTok{for}\NormalTok{(n }\ControlFlowTok{in} \FunctionTok{c}\NormalTok{(}\DecValTok{10}\NormalTok{,}\DecValTok{50}\NormalTok{,}\DecValTok{100}\NormalTok{,}\DecValTok{1000}\NormalTok{))\{}
\NormalTok{  North\_pole }\OtherTok{\textless{}{-}} \FunctionTok{rnorm}\NormalTok{(n,}\DecValTok{0}\NormalTok{,}\DecValTok{1}\NormalTok{)}
\NormalTok{  South\_pole }\OtherTok{\textless{}{-}} \FunctionTok{rnorm}\NormalTok{(n,}\DecValTok{0}\NormalTok{,}\DecValTok{1}\NormalTok{)}
\NormalTok{  t\_df}\OtherTok{\textless{}{-}}\FunctionTok{data.frame}\NormalTok{(}\AttributeTok{nsize=}\FunctionTok{rep}\NormalTok{(n,n),}
                   \AttributeTok{simulation=}\FunctionTok{rep}\NormalTok{(sim,n),}
\NormalTok{                                  North\_pole,}
\NormalTok{                                  South\_pole)}
\NormalTok{  all\_df}\OtherTok{\textless{}{-}}\FunctionTok{rbind}\NormalTok{(all\_df,t\_df)}
\NormalTok{  \}}
\NormalTok{\}}


\FunctionTok{ggplot}\NormalTok{(all\_df,}\FunctionTok{aes}\NormalTok{(}\AttributeTok{x=}\NormalTok{North\_pole,}\AttributeTok{y=}\NormalTok{South\_pole))}\SpecialCharTok{+}
  \FunctionTok{geom\_point}\NormalTok{()}\SpecialCharTok{+}
  \FunctionTok{geom\_smooth}\NormalTok{(}\AttributeTok{method=}\NormalTok{lm, }\AttributeTok{se=}\ConstantTok{FALSE}\NormalTok{)}\SpecialCharTok{+}
  \FunctionTok{theme\_classic}\NormalTok{()}\SpecialCharTok{+}
  \FunctionTok{facet\_wrap}\NormalTok{(}\SpecialCharTok{\textasciitilde{}}\NormalTok{nsize)}\SpecialCharTok{+}
  \FunctionTok{transition\_states}\NormalTok{(}
\NormalTok{    simulation,}
    \AttributeTok{transition\_length =} \DecValTok{2}\NormalTok{,}
    \AttributeTok{state\_length =} \DecValTok{1}
\NormalTok{  )}\SpecialCharTok{+}\FunctionTok{enter\_fade}\NormalTok{() }\SpecialCharTok{+} 
  \FunctionTok{exit\_shrink}\NormalTok{() }\SpecialCharTok{+}
  \FunctionTok{ease\_aes}\NormalTok{(}\StringTok{\textquotesingle{}sine{-}in{-}out\textquotesingle{}}\NormalTok{)}
\end{Highlighting}
\end{Shaded}

\subsection{Correlation between X and Y variables that have a true
correlation as a function of
sample-size}\label{correlation-between-x-and-y-variables-that-have-a-true-correlation-as-a-function-of-sample-size}

\begin{Shaded}
\begin{Highlighting}[]
\FunctionTok{library}\NormalTok{(MASS)}
\NormalTok{r}\OtherTok{\textless{}{-}}\NormalTok{.}\DecValTok{7}

\NormalTok{proportional\_permute}\OtherTok{\textless{}{-}}\ControlFlowTok{function}\NormalTok{(x,prop)\{}
\NormalTok{  indices}\OtherTok{\textless{}{-}}\FunctionTok{seq}\NormalTok{(}\DecValTok{1}\SpecialCharTok{:}\FunctionTok{length}\NormalTok{(x))}
\NormalTok{  s\_indices}\OtherTok{\textless{}{-}}\FunctionTok{sample}\NormalTok{(indices)}
\NormalTok{  n\_shuffle}\OtherTok{\textless{}{-}}\FunctionTok{round}\NormalTok{(}\FunctionTok{length}\NormalTok{(x)}\SpecialCharTok{*}\NormalTok{prop)}
  \ControlFlowTok{switch}\OtherTok{\textless{}{-}}\FunctionTok{sample}\NormalTok{(indices)}
\NormalTok{  x[s\_indices[}\DecValTok{1}\SpecialCharTok{:}\NormalTok{n\_shuffle]]}\OtherTok{\textless{}{-}}\NormalTok{x[}\ControlFlowTok{switch}\NormalTok{[}\DecValTok{1}\SpecialCharTok{:}\NormalTok{n\_shuffle]]}
  \FunctionTok{return}\NormalTok{(x)}
\NormalTok{\}}

\NormalTok{all\_df}\OtherTok{\textless{}{-}}\FunctionTok{data.frame}\NormalTok{()}
\ControlFlowTok{for}\NormalTok{(sim }\ControlFlowTok{in} \DecValTok{1}\SpecialCharTok{:}\DecValTok{10}\NormalTok{)\{}
  \ControlFlowTok{for}\NormalTok{(samples }\ControlFlowTok{in} \FunctionTok{c}\NormalTok{(}\DecValTok{10}\NormalTok{,}\DecValTok{50}\NormalTok{,}\DecValTok{100}\NormalTok{,}\DecValTok{1000}\NormalTok{))\{}
    \CommentTok{\#data \textless{}{-} mvrnorm(n=samples, mu=c(0, 0), Sigma=matrix(c(1, r, r, 1), nrow=2), empirical=TRUE)}
    \CommentTok{\#North\_pole \textless{}{-} data[, 1]  \# standard normal (mu=0, sd=1)}
    \CommentTok{\#South\_pole \textless{}{-} data[, 2] }
    
\NormalTok{    North\_pole }\OtherTok{\textless{}{-}} \FunctionTok{runif}\NormalTok{(samples,}\DecValTok{1}\NormalTok{,}\DecValTok{10}\NormalTok{)}
\NormalTok{    South\_pole }\OtherTok{\textless{}{-}} \FunctionTok{proportional\_permute}\NormalTok{(North\_pole,.}\DecValTok{5}\NormalTok{)}\SpecialCharTok{+}\FunctionTok{runif}\NormalTok{(samples,}\SpecialCharTok{{-}}\DecValTok{5}\NormalTok{,}\DecValTok{5}\NormalTok{)}

\NormalTok{    t\_df}\OtherTok{\textless{}{-}}\FunctionTok{data.frame}\NormalTok{(}\AttributeTok{nsize=}\FunctionTok{rep}\NormalTok{(samples,samples),}
                   \AttributeTok{simulation=}\FunctionTok{rep}\NormalTok{(sim,samples),}
\NormalTok{                                  North\_pole,}
\NormalTok{                                  South\_pole)}
\NormalTok{  all\_df}\OtherTok{\textless{}{-}}\FunctionTok{rbind}\NormalTok{(all\_df,t\_df)}
\NormalTok{  \}}
\NormalTok{\}}

\FunctionTok{ggplot}\NormalTok{(all\_df,}\FunctionTok{aes}\NormalTok{(}\AttributeTok{x=}\NormalTok{North\_pole,}\AttributeTok{y=}\NormalTok{South\_pole))}\SpecialCharTok{+}
  \FunctionTok{geom\_point}\NormalTok{()}\SpecialCharTok{+}
  \FunctionTok{geom\_smooth}\NormalTok{(}\AttributeTok{method=}\NormalTok{lm, }\AttributeTok{se=}\ConstantTok{FALSE}\NormalTok{)}\SpecialCharTok{+}
  \FunctionTok{theme\_classic}\NormalTok{()}\SpecialCharTok{+}
  \FunctionTok{facet\_wrap}\NormalTok{(}\SpecialCharTok{\textasciitilde{}}\NormalTok{nsize)}\SpecialCharTok{+}
  \FunctionTok{transition\_states}\NormalTok{(}
\NormalTok{    simulation,}
    \AttributeTok{transition\_length =} \DecValTok{2}\NormalTok{,}
    \AttributeTok{state\_length =} \DecValTok{1}
\NormalTok{  )}\SpecialCharTok{+}\FunctionTok{enter\_fade}\NormalTok{() }\SpecialCharTok{+} 
  \FunctionTok{exit\_shrink}\NormalTok{() }\SpecialCharTok{+}
  \FunctionTok{ease\_aes}\NormalTok{(}\StringTok{\textquotesingle{}sine{-}in{-}out\textquotesingle{}}\NormalTok{)}
\end{Highlighting}
\end{Shaded}

\subsection{Type I errors, sampling random deviates from normal
distribution with regression
lines}\label{type-i-errors-sampling-random-deviates-from-normal-distribution-with-regression-lines}

These scatter plots only show what would be type I errors (assuming
alpha=.05). The X and Y values were both sampled from the same normal
distribution (mean = 0, sd=1). 1000 simulations were conducted for each
sample size (10,50,100,1000). For each, the animation shows 10 scatter
plots where the observed ``correlation'' would have passed a
significance test. According to definition, these correlations only
arise from random normal deviates 5\% of the time, but when they do
arise for small sample sizes, they look fairly convincing.

\begin{Shaded}
\begin{Highlighting}[]
\NormalTok{all\_df}\OtherTok{\textless{}{-}}\FunctionTok{data.frame}\NormalTok{()}
\ControlFlowTok{for}\NormalTok{(n }\ControlFlowTok{in} \FunctionTok{c}\NormalTok{(}\DecValTok{10}\NormalTok{,}\DecValTok{50}\NormalTok{,}\DecValTok{100}\NormalTok{,}\DecValTok{1000}\NormalTok{))\{}
\NormalTok{  count\_sims}\OtherTok{\textless{}{-}}\DecValTok{0}
  \ControlFlowTok{for}\NormalTok{(sim }\ControlFlowTok{in} \DecValTok{1}\SpecialCharTok{:}\DecValTok{1000}\NormalTok{)\{}
\NormalTok{    North\_pole }\OtherTok{\textless{}{-}} \FunctionTok{rnorm}\NormalTok{(n,}\DecValTok{0}\NormalTok{,}\DecValTok{1}\NormalTok{)}
\NormalTok{    South\_pole }\OtherTok{\textless{}{-}} \FunctionTok{rnorm}\NormalTok{(n,}\DecValTok{0}\NormalTok{,}\DecValTok{1}\NormalTok{)}
    \ControlFlowTok{if}\NormalTok{(}\FunctionTok{cor.test}\NormalTok{(North\_pole,South\_pole)}\SpecialCharTok{$}\NormalTok{p.value}\SpecialCharTok{\textless{}}\NormalTok{.}\DecValTok{05}\NormalTok{)\{}
\NormalTok{      count\_sims}\OtherTok{\textless{}{-}}\NormalTok{count\_sims}\SpecialCharTok{+}\DecValTok{1}
\NormalTok{    t\_df}\OtherTok{\textless{}{-}}\FunctionTok{data.frame}\NormalTok{(}\AttributeTok{nsize=}\FunctionTok{rep}\NormalTok{(n,n),}
                     \AttributeTok{simulation=}\FunctionTok{rep}\NormalTok{(count\_sims,n),}
\NormalTok{                     North\_pole,}
\NormalTok{                     South\_pole)}
\NormalTok{    all\_df}\OtherTok{\textless{}{-}}\FunctionTok{rbind}\NormalTok{(all\_df,t\_df)}
    
    \ControlFlowTok{if}\NormalTok{(count\_sims}\SpecialCharTok{==}\DecValTok{10}\NormalTok{)\{}
      \ControlFlowTok{break}
\NormalTok{    \}}
\NormalTok{    \}}
\NormalTok{  \}}
\NormalTok{\}}


\FunctionTok{ggplot}\NormalTok{(all\_df,}\FunctionTok{aes}\NormalTok{(}\AttributeTok{x=}\NormalTok{North\_pole,}\AttributeTok{y=}\NormalTok{South\_pole))}\SpecialCharTok{+}
  \FunctionTok{geom\_point}\NormalTok{()}\SpecialCharTok{+}
  \FunctionTok{geom\_smooth}\NormalTok{(}\AttributeTok{method=}\NormalTok{lm, }\AttributeTok{se=}\ConstantTok{TRUE}\NormalTok{)}\SpecialCharTok{+}
  \FunctionTok{theme\_classic}\NormalTok{()}\SpecialCharTok{+}
  \FunctionTok{facet\_wrap}\NormalTok{(}\SpecialCharTok{\textasciitilde{}}\NormalTok{nsize)}\SpecialCharTok{+}
  \FunctionTok{transition\_states}\NormalTok{(}
\NormalTok{    simulation,}
    \AttributeTok{transition\_length =} \DecValTok{2}\NormalTok{,}
    \AttributeTok{state\_length =} \DecValTok{1}
\NormalTok{  )}\SpecialCharTok{+}\FunctionTok{enter\_fade}\NormalTok{() }\SpecialCharTok{+} 
  \FunctionTok{exit\_shrink}\NormalTok{() }\SpecialCharTok{+}
  \FunctionTok{ease\_aes}\NormalTok{(}\StringTok{\textquotesingle{}sine{-}in{-}out\textquotesingle{}}\NormalTok{)}
\end{Highlighting}
\end{Shaded}

\subsection{Cell-size and correlation}\label{cell-size-and-correlation}

This simulation illustrates how the behavior of correlating two random
normal samples as a function of cell-size. The sample-size is always set
at N=10. For each panel, the simulation uses an increasing cell-size to
estimate the mean for X and Y. When cell-size is 1, 10 X and Y values
are drawn from the same normal (u=0, sd=1). When cell-size is 5, for
each X,Y score in the plot, 5 samples were drawn from the same normal,
and then the mean of the samples is plotted. The effect of cell-size
shrinks the dot cloud, as both X and Y scores provide better estimates
of the population mean = 0. Cell-size has no effect on the behavior of
r, which swings around because sample-size N is small. These are all
random, so there is always a 5\% type I error rate (alpha =.05).

\begin{Shaded}
\begin{Highlighting}[]
\NormalTok{get\_sampling\_means}\OtherTok{\textless{}{-}}\ControlFlowTok{function}\NormalTok{(m,sd,cell\_size,s\_size)\{}
\NormalTok{  save\_means}\OtherTok{\textless{}{-}}\FunctionTok{length}\NormalTok{(s\_size)}
  \ControlFlowTok{for}\NormalTok{(i }\ControlFlowTok{in} \DecValTok{1}\SpecialCharTok{:}\NormalTok{s\_size)\{}
\NormalTok{    save\_means[i]}\OtherTok{\textless{}{-}}\FunctionTok{mean}\NormalTok{(}\FunctionTok{rnorm}\NormalTok{(cell\_size,m,sd))}
\NormalTok{  \}}
  \FunctionTok{return}\NormalTok{(save\_means)}
\NormalTok{\}}

\NormalTok{all\_df}\OtherTok{\textless{}{-}}\FunctionTok{data.frame}\NormalTok{()}
\ControlFlowTok{for}\NormalTok{(n }\ControlFlowTok{in} \FunctionTok{c}\NormalTok{(}\DecValTok{1}\NormalTok{,}\DecValTok{5}\NormalTok{,}\DecValTok{10}\NormalTok{,}\DecValTok{100}\NormalTok{))\{}
\NormalTok{  count\_sims}\OtherTok{\textless{}{-}}\DecValTok{0}
  \ControlFlowTok{for}\NormalTok{(sim }\ControlFlowTok{in} \DecValTok{1}\SpecialCharTok{:}\DecValTok{10}\NormalTok{)\{}
\NormalTok{    North\_pole }\OtherTok{\textless{}{-}} \FunctionTok{get\_sampling\_means}\NormalTok{(}\DecValTok{0}\NormalTok{,}\DecValTok{1}\NormalTok{,n,}\DecValTok{10}\NormalTok{)}
\NormalTok{    South\_pole }\OtherTok{\textless{}{-}} \FunctionTok{get\_sampling\_means}\NormalTok{(}\DecValTok{0}\NormalTok{,}\DecValTok{1}\NormalTok{,n,}\DecValTok{10}\NormalTok{)}
\NormalTok{      count\_sims}\OtherTok{\textless{}{-}}\NormalTok{count\_sims}\SpecialCharTok{+}\DecValTok{1}
\NormalTok{      t\_df}\OtherTok{\textless{}{-}}\FunctionTok{data.frame}\NormalTok{(}\AttributeTok{nsize=}\FunctionTok{rep}\NormalTok{(n,}\DecValTok{10}\NormalTok{),}
                       \AttributeTok{simulation=}\FunctionTok{rep}\NormalTok{(count\_sims,}\DecValTok{10}\NormalTok{),}
\NormalTok{                       North\_pole,}
\NormalTok{                       South\_pole)}
\NormalTok{      all\_df}\OtherTok{\textless{}{-}}\FunctionTok{rbind}\NormalTok{(all\_df,t\_df)}
\NormalTok{  \}}
\NormalTok{\}}


\FunctionTok{ggplot}\NormalTok{(all\_df,}\FunctionTok{aes}\NormalTok{(}\AttributeTok{x=}\NormalTok{North\_pole,}\AttributeTok{y=}\NormalTok{South\_pole))}\SpecialCharTok{+}
  \FunctionTok{geom\_point}\NormalTok{()}\SpecialCharTok{+}
  \FunctionTok{geom\_smooth}\NormalTok{(}\AttributeTok{method=}\NormalTok{lm, }\AttributeTok{se=}\ConstantTok{TRUE}\NormalTok{)}\SpecialCharTok{+}
  \FunctionTok{theme\_classic}\NormalTok{()}\SpecialCharTok{+}
  \FunctionTok{facet\_wrap}\NormalTok{(}\SpecialCharTok{\textasciitilde{}}\NormalTok{nsize)}\SpecialCharTok{+}
  \FunctionTok{ggtitle}\NormalTok{(}\StringTok{"Random scatterplots, N=10, Cell{-}size = 1,5,10,100"}\NormalTok{)}\SpecialCharTok{+}
  \FunctionTok{transition\_states}\NormalTok{(}
\NormalTok{    simulation,}
    \AttributeTok{transition\_length =} \DecValTok{2}\NormalTok{,}
    \AttributeTok{state\_length =} \DecValTok{1}
\NormalTok{  )}\SpecialCharTok{+}\FunctionTok{enter\_fade}\NormalTok{() }\SpecialCharTok{+} 
  \FunctionTok{exit\_shrink}\NormalTok{() }\SpecialCharTok{+}
  \FunctionTok{ease\_aes}\NormalTok{(}\StringTok{\textquotesingle{}sine{-}in{-}out\textquotesingle{}}\NormalTok{)}
\end{Highlighting}
\end{Shaded}

\subsection{Regression}\label{regression}

We look at how the residuals (error from points to line) behave as the
regression lines moves above and below it's true value. The total error
associated with all of the red lines is represents by the grey area.
This total error is smallest (minimized) when the black line overlaps
with the blue regression line (the best fit line). The total error
expands as the black line moves away from the regression. That's why the
regression line is the least wrong (best fit) line to skewer the data
(according to least squares definition)

\begin{Shaded}
\begin{Highlighting}[]
\NormalTok{d }\OtherTok{\textless{}{-}}\NormalTok{ mtcars}
\NormalTok{fit }\OtherTok{\textless{}{-}} \FunctionTok{lm}\NormalTok{(mpg }\SpecialCharTok{\textasciitilde{}}\NormalTok{ hp, }\AttributeTok{data =}\NormalTok{ d)}
\NormalTok{d}\SpecialCharTok{$}\NormalTok{predicted }\OtherTok{\textless{}{-}} \FunctionTok{predict}\NormalTok{(fit)   }\CommentTok{\# Save the predicted values}
\NormalTok{d}\SpecialCharTok{$}\NormalTok{residuals }\OtherTok{\textless{}{-}} \FunctionTok{residuals}\NormalTok{(fit) }\CommentTok{\# Save the residual values}

\NormalTok{coefs}\OtherTok{\textless{}{-}}\FunctionTok{coef}\NormalTok{(}\FunctionTok{lm}\NormalTok{(mpg }\SpecialCharTok{\textasciitilde{}}\NormalTok{ hp, }\AttributeTok{data =}\NormalTok{ mtcars))}
\NormalTok{coefs[}\DecValTok{1}\NormalTok{]}
\NormalTok{coefs[}\DecValTok{2}\NormalTok{]}

\NormalTok{x}\OtherTok{\textless{}{-}}\NormalTok{d}\SpecialCharTok{$}\NormalTok{hp}
\NormalTok{move\_line}\OtherTok{\textless{}{-}}\FunctionTok{c}\NormalTok{(}\FunctionTok{seq}\NormalTok{(}\SpecialCharTok{{-}}\DecValTok{6}\NormalTok{,}\DecValTok{6}\NormalTok{,.}\DecValTok{5}\NormalTok{),}\FunctionTok{seq}\NormalTok{(}\DecValTok{6}\NormalTok{,}\SpecialCharTok{{-}}\DecValTok{6}\NormalTok{,}\SpecialCharTok{{-}}\NormalTok{.}\DecValTok{5}\NormalTok{))}
\NormalTok{total\_error}\OtherTok{\textless{}{-}}\FunctionTok{length}\NormalTok{(}\FunctionTok{length}\NormalTok{(move\_line))}
\NormalTok{cnt}\OtherTok{\textless{}{-}}\DecValTok{0}
\ControlFlowTok{for}\NormalTok{(i }\ControlFlowTok{in}\NormalTok{ move\_line)\{}
\NormalTok{  cnt}\OtherTok{\textless{}{-}}\NormalTok{cnt}\SpecialCharTok{+}\DecValTok{1}
\NormalTok{  predicted\_y }\OtherTok{\textless{}{-}}\NormalTok{ coefs[}\DecValTok{2}\NormalTok{]}\SpecialCharTok{*}\NormalTok{x }\SpecialCharTok{+}\NormalTok{ coefs[}\DecValTok{1}\NormalTok{]}\SpecialCharTok{+}\NormalTok{i}
\NormalTok{  error\_y }\OtherTok{\textless{}{-}}\NormalTok{ (predicted\_y}\SpecialCharTok{{-}}\NormalTok{d}\SpecialCharTok{$}\NormalTok{mpg)}\SpecialCharTok{\^{}}\DecValTok{2}
\NormalTok{  total\_error[cnt]}\OtherTok{\textless{}{-}}\FunctionTok{sqrt}\NormalTok{(}\FunctionTok{sum}\NormalTok{(error\_y)}\SpecialCharTok{/}\DecValTok{32}\NormalTok{)}
\NormalTok{\}}

\NormalTok{move\_line\_sims}\OtherTok{\textless{}{-}}\FunctionTok{rep}\NormalTok{(move\_line,}\AttributeTok{each=}\DecValTok{32}\NormalTok{)}
\NormalTok{total\_error\_sims}\OtherTok{\textless{}{-}}\FunctionTok{rep}\NormalTok{(total\_error,}\AttributeTok{each=}\DecValTok{32}\NormalTok{)}
\NormalTok{sims}\OtherTok{\textless{}{-}}\FunctionTok{rep}\NormalTok{(}\DecValTok{1}\SpecialCharTok{:}\DecValTok{50}\NormalTok{,}\AttributeTok{each=}\DecValTok{32}\NormalTok{)}

\NormalTok{d}\OtherTok{\textless{}{-}}\NormalTok{d }\SpecialCharTok{\%\textgreater{}\%} \FunctionTok{slice}\NormalTok{(}\FunctionTok{rep}\NormalTok{(}\FunctionTok{row\_number}\NormalTok{(), }\DecValTok{50}\NormalTok{))}

\NormalTok{d}\OtherTok{\textless{}{-}}\FunctionTok{cbind}\NormalTok{(d,sims,move\_line\_sims,total\_error\_sims)}


\NormalTok{anim}\OtherTok{\textless{}{-}}\FunctionTok{ggplot}\NormalTok{(d, }\FunctionTok{aes}\NormalTok{(}\AttributeTok{x =}\NormalTok{ hp, }\AttributeTok{y =}\NormalTok{ mpg, }\AttributeTok{frame=}\NormalTok{sims)) }\SpecialCharTok{+}
  \FunctionTok{geom\_smooth}\NormalTok{(}\AttributeTok{method =} \StringTok{"lm"}\NormalTok{, }\AttributeTok{se =} \ConstantTok{FALSE}\NormalTok{, }\AttributeTok{color =} \StringTok{"lightblue"}\NormalTok{) }\SpecialCharTok{+}  
  \FunctionTok{geom\_abline}\NormalTok{(}\AttributeTok{intercept =} \FloatTok{30.09886}\SpecialCharTok{+}\NormalTok{move\_line\_sims, }\AttributeTok{slope =} \SpecialCharTok{{-}}\FloatTok{0.06822828}\NormalTok{)}\SpecialCharTok{+}
  \FunctionTok{lims}\NormalTok{(}\AttributeTok{x =} \FunctionTok{c}\NormalTok{(}\DecValTok{0}\NormalTok{,}\DecValTok{400}\NormalTok{), }\AttributeTok{y =} \FunctionTok{c}\NormalTok{(}\SpecialCharTok{{-}}\DecValTok{10}\NormalTok{,}\DecValTok{40}\NormalTok{))}\SpecialCharTok{+}
  \FunctionTok{geom\_segment}\NormalTok{(}\FunctionTok{aes}\NormalTok{(}\AttributeTok{xend =}\NormalTok{ hp, }\AttributeTok{yend =}\NormalTok{ predicted}\SpecialCharTok{+}\NormalTok{move\_line\_sims, }\AttributeTok{color=}\StringTok{"red"}\NormalTok{), }\AttributeTok{alpha =}\NormalTok{ .}\DecValTok{5}\NormalTok{) }\SpecialCharTok{+} 
  \FunctionTok{geom\_point}\NormalTok{() }\SpecialCharTok{+}
  \FunctionTok{geom\_ribbon}\NormalTok{(}\FunctionTok{aes}\NormalTok{(}\AttributeTok{ymin =}\NormalTok{ predicted}\SpecialCharTok{+}\NormalTok{move\_line\_sims }\SpecialCharTok{{-}}\NormalTok{ total\_error\_sims, }\AttributeTok{ymax =}\NormalTok{ predicted}\SpecialCharTok{+}\NormalTok{move\_line\_sims }\SpecialCharTok{+}\NormalTok{ total\_error\_sims), }\AttributeTok{fill =} \StringTok{"lightgrey"}\NormalTok{, }\AttributeTok{alpha=}\NormalTok{.}\DecValTok{2}\NormalTok{)}\SpecialCharTok{+} 
  \FunctionTok{theme\_classic}\NormalTok{()}\SpecialCharTok{+}
  \FunctionTok{theme}\NormalTok{(}\AttributeTok{legend.position=}\StringTok{"none"}\NormalTok{)}\SpecialCharTok{+}
  \FunctionTok{xlab}\NormalTok{(}\StringTok{"X"}\NormalTok{)}\SpecialCharTok{+}\FunctionTok{ylab}\NormalTok{(}\StringTok{"Y"}\NormalTok{)}\SpecialCharTok{+}
  \FunctionTok{transition\_manual}\NormalTok{(}\AttributeTok{frames=}\NormalTok{sims)}\SpecialCharTok{+}
  \FunctionTok{enter\_fade}\NormalTok{() }\SpecialCharTok{+} 
  \FunctionTok{exit\_fade}\NormalTok{()}\SpecialCharTok{+}
  \FunctionTok{ease\_aes}\NormalTok{(}\StringTok{\textquotesingle{}sine{-}in{-}out\textquotesingle{}}\NormalTok{)}

\FunctionTok{animate}\NormalTok{(anim,}\AttributeTok{fps=}\DecValTok{5}\NormalTok{)}
\end{Highlighting}
\end{Shaded}

\section{Sampling distributions}\label{sampling-distributions}

\subsection{Sampling from a uniform
distribution}\label{sampling-from-a-uniform-distribution}

Animation shows histograms for N=20, sampled from a uniform
distribution, along with mean (red line). Uniform distribution in this
case is integer values from 1 to 10.

\begin{Shaded}
\begin{Highlighting}[]
\NormalTok{a}\OtherTok{\textless{}{-}}\FunctionTok{round}\NormalTok{(}\FunctionTok{runif}\NormalTok{(}\DecValTok{20}\SpecialCharTok{*}\DecValTok{10}\NormalTok{,}\DecValTok{1}\NormalTok{,}\DecValTok{10}\NormalTok{))}
\NormalTok{df}\OtherTok{\textless{}{-}}\FunctionTok{data.frame}\NormalTok{(a,}\AttributeTok{sample=}\FunctionTok{rep}\NormalTok{(}\DecValTok{1}\SpecialCharTok{:}\DecValTok{10}\NormalTok{,}\AttributeTok{each=}\DecValTok{20}\NormalTok{))}
\NormalTok{df2}\OtherTok{\textless{}{-}}\FunctionTok{aggregate}\NormalTok{(a}\SpecialCharTok{\textasciitilde{}}\NormalTok{sample,df,mean)}
\NormalTok{df}\OtherTok{\textless{}{-}}\FunctionTok{cbind}\NormalTok{(df,}\AttributeTok{mean\_loc=}\FunctionTok{rep}\NormalTok{(df2}\SpecialCharTok{$}\NormalTok{a,}\AttributeTok{each=}\DecValTok{20}\NormalTok{))}

\FunctionTok{library}\NormalTok{(gganimate)}

\FunctionTok{ggplot}\NormalTok{(df,}\FunctionTok{aes}\NormalTok{(}\AttributeTok{x=}\NormalTok{a, }\AttributeTok{group=}\NormalTok{sample,}\AttributeTok{frame=}\NormalTok{sample)) }\SpecialCharTok{+}
  \FunctionTok{geom\_histogram}\NormalTok{() }\SpecialCharTok{+}
  \FunctionTok{geom\_vline}\NormalTok{(}\FunctionTok{aes}\NormalTok{(}\AttributeTok{xintercept=}\NormalTok{mean\_loc,}\AttributeTok{frame =}\NormalTok{ sample),}\AttributeTok{color=}\StringTok{"red"}\NormalTok{)}\SpecialCharTok{+}
  \FunctionTok{scale\_x\_continuous}\NormalTok{(}\AttributeTok{breaks=}\FunctionTok{seq}\NormalTok{(}\DecValTok{1}\NormalTok{,}\DecValTok{10}\NormalTok{,}\DecValTok{1}\NormalTok{))}\SpecialCharTok{+}
  \FunctionTok{theme\_classic}\NormalTok{()}\SpecialCharTok{+}
  \FunctionTok{transition\_states}\NormalTok{(}
\NormalTok{    sample,}
    \AttributeTok{transition\_length =} \DecValTok{2}\NormalTok{,}
    \AttributeTok{state\_length =} \DecValTok{1}
\NormalTok{  )}\SpecialCharTok{+}\FunctionTok{enter\_fade}\NormalTok{() }\SpecialCharTok{+} 
  \FunctionTok{exit\_shrink}\NormalTok{() }\SpecialCharTok{+}
  \FunctionTok{ease\_aes}\NormalTok{(}\StringTok{\textquotesingle{}sine{-}in{-}out\textquotesingle{}}\NormalTok{)}
\end{Highlighting}
\end{Shaded}

\subsection{Sampling from uniform with line showing expected value for
each
number}\label{sampling-from-uniform-with-line-showing-expected-value-for-each-number}

\begin{Shaded}
\begin{Highlighting}[]
\NormalTok{a}\OtherTok{\textless{}{-}}\FunctionTok{round}\NormalTok{(}\FunctionTok{runif}\NormalTok{(}\DecValTok{20}\SpecialCharTok{*}\DecValTok{10}\NormalTok{,}\DecValTok{1}\NormalTok{,}\DecValTok{10}\NormalTok{))}
\NormalTok{df}\OtherTok{\textless{}{-}}\FunctionTok{data.frame}\NormalTok{(a,}\AttributeTok{sample=}\FunctionTok{rep}\NormalTok{(}\DecValTok{1}\SpecialCharTok{:}\DecValTok{10}\NormalTok{,}\AttributeTok{each=}\DecValTok{20}\NormalTok{))}


\FunctionTok{library}\NormalTok{(gganimate)}
\FunctionTok{ggplot}\NormalTok{(df,}\FunctionTok{aes}\NormalTok{(}\AttributeTok{x=}\NormalTok{a))}\SpecialCharTok{+}
  \FunctionTok{geom\_histogram}\NormalTok{(}\AttributeTok{bins=}\DecValTok{10}\NormalTok{, }\AttributeTok{color=}\StringTok{"white"}\NormalTok{)}\SpecialCharTok{+}
  \FunctionTok{theme\_classic}\NormalTok{()}\SpecialCharTok{+}
  \FunctionTok{scale\_x\_continuous}\NormalTok{(}\AttributeTok{breaks=}\FunctionTok{seq}\NormalTok{(}\DecValTok{1}\NormalTok{,}\DecValTok{10}\NormalTok{,}\DecValTok{1}\NormalTok{))}\SpecialCharTok{+}
  \FunctionTok{geom\_hline}\NormalTok{(}\AttributeTok{yintercept=}\DecValTok{2}\NormalTok{)}\SpecialCharTok{+}
  \FunctionTok{ggtitle}\NormalTok{(}\StringTok{"Small N=20 samples from a uniform distribution"}\NormalTok{)}\SpecialCharTok{+}
  \FunctionTok{transition\_states}\NormalTok{(}
\NormalTok{    sample,}
    \AttributeTok{transition\_length =} \DecValTok{2}\NormalTok{,}
    \AttributeTok{state\_length =} \DecValTok{1}
\NormalTok{  )}\SpecialCharTok{+}\FunctionTok{enter\_fade}\NormalTok{() }\SpecialCharTok{+} 
  \FunctionTok{exit\_shrink}\NormalTok{() }\SpecialCharTok{+}
  \FunctionTok{ease\_aes}\NormalTok{(}\StringTok{\textquotesingle{}sine{-}in{-}out\textquotesingle{}}\NormalTok{)}
\end{Highlighting}
\end{Shaded}

\subsection{Sampling distribution of the mean, Normal population
distribution and sample
histograms}\label{sampling-distribution-of-the-mean-normal-population-distribution-and-sample-histograms}

This animation illustrates the relationship between a distribution
(population), samples from the distribution, and the sampling
distribution of the sample means, all as a function of n

Normal distribution in red. Individual sample histograms in grey.
Vertical red line is mean of individual sample. Histograms for sampling
distribution of the sample mean in blue. Vertical blue line is mean of
the sampling distribution of the sample mean.

Note: for purposes of the animation (and because it was easier to do
this way), the histograms for the sampling distribution of the sample
means have different sizes. When sample-size = 10, the histogram shows
10 sample means. When sample size=100, the histogram shows 100 sample
means. I could have simulated many more sample means (say 10000) for
each, but then the histograms for the sample means would be static.

The y-axis is very rough. The heights of the histograms and
distributions were scaled to be in the same range for the animation.

\begin{Shaded}
\begin{Highlighting}[]
\NormalTok{get\_sampling\_means}\OtherTok{\textless{}{-}}\ControlFlowTok{function}\NormalTok{(m,sd,s\_size)\{}
\NormalTok{  save\_means}\OtherTok{\textless{}{-}}\FunctionTok{length}\NormalTok{(s\_size)}
  \ControlFlowTok{for}\NormalTok{(i }\ControlFlowTok{in} \DecValTok{1}\SpecialCharTok{:}\NormalTok{s\_size)\{}
\NormalTok{    save\_means[i]}\OtherTok{\textless{}{-}}\FunctionTok{mean}\NormalTok{(}\FunctionTok{rnorm}\NormalTok{(s\_size,m,sd))}
\NormalTok{  \}}
  \FunctionTok{return}\NormalTok{(save\_means)}
\NormalTok{\}}

\NormalTok{all\_df}\OtherTok{\textless{}{-}}\FunctionTok{data.frame}\NormalTok{()}
\ControlFlowTok{for}\NormalTok{(sims }\ControlFlowTok{in} \DecValTok{1}\SpecialCharTok{:}\DecValTok{10}\NormalTok{)\{}
  \ControlFlowTok{for}\NormalTok{(n }\ControlFlowTok{in} \FunctionTok{c}\NormalTok{(}\DecValTok{10}\NormalTok{,}\DecValTok{50}\NormalTok{,}\DecValTok{100}\NormalTok{,}\DecValTok{1000}\NormalTok{))\{}
\NormalTok{    sample}\OtherTok{\textless{}{-}}\FunctionTok{rnorm}\NormalTok{(n,}\DecValTok{0}\NormalTok{,}\DecValTok{1}\NormalTok{)}
\NormalTok{    sample\_means}\OtherTok{\textless{}{-}}\FunctionTok{get\_sampling\_means}\NormalTok{(}\DecValTok{0}\NormalTok{,}\DecValTok{1}\NormalTok{,n)}
\NormalTok{    t\_df}\OtherTok{\textless{}{-}}\FunctionTok{data.frame}\NormalTok{(}\AttributeTok{sims=}\FunctionTok{rep}\NormalTok{(sims,n),}
\NormalTok{                     sample,}
\NormalTok{                     sample\_means,}
                     \AttributeTok{sample\_size=}\FunctionTok{rep}\NormalTok{(n,n),}
                     \AttributeTok{sample\_mean=}\FunctionTok{rep}\NormalTok{(}\FunctionTok{mean}\NormalTok{(sample),n),}
                     \AttributeTok{sampling\_mean=}\FunctionTok{rep}\NormalTok{(}\FunctionTok{mean}\NormalTok{(sample\_means),n)}
\NormalTok{                     )}
\NormalTok{    all\_df}\OtherTok{\textless{}{-}}\FunctionTok{rbind}\NormalTok{(all\_df,t\_df)}
\NormalTok{  \}}
\NormalTok{\}}


\FunctionTok{ggplot}\NormalTok{(all\_df, }\FunctionTok{aes}\NormalTok{(}\AttributeTok{x=}\NormalTok{sample))}\SpecialCharTok{+}
  \FunctionTok{geom\_histogram}\NormalTok{(}\FunctionTok{aes}\NormalTok{(}\AttributeTok{y=}\NormalTok{(..density..)}\SpecialCharTok{/}\FunctionTok{max}\NormalTok{(..density..)}\SpecialCharTok{\^{}}\NormalTok{.}\DecValTok{8}\NormalTok{),}\AttributeTok{color=}\StringTok{"white"}\NormalTok{,}\AttributeTok{fill=}\StringTok{"grey"}\NormalTok{)}\SpecialCharTok{+}
  \FunctionTok{geom\_histogram}\NormalTok{(}\FunctionTok{aes}\NormalTok{(}\AttributeTok{x=}\NormalTok{sample\_means,}\AttributeTok{y=}\NormalTok{(..density..)}\SpecialCharTok{/}\FunctionTok{max}\NormalTok{(..density..)),}\AttributeTok{fill=}\StringTok{"blue"}\NormalTok{,}\AttributeTok{color=}\StringTok{"white"}\NormalTok{,}\AttributeTok{alpha=}\NormalTok{.}\DecValTok{5}\NormalTok{)}\SpecialCharTok{+}
  \FunctionTok{stat\_function}\NormalTok{(}\AttributeTok{fun =}\NormalTok{ dnorm, }
                \AttributeTok{args =} \FunctionTok{list}\NormalTok{(}\AttributeTok{mean =} \DecValTok{0}\NormalTok{, }\AttributeTok{sd =} \DecValTok{1}\NormalTok{), }
                \AttributeTok{lwd =}\NormalTok{ .}\DecValTok{75}\NormalTok{, }
                \AttributeTok{col =} \StringTok{\textquotesingle{}red\textquotesingle{}}\NormalTok{)}\SpecialCharTok{+}
  \FunctionTok{geom\_vline}\NormalTok{(}\FunctionTok{aes}\NormalTok{(}\AttributeTok{xintercept=}\NormalTok{sample\_mean,}\AttributeTok{frame=}\NormalTok{sims),}\AttributeTok{color=}\StringTok{"red"}\NormalTok{)}\SpecialCharTok{+}
  \FunctionTok{geom\_vline}\NormalTok{(}\FunctionTok{aes}\NormalTok{(}\AttributeTok{xintercept=}\NormalTok{sampling\_mean,}\AttributeTok{frame=}\NormalTok{sims),}\AttributeTok{color=}\StringTok{"blue"}\NormalTok{)}\SpecialCharTok{+}
  \FunctionTok{facet\_wrap}\NormalTok{(}\SpecialCharTok{\textasciitilde{}}\NormalTok{sample\_size)}\SpecialCharTok{+}\FunctionTok{xlim}\NormalTok{(}\SpecialCharTok{{-}}\DecValTok{3}\NormalTok{,}\DecValTok{3}\NormalTok{)}\SpecialCharTok{+}
  \FunctionTok{theme\_classic}\NormalTok{()}\SpecialCharTok{+}\FunctionTok{ggtitle}\NormalTok{(}\StringTok{"Population (red), Samples (grey), }\SpecialCharTok{\textbackslash{}n}\StringTok{ and Sampling distribution of the mean (blue)"}\NormalTok{)}\SpecialCharTok{+}\FunctionTok{ylab}\NormalTok{(}\StringTok{"Rough likelihoods"}\NormalTok{)}\SpecialCharTok{+}
  \FunctionTok{xlab}\NormalTok{(}\StringTok{"value"}\NormalTok{)}\SpecialCharTok{+}
  \FunctionTok{transition\_states}\NormalTok{(}
\NormalTok{    sims,}
    \AttributeTok{transition\_length =} \DecValTok{2}\NormalTok{,}
    \AttributeTok{state\_length =} \DecValTok{1}
\NormalTok{  )}\SpecialCharTok{+}\FunctionTok{enter\_fade}\NormalTok{() }\SpecialCharTok{+} 
  \FunctionTok{exit\_shrink}\NormalTok{() }\SpecialCharTok{+}
  \FunctionTok{ease\_aes}\NormalTok{(}\StringTok{\textquotesingle{}sine{-}in{-}out\textquotesingle{}}\NormalTok{)}
\end{Highlighting}
\end{Shaded}

\subsection{Null and True effect samples and sampling
means}\label{null-and-true-effect-samples-and-sampling-means}

The null dots show 50 different samples, with the red dot as the mean
for each sample. Null dots are all sampled from normal (u=0, sd=1). The
true dots show 50 more samples, with red dots for their means. However,
the mean of the true shifts between -1.5 and +1.5 standard deviations of
0. This illustrates how a true effect moves in and out of the null
range.

\begin{Shaded}
\begin{Highlighting}[]
\NormalTok{all\_df}\OtherTok{\textless{}{-}}\FunctionTok{data.frame}\NormalTok{()}
\NormalTok{all\_df\_means}\OtherTok{\textless{}{-}}\FunctionTok{data.frame}\NormalTok{()}
\NormalTok{dif\_sim}\OtherTok{\textless{}{-}}\FunctionTok{seq}\NormalTok{(}\SpecialCharTok{{-}}\FloatTok{1.5}\NormalTok{,}\FloatTok{1.5}\NormalTok{,.}\DecValTok{25}\NormalTok{)}
\ControlFlowTok{for}\NormalTok{(sim }\ControlFlowTok{in} \DecValTok{1}\SpecialCharTok{:}\DecValTok{13}\NormalTok{)\{}
\NormalTok{  values}\OtherTok{\textless{}{-}}\FunctionTok{c}\NormalTok{(}\FunctionTok{rnorm}\NormalTok{(}\DecValTok{25}\SpecialCharTok{*}\DecValTok{25}\NormalTok{,}\DecValTok{0}\NormalTok{,}\DecValTok{1}\NormalTok{),}\FunctionTok{rnorm}\NormalTok{(}\DecValTok{25}\SpecialCharTok{*}\DecValTok{25}\NormalTok{,dif\_sim[sim],}\DecValTok{1}\NormalTok{))}
\NormalTok{  samples}\OtherTok{\textless{}{-}}\FunctionTok{c}\NormalTok{(}\FunctionTok{rep}\NormalTok{(}\FunctionTok{seq}\NormalTok{(}\DecValTok{1}\SpecialCharTok{:}\DecValTok{25}\NormalTok{),}\AttributeTok{each=}\DecValTok{25}\NormalTok{),}\FunctionTok{rep}\NormalTok{(}\FunctionTok{seq}\NormalTok{(}\DecValTok{1}\SpecialCharTok{:}\DecValTok{25}\NormalTok{),}\AttributeTok{each=}\DecValTok{25}\NormalTok{))}
\NormalTok{  df}\OtherTok{\textless{}{-}}\FunctionTok{data.frame}\NormalTok{(samples,values,}\AttributeTok{sims=}\FunctionTok{rep}\NormalTok{(sim,}\DecValTok{50}\SpecialCharTok{*}\DecValTok{25}\NormalTok{),}\AttributeTok{type=}\FunctionTok{rep}\NormalTok{(}\FunctionTok{c}\NormalTok{(}\StringTok{"null"}\NormalTok{,}\StringTok{"true"}\NormalTok{),}\AttributeTok{each=}\DecValTok{625}\NormalTok{))}
\NormalTok{  df\_means}\OtherTok{\textless{}{-}}\FunctionTok{aggregate}\NormalTok{(values}\SpecialCharTok{\textasciitilde{}}\NormalTok{samples}\SpecialCharTok{*}\NormalTok{type,df,mean, }\AttributeTok{sims=}\FunctionTok{rep}\NormalTok{(sim,}\DecValTok{50}\NormalTok{))}
\NormalTok{  all\_df}\OtherTok{\textless{}{-}}\FunctionTok{rbind}\NormalTok{(all\_df,df)}
\NormalTok{  all\_df\_means}\OtherTok{\textless{}{-}}\FunctionTok{rbind}\NormalTok{(all\_df\_means,df\_means)}
\NormalTok{\}}

\NormalTok{all\_df}\OtherTok{\textless{}{-}}\FunctionTok{cbind}\NormalTok{(all\_df,}\AttributeTok{means=}\FunctionTok{rep}\NormalTok{(all\_df\_means}\SpecialCharTok{$}\NormalTok{values,}\AttributeTok{each=}\DecValTok{25}\NormalTok{))}

\FunctionTok{ggplot}\NormalTok{(all\_df,}\FunctionTok{aes}\NormalTok{(}\AttributeTok{y=}\NormalTok{values,}\AttributeTok{x=}\NormalTok{samples))}\SpecialCharTok{+}
  \FunctionTok{geom\_point}\NormalTok{(}\FunctionTok{aes}\NormalTok{(}\AttributeTok{color=}\FunctionTok{abs}\NormalTok{(values)), }\AttributeTok{alpha=}\NormalTok{.}\DecValTok{25}\NormalTok{)}\SpecialCharTok{+}
  \FunctionTok{geom\_point}\NormalTok{(}\FunctionTok{aes}\NormalTok{(}\AttributeTok{y=}\NormalTok{means,}\AttributeTok{x=}\NormalTok{samples),}\AttributeTok{color=}\StringTok{"red"}\NormalTok{)}\SpecialCharTok{+}
  \FunctionTok{theme\_classic}\NormalTok{()}\SpecialCharTok{+}
  \FunctionTok{geom\_vline}\NormalTok{(}\AttributeTok{xintercept=}\FloatTok{25.5}\NormalTok{)}\SpecialCharTok{+}
  \FunctionTok{facet\_wrap}\NormalTok{(}\SpecialCharTok{\textasciitilde{}}\NormalTok{type)}\SpecialCharTok{+}
  \FunctionTok{geom\_hline}\NormalTok{(}\AttributeTok{yintercept=}\DecValTok{0}\NormalTok{)}\SpecialCharTok{+}
  \FunctionTok{theme}\NormalTok{(}\AttributeTok{legend.position=}\StringTok{"none"}\NormalTok{) }\SpecialCharTok{+}
  \FunctionTok{ggtitle}\NormalTok{(}\StringTok{"null=0, True effect moves from {-}1.5 sd to 1.5 sd"}\NormalTok{)}\SpecialCharTok{+}
  \FunctionTok{transition\_states}\NormalTok{(}
\NormalTok{    sims,}
    \AttributeTok{transition\_length =} \DecValTok{2}\NormalTok{,}
    \AttributeTok{state\_length =} \DecValTok{1}
\NormalTok{  )}\SpecialCharTok{+}\FunctionTok{enter\_fade}\NormalTok{() }\SpecialCharTok{+} 
  \FunctionTok{exit\_shrink}\NormalTok{() }\SpecialCharTok{+}
  \FunctionTok{ease\_aes}\NormalTok{(}\StringTok{\textquotesingle{}sine{-}in{-}out\textquotesingle{}}\NormalTok{)}
\end{Highlighting}
\end{Shaded}

\section{Statistical Inference}\label{statistical-inference}

\subsection{Randomization Test}\label{randomization-test}

This is an attempt at visualizing a randomization test. Samples are
taken under two conditions of the IV (A and B). At the beginning of the
animation, the original scores in the first condition are shown as green
dots on the left, and the original scores in the second condition are
the red dots on the right. The means for each group are the purple dots.
During the randomization, the original scores are shuffled randomly
between the two conditions. After each shuffle, two new means are
computed and displayed as the yellow dots. This occurs either for all
permutations, or for a large random sample of them. The animation shows
the original scores being shuffled around across the randomizations (the
colored dots switch their original condition, appearing from side to
side).

For intuitive inference, one might look at the range of motion of the
yellow dots. This is how the mean difference between group 1 and group 2
behaves under randomization. It's what chance can do. If the difference
between the purple dots is well outside the range of motion of the
yellow dots, then the mean difference observed in the beginning is not
likely produced by chance.

\begin{Shaded}
\begin{Highlighting}[]
\NormalTok{study}\OtherTok{\textless{}{-}}\FunctionTok{round}\NormalTok{(}\FunctionTok{runif}\NormalTok{(}\DecValTok{10}\NormalTok{,}\DecValTok{80}\NormalTok{,}\DecValTok{100}\NormalTok{))}
\NormalTok{no\_study}\OtherTok{\textless{}{-}}\FunctionTok{round}\NormalTok{(}\FunctionTok{runif}\NormalTok{(}\DecValTok{10}\NormalTok{,}\DecValTok{40}\NormalTok{,}\DecValTok{90}\NormalTok{))}

\NormalTok{study\_df}\OtherTok{\textless{}{-}}\FunctionTok{data.frame}\NormalTok{(}\AttributeTok{student=}\FunctionTok{seq}\NormalTok{(}\DecValTok{1}\SpecialCharTok{:}\DecValTok{10}\NormalTok{),study,no\_study)}
\NormalTok{mean\_original}\OtherTok{\textless{}{-}}\FunctionTok{data.frame}\NormalTok{(}\AttributeTok{IV=}\FunctionTok{c}\NormalTok{(}\StringTok{"studied"}\NormalTok{,}\StringTok{"didnt\_study"}\NormalTok{),}
                          \AttributeTok{means=}\FunctionTok{c}\NormalTok{(}\FunctionTok{mean}\NormalTok{(study),}\FunctionTok{mean}\NormalTok{(no\_study)))}
\NormalTok{t\_df}\OtherTok{\textless{}{-}}\FunctionTok{data.frame}\NormalTok{(}\AttributeTok{sims=}\FunctionTok{rep}\NormalTok{(}\DecValTok{1}\NormalTok{,}\DecValTok{20}\NormalTok{),}
                 \AttributeTok{IV=}\FunctionTok{rep}\NormalTok{(}\FunctionTok{c}\NormalTok{(}\StringTok{"studied"}\NormalTok{,}\StringTok{"didnt\_study"}\NormalTok{),}\AttributeTok{each=}\DecValTok{10}\NormalTok{),}
                 \AttributeTok{values=}\FunctionTok{c}\NormalTok{(study,no\_study),}
                 \AttributeTok{rand\_order=}\FunctionTok{rep}\NormalTok{(}\FunctionTok{c}\NormalTok{(}\DecValTok{0}\NormalTok{,}\DecValTok{1}\NormalTok{),}\AttributeTok{each=}\DecValTok{10}\NormalTok{))}

\NormalTok{raw\_df}\OtherTok{\textless{}{-}}\NormalTok{t\_df}
\ControlFlowTok{for}\NormalTok{(i }\ControlFlowTok{in} \DecValTok{2}\SpecialCharTok{:}\DecValTok{10}\NormalTok{)\{}
\NormalTok{  new\_index}\OtherTok{\textless{}{-}}\FunctionTok{sample}\NormalTok{(}\DecValTok{1}\SpecialCharTok{:}\DecValTok{20}\NormalTok{)}
\NormalTok{  t\_df}\SpecialCharTok{$}\NormalTok{values}\OtherTok{\textless{}{-}}\NormalTok{t\_df}\SpecialCharTok{$}\NormalTok{values[new\_index]}
\NormalTok{  t\_df}\SpecialCharTok{$}\NormalTok{rand\_order}\OtherTok{\textless{}{-}}\NormalTok{t\_df}\SpecialCharTok{$}\NormalTok{rand\_order[new\_index]}
\NormalTok{  t\_df}\SpecialCharTok{$}\NormalTok{sims}\OtherTok{\textless{}{-}}\FunctionTok{rep}\NormalTok{(i,}\DecValTok{20}\NormalTok{)}
\NormalTok{  raw\_df}\OtherTok{\textless{}{-}}\FunctionTok{rbind}\NormalTok{(raw\_df,t\_df)}
\NormalTok{\}}

\NormalTok{raw\_df}\SpecialCharTok{$}\NormalTok{rand\_order}\OtherTok{\textless{}{-}}\FunctionTok{as.factor}\NormalTok{(raw\_df}\SpecialCharTok{$}\NormalTok{rand\_order)}
\NormalTok{rand\_df}\OtherTok{\textless{}{-}}\FunctionTok{aggregate}\NormalTok{(values}\SpecialCharTok{\textasciitilde{}}\NormalTok{sims}\SpecialCharTok{*}\NormalTok{IV,raw\_df,mean)}
\FunctionTok{names}\NormalTok{(rand\_df)}\OtherTok{\textless{}{-}}\FunctionTok{c}\NormalTok{(}\StringTok{"sims"}\NormalTok{,}\StringTok{"IV"}\NormalTok{,}\StringTok{"means"}\NormalTok{)}


\NormalTok{a}\OtherTok{\textless{}{-}}\FunctionTok{ggplot}\NormalTok{(raw\_df,}\FunctionTok{aes}\NormalTok{(}\AttributeTok{x=}\NormalTok{IV,}\AttributeTok{y=}\NormalTok{values,}\AttributeTok{color=}\NormalTok{rand\_order,}\AttributeTok{size=}\DecValTok{3}\NormalTok{))}\SpecialCharTok{+}
  \FunctionTok{geom\_point}\NormalTok{(}\AttributeTok{stat=}\StringTok{"identity"}\NormalTok{,}\AttributeTok{alpha=}\NormalTok{.}\DecValTok{5}\NormalTok{)}\SpecialCharTok{+}
  \FunctionTok{geom\_point}\NormalTok{(}\AttributeTok{data=}\NormalTok{mean\_original,}\FunctionTok{aes}\NormalTok{(}\AttributeTok{x=}\NormalTok{IV,}\AttributeTok{y=}\NormalTok{means),}\AttributeTok{stat=}\StringTok{"identity"}\NormalTok{,}\AttributeTok{shape=}\DecValTok{21}\NormalTok{,}\AttributeTok{size=}\DecValTok{6}\NormalTok{,}\AttributeTok{color=}\StringTok{"black"}\NormalTok{,}\AttributeTok{fill=}\StringTok{"mediumorchid2"}\NormalTok{)}\SpecialCharTok{+}
  \FunctionTok{geom\_point}\NormalTok{(}\AttributeTok{data=}\NormalTok{rand\_df,}\FunctionTok{aes}\NormalTok{(}\AttributeTok{x=}\NormalTok{IV,}\AttributeTok{y=}\NormalTok{means),}\AttributeTok{stat=}\StringTok{"identity"}\NormalTok{,}\AttributeTok{shape=}\DecValTok{21}\NormalTok{,}\AttributeTok{size=}\DecValTok{6}\NormalTok{,}\AttributeTok{color=}\StringTok{"black"}\NormalTok{,}\AttributeTok{fill=}\StringTok{"gold"}\NormalTok{)}\SpecialCharTok{+}
  \FunctionTok{theme\_classic}\NormalTok{(}\AttributeTok{base\_size =} \DecValTok{15}\NormalTok{)}\SpecialCharTok{+}
  \FunctionTok{coord\_cartesian}\NormalTok{(}\AttributeTok{ylim=}\FunctionTok{c}\NormalTok{(}\DecValTok{40}\NormalTok{, }\DecValTok{100}\NormalTok{))}\SpecialCharTok{+}
  \FunctionTok{theme}\NormalTok{(}\AttributeTok{legend.position=}\StringTok{"none"}\NormalTok{) }\SpecialCharTok{+}
  \FunctionTok{ggtitle}\NormalTok{(}\StringTok{"Randomization test: Original Means (purple), }
\StringTok{          }\SpecialCharTok{\textbackslash{}n}\StringTok{ Randomized means (yellow)}
\StringTok{          }\SpecialCharTok{\textbackslash{}n}\StringTok{ Original scores (red,greenish)"}\NormalTok{)}\SpecialCharTok{+}
  \FunctionTok{transition\_states}\NormalTok{(}
\NormalTok{    sims,}
    \AttributeTok{transition\_length =} \DecValTok{1}\NormalTok{,}
    \AttributeTok{state\_length =} \DecValTok{2}
\NormalTok{  )}\SpecialCharTok{+}\FunctionTok{enter\_fade}\NormalTok{() }\SpecialCharTok{+} 
  \FunctionTok{exit\_shrink}\NormalTok{() }\SpecialCharTok{+}
  \FunctionTok{ease\_aes}\NormalTok{(}\StringTok{\textquotesingle{}sine{-}in{-}out\textquotesingle{}}\NormalTok{)}

\FunctionTok{animate}\NormalTok{(a,}\AttributeTok{nframes=}\DecValTok{100}\NormalTok{,}\AttributeTok{fps=}\DecValTok{5}\NormalTok{)}
\end{Highlighting}
\end{Shaded}

\subsection{Independent t-test Null}\label{independent-t-test-null}

This is a simulation of the null distribution for an independent samples
t-test, two groups, 10 observations per group.

This animation has two panels. The left panel shows means for group A
and B, sampled from the same normal distribution (mu=50, sd =10). The
dots represent individual scores for each of 10 observations per group.

The right panel shows a t-distribution (df=18) along with the observed
t-statistic for each simulation.

\texttt{gganimate} does not yet directly support multiple panels as
shown in this gif. I hacked together these two gifs using the
\texttt{magick} package. Apologies for the hackiness.

\begin{Shaded}
\begin{Highlighting}[]
\FunctionTok{library}\NormalTok{(dplyr)}
\FunctionTok{library}\NormalTok{(ggplot2)}
\FunctionTok{library}\NormalTok{(magick)}
\FunctionTok{library}\NormalTok{(gganimate)}

\NormalTok{A}\OtherTok{\textless{}{-}}\FunctionTok{rnorm}\NormalTok{(}\DecValTok{100}\NormalTok{,}\DecValTok{50}\NormalTok{,}\DecValTok{10}\NormalTok{)}
\NormalTok{B}\OtherTok{\textless{}{-}}\FunctionTok{rnorm}\NormalTok{(}\DecValTok{100}\NormalTok{,}\DecValTok{50}\NormalTok{,}\DecValTok{10}\NormalTok{)}
\NormalTok{DV }\OtherTok{\textless{}{-}} \FunctionTok{c}\NormalTok{(A,B)}
\NormalTok{IV }\OtherTok{\textless{}{-}} \FunctionTok{rep}\NormalTok{(}\FunctionTok{c}\NormalTok{(}\StringTok{"A"}\NormalTok{,}\StringTok{"B"}\NormalTok{),}\AttributeTok{each=}\DecValTok{100}\NormalTok{)}
\NormalTok{sims }\OtherTok{\textless{}{-}} \FunctionTok{rep}\NormalTok{(}\FunctionTok{rep}\NormalTok{(}\DecValTok{1}\SpecialCharTok{:}\DecValTok{10}\NormalTok{,}\AttributeTok{each=}\DecValTok{10}\NormalTok{),}\DecValTok{2}\NormalTok{)}
\NormalTok{df}\OtherTok{\textless{}{-}}\FunctionTok{data.frame}\NormalTok{(sims,IV,DV)}

\NormalTok{means\_df }\OtherTok{\textless{}{-}}\NormalTok{ df }\SpecialCharTok{\%\textgreater{}\%}
               \FunctionTok{group\_by}\NormalTok{(sims,IV) }\SpecialCharTok{\%\textgreater{}\%}
               \FunctionTok{summarize}\NormalTok{(}\AttributeTok{means=}\FunctionTok{mean}\NormalTok{(DV),}
                         \AttributeTok{sem =} \FunctionTok{sd}\NormalTok{(DV)}\SpecialCharTok{/}\FunctionTok{sqrt}\NormalTok{(}\FunctionTok{length}\NormalTok{(DV)))}

\NormalTok{stats\_df }\OtherTok{\textless{}{-}}\NormalTok{ df }\SpecialCharTok{\%\textgreater{}\%}
              \FunctionTok{group\_by}\NormalTok{(sims) }\SpecialCharTok{\%\textgreater{}\%}
              \FunctionTok{summarize}\NormalTok{(}\AttributeTok{ts =} \FunctionTok{t.test}\NormalTok{(DV}\SpecialCharTok{\textasciitilde{}}\NormalTok{IV,}\AttributeTok{var.equal=}\ConstantTok{TRUE}\NormalTok{)}\SpecialCharTok{$}\NormalTok{statistic)}

\NormalTok{a}\OtherTok{\textless{}{-}}\FunctionTok{ggplot}\NormalTok{(means\_df, }\FunctionTok{aes}\NormalTok{(}\AttributeTok{x=}\NormalTok{IV,}\AttributeTok{y=}\NormalTok{means, }\AttributeTok{fill=}\NormalTok{IV))}\SpecialCharTok{+}
  \FunctionTok{geom\_bar}\NormalTok{(}\AttributeTok{stat=}\StringTok{"identity"}\NormalTok{)}\SpecialCharTok{+}
  \FunctionTok{geom\_point}\NormalTok{(}\AttributeTok{data=}\NormalTok{df,}\FunctionTok{aes}\NormalTok{(}\AttributeTok{x=}\NormalTok{IV, }\AttributeTok{y=}\NormalTok{DV), }\AttributeTok{alpha=}\NormalTok{.}\DecValTok{25}\NormalTok{)}\SpecialCharTok{+}
  \FunctionTok{geom\_errorbar}\NormalTok{(}\FunctionTok{aes}\NormalTok{(}\AttributeTok{ymin=}\NormalTok{means}\SpecialCharTok{{-}}\NormalTok{sem, }\AttributeTok{ymax=}\NormalTok{means}\SpecialCharTok{+}\NormalTok{sem),}\AttributeTok{width=}\NormalTok{.}\DecValTok{2}\NormalTok{)}\SpecialCharTok{+}
  \FunctionTok{theme\_classic}\NormalTok{()}\SpecialCharTok{+}
  \FunctionTok{transition\_states}\NormalTok{(}
    \AttributeTok{states=}\NormalTok{sims,}
    \AttributeTok{transition\_length =} \DecValTok{2}\NormalTok{,}
    \AttributeTok{state\_length =} \DecValTok{1}
\NormalTok{  )}\SpecialCharTok{+}\FunctionTok{enter\_fade}\NormalTok{() }\SpecialCharTok{+} 
  \FunctionTok{exit\_shrink}\NormalTok{() }\SpecialCharTok{+}
  \FunctionTok{ease\_aes}\NormalTok{(}\StringTok{\textquotesingle{}sine{-}in{-}out\textquotesingle{}}\NormalTok{)}
  
\NormalTok{a\_gif}\OtherTok{\textless{}{-}}\FunctionTok{animate}\NormalTok{(a, }\AttributeTok{width =} \DecValTok{240}\NormalTok{, }\AttributeTok{height =} \DecValTok{240}\NormalTok{)}

\NormalTok{b}\OtherTok{\textless{}{-}}\FunctionTok{ggplot}\NormalTok{(stats\_df,}\FunctionTok{aes}\NormalTok{(}\AttributeTok{x=}\NormalTok{ts))}\SpecialCharTok{+}
  \FunctionTok{geom\_vline}\NormalTok{(}\FunctionTok{aes}\NormalTok{(}\AttributeTok{xintercept=}\NormalTok{ts, }\AttributeTok{frame=}\NormalTok{sims))}\SpecialCharTok{+}
  \FunctionTok{geom\_line}\NormalTok{(}\AttributeTok{data=}\FunctionTok{data.frame}\NormalTok{(}\AttributeTok{x=}\FunctionTok{seq}\NormalTok{(}\SpecialCharTok{{-}}\DecValTok{5}\NormalTok{,}\DecValTok{5}\NormalTok{,.}\DecValTok{1}\NormalTok{),}
                            \AttributeTok{y=}\FunctionTok{dt}\NormalTok{(}\FunctionTok{seq}\NormalTok{(}\SpecialCharTok{{-}}\DecValTok{5}\NormalTok{,}\DecValTok{5}\NormalTok{,.}\DecValTok{1}\NormalTok{),}\AttributeTok{df=}\DecValTok{18}\NormalTok{)),}
            \FunctionTok{aes}\NormalTok{(}\AttributeTok{x=}\NormalTok{x,}\AttributeTok{y=}\NormalTok{y))}\SpecialCharTok{+}
  \FunctionTok{theme\_classic}\NormalTok{()}\SpecialCharTok{+}
  \FunctionTok{ylab}\NormalTok{(}\StringTok{"density"}\NormalTok{)}\SpecialCharTok{+}
  \FunctionTok{xlab}\NormalTok{(}\StringTok{"t value"}\NormalTok{)}\SpecialCharTok{+}
  \FunctionTok{transition\_states}\NormalTok{(}
    \AttributeTok{states=}\NormalTok{sims,}
    \AttributeTok{transition\_length =} \DecValTok{2}\NormalTok{,}
    \AttributeTok{state\_length =} \DecValTok{1}
\NormalTok{  )}\SpecialCharTok{+}\FunctionTok{enter\_fade}\NormalTok{() }\SpecialCharTok{+} 
  \FunctionTok{exit\_shrink}\NormalTok{() }\SpecialCharTok{+}
  \FunctionTok{ease\_aes}\NormalTok{(}\StringTok{\textquotesingle{}sine{-}in{-}out\textquotesingle{}}\NormalTok{)}

\NormalTok{b\_gif}\OtherTok{\textless{}{-}}\FunctionTok{animate}\NormalTok{(b, }\AttributeTok{width =} \DecValTok{240}\NormalTok{, }\AttributeTok{height =} \DecValTok{240}\NormalTok{)}


\NormalTok{d}\OtherTok{\textless{}{-}}\FunctionTok{image\_blank}\NormalTok{(}\DecValTok{240}\SpecialCharTok{*}\DecValTok{2}\NormalTok{,}\DecValTok{240}\NormalTok{)}

\NormalTok{the\_frame}\OtherTok{\textless{}{-}}\NormalTok{d}
\ControlFlowTok{for}\NormalTok{(i }\ControlFlowTok{in} \DecValTok{2}\SpecialCharTok{:}\DecValTok{100}\NormalTok{)\{}
\NormalTok{  the\_frame}\OtherTok{\textless{}{-}}\FunctionTok{c}\NormalTok{(the\_frame,d)}
\NormalTok{\}}

\NormalTok{a\_mgif}\OtherTok{\textless{}{-}}\FunctionTok{image\_read}\NormalTok{(a\_gif)}
\NormalTok{b\_mgif}\OtherTok{\textless{}{-}}\FunctionTok{image\_read}\NormalTok{(b\_gif)}

\NormalTok{new\_gif}\OtherTok{\textless{}{-}}\FunctionTok{image\_append}\NormalTok{(}\FunctionTok{c}\NormalTok{(a\_mgif[}\DecValTok{1}\NormalTok{], b\_mgif[}\DecValTok{1}\NormalTok{]))}
\ControlFlowTok{for}\NormalTok{(i }\ControlFlowTok{in} \DecValTok{2}\SpecialCharTok{:}\DecValTok{100}\NormalTok{)\{}
\NormalTok{  combined }\OtherTok{\textless{}{-}} \FunctionTok{image\_append}\NormalTok{(}\FunctionTok{c}\NormalTok{(a\_mgif[i], b\_mgif[i]))}
\NormalTok{  new\_gif}\OtherTok{\textless{}{-}}\FunctionTok{c}\NormalTok{(new\_gif,combined)}
\NormalTok{\}}

\NormalTok{new\_gif}
\end{Highlighting}
\end{Shaded}

\subsection{Independent t-test True}\label{independent-t-test-true}

This is a simulation of an independent samples t-test, two groups, 10
observations per group, assuming a true difference of 2 standard
deviations between groups

This animation has two panels. The left panel shows means for group A
(normal, mu=50, sd=10) and B (normal, mu=70, sd=10). The dots represent
individual scores for each of 10 observations per group.

The right panel shows a t-distribution (df=18) along with the observed
t-statistic for each simulation.

\begin{Shaded}
\begin{Highlighting}[]
\FunctionTok{library}\NormalTok{(dplyr)}
\FunctionTok{library}\NormalTok{(ggplot2)}
\FunctionTok{library}\NormalTok{(magick)}
\FunctionTok{library}\NormalTok{(gganimate)}

\NormalTok{A}\OtherTok{\textless{}{-}}\FunctionTok{rnorm}\NormalTok{(}\DecValTok{100}\NormalTok{,}\DecValTok{70}\NormalTok{,}\DecValTok{10}\NormalTok{)}
\NormalTok{B}\OtherTok{\textless{}{-}}\FunctionTok{rnorm}\NormalTok{(}\DecValTok{100}\NormalTok{,}\DecValTok{50}\NormalTok{,}\DecValTok{10}\NormalTok{)}
\NormalTok{DV }\OtherTok{\textless{}{-}} \FunctionTok{c}\NormalTok{(A,B)}
\NormalTok{IV }\OtherTok{\textless{}{-}} \FunctionTok{rep}\NormalTok{(}\FunctionTok{c}\NormalTok{(}\StringTok{"A"}\NormalTok{,}\StringTok{"B"}\NormalTok{),}\AttributeTok{each=}\DecValTok{100}\NormalTok{)}
\NormalTok{sims }\OtherTok{\textless{}{-}} \FunctionTok{rep}\NormalTok{(}\FunctionTok{rep}\NormalTok{(}\DecValTok{1}\SpecialCharTok{:}\DecValTok{10}\NormalTok{,}\AttributeTok{each=}\DecValTok{10}\NormalTok{),}\DecValTok{2}\NormalTok{)}
\NormalTok{df}\OtherTok{\textless{}{-}}\FunctionTok{data.frame}\NormalTok{(sims,IV,DV)}

\NormalTok{means\_df }\OtherTok{\textless{}{-}}\NormalTok{ df }\SpecialCharTok{\%\textgreater{}\%}
               \FunctionTok{group\_by}\NormalTok{(sims,IV) }\SpecialCharTok{\%\textgreater{}\%}
               \FunctionTok{summarize}\NormalTok{(}\AttributeTok{means=}\FunctionTok{mean}\NormalTok{(DV),}
                         \AttributeTok{sem =} \FunctionTok{sd}\NormalTok{(DV)}\SpecialCharTok{/}\FunctionTok{sqrt}\NormalTok{(}\FunctionTok{length}\NormalTok{(DV)))}

\NormalTok{stats\_df }\OtherTok{\textless{}{-}}\NormalTok{ df }\SpecialCharTok{\%\textgreater{}\%}
              \FunctionTok{group\_by}\NormalTok{(sims) }\SpecialCharTok{\%\textgreater{}\%}
              \FunctionTok{summarize}\NormalTok{(}\AttributeTok{ts =} \FunctionTok{t.test}\NormalTok{(DV}\SpecialCharTok{\textasciitilde{}}\NormalTok{IV,}\AttributeTok{var.equal=}\ConstantTok{TRUE}\NormalTok{)}\SpecialCharTok{$}\NormalTok{statistic)}

\NormalTok{a}\OtherTok{\textless{}{-}}\FunctionTok{ggplot}\NormalTok{(means\_df, }\FunctionTok{aes}\NormalTok{(}\AttributeTok{x=}\NormalTok{IV,}\AttributeTok{y=}\NormalTok{means, }\AttributeTok{fill=}\NormalTok{IV))}\SpecialCharTok{+}
  \FunctionTok{geom\_bar}\NormalTok{(}\AttributeTok{stat=}\StringTok{"identity"}\NormalTok{)}\SpecialCharTok{+}
  \FunctionTok{geom\_point}\NormalTok{(}\AttributeTok{data=}\NormalTok{df,}\FunctionTok{aes}\NormalTok{(}\AttributeTok{x=}\NormalTok{IV, }\AttributeTok{y=}\NormalTok{DV), }\AttributeTok{alpha=}\NormalTok{.}\DecValTok{25}\NormalTok{)}\SpecialCharTok{+}
  \FunctionTok{geom\_errorbar}\NormalTok{(}\FunctionTok{aes}\NormalTok{(}\AttributeTok{ymin=}\NormalTok{means}\SpecialCharTok{{-}}\NormalTok{sem, }\AttributeTok{ymax=}\NormalTok{means}\SpecialCharTok{+}\NormalTok{sem),}\AttributeTok{width=}\NormalTok{.}\DecValTok{2}\NormalTok{)}\SpecialCharTok{+}
  \FunctionTok{theme\_classic}\NormalTok{()}\SpecialCharTok{+}
  \FunctionTok{transition\_states}\NormalTok{(}
    \AttributeTok{states=}\NormalTok{sims,}
    \AttributeTok{transition\_length =} \DecValTok{2}\NormalTok{,}
    \AttributeTok{state\_length =} \DecValTok{1}
\NormalTok{  )}\SpecialCharTok{+}\FunctionTok{enter\_fade}\NormalTok{() }\SpecialCharTok{+} 
  \FunctionTok{exit\_shrink}\NormalTok{() }\SpecialCharTok{+}
  \FunctionTok{ease\_aes}\NormalTok{(}\StringTok{\textquotesingle{}sine{-}in{-}out\textquotesingle{}}\NormalTok{)}
  
\NormalTok{a\_gif}\OtherTok{\textless{}{-}}\FunctionTok{animate}\NormalTok{(a, }\AttributeTok{width =} \DecValTok{240}\NormalTok{, }\AttributeTok{height =} \DecValTok{240}\NormalTok{)}

\NormalTok{b}\OtherTok{\textless{}{-}}\FunctionTok{ggplot}\NormalTok{(stats\_df,}\FunctionTok{aes}\NormalTok{(}\AttributeTok{x=}\NormalTok{ts))}\SpecialCharTok{+}
  \FunctionTok{geom\_vline}\NormalTok{(}\FunctionTok{aes}\NormalTok{(}\AttributeTok{xintercept=}\NormalTok{ts, }\AttributeTok{frame=}\NormalTok{sims))}\SpecialCharTok{+}
  \FunctionTok{geom\_vline}\NormalTok{(}\AttributeTok{xintercept=}\FunctionTok{qt}\NormalTok{(}\FunctionTok{c}\NormalTok{(.}\DecValTok{025}\NormalTok{, .}\DecValTok{975}\NormalTok{), }\AttributeTok{df=}\DecValTok{18}\NormalTok{),}\AttributeTok{color=}\StringTok{"green"}\NormalTok{)}\SpecialCharTok{+}
  \FunctionTok{geom\_line}\NormalTok{(}\AttributeTok{data=}\FunctionTok{data.frame}\NormalTok{(}\AttributeTok{x=}\FunctionTok{seq}\NormalTok{(}\SpecialCharTok{{-}}\DecValTok{5}\NormalTok{,}\DecValTok{5}\NormalTok{,.}\DecValTok{1}\NormalTok{),}
                            \AttributeTok{y=}\FunctionTok{dt}\NormalTok{(}\FunctionTok{seq}\NormalTok{(}\SpecialCharTok{{-}}\DecValTok{5}\NormalTok{,}\DecValTok{5}\NormalTok{,.}\DecValTok{1}\NormalTok{),}\AttributeTok{df=}\DecValTok{18}\NormalTok{)),}
            \FunctionTok{aes}\NormalTok{(}\AttributeTok{x=}\NormalTok{x,}\AttributeTok{y=}\NormalTok{y))}\SpecialCharTok{+}
  \FunctionTok{theme\_classic}\NormalTok{()}\SpecialCharTok{+}
  \FunctionTok{ylab}\NormalTok{(}\StringTok{"density"}\NormalTok{)}\SpecialCharTok{+}
  \FunctionTok{xlab}\NormalTok{(}\StringTok{"t value"}\NormalTok{)}\SpecialCharTok{+}
  \FunctionTok{transition\_states}\NormalTok{(}
    \AttributeTok{states=}\NormalTok{sims,}
    \AttributeTok{transition\_length =} \DecValTok{2}\NormalTok{,}
    \AttributeTok{state\_length =} \DecValTok{1}
\NormalTok{  )}\SpecialCharTok{+}\FunctionTok{enter\_fade}\NormalTok{() }\SpecialCharTok{+} 
  \FunctionTok{exit\_shrink}\NormalTok{() }\SpecialCharTok{+}
  \FunctionTok{ease\_aes}\NormalTok{(}\StringTok{\textquotesingle{}sine{-}in{-}out\textquotesingle{}}\NormalTok{)}

\NormalTok{b\_gif}\OtherTok{\textless{}{-}}\FunctionTok{animate}\NormalTok{(b, }\AttributeTok{width =} \DecValTok{240}\NormalTok{, }\AttributeTok{height =} \DecValTok{240}\NormalTok{)}


\NormalTok{d}\OtherTok{\textless{}{-}}\FunctionTok{image\_blank}\NormalTok{(}\DecValTok{240}\SpecialCharTok{*}\DecValTok{2}\NormalTok{,}\DecValTok{240}\NormalTok{)}

\NormalTok{the\_frame}\OtherTok{\textless{}{-}}\NormalTok{d}
\ControlFlowTok{for}\NormalTok{(i }\ControlFlowTok{in} \DecValTok{2}\SpecialCharTok{:}\DecValTok{100}\NormalTok{)\{}
\NormalTok{  the\_frame}\OtherTok{\textless{}{-}}\FunctionTok{c}\NormalTok{(the\_frame,d)}
\NormalTok{\}}

\NormalTok{a\_mgif}\OtherTok{\textless{}{-}}\FunctionTok{image\_read}\NormalTok{(a\_gif)}
\NormalTok{b\_mgif}\OtherTok{\textless{}{-}}\FunctionTok{image\_read}\NormalTok{(b\_gif)}

\NormalTok{new\_gif}\OtherTok{\textless{}{-}}\FunctionTok{image\_append}\NormalTok{(}\FunctionTok{c}\NormalTok{(a\_mgif[}\DecValTok{1}\NormalTok{], b\_mgif[}\DecValTok{1}\NormalTok{]))}
\ControlFlowTok{for}\NormalTok{(i }\ControlFlowTok{in} \DecValTok{2}\SpecialCharTok{:}\DecValTok{100}\NormalTok{)\{}
\NormalTok{  combined }\OtherTok{\textless{}{-}} \FunctionTok{image\_append}\NormalTok{(}\FunctionTok{c}\NormalTok{(a\_mgif[i], b\_mgif[i]))}
\NormalTok{  new\_gif}\OtherTok{\textless{}{-}}\FunctionTok{c}\NormalTok{(new\_gif,combined)}
\NormalTok{\}}

\NormalTok{new\_gif}
\end{Highlighting}
\end{Shaded}

\subsection{T-test True sample-size}\label{t-test-true-sample-size}

The top row shows 10 simulations of an independent sample t-test, with
N=10, and true difference of 1 sd.

The bottom row shows 10 simulations with N=50.

The observed t-value occurs past the critical value (green) line much
more reliably and often when sample size is larger than smaller.

\begin{Shaded}
\begin{Highlighting}[]
\FunctionTok{library}\NormalTok{(dplyr)}
\FunctionTok{library}\NormalTok{(ggplot2)}
\FunctionTok{library}\NormalTok{(magick)}
\FunctionTok{library}\NormalTok{(gganimate)}

\NormalTok{A}\OtherTok{\textless{}{-}}\FunctionTok{rnorm}\NormalTok{(}\DecValTok{100}\NormalTok{,}\DecValTok{60}\NormalTok{,}\DecValTok{10}\NormalTok{)}
\NormalTok{B}\OtherTok{\textless{}{-}}\FunctionTok{rnorm}\NormalTok{(}\DecValTok{100}\NormalTok{,}\DecValTok{50}\NormalTok{,}\DecValTok{10}\NormalTok{)}
\NormalTok{DV }\OtherTok{\textless{}{-}} \FunctionTok{c}\NormalTok{(A,B)}
\NormalTok{IV }\OtherTok{\textless{}{-}} \FunctionTok{rep}\NormalTok{(}\FunctionTok{c}\NormalTok{(}\StringTok{"A"}\NormalTok{,}\StringTok{"B"}\NormalTok{),}\AttributeTok{each=}\DecValTok{100}\NormalTok{)}
\NormalTok{sims }\OtherTok{\textless{}{-}} \FunctionTok{rep}\NormalTok{(}\FunctionTok{rep}\NormalTok{(}\DecValTok{1}\SpecialCharTok{:}\DecValTok{10}\NormalTok{,}\AttributeTok{each=}\DecValTok{10}\NormalTok{),}\DecValTok{2}\NormalTok{)}
\NormalTok{df}\OtherTok{\textless{}{-}}\FunctionTok{data.frame}\NormalTok{(sims,IV,DV)}

\NormalTok{means\_df }\OtherTok{\textless{}{-}}\NormalTok{ df }\SpecialCharTok{\%\textgreater{}\%}
               \FunctionTok{group\_by}\NormalTok{(sims,IV) }\SpecialCharTok{\%\textgreater{}\%}
               \FunctionTok{summarize}\NormalTok{(}\AttributeTok{means=}\FunctionTok{mean}\NormalTok{(DV),}
                         \AttributeTok{sem =} \FunctionTok{sd}\NormalTok{(DV)}\SpecialCharTok{/}\FunctionTok{sqrt}\NormalTok{(}\FunctionTok{length}\NormalTok{(DV)))}

\NormalTok{stats\_df }\OtherTok{\textless{}{-}}\NormalTok{ df }\SpecialCharTok{\%\textgreater{}\%}
              \FunctionTok{group\_by}\NormalTok{(sims) }\SpecialCharTok{\%\textgreater{}\%}
              \FunctionTok{summarize}\NormalTok{(}\AttributeTok{ts =} \FunctionTok{t.test}\NormalTok{(DV}\SpecialCharTok{\textasciitilde{}}\NormalTok{IV,}\AttributeTok{var.equal=}\ConstantTok{TRUE}\NormalTok{)}\SpecialCharTok{$}\NormalTok{statistic)}

\NormalTok{a}\OtherTok{\textless{}{-}}\FunctionTok{ggplot}\NormalTok{(means\_df, }\FunctionTok{aes}\NormalTok{(}\AttributeTok{x=}\NormalTok{IV,}\AttributeTok{y=}\NormalTok{means, }\AttributeTok{fill=}\NormalTok{IV))}\SpecialCharTok{+}
  \FunctionTok{geom\_bar}\NormalTok{(}\AttributeTok{stat=}\StringTok{"identity"}\NormalTok{)}\SpecialCharTok{+}
  \FunctionTok{geom\_point}\NormalTok{(}\AttributeTok{data=}\NormalTok{df,}\FunctionTok{aes}\NormalTok{(}\AttributeTok{x=}\NormalTok{IV, }\AttributeTok{y=}\NormalTok{DV), }\AttributeTok{alpha=}\NormalTok{.}\DecValTok{25}\NormalTok{)}\SpecialCharTok{+}
  \FunctionTok{geom\_errorbar}\NormalTok{(}\FunctionTok{aes}\NormalTok{(}\AttributeTok{ymin=}\NormalTok{means}\SpecialCharTok{{-}}\NormalTok{sem, }\AttributeTok{ymax=}\NormalTok{means}\SpecialCharTok{+}\NormalTok{sem),}\AttributeTok{width=}\NormalTok{.}\DecValTok{2}\NormalTok{)}\SpecialCharTok{+}
  \FunctionTok{theme\_classic}\NormalTok{()}\SpecialCharTok{+}
  \FunctionTok{transition\_states}\NormalTok{(}
    \AttributeTok{states=}\NormalTok{sims,}
    \AttributeTok{transition\_length =} \DecValTok{2}\NormalTok{,}
    \AttributeTok{state\_length =} \DecValTok{1}
\NormalTok{  )}\SpecialCharTok{+}\FunctionTok{enter\_fade}\NormalTok{() }\SpecialCharTok{+} 
  \FunctionTok{exit\_shrink}\NormalTok{() }\SpecialCharTok{+}
  \FunctionTok{ease\_aes}\NormalTok{(}\StringTok{\textquotesingle{}sine{-}in{-}out\textquotesingle{}}\NormalTok{)}
  
\NormalTok{a\_gif}\OtherTok{\textless{}{-}}\FunctionTok{animate}\NormalTok{(a, }\AttributeTok{width =} \DecValTok{240}\NormalTok{, }\AttributeTok{height =} \DecValTok{240}\NormalTok{)}

\NormalTok{b}\OtherTok{\textless{}{-}}\FunctionTok{ggplot}\NormalTok{(stats\_df,}\FunctionTok{aes}\NormalTok{(}\AttributeTok{x=}\NormalTok{ts))}\SpecialCharTok{+}
  \FunctionTok{geom\_vline}\NormalTok{(}\FunctionTok{aes}\NormalTok{(}\AttributeTok{xintercept=}\NormalTok{ts, }\AttributeTok{frame=}\NormalTok{sims))}\SpecialCharTok{+}
  \FunctionTok{geom\_vline}\NormalTok{(}\AttributeTok{xintercept=}\FunctionTok{qt}\NormalTok{(}\FunctionTok{c}\NormalTok{(.}\DecValTok{025}\NormalTok{, .}\DecValTok{975}\NormalTok{), }\AttributeTok{df=}\DecValTok{18}\NormalTok{),}\AttributeTok{color=}\StringTok{"green"}\NormalTok{)}\SpecialCharTok{+}
  \FunctionTok{geom\_line}\NormalTok{(}\AttributeTok{data=}\FunctionTok{data.frame}\NormalTok{(}\AttributeTok{x=}\FunctionTok{seq}\NormalTok{(}\SpecialCharTok{{-}}\DecValTok{5}\NormalTok{,}\DecValTok{5}\NormalTok{,.}\DecValTok{1}\NormalTok{),}
                            \AttributeTok{y=}\FunctionTok{dt}\NormalTok{(}\FunctionTok{seq}\NormalTok{(}\SpecialCharTok{{-}}\DecValTok{5}\NormalTok{,}\DecValTok{5}\NormalTok{,.}\DecValTok{1}\NormalTok{),}\AttributeTok{df=}\DecValTok{18}\NormalTok{)),}
            \FunctionTok{aes}\NormalTok{(}\AttributeTok{x=}\NormalTok{x,}\AttributeTok{y=}\NormalTok{y))}\SpecialCharTok{+}
  \FunctionTok{theme\_classic}\NormalTok{()}\SpecialCharTok{+}
  \FunctionTok{ylab}\NormalTok{(}\StringTok{"density"}\NormalTok{)}\SpecialCharTok{+}
  \FunctionTok{xlab}\NormalTok{(}\StringTok{"t value"}\NormalTok{)}\SpecialCharTok{+}
  \FunctionTok{transition\_states}\NormalTok{(}
    \AttributeTok{states=}\NormalTok{sims,}
    \AttributeTok{transition\_length =} \DecValTok{2}\NormalTok{,}
    \AttributeTok{state\_length =} \DecValTok{1}
\NormalTok{  )}\SpecialCharTok{+}\FunctionTok{enter\_fade}\NormalTok{() }\SpecialCharTok{+} 
  \FunctionTok{exit\_shrink}\NormalTok{() }\SpecialCharTok{+}
  \FunctionTok{ease\_aes}\NormalTok{(}\StringTok{\textquotesingle{}sine{-}in{-}out\textquotesingle{}}\NormalTok{)}

\NormalTok{b\_gif}\OtherTok{\textless{}{-}}\FunctionTok{animate}\NormalTok{(b, }\AttributeTok{width =} \DecValTok{240}\NormalTok{, }\AttributeTok{height =} \DecValTok{240}\NormalTok{)}


\NormalTok{d}\OtherTok{\textless{}{-}}\FunctionTok{image\_blank}\NormalTok{(}\DecValTok{240}\SpecialCharTok{*}\DecValTok{2}\NormalTok{,}\DecValTok{240}\NormalTok{)}

\NormalTok{the\_frame}\OtherTok{\textless{}{-}}\NormalTok{d}
\ControlFlowTok{for}\NormalTok{(i }\ControlFlowTok{in} \DecValTok{2}\SpecialCharTok{:}\DecValTok{100}\NormalTok{)\{}
\NormalTok{  the\_frame}\OtherTok{\textless{}{-}}\FunctionTok{c}\NormalTok{(the\_frame,d)}
\NormalTok{\}}

\NormalTok{a\_mgif}\OtherTok{\textless{}{-}}\FunctionTok{image\_read}\NormalTok{(a\_gif)}
\NormalTok{b\_mgif}\OtherTok{\textless{}{-}}\FunctionTok{image\_read}\NormalTok{(b\_gif)}

\NormalTok{new\_gif}\OtherTok{\textless{}{-}}\FunctionTok{image\_append}\NormalTok{(}\FunctionTok{c}\NormalTok{(a\_mgif[}\DecValTok{1}\NormalTok{], b\_mgif[}\DecValTok{1}\NormalTok{]))}
\ControlFlowTok{for}\NormalTok{(i }\ControlFlowTok{in} \DecValTok{2}\SpecialCharTok{:}\DecValTok{100}\NormalTok{)\{}
\NormalTok{  combined }\OtherTok{\textless{}{-}} \FunctionTok{image\_append}\NormalTok{(}\FunctionTok{c}\NormalTok{(a\_mgif[i], b\_mgif[i]))}
\NormalTok{  new\_gif}\OtherTok{\textless{}{-}}\FunctionTok{c}\NormalTok{(new\_gif,combined)}
\NormalTok{\}}

\NormalTok{new\_gif}

\DocumentationTok{\#\# increase sample{-}size}

\NormalTok{A}\OtherTok{\textless{}{-}}\FunctionTok{rnorm}\NormalTok{(}\DecValTok{50}\SpecialCharTok{*}\DecValTok{10}\NormalTok{,}\DecValTok{60}\NormalTok{,}\DecValTok{10}\NormalTok{)}
\NormalTok{B}\OtherTok{\textless{}{-}}\FunctionTok{rnorm}\NormalTok{(}\DecValTok{50}\SpecialCharTok{*}\DecValTok{10}\NormalTok{,}\DecValTok{50}\NormalTok{,}\DecValTok{10}\NormalTok{)}
\NormalTok{DV }\OtherTok{\textless{}{-}} \FunctionTok{c}\NormalTok{(A,B)}
\NormalTok{IV }\OtherTok{\textless{}{-}} \FunctionTok{rep}\NormalTok{(}\FunctionTok{c}\NormalTok{(}\StringTok{"A"}\NormalTok{,}\StringTok{"B"}\NormalTok{),}\AttributeTok{each=}\DecValTok{50}\SpecialCharTok{*}\DecValTok{10}\NormalTok{)}
\NormalTok{sims }\OtherTok{\textless{}{-}} \FunctionTok{rep}\NormalTok{(}\FunctionTok{rep}\NormalTok{(}\DecValTok{1}\SpecialCharTok{:}\DecValTok{10}\NormalTok{,}\AttributeTok{each=}\DecValTok{50}\NormalTok{),}\DecValTok{2}\NormalTok{)}
\NormalTok{df}\OtherTok{\textless{}{-}}\FunctionTok{data.frame}\NormalTok{(sims,IV,DV)}

\NormalTok{means\_df }\OtherTok{\textless{}{-}}\NormalTok{ df }\SpecialCharTok{\%\textgreater{}\%}
               \FunctionTok{group\_by}\NormalTok{(sims,IV) }\SpecialCharTok{\%\textgreater{}\%}
               \FunctionTok{summarize}\NormalTok{(}\AttributeTok{means=}\FunctionTok{mean}\NormalTok{(DV),}
                         \AttributeTok{sem =} \FunctionTok{sd}\NormalTok{(DV)}\SpecialCharTok{/}\FunctionTok{sqrt}\NormalTok{(}\FunctionTok{length}\NormalTok{(DV)))}

\NormalTok{stats\_df }\OtherTok{\textless{}{-}}\NormalTok{ df }\SpecialCharTok{\%\textgreater{}\%}
              \FunctionTok{group\_by}\NormalTok{(sims) }\SpecialCharTok{\%\textgreater{}\%}
              \FunctionTok{summarize}\NormalTok{(}\AttributeTok{ts =} \FunctionTok{t.test}\NormalTok{(DV}\SpecialCharTok{\textasciitilde{}}\NormalTok{IV,}\AttributeTok{var.equal=}\ConstantTok{TRUE}\NormalTok{)}\SpecialCharTok{$}\NormalTok{statistic)}

\NormalTok{a}\OtherTok{\textless{}{-}}\FunctionTok{ggplot}\NormalTok{(means\_df, }\FunctionTok{aes}\NormalTok{(}\AttributeTok{x=}\NormalTok{IV,}\AttributeTok{y=}\NormalTok{means, }\AttributeTok{fill=}\NormalTok{IV))}\SpecialCharTok{+}
  \FunctionTok{geom\_bar}\NormalTok{(}\AttributeTok{stat=}\StringTok{"identity"}\NormalTok{)}\SpecialCharTok{+}
  \FunctionTok{geom\_point}\NormalTok{(}\AttributeTok{data=}\NormalTok{df,}\FunctionTok{aes}\NormalTok{(}\AttributeTok{x=}\NormalTok{IV, }\AttributeTok{y=}\NormalTok{DV), }\AttributeTok{alpha=}\NormalTok{.}\DecValTok{25}\NormalTok{)}\SpecialCharTok{+}
  \FunctionTok{geom\_errorbar}\NormalTok{(}\FunctionTok{aes}\NormalTok{(}\AttributeTok{ymin=}\NormalTok{means}\SpecialCharTok{{-}}\NormalTok{sem, }\AttributeTok{ymax=}\NormalTok{means}\SpecialCharTok{+}\NormalTok{sem),}\AttributeTok{width=}\NormalTok{.}\DecValTok{2}\NormalTok{)}\SpecialCharTok{+}
  \FunctionTok{theme\_classic}\NormalTok{()}\SpecialCharTok{+}
  \FunctionTok{transition\_states}\NormalTok{(}
    \AttributeTok{states=}\NormalTok{sims,}
    \AttributeTok{transition\_length =} \DecValTok{2}\NormalTok{,}
    \AttributeTok{state\_length =} \DecValTok{1}
\NormalTok{  )}\SpecialCharTok{+}\FunctionTok{enter\_fade}\NormalTok{() }\SpecialCharTok{+} 
  \FunctionTok{exit\_shrink}\NormalTok{() }\SpecialCharTok{+}
  \FunctionTok{ease\_aes}\NormalTok{(}\StringTok{\textquotesingle{}sine{-}in{-}out\textquotesingle{}}\NormalTok{)}
  
\NormalTok{a\_gif}\OtherTok{\textless{}{-}}\FunctionTok{animate}\NormalTok{(a, }\AttributeTok{width =} \DecValTok{240}\NormalTok{, }\AttributeTok{height =} \DecValTok{240}\NormalTok{)}

\NormalTok{b}\OtherTok{\textless{}{-}}\FunctionTok{ggplot}\NormalTok{(stats\_df,}\FunctionTok{aes}\NormalTok{(}\AttributeTok{x=}\NormalTok{ts))}\SpecialCharTok{+}
  \FunctionTok{geom\_vline}\NormalTok{(}\FunctionTok{aes}\NormalTok{(}\AttributeTok{xintercept=}\NormalTok{ts, }\AttributeTok{frame=}\NormalTok{sims))}\SpecialCharTok{+}
  \FunctionTok{geom\_vline}\NormalTok{(}\AttributeTok{xintercept=}\FunctionTok{qt}\NormalTok{(}\FunctionTok{c}\NormalTok{(.}\DecValTok{025}\NormalTok{, .}\DecValTok{975}\NormalTok{), }\AttributeTok{df=}\DecValTok{98}\NormalTok{),}\AttributeTok{color=}\StringTok{"green"}\NormalTok{)}\SpecialCharTok{+}
  \FunctionTok{geom\_line}\NormalTok{(}\AttributeTok{data=}\FunctionTok{data.frame}\NormalTok{(}\AttributeTok{x=}\FunctionTok{seq}\NormalTok{(}\SpecialCharTok{{-}}\DecValTok{5}\NormalTok{,}\DecValTok{5}\NormalTok{,.}\DecValTok{1}\NormalTok{),}
                            \AttributeTok{y=}\FunctionTok{dt}\NormalTok{(}\FunctionTok{seq}\NormalTok{(}\SpecialCharTok{{-}}\DecValTok{5}\NormalTok{,}\DecValTok{5}\NormalTok{,.}\DecValTok{1}\NormalTok{),}\AttributeTok{df=}\DecValTok{98}\NormalTok{)),}
            \FunctionTok{aes}\NormalTok{(}\AttributeTok{x=}\NormalTok{x,}\AttributeTok{y=}\NormalTok{y))}\SpecialCharTok{+}
  \FunctionTok{theme\_classic}\NormalTok{()}\SpecialCharTok{+}
  \FunctionTok{ylab}\NormalTok{(}\StringTok{"density"}\NormalTok{)}\SpecialCharTok{+}
  \FunctionTok{xlab}\NormalTok{(}\StringTok{"t value"}\NormalTok{)}\SpecialCharTok{+}
  \FunctionTok{transition\_states}\NormalTok{(}
    \AttributeTok{states=}\NormalTok{sims,}
    \AttributeTok{transition\_length =} \DecValTok{2}\NormalTok{,}
    \AttributeTok{state\_length =} \DecValTok{1}
\NormalTok{  )}\SpecialCharTok{+}\FunctionTok{enter\_fade}\NormalTok{() }\SpecialCharTok{+} 
  \FunctionTok{exit\_shrink}\NormalTok{() }\SpecialCharTok{+}
  \FunctionTok{ease\_aes}\NormalTok{(}\StringTok{\textquotesingle{}sine{-}in{-}out\textquotesingle{}}\NormalTok{)}

\NormalTok{b\_gif}\OtherTok{\textless{}{-}}\FunctionTok{animate}\NormalTok{(b, }\AttributeTok{width =} \DecValTok{240}\NormalTok{, }\AttributeTok{height =} \DecValTok{240}\NormalTok{)}


\NormalTok{d}\OtherTok{\textless{}{-}}\FunctionTok{image\_blank}\NormalTok{(}\DecValTok{240}\SpecialCharTok{*}\DecValTok{2}\NormalTok{,}\DecValTok{240}\NormalTok{)}

\NormalTok{the\_frame}\OtherTok{\textless{}{-}}\NormalTok{d}
\ControlFlowTok{for}\NormalTok{(i }\ControlFlowTok{in} \DecValTok{2}\SpecialCharTok{:}\DecValTok{100}\NormalTok{)\{}
\NormalTok{  the\_frame}\OtherTok{\textless{}{-}}\FunctionTok{c}\NormalTok{(the\_frame,d)}
\NormalTok{\}}

\NormalTok{a\_mgif}\OtherTok{\textless{}{-}}\FunctionTok{image\_read}\NormalTok{(a\_gif)}
\NormalTok{b\_mgif}\OtherTok{\textless{}{-}}\FunctionTok{image\_read}\NormalTok{(b\_gif)}

\NormalTok{new\_gif2}\OtherTok{\textless{}{-}}\FunctionTok{image\_append}\NormalTok{(}\FunctionTok{c}\NormalTok{(a\_mgif[}\DecValTok{1}\NormalTok{], b\_mgif[}\DecValTok{1}\NormalTok{]))}
\ControlFlowTok{for}\NormalTok{(i }\ControlFlowTok{in} \DecValTok{2}\SpecialCharTok{:}\DecValTok{100}\NormalTok{)\{}
\NormalTok{  combined }\OtherTok{\textless{}{-}} \FunctionTok{image\_append}\NormalTok{(}\FunctionTok{c}\NormalTok{(a\_mgif[i], b\_mgif[i]))}
\NormalTok{  new\_gif2}\OtherTok{\textless{}{-}}\FunctionTok{c}\NormalTok{(new\_gif2,combined)}
\NormalTok{\}}

\DocumentationTok{\#\# add new row}

\NormalTok{final\_gif }\OtherTok{\textless{}{-}} \FunctionTok{image\_append}\NormalTok{(}\FunctionTok{c}\NormalTok{(new\_gif[}\DecValTok{1}\NormalTok{], new\_gif2[}\DecValTok{1}\NormalTok{]),}\AttributeTok{stack=}\ConstantTok{TRUE}\NormalTok{)}
\ControlFlowTok{for}\NormalTok{(i }\ControlFlowTok{in} \DecValTok{2}\SpecialCharTok{:}\DecValTok{100}\NormalTok{)\{}
\NormalTok{  combined }\OtherTok{\textless{}{-}} \FunctionTok{image\_append}\NormalTok{(}\FunctionTok{c}\NormalTok{(new\_gif[i], new\_gif2[i]),}\AttributeTok{stack=}\ConstantTok{TRUE}\NormalTok{)}
\NormalTok{  final\_gif}\OtherTok{\textless{}{-}}\FunctionTok{c}\NormalTok{(final\_gif,combined)}
\NormalTok{\}}

\NormalTok{final\_gif}
\end{Highlighting}
\end{Shaded}

\subsection{one-factor ANOVA Null}\label{one-factor-anova-null}

Three groups, N=10, all observations sampled from same normal
distribution (mu=50, sd = 10)

\begin{Shaded}
\begin{Highlighting}[]
\FunctionTok{library}\NormalTok{(dplyr)}
\FunctionTok{library}\NormalTok{(ggplot2)}
\FunctionTok{library}\NormalTok{(magick)}
\FunctionTok{library}\NormalTok{(gganimate)}


\NormalTok{A}\OtherTok{\textless{}{-}}\FunctionTok{rnorm}\NormalTok{(}\DecValTok{100}\NormalTok{,}\DecValTok{50}\NormalTok{,}\DecValTok{10}\NormalTok{)}
\NormalTok{B}\OtherTok{\textless{}{-}}\FunctionTok{rnorm}\NormalTok{(}\DecValTok{100}\NormalTok{,}\DecValTok{50}\NormalTok{,}\DecValTok{10}\NormalTok{)}
\NormalTok{C}\OtherTok{\textless{}{-}}\FunctionTok{rnorm}\NormalTok{(}\DecValTok{100}\NormalTok{,}\DecValTok{50}\NormalTok{,}\DecValTok{10}\NormalTok{)}
\NormalTok{DV }\OtherTok{\textless{}{-}} \FunctionTok{c}\NormalTok{(A,B,C)}
\NormalTok{IV }\OtherTok{\textless{}{-}} \FunctionTok{rep}\NormalTok{(}\FunctionTok{rep}\NormalTok{(}\FunctionTok{c}\NormalTok{(}\StringTok{"A"}\NormalTok{,}\StringTok{"B"}\NormalTok{,}\StringTok{"C"}\NormalTok{),}\AttributeTok{each=}\DecValTok{10}\NormalTok{),}\DecValTok{10}\NormalTok{)}
\NormalTok{sims }\OtherTok{\textless{}{-}} \FunctionTok{rep}\NormalTok{(}\DecValTok{1}\SpecialCharTok{:}\DecValTok{10}\NormalTok{,}\AttributeTok{each=}\DecValTok{30}\NormalTok{)}
\NormalTok{df}\OtherTok{\textless{}{-}}\FunctionTok{data.frame}\NormalTok{(sims,IV,DV)}

\NormalTok{means\_df }\OtherTok{\textless{}{-}}\NormalTok{ df }\SpecialCharTok{\%\textgreater{}\%}
  \FunctionTok{group\_by}\NormalTok{(sims,IV) }\SpecialCharTok{\%\textgreater{}\%}
  \FunctionTok{summarize}\NormalTok{(}\AttributeTok{means=}\FunctionTok{mean}\NormalTok{(DV),}
            \AttributeTok{sem =} \FunctionTok{sd}\NormalTok{(DV)}\SpecialCharTok{/}\FunctionTok{sqrt}\NormalTok{(}\FunctionTok{length}\NormalTok{(DV)))}

\NormalTok{stats\_df }\OtherTok{\textless{}{-}}\NormalTok{ df }\SpecialCharTok{\%\textgreater{}\%}
  \FunctionTok{group\_by}\NormalTok{(sims) }\SpecialCharTok{\%\textgreater{}\%}
  \FunctionTok{summarize}\NormalTok{(}\AttributeTok{Fs =} \FunctionTok{summary}\NormalTok{(}\FunctionTok{aov}\NormalTok{(DV}\SpecialCharTok{\textasciitilde{}}\NormalTok{IV))[[}\DecValTok{1}\NormalTok{]][[}\DecValTok{4}\NormalTok{]][}\DecValTok{1}\NormalTok{])}

\NormalTok{a}\OtherTok{\textless{}{-}}\FunctionTok{ggplot}\NormalTok{(means\_df, }\FunctionTok{aes}\NormalTok{(}\AttributeTok{x=}\NormalTok{IV,}\AttributeTok{y=}\NormalTok{means, }\AttributeTok{fill=}\NormalTok{IV))}\SpecialCharTok{+}
  \FunctionTok{geom\_bar}\NormalTok{(}\AttributeTok{stat=}\StringTok{"identity"}\NormalTok{)}\SpecialCharTok{+}
  \FunctionTok{geom\_point}\NormalTok{(}\AttributeTok{data=}\NormalTok{df,}\FunctionTok{aes}\NormalTok{(}\AttributeTok{x=}\NormalTok{IV, }\AttributeTok{y=}\NormalTok{DV), }\AttributeTok{alpha=}\NormalTok{.}\DecValTok{25}\NormalTok{)}\SpecialCharTok{+}
  \FunctionTok{geom\_errorbar}\NormalTok{(}\FunctionTok{aes}\NormalTok{(}\AttributeTok{ymin=}\NormalTok{means}\SpecialCharTok{{-}}\NormalTok{sem, }\AttributeTok{ymax=}\NormalTok{means}\SpecialCharTok{+}\NormalTok{sem),}\AttributeTok{width=}\NormalTok{.}\DecValTok{2}\NormalTok{)}\SpecialCharTok{+}
  \FunctionTok{theme\_classic}\NormalTok{(}\AttributeTok{base\_size =} \DecValTok{20}\NormalTok{)}\SpecialCharTok{+}
  \FunctionTok{transition\_states}\NormalTok{(}
    \AttributeTok{states=}\NormalTok{sims,}
    \AttributeTok{transition\_length =} \DecValTok{2}\NormalTok{,}
    \AttributeTok{state\_length =} \DecValTok{1}
\NormalTok{  )}\SpecialCharTok{+}\FunctionTok{enter\_fade}\NormalTok{() }\SpecialCharTok{+} 
  \FunctionTok{exit\_shrink}\NormalTok{() }\SpecialCharTok{+}
  \FunctionTok{ease\_aes}\NormalTok{(}\StringTok{\textquotesingle{}sine{-}in{-}out\textquotesingle{}}\NormalTok{)}

\NormalTok{b}\OtherTok{\textless{}{-}}\FunctionTok{ggplot}\NormalTok{(stats\_df,}\FunctionTok{aes}\NormalTok{(}\AttributeTok{x=}\NormalTok{Fs))}\SpecialCharTok{+}
  \FunctionTok{geom\_vline}\NormalTok{(}\FunctionTok{aes}\NormalTok{(}\AttributeTok{xintercept=}\NormalTok{Fs))}\SpecialCharTok{+}
  \FunctionTok{geom\_vline}\NormalTok{(}\AttributeTok{xintercept=}\FunctionTok{qf}\NormalTok{(.}\DecValTok{95}\NormalTok{, }\AttributeTok{df1=}\DecValTok{2}\NormalTok{,}\AttributeTok{df2=}\DecValTok{27}\NormalTok{),}\AttributeTok{color=}\StringTok{"green"}\NormalTok{)}\SpecialCharTok{+}
  \FunctionTok{geom\_line}\NormalTok{(}\AttributeTok{data=}\FunctionTok{data.frame}\NormalTok{(}\AttributeTok{x=}\FunctionTok{seq}\NormalTok{(}\DecValTok{0}\NormalTok{,}\DecValTok{6}\NormalTok{,.}\DecValTok{1}\NormalTok{),}
                            \AttributeTok{y=}\FunctionTok{df}\NormalTok{(}\FunctionTok{seq}\NormalTok{(}\DecValTok{0}\NormalTok{,}\DecValTok{6}\NormalTok{,.}\DecValTok{1}\NormalTok{),}\AttributeTok{df1=}\DecValTok{2}\NormalTok{,}\AttributeTok{df2=}\DecValTok{27}\NormalTok{)),}
            \FunctionTok{aes}\NormalTok{(}\AttributeTok{x=}\NormalTok{x,}\AttributeTok{y=}\NormalTok{y))}\SpecialCharTok{+}
  \FunctionTok{theme\_classic}\NormalTok{(}\AttributeTok{base\_size =} \DecValTok{20}\NormalTok{)}\SpecialCharTok{+}
  \FunctionTok{ylab}\NormalTok{(}\StringTok{"density"}\NormalTok{)}\SpecialCharTok{+}
  \FunctionTok{xlab}\NormalTok{(}\StringTok{"F value"}\NormalTok{)}\SpecialCharTok{+}
  \FunctionTok{transition\_states}\NormalTok{(}
    \AttributeTok{states=}\NormalTok{sims,}
    \AttributeTok{transition\_length =} \DecValTok{2}\NormalTok{,}
    \AttributeTok{state\_length =} \DecValTok{1}
\NormalTok{  )}\SpecialCharTok{+}\FunctionTok{enter\_fade}\NormalTok{() }\SpecialCharTok{+} 
  \FunctionTok{exit\_shrink}\NormalTok{() }\SpecialCharTok{+}
  \FunctionTok{ease\_aes}\NormalTok{(}\StringTok{\textquotesingle{}sine{-}in{-}out\textquotesingle{}}\NormalTok{)}

\NormalTok{a\_gif}\OtherTok{\textless{}{-}}\FunctionTok{animate}\NormalTok{(a,}\AttributeTok{width=}\DecValTok{480}\NormalTok{,}\AttributeTok{height=}\DecValTok{480}\NormalTok{)}
\NormalTok{b\_gif}\OtherTok{\textless{}{-}}\FunctionTok{animate}\NormalTok{(b,}\AttributeTok{width=}\DecValTok{480}\NormalTok{,}\AttributeTok{height=}\DecValTok{480}\NormalTok{)}

\NormalTok{a\_mgif}\OtherTok{\textless{}{-}}\FunctionTok{image\_read}\NormalTok{(a\_gif)}
\NormalTok{b\_mgif}\OtherTok{\textless{}{-}}\FunctionTok{image\_read}\NormalTok{(b\_gif)}

\NormalTok{new\_gif}\OtherTok{\textless{}{-}}\FunctionTok{image\_append}\NormalTok{(}\FunctionTok{c}\NormalTok{(a\_mgif[}\DecValTok{1}\NormalTok{], b\_mgif[}\DecValTok{1}\NormalTok{]))}
\ControlFlowTok{for}\NormalTok{(i }\ControlFlowTok{in} \DecValTok{2}\SpecialCharTok{:}\DecValTok{100}\NormalTok{)\{}
\NormalTok{  combined }\OtherTok{\textless{}{-}} \FunctionTok{image\_append}\NormalTok{(}\FunctionTok{c}\NormalTok{(a\_mgif[i], b\_mgif[i]))}
\NormalTok{  new\_gif}\OtherTok{\textless{}{-}}\FunctionTok{c}\NormalTok{(new\_gif,combined)}
\NormalTok{\}}

\NormalTok{new\_gif}
\end{Highlighting}
\end{Shaded}

\subsection{Factorial Null}\label{factorial-null}

10 simulations, N=10 in each of 4 conditions in a 2x2
(between-subjects). All observations taken from the same normal
distribution (mu=50, sd =10).

\begin{Shaded}
\begin{Highlighting}[]
\NormalTok{A}\OtherTok{\textless{}{-}}\FunctionTok{rnorm}\NormalTok{(}\DecValTok{100}\NormalTok{,}\DecValTok{50}\NormalTok{,}\DecValTok{10}\NormalTok{)}
\NormalTok{B}\OtherTok{\textless{}{-}}\FunctionTok{rnorm}\NormalTok{(}\DecValTok{100}\NormalTok{,}\DecValTok{50}\NormalTok{,}\DecValTok{10}\NormalTok{)}
\NormalTok{C}\OtherTok{\textless{}{-}}\FunctionTok{rnorm}\NormalTok{(}\DecValTok{100}\NormalTok{,}\DecValTok{50}\NormalTok{,}\DecValTok{10}\NormalTok{)}
\NormalTok{D}\OtherTok{\textless{}{-}}\FunctionTok{rnorm}\NormalTok{(}\DecValTok{100}\NormalTok{,}\DecValTok{50}\NormalTok{,}\DecValTok{10}\NormalTok{)}
\NormalTok{DV }\OtherTok{\textless{}{-}} \FunctionTok{c}\NormalTok{(A,B,C,D)}
\NormalTok{IV1 }\OtherTok{\textless{}{-}} \FunctionTok{rep}\NormalTok{(}\FunctionTok{c}\NormalTok{(}\StringTok{"A"}\NormalTok{,}\StringTok{"B"}\NormalTok{),}\AttributeTok{each=}\DecValTok{200}\NormalTok{)}
\NormalTok{IV2}\OtherTok{\textless{}{-}}\FunctionTok{rep}\NormalTok{(}\FunctionTok{rep}\NormalTok{(}\FunctionTok{c}\NormalTok{(}\StringTok{"1"}\NormalTok{,}\StringTok{"2"}\NormalTok{),}\AttributeTok{each=}\DecValTok{100}\NormalTok{),}\DecValTok{2}\NormalTok{)}
\NormalTok{sims }\OtherTok{\textless{}{-}} \FunctionTok{rep}\NormalTok{(}\DecValTok{1}\SpecialCharTok{:}\DecValTok{10}\NormalTok{,}\DecValTok{40}\NormalTok{)}
\NormalTok{df}\OtherTok{\textless{}{-}}\FunctionTok{data.frame}\NormalTok{(sims,IV1,IV2,DV)}

\NormalTok{means\_df }\OtherTok{\textless{}{-}}\NormalTok{ df }\SpecialCharTok{\%\textgreater{}\%}
  \FunctionTok{group\_by}\NormalTok{(sims,IV1,IV2) }\SpecialCharTok{\%\textgreater{}\%}
  \FunctionTok{summarize}\NormalTok{(}\AttributeTok{means=}\FunctionTok{mean}\NormalTok{(DV),}
            \AttributeTok{sem =} \FunctionTok{sd}\NormalTok{(DV)}\SpecialCharTok{/}\FunctionTok{sqrt}\NormalTok{(}\FunctionTok{length}\NormalTok{(DV)))}

\NormalTok{stats\_df }\OtherTok{\textless{}{-}}\NormalTok{ df }\SpecialCharTok{\%\textgreater{}\%}
  \FunctionTok{group\_by}\NormalTok{(sims) }\SpecialCharTok{\%\textgreater{}\%}
  \FunctionTok{summarize}\NormalTok{(}\AttributeTok{FIV1 =} \FunctionTok{summary}\NormalTok{(}\FunctionTok{aov}\NormalTok{(DV}\SpecialCharTok{\textasciitilde{}}\NormalTok{IV1}\SpecialCharTok{*}\NormalTok{IV2))[[}\DecValTok{1}\NormalTok{]][[}\DecValTok{4}\NormalTok{]][}\DecValTok{1}\NormalTok{],}
            \AttributeTok{FIV2 =} \FunctionTok{summary}\NormalTok{(}\FunctionTok{aov}\NormalTok{(DV}\SpecialCharTok{\textasciitilde{}}\NormalTok{IV1}\SpecialCharTok{*}\NormalTok{IV2))[[}\DecValTok{1}\NormalTok{]][[}\DecValTok{4}\NormalTok{]][}\DecValTok{2}\NormalTok{],}
            \AttributeTok{F1x2 =} \FunctionTok{summary}\NormalTok{(}\FunctionTok{aov}\NormalTok{(DV}\SpecialCharTok{\textasciitilde{}}\NormalTok{IV1}\SpecialCharTok{*}\NormalTok{IV2))[[}\DecValTok{1}\NormalTok{]][[}\DecValTok{4}\NormalTok{]][}\DecValTok{3}\NormalTok{]}
\NormalTok{            )}

\NormalTok{a}\OtherTok{\textless{}{-}}\FunctionTok{ggplot}\NormalTok{(means\_df, }\FunctionTok{aes}\NormalTok{(}\AttributeTok{x=}\NormalTok{IV1,}\AttributeTok{y=}\NormalTok{means, }
                                           \AttributeTok{group=}\NormalTok{IV2,}
                                           \AttributeTok{color=}\NormalTok{IV2))}\SpecialCharTok{+}
  \FunctionTok{geom\_point}\NormalTok{(}\AttributeTok{data=}\NormalTok{df,}\FunctionTok{aes}\NormalTok{(}\AttributeTok{x=}\NormalTok{IV1, }\AttributeTok{y=}\NormalTok{DV,}\AttributeTok{group=}\NormalTok{IV2), }
             \AttributeTok{position=}\FunctionTok{position\_dodge}\NormalTok{(}\AttributeTok{width=}\NormalTok{.}\DecValTok{2}\NormalTok{),}
             \AttributeTok{size=}\DecValTok{2}\NormalTok{,}
             \AttributeTok{alpha=}\NormalTok{.}\DecValTok{25}\NormalTok{)}\SpecialCharTok{+}
  \FunctionTok{geom\_point}\NormalTok{(}\AttributeTok{size=}\DecValTok{4}\NormalTok{)}\SpecialCharTok{+}
  \FunctionTok{geom\_line}\NormalTok{(}\AttributeTok{size=}\FloatTok{1.3}\NormalTok{)}\SpecialCharTok{+}
  \FunctionTok{geom\_errorbar}\NormalTok{(}\FunctionTok{aes}\NormalTok{(}\AttributeTok{ymin=}\NormalTok{means}\SpecialCharTok{{-}}\NormalTok{sem, }\AttributeTok{ymax=}\NormalTok{means}\SpecialCharTok{+}\NormalTok{sem),}\AttributeTok{width=}\NormalTok{.}\DecValTok{2}\NormalTok{,}
                \AttributeTok{color=}\StringTok{"black"}\NormalTok{)}\SpecialCharTok{+}
  \FunctionTok{theme\_classic}\NormalTok{(}\AttributeTok{base\_size =} \DecValTok{20}\NormalTok{)}\SpecialCharTok{+}
  \FunctionTok{transition\_states}\NormalTok{(}
    \AttributeTok{states=}\NormalTok{sims,}
    \AttributeTok{transition\_length =} \DecValTok{2}\NormalTok{,}
    \AttributeTok{state\_length =} \DecValTok{1}
\NormalTok{  )}\SpecialCharTok{+}\FunctionTok{enter\_fade}\NormalTok{() }\SpecialCharTok{+} 
  \FunctionTok{exit\_shrink}\NormalTok{() }\SpecialCharTok{+}
  \FunctionTok{ease\_aes}\NormalTok{(}\StringTok{\textquotesingle{}sine{-}in{-}out\textquotesingle{}}\NormalTok{)}

\NormalTok{b}\OtherTok{\textless{}{-}}\FunctionTok{ggplot}\NormalTok{(stats\_df,}\FunctionTok{aes}\NormalTok{(}\AttributeTok{x=}\NormalTok{FIV1))}\SpecialCharTok{+}
  \FunctionTok{geom\_vline}\NormalTok{(}\FunctionTok{aes}\NormalTok{(}\AttributeTok{xintercept=}\NormalTok{FIV1),}\AttributeTok{color=}\StringTok{"red"}\NormalTok{,}\AttributeTok{size=}\FloatTok{1.2}\NormalTok{)}\SpecialCharTok{+}
  \FunctionTok{geom\_vline}\NormalTok{(}\FunctionTok{aes}\NormalTok{(}\AttributeTok{xintercept=}\NormalTok{FIV2),}\AttributeTok{color=}\StringTok{"blue"}\NormalTok{,}\AttributeTok{size=}\FloatTok{1.2}\NormalTok{)}\SpecialCharTok{+}
  \FunctionTok{geom\_vline}\NormalTok{(}\FunctionTok{aes}\NormalTok{(}\AttributeTok{xintercept=}\NormalTok{F1x2),}\AttributeTok{color=}\StringTok{"purple"}\NormalTok{,}\AttributeTok{size=}\FloatTok{1.2}\NormalTok{)}\SpecialCharTok{+}
  \FunctionTok{geom\_vline}\NormalTok{(}\AttributeTok{xintercept=}\FunctionTok{qf}\NormalTok{(.}\DecValTok{95}\NormalTok{, }\AttributeTok{df1=}\DecValTok{1}\NormalTok{,}\AttributeTok{df2=}\DecValTok{36}\NormalTok{),}\AttributeTok{color=}\StringTok{"green"}\NormalTok{,}\AttributeTok{size=}\FloatTok{1.2}\NormalTok{)}\SpecialCharTok{+}
  \FunctionTok{geom\_line}\NormalTok{(}\AttributeTok{data=}\FunctionTok{data.frame}\NormalTok{(}\AttributeTok{x=}\FunctionTok{seq}\NormalTok{(}\DecValTok{0}\NormalTok{,}\DecValTok{20}\NormalTok{,.}\DecValTok{1}\NormalTok{),}
                            \AttributeTok{y=}\FunctionTok{df}\NormalTok{(}\FunctionTok{seq}\NormalTok{(}\DecValTok{0}\NormalTok{,}\DecValTok{20}\NormalTok{,.}\DecValTok{1}\NormalTok{),}\AttributeTok{df1=}\DecValTok{1}\NormalTok{,}\AttributeTok{df2=}\DecValTok{36}\NormalTok{)),}
            \FunctionTok{aes}\NormalTok{(}\AttributeTok{x=}\NormalTok{x,}\AttributeTok{y=}\NormalTok{y))}\SpecialCharTok{+}
  \FunctionTok{theme\_classic}\NormalTok{(}\AttributeTok{base\_size =} \DecValTok{20}\NormalTok{)}\SpecialCharTok{+}
  \FunctionTok{ylab}\NormalTok{(}\StringTok{"density"}\NormalTok{)}\SpecialCharTok{+}
  \FunctionTok{xlab}\NormalTok{(}\StringTok{"F value"}\NormalTok{)}\SpecialCharTok{+}
  \FunctionTok{ggtitle}\NormalTok{(}\AttributeTok{label=}\StringTok{""}\NormalTok{,}\AttributeTok{subtitle=}\StringTok{"red=IV1, blue=IV2, }\SpecialCharTok{\textbackslash{}n}\StringTok{ purple=Interaction"}\NormalTok{)}\SpecialCharTok{+}
  \FunctionTok{transition\_states}\NormalTok{(}
    \AttributeTok{states=}\NormalTok{sims,}
    \AttributeTok{transition\_length =} \DecValTok{2}\NormalTok{,}
    \AttributeTok{state\_length =} \DecValTok{1}
\NormalTok{  )}

\NormalTok{a\_gif}\OtherTok{\textless{}{-}}\FunctionTok{animate}\NormalTok{(a,}\AttributeTok{width=}\DecValTok{480}\NormalTok{,}\AttributeTok{height=}\DecValTok{480}\NormalTok{)}
\NormalTok{b\_gif}\OtherTok{\textless{}{-}}\FunctionTok{animate}\NormalTok{(b,}\AttributeTok{width=}\DecValTok{480}\NormalTok{,}\AttributeTok{height=}\DecValTok{480}\NormalTok{)}

\NormalTok{a\_mgif}\OtherTok{\textless{}{-}}\FunctionTok{image\_read}\NormalTok{(a\_gif)}
\NormalTok{b\_mgif}\OtherTok{\textless{}{-}}\FunctionTok{image\_read}\NormalTok{(b\_gif)}

\NormalTok{new\_gif}\OtherTok{\textless{}{-}}\FunctionTok{image\_append}\NormalTok{(}\FunctionTok{c}\NormalTok{(a\_mgif[}\DecValTok{1}\NormalTok{], b\_mgif[}\DecValTok{1}\NormalTok{]))}
\ControlFlowTok{for}\NormalTok{(i }\ControlFlowTok{in} \DecValTok{2}\SpecialCharTok{:}\DecValTok{100}\NormalTok{)\{}
\NormalTok{  combined }\OtherTok{\textless{}{-}} \FunctionTok{image\_append}\NormalTok{(}\FunctionTok{c}\NormalTok{(a\_mgif[i], b\_mgif[i]))}
\NormalTok{  new\_gif}\OtherTok{\textless{}{-}}\FunctionTok{c}\NormalTok{(new\_gif,combined)}
\NormalTok{\}}

\FunctionTok{image\_animate}\NormalTok{(new\_gif, }\AttributeTok{fps =} \DecValTok{10}\NormalTok{,}\AttributeTok{dispose=}\StringTok{"none"}\NormalTok{)}
\end{Highlighting}
\end{Shaded}

\section{Distributions}\label{distributions}

\subsection{Normal changing mean}\label{normal-changing-mean}

\begin{Shaded}
\begin{Highlighting}[]
\NormalTok{some\_means}\OtherTok{\textless{}{-}}\FunctionTok{c}\NormalTok{(}\DecValTok{0}\NormalTok{,}\DecValTok{1}\NormalTok{,}\DecValTok{2}\NormalTok{,}\DecValTok{3}\NormalTok{,}\DecValTok{4}\NormalTok{,}\DecValTok{5}\NormalTok{,}\DecValTok{4}\NormalTok{,}\DecValTok{3}\NormalTok{,}\DecValTok{2}\NormalTok{,}\DecValTok{1}\NormalTok{)}
\NormalTok{all\_df}\OtherTok{\textless{}{-}}\FunctionTok{data.frame}\NormalTok{()}
\ControlFlowTok{for}\NormalTok{(i }\ControlFlowTok{in} \DecValTok{1}\SpecialCharTok{:}\DecValTok{10}\NormalTok{)\{}
\NormalTok{  dnorm\_vec }\OtherTok{\textless{}{-}} \FunctionTok{dnorm}\NormalTok{(}\FunctionTok{seq}\NormalTok{(}\SpecialCharTok{{-}}\DecValTok{10}\NormalTok{,}\DecValTok{10}\NormalTok{,.}\DecValTok{1}\NormalTok{),}\AttributeTok{mean=}\NormalTok{some\_means[i],}\AttributeTok{sd=}\DecValTok{1}\NormalTok{)}
\NormalTok{  x\_range   }\OtherTok{\textless{}{-}} \FunctionTok{seq}\NormalTok{(}\SpecialCharTok{{-}}\DecValTok{10}\NormalTok{,}\DecValTok{10}\NormalTok{,.}\DecValTok{1}\NormalTok{)}
\NormalTok{  means }\OtherTok{\textless{}{-}} \FunctionTok{rep}\NormalTok{(some\_means[i], }\FunctionTok{length}\NormalTok{(x\_range))}
\NormalTok{  sims }\OtherTok{\textless{}{-}} \FunctionTok{rep}\NormalTok{(i, }\FunctionTok{length}\NormalTok{(x\_range))}
\NormalTok{  t\_df}\OtherTok{\textless{}{-}}\FunctionTok{data.frame}\NormalTok{(sims,means,x\_range,dnorm\_vec)}
\NormalTok{  all\_df}\OtherTok{\textless{}{-}}\FunctionTok{rbind}\NormalTok{(all\_df,t\_df)}
\NormalTok{\}}

\FunctionTok{ggplot}\NormalTok{(all\_df, }\FunctionTok{aes}\NormalTok{(}\AttributeTok{x=}\NormalTok{x\_range,}\AttributeTok{y=}\NormalTok{dnorm\_vec))}\SpecialCharTok{+}
  \FunctionTok{geom\_line}\NormalTok{()}\SpecialCharTok{+}
  \FunctionTok{theme\_classic}\NormalTok{()}\SpecialCharTok{+}
  \FunctionTok{ylab}\NormalTok{(}\StringTok{"probability density"}\NormalTok{)}\SpecialCharTok{+}
  \FunctionTok{xlab}\NormalTok{(}\StringTok{"value"}\NormalTok{)}\SpecialCharTok{+}
  \FunctionTok{ggtitle}\NormalTok{(}\StringTok{"Normal Distribution with changing Mean"}\NormalTok{)}\SpecialCharTok{+}
   \FunctionTok{transition\_states}\NormalTok{(}
\NormalTok{    sims,}
    \AttributeTok{transition\_length =} \DecValTok{1}\NormalTok{,}
    \AttributeTok{state\_length =} \DecValTok{1}
\NormalTok{  )}
  \CommentTok{\#enter\_fade() + }
  \CommentTok{\#exit\_shrink() +}
  \CommentTok{\#ease\_aes(\textquotesingle{}sine{-}in{-}out\textquotesingle{})}
\end{Highlighting}
\end{Shaded}

\subsection{Normal changing sd}\label{normal-changing-sd}

\begin{Shaded}
\begin{Highlighting}[]
\NormalTok{some\_sds}\OtherTok{\textless{}{-}}\FunctionTok{seq}\NormalTok{(}\FloatTok{0.5}\NormalTok{,}\DecValTok{5}\NormalTok{,.}\DecValTok{5}\NormalTok{)}
\NormalTok{all\_df}\OtherTok{\textless{}{-}}\FunctionTok{data.frame}\NormalTok{()}
\ControlFlowTok{for}\NormalTok{(i }\ControlFlowTok{in} \DecValTok{1}\SpecialCharTok{:}\DecValTok{10}\NormalTok{)\{}
\NormalTok{  dnorm\_vec }\OtherTok{\textless{}{-}} \FunctionTok{dnorm}\NormalTok{(}\FunctionTok{seq}\NormalTok{(}\SpecialCharTok{{-}}\DecValTok{10}\NormalTok{,}\DecValTok{10}\NormalTok{,.}\DecValTok{1}\NormalTok{),}\AttributeTok{mean=}\DecValTok{0}\NormalTok{,}\AttributeTok{sd=}\NormalTok{some\_sds[i])}
\NormalTok{  x\_range   }\OtherTok{\textless{}{-}} \FunctionTok{seq}\NormalTok{(}\SpecialCharTok{{-}}\DecValTok{10}\NormalTok{,}\DecValTok{10}\NormalTok{,.}\DecValTok{1}\NormalTok{)}
\NormalTok{  sds }\OtherTok{\textless{}{-}} \FunctionTok{rep}\NormalTok{(some\_sds[i], }\FunctionTok{length}\NormalTok{(x\_range))}
\NormalTok{  sims }\OtherTok{\textless{}{-}} \FunctionTok{rep}\NormalTok{(i, }\FunctionTok{length}\NormalTok{(x\_range))}
\NormalTok{  t\_df}\OtherTok{\textless{}{-}}\FunctionTok{data.frame}\NormalTok{(sims,sds,x\_range,dnorm\_vec)}
\NormalTok{  all\_df}\OtherTok{\textless{}{-}}\FunctionTok{rbind}\NormalTok{(all\_df,t\_df)}
\NormalTok{\}}

\NormalTok{labs\_df}\OtherTok{\textless{}{-}}\FunctionTok{data.frame}\NormalTok{(}\AttributeTok{sims=}\DecValTok{1}\SpecialCharTok{:}\DecValTok{10}\NormalTok{,}
                    \AttributeTok{sds=}\FunctionTok{as.character}\NormalTok{(}\FunctionTok{seq}\NormalTok{(}\FloatTok{0.5}\NormalTok{,}\DecValTok{5}\NormalTok{,.}\DecValTok{5}\NormalTok{)))}

\FunctionTok{ggplot}\NormalTok{(all\_df, }\FunctionTok{aes}\NormalTok{(}\AttributeTok{x=}\NormalTok{x\_range,}\AttributeTok{y=}\NormalTok{dnorm\_vec, }\AttributeTok{frame=}\NormalTok{sims))}\SpecialCharTok{+}
  \FunctionTok{geom\_line}\NormalTok{()}\SpecialCharTok{+}
  \FunctionTok{theme\_classic}\NormalTok{()}\SpecialCharTok{+}
  \FunctionTok{ylab}\NormalTok{(}\StringTok{"probability density"}\NormalTok{)}\SpecialCharTok{+}
  \FunctionTok{xlab}\NormalTok{(}\StringTok{"value"}\NormalTok{)}\SpecialCharTok{+}
  \FunctionTok{ggtitle}\NormalTok{(}\StringTok{"Normal Distribution with changing sd"}\NormalTok{)}\SpecialCharTok{+}
  \FunctionTok{geom\_label}\NormalTok{(}\AttributeTok{data =}\NormalTok{ labs\_df, }\FunctionTok{aes}\NormalTok{(}\AttributeTok{x =} \DecValTok{5}\NormalTok{, }\AttributeTok{y =}\NormalTok{ .}\DecValTok{5}\NormalTok{, }\AttributeTok{label =}\NormalTok{ sds))}\SpecialCharTok{+}
   \FunctionTok{transition\_states}\NormalTok{(}
\NormalTok{    sims,}
    \AttributeTok{transition\_length =} \DecValTok{2}\NormalTok{,}
    \AttributeTok{state\_length =} \DecValTok{1}
\NormalTok{  )}\SpecialCharTok{+}
  \FunctionTok{enter\_fade}\NormalTok{() }\SpecialCharTok{+} 
  \FunctionTok{exit\_shrink}\NormalTok{() }\SpecialCharTok{+}
  \FunctionTok{ease\_aes}\NormalTok{(}\StringTok{\textquotesingle{}sine{-}in{-}out\textquotesingle{}}\NormalTok{)}
\end{Highlighting}
\end{Shaded}

\bookmarksetup{startatroot}

\chapter*{References}\label{references}
\addcontentsline{toc}{chapter}{References}

\markboth{References}{References}

\phantomsection\label{refs}
\begin{CSLReferences}{1}{0}
\bibitem[\citeproctext]{ref-Anscombe1973}
Anscombe, F. J. 1973. {``Graphs in Statistical Analysis.''}
\emph{American Statistician} 27: 17--21.

\bibitem[\citeproctext]{ref-mallory_barnes}
Barnes, Mallory L. 2023. \emph{Statistics for Environmental Science}.

\bibitem[\citeproctext]{ref-behmer2017spatial}
Behmer, Lawrence P, and Matthew JC Crump. 2017. {``Spatial Knowledge
During Skilled Action Sequencing: {Hierarchical} Versus Nonhierarchical
Representations.''} \emph{Attention, Perception, \& Psychophysics} 79
(8): 2435--48. \url{https://doi.org/10.3758/s13414-017-1389-3}.

\bibitem[\citeproctext]{ref-Campbell1963}
Campbell, D. T., and J. C. Stanley. 1963. \emph{Experimental and
Quasi-Experimental Designs for Research}. {Boston, MA}: {Houghton
Mifflin}.

\bibitem[\citeproctext]{ref-Cohen1988}
Cohen, J. 1988. \emph{Statistical Power Analysis for the Behavioral
Sciences}. Second. {Lawrence Erlbaum}.

\bibitem[\citeproctext]{ref-Fisher1922b}
Fisher, R. A. 1922. {``On the Mathematical Foundation of Theoretical
Statistics.''} \emph{Philosophical Transactions of the Royal Society A}
222: 309--68.

\bibitem[\citeproctext]{ref-Hothersall2004}
Hothersall, D. 2004. \emph{History of Psychology}. {McGraw-Hill}.

\bibitem[\citeproctext]{ref-james2015computer}
James, Ella L, Michael B Bonsall, Laura Hoppitt, Elizabeth M Tunbridge,
John R Geddes, Amy L Milton, and Emily A Holmes. 2015. {``Computer Game
Play Reduces Intrusive Memories of Experimental Trauma via
Reconsolidation-Update Mechanisms.''} \emph{Psychological Science} 26
(8): 1201--15. \url{https://doi.org/10.1177/0956797615583071}.

\bibitem[\citeproctext]{ref-Keynes1923}
Keynes, John Maynard. 1923. \emph{A Tract on Monetary Reform}. {London}:
{Macmillan and Company}.

\bibitem[\citeproctext]{ref-matejka2017same}
Matejka, Justin, and George Fitzmaurice. 2017. {``Same Stats, Different
Graphs: Generating Datasets with Varied Appearance and Identical
Statistics Through Simulated Annealing.''} In \emph{Proceedings of the
2017 {CHI} Conference on Human Factors in Computing Systems}, 1290--94.
{ACM}. \url{https://doi.org/10.1145/3025453.3025912}.

\bibitem[\citeproctext]{ref-maul_rethinking_2017}
Maul, Andrew. 2017. {``Rethinking {Traditional Methods} of {Survey
Validation}.''} \emph{Measurement: Interdisciplinary Research and
Perspectives} 15 (2): 51--69.
\url{https://doi.org/10.1080/15366367.2017.1348108}.

\bibitem[\citeproctext]{ref-Meehl1967}
Meehl, P. H. 1967. {``Theory Testing in Psychology and Physics: {A}
Methodological Paradox.''} \emph{Philosophy of Science} 34: 103--15.
\url{https://doi.org/10.1086/288135}.

\bibitem[\citeproctext]{ref-mehr20165}
Mehr, Samuel A, Lee Ann Song, and Elizabeth S Spelke. 2016. {``For
5-Month-Old Infants, Melodies Are Social.''} \emph{Psychological
Science} 27 (4): 486--501.
\url{https://doi.org/10.1177/0956797615626691}.

\bibitem[\citeproctext]{ref-Pfungst1911}
Pfungst, O. 1911. \emph{Clever {Hans} ({The} Horse of {Mr}. Von
{Osten}): {A} Contribution to Experimental Animal and Human Psychology}.
Translated by C. L. Rahn. {New York}: {Henry Holt}.

\bibitem[\citeproctext]{ref-salsburg2001lady}
Salsburg, David. 2001. \emph{The Lady Tasting Tea: {How} Statistics
Revolutionized Science in the Twentieth Century}. {Macmillan}.

\bibitem[\citeproctext]{ref-skilleter1996}
Skilleter, G. A. 1996. {``An Experimental Test of Artifacts from
Repeated Sampling in Soft-Sediments.''} \emph{Journal of Experimental
Marine Biology and Ecology} 205 (1): 137--48.
\url{https://doi.org/10.1016/S0022-0981(96)02617-2}.

\bibitem[\citeproctext]{ref-Student1908}
Student, A. 1908. {``The Probable Error of a Mean.''} \emph{Biometrika}
6: 1--2.

\bibitem[\citeproctext]{ref-vonkuxfcgelgen2021}
von Kügelgen, Julius, Luigi Gresele, and Bernhard Schölkopf. 2021.
{``Simpson's Paradox in COVID-19 Case Fatality Rates: A Mediation
Analysis of Age-Related Causal Effects.''} \emph{IEEE Transactions on
Artificial Intelligence} 2 (1): 18--27.
\url{https://doi.org/10.1109/TAI.2021.3073088}.

\end{CSLReferences}




\end{document}
